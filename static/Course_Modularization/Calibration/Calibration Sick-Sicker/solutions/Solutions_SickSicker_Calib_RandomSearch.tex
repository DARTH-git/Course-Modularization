% Options for packages loaded elsewhere
\PassOptionsToPackage{unicode}{hyperref}
\PassOptionsToPackage{hyphens}{url}
%
\documentclass[
]{article}
\usepackage{lmodern}
\usepackage{amssymb,amsmath}
\usepackage{ifxetex,ifluatex}
\ifnum 0\ifxetex 1\fi\ifluatex 1\fi=0 % if pdftex
  \usepackage[T1]{fontenc}
  \usepackage[utf8]{inputenc}
  \usepackage{textcomp} % provide euro and other symbols
\else % if luatex or xetex
  \usepackage{unicode-math}
  \defaultfontfeatures{Scale=MatchLowercase}
  \defaultfontfeatures[\rmfamily]{Ligatures=TeX,Scale=1}
\fi
% Use upquote if available, for straight quotes in verbatim environments
\IfFileExists{upquote.sty}{\usepackage{upquote}}{}
\IfFileExists{microtype.sty}{% use microtype if available
  \usepackage[]{microtype}
  \UseMicrotypeSet[protrusion]{basicmath} % disable protrusion for tt fonts
}{}
\makeatletter
\@ifundefined{KOMAClassName}{% if non-KOMA class
  \IfFileExists{parskip.sty}{%
    \usepackage{parskip}
  }{% else
    \setlength{\parindent}{0pt}
    \setlength{\parskip}{6pt plus 2pt minus 1pt}}
}{% if KOMA class
  \KOMAoptions{parskip=half}}
\makeatother
\usepackage{xcolor}
\IfFileExists{xurl.sty}{\usepackage{xurl}}{} % add URL line breaks if available
\IfFileExists{bookmark.sty}{\usepackage{bookmark}}{\usepackage{hyperref}}
\hypersetup{
  pdftitle={Calibrating the Sick-Sicker model},
  pdfauthor={The DARTH workgroup},
  hidelinks,
  pdfcreator={LaTeX via pandoc}}
\urlstyle{same} % disable monospaced font for URLs
\usepackage[margin=1in]{geometry}
\usepackage{color}
\usepackage{fancyvrb}
\newcommand{\VerbBar}{|}
\newcommand{\VERB}{\Verb[commandchars=\\\{\}]}
\DefineVerbatimEnvironment{Highlighting}{Verbatim}{commandchars=\\\{\}}
% Add ',fontsize=\small' for more characters per line
\usepackage{framed}
\definecolor{shadecolor}{RGB}{248,248,248}
\newenvironment{Shaded}{\begin{snugshade}}{\end{snugshade}}
\newcommand{\AlertTok}[1]{\textcolor[rgb]{0.94,0.16,0.16}{#1}}
\newcommand{\AnnotationTok}[1]{\textcolor[rgb]{0.56,0.35,0.01}{\textbf{\textit{#1}}}}
\newcommand{\AttributeTok}[1]{\textcolor[rgb]{0.77,0.63,0.00}{#1}}
\newcommand{\BaseNTok}[1]{\textcolor[rgb]{0.00,0.00,0.81}{#1}}
\newcommand{\BuiltInTok}[1]{#1}
\newcommand{\CharTok}[1]{\textcolor[rgb]{0.31,0.60,0.02}{#1}}
\newcommand{\CommentTok}[1]{\textcolor[rgb]{0.56,0.35,0.01}{\textit{#1}}}
\newcommand{\CommentVarTok}[1]{\textcolor[rgb]{0.56,0.35,0.01}{\textbf{\textit{#1}}}}
\newcommand{\ConstantTok}[1]{\textcolor[rgb]{0.00,0.00,0.00}{#1}}
\newcommand{\ControlFlowTok}[1]{\textcolor[rgb]{0.13,0.29,0.53}{\textbf{#1}}}
\newcommand{\DataTypeTok}[1]{\textcolor[rgb]{0.13,0.29,0.53}{#1}}
\newcommand{\DecValTok}[1]{\textcolor[rgb]{0.00,0.00,0.81}{#1}}
\newcommand{\DocumentationTok}[1]{\textcolor[rgb]{0.56,0.35,0.01}{\textbf{\textit{#1}}}}
\newcommand{\ErrorTok}[1]{\textcolor[rgb]{0.64,0.00,0.00}{\textbf{#1}}}
\newcommand{\ExtensionTok}[1]{#1}
\newcommand{\FloatTok}[1]{\textcolor[rgb]{0.00,0.00,0.81}{#1}}
\newcommand{\FunctionTok}[1]{\textcolor[rgb]{0.00,0.00,0.00}{#1}}
\newcommand{\ImportTok}[1]{#1}
\newcommand{\InformationTok}[1]{\textcolor[rgb]{0.56,0.35,0.01}{\textbf{\textit{#1}}}}
\newcommand{\KeywordTok}[1]{\textcolor[rgb]{0.13,0.29,0.53}{\textbf{#1}}}
\newcommand{\NormalTok}[1]{#1}
\newcommand{\OperatorTok}[1]{\textcolor[rgb]{0.81,0.36,0.00}{\textbf{#1}}}
\newcommand{\OtherTok}[1]{\textcolor[rgb]{0.56,0.35,0.01}{#1}}
\newcommand{\PreprocessorTok}[1]{\textcolor[rgb]{0.56,0.35,0.01}{\textit{#1}}}
\newcommand{\RegionMarkerTok}[1]{#1}
\newcommand{\SpecialCharTok}[1]{\textcolor[rgb]{0.00,0.00,0.00}{#1}}
\newcommand{\SpecialStringTok}[1]{\textcolor[rgb]{0.31,0.60,0.02}{#1}}
\newcommand{\StringTok}[1]{\textcolor[rgb]{0.31,0.60,0.02}{#1}}
\newcommand{\VariableTok}[1]{\textcolor[rgb]{0.00,0.00,0.00}{#1}}
\newcommand{\VerbatimStringTok}[1]{\textcolor[rgb]{0.31,0.60,0.02}{#1}}
\newcommand{\WarningTok}[1]{\textcolor[rgb]{0.56,0.35,0.01}{\textbf{\textit{#1}}}}
\usepackage{graphicx,grffile}
\makeatletter
\def\maxwidth{\ifdim\Gin@nat@width>\linewidth\linewidth\else\Gin@nat@width\fi}
\def\maxheight{\ifdim\Gin@nat@height>\textheight\textheight\else\Gin@nat@height\fi}
\makeatother
% Scale images if necessary, so that they will not overflow the page
% margins by default, and it is still possible to overwrite the defaults
% using explicit options in \includegraphics[width, height, ...]{}
\setkeys{Gin}{width=\maxwidth,height=\maxheight,keepaspectratio}
% Set default figure placement to htbp
\makeatletter
\def\fps@figure{htbp}
\makeatother
\setlength{\emergencystretch}{3em} % prevent overfull lines
\providecommand{\tightlist}{%
  \setlength{\itemsep}{0pt}\setlength{\parskip}{0pt}}
\setcounter{secnumdepth}{-\maxdimen} % remove section numbering

\title{Calibrating the Sick-Sicker model}
\usepackage{etoolbox}
\makeatletter
\providecommand{\subtitle}[1]{% add subtitle to \maketitle
  \apptocmd{\@title}{\par {\large #1 \par}}{}{}
}
\makeatother
\subtitle{Random search using Latin-Hypercube Sampling}
\author{The DARTH workgroup}
\date{}

\begin{document}
\maketitle

Developed by the Decision Analysis in R for Technologies in Health
(DARTH) workgroup:

Fernando Alarid-Escudero, PhD (1)

Eva A. Enns, MS, PhD (2)

M.G. Myriam Hunink, MD, PhD (3,4)

Hawre J. Jalal, MD, PhD (5)

Eline M. Krijkamp, MSc (3)

Petros Pechlivanoglou, PhD (6)

Alan Yang, MSc (7)

In collaboration of:

\begin{enumerate}
\def\labelenumi{\arabic{enumi}.}
\tightlist
\item
  Drug Policy Program, Center for Research and Teaching in Economics
  (CIDE) - CONACyT, Aguascalientes, Mexico
\item
  University of Minnesota School of Public Health, Minneapolis, MN, USA
\item
  Erasmus MC, Rotterdam, The Netherlands
\item
  Harvard T.H. Chan School of Public Health, Boston, USA
\item
  University of Pittsburgh Graduate School of Public Health, Pittsburgh,
  PA, USA
\item
  The Hospital for Sick Children, Toronto and University of Toronto,
  Toronto ON, Canada
\item
  The Hospital for Sick Children, Toronto ON, Canada
\end{enumerate}

Please cite our publications when using this code:

\begin{itemize}
\item
  Alarid-Escudero F, Maclehose RF, Peralta Y, Kuntz KM, Enns EA.
  Non-identifiability in model calibration and implications for medical
  decision making. Med Decis Making. 2018; 38(7):810-821.
\item
  Jalal H, Pechlivanoglou P, Krijkamp E, Alarid-Escudero F, Enns E,
  Hunink MG. An Overview of R in Health Decision Sciences. Med Decis
  Making. 2017; 37(3): 735-746.
  \url{https://journals.sagepub.com/doi/abs/10.1177/0272989X16686559}
\end{itemize}

A walkthrough of the code could be found in the follwing link: -
\url{https://darth-git.github.io/calibSMDM2018-materials/}

Copyright 2017, THE HOSPITAL FOR SICK CHILDREN AND THE COLLABORATING
INSTITUTIONS. All rights reserved in Canada, the United States and
worldwide. Copyright, trademarks, trade names and any and all associated
intellectual property are exclusively owned by THE HOSPITAL FOR Sick
CHILDREN and the collaborating institutions. These materials may be
used, reproduced, modified, distributed and adapted with proper
attribution.

\newpage

Change \texttt{eval} to \texttt{TRUE} if you want to knit this document.

\begin{Shaded}
\begin{Highlighting}[]
\KeywordTok{rm}\NormalTok{(}\DataTypeTok{list =} \KeywordTok{ls}\NormalTok{())      }\CommentTok{# clear memory (removes all the variables from the workspace)}
\end{Highlighting}
\end{Shaded}

\hypertarget{calibration-specifications}{%
\section{00 Calibration
Specifications}\label{calibration-specifications}}

Model: Sick-Sicker 4-state Markov Model

Inputs to be calibrated: p\_S1S2, hr\_S1, hr\_S2

Targets: Surv, Prev, PropSick

Calibration method: Random search using Latin-Hypercube Sampling

Goodness-of-fit measure: Sum of Log-Likelihood

\hypertarget{load-packages}{%
\section{01 Load packages}\label{load-packages}}

\begin{Shaded}
\begin{Highlighting}[]
\ControlFlowTok{if}\NormalTok{ (}\OperatorTok{!}\KeywordTok{require}\NormalTok{(}\StringTok{'pacman'}\NormalTok{)) }\KeywordTok{install.packages}\NormalTok{(}\StringTok{'pacman'}\NormalTok{); }\KeywordTok{library}\NormalTok{(pacman) }\CommentTok{# use this package to conveniently install other packages}
\CommentTok{# load (install if required) packages from CRAN}
\KeywordTok{p_load}\NormalTok{(}\StringTok{"lhs"}\NormalTok{, }\StringTok{"plotrix"}\NormalTok{, }\StringTok{"psych"}\NormalTok{, }\StringTok{"scatterplot3d"}\NormalTok{)  }
\CommentTok{# install_github("DARTH-git/darthtools", force = TRUE) Uncomment if there is a newer version}
\KeywordTok{p_load_gh}\NormalTok{(}\StringTok{"DARTH-git/darthtools"}\NormalTok{)}
\end{Highlighting}
\end{Shaded}

\hypertarget{load-target-data}{%
\section{02 Load target data}\label{load-target-data}}

\begin{Shaded}
\begin{Highlighting}[]
\KeywordTok{load}\NormalTok{(}\StringTok{"SickSicker_CalibTargets.RData"}\NormalTok{)}
\NormalTok{lst_targets <-}\StringTok{ }\NormalTok{SickSicker_targets}

\CommentTok{# Plot the targets}

\CommentTok{# TARGET 1: Survival ("Surv")}
\NormalTok{plotrix}\OperatorTok{::}\KeywordTok{plotCI}\NormalTok{(}\DataTypeTok{x =}\NormalTok{ lst_targets}\OperatorTok{$}\NormalTok{Surv}\OperatorTok{$}\NormalTok{time, }\DataTypeTok{y =}\NormalTok{ lst_targets}\OperatorTok{$}\NormalTok{Surv}\OperatorTok{$}\NormalTok{value, }
                \DataTypeTok{ui =}\NormalTok{ lst_targets}\OperatorTok{$}\NormalTok{Surv}\OperatorTok{$}\NormalTok{ub,}
                \DataTypeTok{li =}\NormalTok{ lst_targets}\OperatorTok{$}\NormalTok{Surv}\OperatorTok{$}\NormalTok{lb,}
                \DataTypeTok{ylim =} \KeywordTok{c}\NormalTok{(}\DecValTok{0}\NormalTok{, }\DecValTok{1}\NormalTok{), }
                \DataTypeTok{xlab =} \StringTok{"Time"}\NormalTok{, }\DataTypeTok{ylab =} \StringTok{"Pr Survive"}\NormalTok{)}

\CommentTok{# TARGET 2: Prevalence ("Prev")}
\NormalTok{plotrix}\OperatorTok{::}\KeywordTok{plotCI}\NormalTok{(}\DataTypeTok{x =}\NormalTok{ lst_targets}\OperatorTok{$}\NormalTok{Prev}\OperatorTok{$}\NormalTok{time, }\DataTypeTok{y =}\NormalTok{ lst_targets}\OperatorTok{$}\NormalTok{Prev}\OperatorTok{$}\NormalTok{value,}
                \DataTypeTok{ui =}\NormalTok{ lst_targets}\OperatorTok{$}\NormalTok{Prev}\OperatorTok{$}\NormalTok{ub,}
                \DataTypeTok{li =}\NormalTok{ lst_targets}\OperatorTok{$}\NormalTok{Prev}\OperatorTok{$}\NormalTok{lb,}
                \DataTypeTok{ylim =} \KeywordTok{c}\NormalTok{(}\DecValTok{0}\NormalTok{, }\DecValTok{1}\NormalTok{),}
                \DataTypeTok{xlab =} \StringTok{"Time"}\NormalTok{, }\DataTypeTok{ylab =} \StringTok{"Prev"}\NormalTok{)}

\CommentTok{# TARGET 3: Proportion who are Sick ("PropSick"), among all those afflicted (Sick+Sicker)}
\NormalTok{plotrix}\OperatorTok{::}\KeywordTok{plotCI}\NormalTok{(}\DataTypeTok{x =}\NormalTok{ lst_targets}\OperatorTok{$}\NormalTok{PropSick}\OperatorTok{$}\NormalTok{time, }\DataTypeTok{y =}\NormalTok{ lst_targets}\OperatorTok{$}\NormalTok{PropSick}\OperatorTok{$}\NormalTok{value,}
                \DataTypeTok{ui =}\NormalTok{ lst_targets}\OperatorTok{$}\NormalTok{PropSick}\OperatorTok{$}\NormalTok{ub,}
                \DataTypeTok{li =}\NormalTok{ lst_targets}\OperatorTok{$}\NormalTok{PropSick}\OperatorTok{$}\NormalTok{lb,}
                \DataTypeTok{ylim =} \KeywordTok{c}\NormalTok{(}\DecValTok{0}\NormalTok{, }\DecValTok{1}\NormalTok{),}
                \DataTypeTok{xlab =} \StringTok{"Time"}\NormalTok{, }\DataTypeTok{ylab =} \StringTok{"PropSick"}\NormalTok{)}
\end{Highlighting}
\end{Shaded}

\hypertarget{load-model-as-a-function}{%
\section{03 Load model as a function}\label{load-model-as-a-function}}

\begin{Shaded}
\begin{Highlighting}[]
\CommentTok{# - inputs are parameters to be estimated through calibration}
\CommentTok{# - outputs correspond to the target data}

\CommentTok{# creates the function run_sick_sicker_markov()}
\KeywordTok{source}\NormalTok{(}\StringTok{"SickSicker_MarkovModel_Function.R"}\NormalTok{)}

\CommentTok{# Check that it works}
\NormalTok{v_params_test <-}\StringTok{ }\KeywordTok{c}\NormalTok{(}\DataTypeTok{p_S1S2 =} \FloatTok{0.105}\NormalTok{, }\DataTypeTok{hr_S1 =} \DecValTok{3}\NormalTok{, }\DataTypeTok{hr_S2 =} \DecValTok{10}\NormalTok{)}
\KeywordTok{run_sick_sicker_markov}\NormalTok{(v_params_test) }\CommentTok{# It works!}
\end{Highlighting}
\end{Shaded}

\hypertarget{specify-calibration-parameters}{%
\section{04 Specify calibration
parameters}\label{specify-calibration-parameters}}

\begin{Shaded}
\begin{Highlighting}[]
\CommentTok{# Specify seed (for reproducible sequence of random numbers)}
\KeywordTok{set.seed}\NormalTok{(}\DecValTok{072218}\NormalTok{)}

\CommentTok{# number of random samples}
\NormalTok{n_samp <-}\StringTok{ }\DecValTok{1000}

\CommentTok{# names and number of input parameters to be calibrated}
\NormalTok{v_param_names <-}\StringTok{ }\KeywordTok{c}\NormalTok{(}\StringTok{"p_S1S2"}\NormalTok{,}\StringTok{"hr_S1"}\NormalTok{,}\StringTok{"hr_S2"}\NormalTok{)}
\NormalTok{n_param <-}\StringTok{ }\KeywordTok{length}\NormalTok{(v_param_names)}

\CommentTok{# range on input search space}
\NormalTok{lb <-}\StringTok{ }\KeywordTok{c}\NormalTok{(}\DataTypeTok{p_S1S2 =} \FloatTok{0.01}\NormalTok{, }\DataTypeTok{hr_S1 =} \FloatTok{1.0}\NormalTok{, }\DataTypeTok{hr_S2 =} \DecValTok{5}\NormalTok{) }\CommentTok{# lower bound}
\NormalTok{ub <-}\StringTok{ }\KeywordTok{c}\NormalTok{(}\DataTypeTok{p_S1S2 =} \FloatTok{0.50}\NormalTok{, }\DataTypeTok{hr_S1 =} \FloatTok{4.5}\NormalTok{, }\DataTypeTok{hr_S2 =} \DecValTok{15}\NormalTok{) }\CommentTok{# upper bound}

\CommentTok{# number of calibration targets}
\NormalTok{v_target_names <-}\StringTok{ }\KeywordTok{c}\NormalTok{(}\StringTok{"Surv"}\NormalTok{, }\StringTok{"Prev"}\NormalTok{, }\StringTok{"PropSick"}\NormalTok{)}
\NormalTok{n_target <-}\StringTok{ }\KeywordTok{length}\NormalTok{(v_target_names)}
\end{Highlighting}
\end{Shaded}

\hypertarget{calibrate}{%
\section{05 Calibrate!}\label{calibrate}}

\begin{Shaded}
\begin{Highlighting}[]
\CommentTok{# record start time of calibration}
\NormalTok{t_init <-}\StringTok{ }\KeywordTok{Sys.time}\NormalTok{()}

\CommentTok{###  Generate a random sample of input values  }\AlertTok{###}

\CommentTok{# Sample unit Latin Hypercube}
\NormalTok{m_lhs_unit <-}\StringTok{ }\KeywordTok{randomLHS}\NormalTok{(n_samp, n_param)}

\CommentTok{# Rescale to min/max of each parameter}
\NormalTok{m_param_samp <-}\StringTok{ }\KeywordTok{matrix}\NormalTok{(}\DataTypeTok{nrow=}\NormalTok{n_samp,}\DataTypeTok{ncol=}\NormalTok{n_param)}
\ControlFlowTok{for}\NormalTok{ (i }\ControlFlowTok{in} \DecValTok{1}\OperatorTok{:}\NormalTok{n_param)\{}
\NormalTok{  m_param_samp[,i] <-}\StringTok{ }\KeywordTok{qunif}\NormalTok{(m_lhs_unit[,i],}
                           \DataTypeTok{min =}\NormalTok{ lb[i],}
                           \DataTypeTok{max =}\NormalTok{ ub[i])}
\NormalTok{\}}
\KeywordTok{colnames}\NormalTok{(m_param_samp) <-}\StringTok{ }\NormalTok{v_param_names}

\CommentTok{# view resulting parameter set samples}
\KeywordTok{pairs.panels}\NormalTok{(m_param_samp)}


\CommentTok{###  Run the model for each set of input values }\AlertTok{###}

\CommentTok{# initialize goodness-of-fit vector}
\NormalTok{m_GOF <-}\StringTok{ }\KeywordTok{matrix}\NormalTok{(}\DataTypeTok{nrow =}\NormalTok{ n_samp, }\DataTypeTok{ncol =}\NormalTok{ n_target)}
\KeywordTok{colnames}\NormalTok{(m_GOF) <-}\StringTok{ }\KeywordTok{paste0}\NormalTok{(v_target_names, }\StringTok{"_fit"}\NormalTok{)}

\CommentTok{# loop through sampled sets of input values}
\ControlFlowTok{for}\NormalTok{ (j }\ControlFlowTok{in} \DecValTok{1}\OperatorTok{:}\NormalTok{n_samp)\{}
  
  \CommentTok{###  Run model for a given parameter set  }\AlertTok{###}
\NormalTok{  model_res <-}\StringTok{ }\KeywordTok{run_sick_sicker_markov}\NormalTok{(}\DataTypeTok{v_params =}\NormalTok{ m_param_samp[j, ])}
  
  
  \CommentTok{###  Calculate goodness-of-fit of model outputs to targets  }\AlertTok{###}

  \CommentTok{# TARGET 1: Survival ("Surv")}
  \CommentTok{# log likelihood  }
\NormalTok{  m_GOF[j,}\DecValTok{1}\NormalTok{] <-}\StringTok{ }\KeywordTok{sum}\NormalTok{(}\KeywordTok{dnorm}\NormalTok{(}\DataTypeTok{x =}\NormalTok{ lst_targets}\OperatorTok{$}\NormalTok{Surv}\OperatorTok{$}\NormalTok{value,}
                       \DataTypeTok{mean =}\NormalTok{ model_res}\OperatorTok{$}\NormalTok{Surv,}
                       \DataTypeTok{sd =}\NormalTok{ lst_targets}\OperatorTok{$}\NormalTok{Surv}\OperatorTok{$}\NormalTok{se,}
                       \DataTypeTok{log =}\NormalTok{ T))}
  
  \CommentTok{# weighted sum of squared errors (alternative to log likelihood)}
  \CommentTok{# w <- 1/(lst_targets$Surv$se^2)}
  \CommentTok{# m_GOF[j,1] <- -sum(w*(lst_targets$Surv$value - v_res)^2)}
  
  
  \CommentTok{# TARGET 2: "Prev"}
  \CommentTok{# log likelihood}
\NormalTok{  m_GOF[j,}\DecValTok{2}\NormalTok{] <-}\StringTok{ }\KeywordTok{sum}\NormalTok{(}\KeywordTok{dnorm}\NormalTok{(}\DataTypeTok{x =}\NormalTok{ lst_targets}\OperatorTok{$}\NormalTok{Prev}\OperatorTok{$}\NormalTok{value,}
                         \DataTypeTok{mean =}\NormalTok{ model_res}\OperatorTok{$}\NormalTok{Prev,}
                         \DataTypeTok{sd =}\NormalTok{ lst_targets}\OperatorTok{$}\NormalTok{Prev}\OperatorTok{$}\NormalTok{se,}
                         \DataTypeTok{log =}\NormalTok{ T))}
  
  \CommentTok{# TARGET 3: "PropSick"}
  \CommentTok{# log likelihood}
\NormalTok{  m_GOF[j,}\DecValTok{3}\NormalTok{] <-}\StringTok{ }\KeywordTok{sum}\NormalTok{(}\KeywordTok{dnorm}\NormalTok{(}\DataTypeTok{x =}\NormalTok{ lst_targets}\OperatorTok{$}\NormalTok{PropSick}\OperatorTok{$}\NormalTok{value,}
                         \DataTypeTok{mean =}\NormalTok{ model_res}\OperatorTok{$}\NormalTok{PropSick,}
                         \DataTypeTok{sd =}\NormalTok{ lst_targets}\OperatorTok{$}\NormalTok{PropSick}\OperatorTok{$}\NormalTok{se,}
                         \DataTypeTok{log =}\NormalTok{ T))}
  
  
\NormalTok{\} }\CommentTok{# End loop over sampled parameter sets}


\CommentTok{###  Combine fits to the different targets into single GOF  }\AlertTok{###}
\CommentTok{# can give different targets different weights}
\NormalTok{v_weights <-}\StringTok{ }\KeywordTok{matrix}\NormalTok{(}\DecValTok{1}\NormalTok{, }\DataTypeTok{nrow =}\NormalTok{ n_target, }\DataTypeTok{ncol =} \DecValTok{1}\NormalTok{)}
\CommentTok{# matrix multiplication to calculate weight sum of each GOF matrix row}
\NormalTok{v_GOF_overall <-}\StringTok{ }\KeywordTok{c}\NormalTok{(m_GOF}\OperatorTok\NormalTok{v_weights)}
\CommentTok{# Store in GOF matrix with column name "Overall"}
\NormalTok{m_GOF <-}\StringTok{ }\KeywordTok{cbind}\NormalTok{(m_GOF,}\DataTypeTok{Overall_fit=}\NormalTok{v_GOF_overall)}

\CommentTok{# Calculate computation time}
\NormalTok{comp_time <-}\StringTok{ }\KeywordTok{Sys.time}\NormalTok{() }\OperatorTok{-}\StringTok{ }\NormalTok{t_init}
\end{Highlighting}
\end{Shaded}

\hypertarget{exploring-best-fitting-input-sets}{%
\section{06 Exploring best-fitting input
sets}\label{exploring-best-fitting-input-sets}}

\begin{Shaded}
\begin{Highlighting}[]
\CommentTok{# Arrange parameter sets in order of fit}
\NormalTok{m_calib_res <-}\StringTok{ }\KeywordTok{cbind}\NormalTok{(m_param_samp,m_GOF)}
\NormalTok{m_calib_res <-}\StringTok{ }\NormalTok{m_calib_res[}\KeywordTok{order}\NormalTok{(}\OperatorTok{-}\NormalTok{m_calib_res[,}\StringTok{"Overall_fit"}\NormalTok{]),]}

\CommentTok{# Examine the top 10 best-fitting sets}
\NormalTok{m_calib_res[}\DecValTok{1}\OperatorTok{:}\DecValTok{10}\NormalTok{,]}

\CommentTok{# Plot the top 100 (top 10%)}
\KeywordTok{scatterplot3d}\NormalTok{(}\DataTypeTok{x =}\NormalTok{ m_calib_res[}\DecValTok{1}\OperatorTok{:}\DecValTok{100}\NormalTok{, }\DecValTok{1}\NormalTok{],}
              \DataTypeTok{y =}\NormalTok{ m_calib_res[}\DecValTok{1}\OperatorTok{:}\DecValTok{100}\NormalTok{, }\DecValTok{2}\NormalTok{],}
              \DataTypeTok{z =}\NormalTok{ m_calib_res[}\DecValTok{1}\OperatorTok{:}\DecValTok{100}\NormalTok{, }\DecValTok{3}\NormalTok{],}
              \DataTypeTok{xlim =} \KeywordTok{c}\NormalTok{(lb[}\DecValTok{1}\NormalTok{],ub[}\DecValTok{1}\NormalTok{]), }\DataTypeTok{ylim =} \KeywordTok{c}\NormalTok{(lb[}\DecValTok{2}\NormalTok{],ub[}\DecValTok{2}\NormalTok{]), }\DataTypeTok{zlim =} \KeywordTok{c}\NormalTok{(lb[}\DecValTok{3}\NormalTok{],ub[}\DecValTok{3}\NormalTok{]),}
              \DataTypeTok{xlab =}\NormalTok{ v_param_names[}\DecValTok{1}\NormalTok{], }\DataTypeTok{ylab =}\NormalTok{ v_param_names[}\DecValTok{2}\NormalTok{], }\DataTypeTok{zlab =}\NormalTok{ v_param_names[}\DecValTok{3}\NormalTok{])}

\CommentTok{# Pairwise comparison of top 100 sets}
\KeywordTok{pairs.panels}\NormalTok{(m_calib_res[}\DecValTok{1}\OperatorTok{:}\DecValTok{100}\NormalTok{,v_param_names])}

\CommentTok{### Plot model-predicted output at best set vs targets }\AlertTok{###}
\NormalTok{v_out_best <-}\StringTok{ }\KeywordTok{run_sick_sicker_markov}\NormalTok{(m_calib_res[}\DecValTok{1}\NormalTok{,])}

\CommentTok{# TARGET 1: Survival ("Surv")}
\NormalTok{plotrix}\OperatorTok{::}\KeywordTok{plotCI}\NormalTok{(}\DataTypeTok{x =}\NormalTok{ lst_targets}\OperatorTok{$}\NormalTok{Surv}\OperatorTok{$}\NormalTok{time, }\DataTypeTok{y =}\NormalTok{ lst_targets}\OperatorTok{$}\NormalTok{Surv}\OperatorTok{$}\NormalTok{value, }
                \DataTypeTok{ui =}\NormalTok{ lst_targets}\OperatorTok{$}\NormalTok{Surv}\OperatorTok{$}\NormalTok{ub,}
                \DataTypeTok{li =}\NormalTok{ lst_targets}\OperatorTok{$}\NormalTok{Surv}\OperatorTok{$}\NormalTok{lb,}
                \DataTypeTok{ylim =} \KeywordTok{c}\NormalTok{(}\DecValTok{0}\NormalTok{, }\DecValTok{1}\NormalTok{), }
                \DataTypeTok{xlab =} \StringTok{"Time"}\NormalTok{, }\DataTypeTok{ylab =} \StringTok{"Pr Survive"}\NormalTok{)}
\KeywordTok{points}\NormalTok{(}\DataTypeTok{x =}\NormalTok{ lst_targets}\OperatorTok{$}\NormalTok{Surv}\OperatorTok{$}\NormalTok{time, }
       \DataTypeTok{y =}\NormalTok{ v_out_best}\OperatorTok{$}\NormalTok{Surv, }
       \DataTypeTok{pch =} \DecValTok{8}\NormalTok{, }\DataTypeTok{col =} \StringTok{"red"}\NormalTok{)}
\KeywordTok{legend}\NormalTok{(}\StringTok{"topright"}\NormalTok{, }
       \DataTypeTok{legend =} \KeywordTok{c}\NormalTok{(}\StringTok{"Target"}\NormalTok{, }\StringTok{"Model-predicted output"}\NormalTok{),}
       \DataTypeTok{col =} \KeywordTok{c}\NormalTok{(}\StringTok{"black"}\NormalTok{, }\StringTok{"red"}\NormalTok{), }\DataTypeTok{pch =} \KeywordTok{c}\NormalTok{(}\DecValTok{1}\NormalTok{, }\DecValTok{8}\NormalTok{))}

\CommentTok{# TARGET 2: "Prev"}
\NormalTok{plotrix}\OperatorTok{::}\KeywordTok{plotCI}\NormalTok{(}\DataTypeTok{x =}\NormalTok{ lst_targets}\OperatorTok{$}\NormalTok{Prev}\OperatorTok{$}\NormalTok{time, }\DataTypeTok{y =}\NormalTok{ lst_targets}\OperatorTok{$}\NormalTok{Prev}\OperatorTok{$}\NormalTok{value,}
                \DataTypeTok{ui =}\NormalTok{ lst_targets}\OperatorTok{$}\NormalTok{Prev}\OperatorTok{$}\NormalTok{ub,}
                \DataTypeTok{li =}\NormalTok{ lst_targets}\OperatorTok{$}\NormalTok{Prev}\OperatorTok{$}\NormalTok{lb,}
                \DataTypeTok{ylim =} \KeywordTok{c}\NormalTok{(}\DecValTok{0}\NormalTok{, }\DecValTok{1}\NormalTok{),}
                \DataTypeTok{xlab =} \StringTok{"Time"}\NormalTok{, }\DataTypeTok{ylab =} \StringTok{"Prev"}\NormalTok{)}
\KeywordTok{points}\NormalTok{(}\DataTypeTok{x =}\NormalTok{ lst_targets}\OperatorTok{$}\NormalTok{Prev}\OperatorTok{$}\NormalTok{time,}
       \DataTypeTok{y =}\NormalTok{ v_out_best}\OperatorTok{$}\NormalTok{Prev,}
       \DataTypeTok{pch =} \DecValTok{8}\NormalTok{, }\DataTypeTok{col =} \StringTok{"red"}\NormalTok{)}
\KeywordTok{legend}\NormalTok{(}\StringTok{"topright"}\NormalTok{,}
       \DataTypeTok{legend =} \KeywordTok{c}\NormalTok{(}\StringTok{"Target"}\NormalTok{, }\StringTok{"Model-predicted output"}\NormalTok{),}
       \DataTypeTok{col =} \KeywordTok{c}\NormalTok{(}\StringTok{"black"}\NormalTok{, }\StringTok{"red"}\NormalTok{), }\DataTypeTok{pch =} \KeywordTok{c}\NormalTok{(}\DecValTok{1}\NormalTok{, }\DecValTok{8}\NormalTok{))}

\CommentTok{# TARGET 3: "PropSick"}
\NormalTok{plotrix}\OperatorTok{::}\KeywordTok{plotCI}\NormalTok{(}\DataTypeTok{x =}\NormalTok{ lst_targets}\OperatorTok{$}\NormalTok{PropSick}\OperatorTok{$}\NormalTok{time, }\DataTypeTok{y =}\NormalTok{ lst_targets}\OperatorTok{$}\NormalTok{PropSick}\OperatorTok{$}\NormalTok{value,}
                \DataTypeTok{ui =}\NormalTok{ lst_targets}\OperatorTok{$}\NormalTok{PropSick}\OperatorTok{$}\NormalTok{ub,}
                \DataTypeTok{li =}\NormalTok{ lst_targets}\OperatorTok{$}\NormalTok{PropSick}\OperatorTok{$}\NormalTok{lb,}
                \DataTypeTok{ylim =} \KeywordTok{c}\NormalTok{(}\DecValTok{0}\NormalTok{, }\DecValTok{1}\NormalTok{),}
                \DataTypeTok{xlab =} \StringTok{"Time"}\NormalTok{, }\DataTypeTok{ylab =} \StringTok{"PropSick"}\NormalTok{)}
\KeywordTok{points}\NormalTok{(}\DataTypeTok{x =}\NormalTok{ lst_targets}\OperatorTok{$}\NormalTok{PropSick}\OperatorTok{$}\NormalTok{time,}
       \DataTypeTok{y =}\NormalTok{ v_out_best}\OperatorTok{$}\NormalTok{PropSick,}
       \DataTypeTok{pch =} \DecValTok{8}\NormalTok{, }\DataTypeTok{col =} \StringTok{"red"}\NormalTok{)}
\KeywordTok{legend}\NormalTok{(}\StringTok{"topright"}\NormalTok{,}
       \DataTypeTok{legend =} \KeywordTok{c}\NormalTok{(}\StringTok{"Target"}\NormalTok{, }\StringTok{"Model-predicted output"}\NormalTok{),}
       \DataTypeTok{col =} \KeywordTok{c}\NormalTok{(}\StringTok{"black"}\NormalTok{, }\StringTok{"red"}\NormalTok{), }\DataTypeTok{pch =} \KeywordTok{c}\NormalTok{(}\DecValTok{1}\NormalTok{, }\DecValTok{8}\NormalTok{))}
\end{Highlighting}
\end{Shaded}

\end{document}
