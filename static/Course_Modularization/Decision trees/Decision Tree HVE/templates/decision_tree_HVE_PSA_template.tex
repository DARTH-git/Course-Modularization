% Options for packages loaded elsewhere
\PassOptionsToPackage{unicode}{hyperref}
\PassOptionsToPackage{hyphens}{url}
%
\documentclass[
]{article}
\usepackage{lmodern}
\usepackage{amssymb,amsmath}
\usepackage{ifxetex,ifluatex}
\ifnum 0\ifxetex 1\fi\ifluatex 1\fi=0 % if pdftex
  \usepackage[T1]{fontenc}
  \usepackage[utf8]{inputenc}
  \usepackage{textcomp} % provide euro and other symbols
\else % if luatex or xetex
  \usepackage{unicode-math}
  \defaultfontfeatures{Scale=MatchLowercase}
  \defaultfontfeatures[\rmfamily]{Ligatures=TeX,Scale=1}
\fi
% Use upquote if available, for straight quotes in verbatim environments
\IfFileExists{upquote.sty}{\usepackage{upquote}}{}
\IfFileExists{microtype.sty}{% use microtype if available
  \usepackage[]{microtype}
  \UseMicrotypeSet[protrusion]{basicmath} % disable protrusion for tt fonts
}{}
\makeatletter
\@ifundefined{KOMAClassName}{% if non-KOMA class
  \IfFileExists{parskip.sty}{%
    \usepackage{parskip}
  }{% else
    \setlength{\parindent}{0pt}
    \setlength{\parskip}{6pt plus 2pt minus 1pt}}
}{% if KOMA class
  \KOMAoptions{parskip=half}}
\makeatother
\usepackage{xcolor}
\IfFileExists{xurl.sty}{\usepackage{xurl}}{} % add URL line breaks if available
\IfFileExists{bookmark.sty}{\usepackage{bookmark}}{\usepackage{hyperref}}
\hypersetup{
  pdftitle={PSA: Three-strategy decision tree in R - HVE},
  pdfauthor={The DARTH workgroup},
  hidelinks,
  pdfcreator={LaTeX via pandoc}}
\urlstyle{same} % disable monospaced font for URLs
\usepackage[margin=1in]{geometry}
\usepackage{color}
\usepackage{fancyvrb}
\newcommand{\VerbBar}{|}
\newcommand{\VERB}{\Verb[commandchars=\\\{\}]}
\DefineVerbatimEnvironment{Highlighting}{Verbatim}{commandchars=\\\{\}}
% Add ',fontsize=\small' for more characters per line
\usepackage{framed}
\definecolor{shadecolor}{RGB}{248,248,248}
\newenvironment{Shaded}{\begin{snugshade}}{\end{snugshade}}
\newcommand{\AlertTok}[1]{\textcolor[rgb]{0.94,0.16,0.16}{#1}}
\newcommand{\AnnotationTok}[1]{\textcolor[rgb]{0.56,0.35,0.01}{\textbf{\textit{#1}}}}
\newcommand{\AttributeTok}[1]{\textcolor[rgb]{0.77,0.63,0.00}{#1}}
\newcommand{\BaseNTok}[1]{\textcolor[rgb]{0.00,0.00,0.81}{#1}}
\newcommand{\BuiltInTok}[1]{#1}
\newcommand{\CharTok}[1]{\textcolor[rgb]{0.31,0.60,0.02}{#1}}
\newcommand{\CommentTok}[1]{\textcolor[rgb]{0.56,0.35,0.01}{\textit{#1}}}
\newcommand{\CommentVarTok}[1]{\textcolor[rgb]{0.56,0.35,0.01}{\textbf{\textit{#1}}}}
\newcommand{\ConstantTok}[1]{\textcolor[rgb]{0.00,0.00,0.00}{#1}}
\newcommand{\ControlFlowTok}[1]{\textcolor[rgb]{0.13,0.29,0.53}{\textbf{#1}}}
\newcommand{\DataTypeTok}[1]{\textcolor[rgb]{0.13,0.29,0.53}{#1}}
\newcommand{\DecValTok}[1]{\textcolor[rgb]{0.00,0.00,0.81}{#1}}
\newcommand{\DocumentationTok}[1]{\textcolor[rgb]{0.56,0.35,0.01}{\textbf{\textit{#1}}}}
\newcommand{\ErrorTok}[1]{\textcolor[rgb]{0.64,0.00,0.00}{\textbf{#1}}}
\newcommand{\ExtensionTok}[1]{#1}
\newcommand{\FloatTok}[1]{\textcolor[rgb]{0.00,0.00,0.81}{#1}}
\newcommand{\FunctionTok}[1]{\textcolor[rgb]{0.00,0.00,0.00}{#1}}
\newcommand{\ImportTok}[1]{#1}
\newcommand{\InformationTok}[1]{\textcolor[rgb]{0.56,0.35,0.01}{\textbf{\textit{#1}}}}
\newcommand{\KeywordTok}[1]{\textcolor[rgb]{0.13,0.29,0.53}{\textbf{#1}}}
\newcommand{\NormalTok}[1]{#1}
\newcommand{\OperatorTok}[1]{\textcolor[rgb]{0.81,0.36,0.00}{\textbf{#1}}}
\newcommand{\OtherTok}[1]{\textcolor[rgb]{0.56,0.35,0.01}{#1}}
\newcommand{\PreprocessorTok}[1]{\textcolor[rgb]{0.56,0.35,0.01}{\textit{#1}}}
\newcommand{\RegionMarkerTok}[1]{#1}
\newcommand{\SpecialCharTok}[1]{\textcolor[rgb]{0.00,0.00,0.00}{#1}}
\newcommand{\SpecialStringTok}[1]{\textcolor[rgb]{0.31,0.60,0.02}{#1}}
\newcommand{\StringTok}[1]{\textcolor[rgb]{0.31,0.60,0.02}{#1}}
\newcommand{\VariableTok}[1]{\textcolor[rgb]{0.00,0.00,0.00}{#1}}
\newcommand{\VerbatimStringTok}[1]{\textcolor[rgb]{0.31,0.60,0.02}{#1}}
\newcommand{\WarningTok}[1]{\textcolor[rgb]{0.56,0.35,0.01}{\textbf{\textit{#1}}}}
\usepackage{graphicx,grffile}
\makeatletter
\def\maxwidth{\ifdim\Gin@nat@width>\linewidth\linewidth\else\Gin@nat@width\fi}
\def\maxheight{\ifdim\Gin@nat@height>\textheight\textheight\else\Gin@nat@height\fi}
\makeatother
% Scale images if necessary, so that they will not overflow the page
% margins by default, and it is still possible to overwrite the defaults
% using explicit options in \includegraphics[width, height, ...]{}
\setkeys{Gin}{width=\maxwidth,height=\maxheight,keepaspectratio}
% Set default figure placement to htbp
\makeatletter
\def\fps@figure{htbp}
\makeatother
\setlength{\emergencystretch}{3em} % prevent overfull lines
\providecommand{\tightlist}{%
  \setlength{\itemsep}{0pt}\setlength{\parskip}{0pt}}
\setcounter{secnumdepth}{-\maxdimen} % remove section numbering

\title{PSA: Three-strategy decision tree in R - HVE}
\author{The DARTH workgroup}
\date{}

\begin{document}
\maketitle

Developed by the Decision Analysis in R for Technologies in Health
(DARTH) workgroup:

Fernando Alarid-Escudero, PhD (1)

Eva A. Enns, MS, PhD (2)

M.G. Myriam Hunink, MD, PhD (3,4)

Hawre J. Jalal, MD, PhD (5)

Eline M. Krijkamp, MSc (3)

Petros Pechlivanoglou, PhD (6,7)

Alan Yang, MSc (7)

In collaboration of:

\begin{enumerate}
\def\labelenumi{\arabic{enumi}.}
\tightlist
\item
  Drug Policy Program, Center for Research and Teaching in Economics
  (CIDE) - CONACyT, Aguascalientes, Mexico
\item
  University of Minnesota School of Public Health, Minneapolis, MN, USA
\item
  Erasmus MC, Rotterdam, The Netherlands
\item
  Harvard T.H. Chan School of Public Health, Boston, USA
\item
  University of Pittsburgh Graduate School of Public Health, Pittsburgh,
  PA, USA
\item
  University of Toronto, Toronto ON, Canada
\item
  The Hospital for Sick Children, Toronto ON, Canada
\end{enumerate}

Please cite our publications when using this code:

\begin{itemize}
\item
  Jalal H, Pechlivanoglou P, Krijkamp E, Alarid-Escudero F, Enns E,
  Hunink MG. An Overview of R in Health Decision Sciences. Med Decis
  Making. 2017; 37(3): 735-746.
  \url{https://journals.sagepub.com/doi/abs/10.1177/0272989X16686559}
\item
  Krijkamp EM, Alarid-Escudero F, Enns EA, Jalal HJ, Hunink MGM,
  Pechlivanoglou P. Microsimulation modeling for health decision
  sciences using R: A tutorial. Med Decis Making. 2018;38(3):400--22.
  \url{https://journals.sagepub.com/doi/abs/10.1177/0272989X18754513}
\item
  Krijkamp EM, Alarid-Escudero F, Enns E, Pechlivanoglou P, Hunink MM,
  Jalal H. A Multidimensional Array Representation of State-Transition
  Model Dynamics. Med Decis Making. 2020 Online first.
  \url{https://doi.org/10.1177/0272989X19893973}
\end{itemize}

Copyright 2017, THE HOSPITAL FOR SICK CHILDREN AND THE COLLABORATING
INSTITUTIONS. All rights reserved in Canada, the United States and
worldwide. Copyright, trademarks, trade names and any and all associated
intellectual property are exclusively owned by THE HOSPITAL FOR Sick
CHILDREN and the collaborating institutions. These materials may be
used, reproduced, modified, distributed and adapted with proper
attribution.

\newpage

Change \texttt{eval} to \texttt{TRUE} if you want to knit this document.

\begin{Shaded}
\begin{Highlighting}[]
\KeywordTok{rm}\NormalTok{(}\DataTypeTok{list =} \KeywordTok{ls}\NormalTok{())      }\CommentTok{# clear memory (removes all the variables from the workspace)}
\end{Highlighting}
\end{Shaded}

\hypertarget{load-packages}{%
\section{01 Load packages}\label{load-packages}}

\begin{Shaded}
\begin{Highlighting}[]
\ControlFlowTok{if}\NormalTok{ (}\OperatorTok{!}\KeywordTok{require}\NormalTok{(}\StringTok{'pacman'}\NormalTok{)) }\KeywordTok{install.packages}\NormalTok{(}\StringTok{'pacman'}\NormalTok{); }\KeywordTok{library}\NormalTok{(pacman) }\CommentTok{# use this package to conveniently install other packages}
\CommentTok{# load (install if required) packages from CRAN}
\KeywordTok{p_load}\NormalTok{(}\StringTok{"here"}\NormalTok{, }\StringTok{"dplyr"}\NormalTok{, }\StringTok{"devtools"}\NormalTok{, }\StringTok{"scales"}\NormalTok{, }\StringTok{"ellipse"}\NormalTok{, }\StringTok{"ggplot2"}\NormalTok{, }\StringTok{"lazyeval"}\NormalTok{, }\StringTok{"igraph"}\NormalTok{, }\StringTok{"truncnorm"}\NormalTok{, }\StringTok{"ggraph"}\NormalTok{, }\StringTok{"reshape2"}\NormalTok{, }\StringTok{"knitr"}\NormalTok{, }\StringTok{"stringr"}\NormalTok{)                                           }
\CommentTok{# load (install if required) packages from GitHub}
\CommentTok{# install_github("DARTH-git/dampack", force = TRUE) # Uncomment if there is a newer version}
\CommentTok{# install_github("DARTH-git/dectree", force = TRUE) # Uncomment if there is a newer version}
\KeywordTok{p_load_gh}\NormalTok{(}\StringTok{"DARTH-git/dampack"}\NormalTok{, }\StringTok{"DARTH-git/dectree"}\NormalTok{) }\CommentTok{# load one or more GitHub packages  }
\end{Highlighting}
\end{Shaded}

\hypertarget{load-functions}{%
\section{02 Load functions}\label{load-functions}}

\begin{Shaded}
\begin{Highlighting}[]
\KeywordTok{source}\NormalTok{(}\KeywordTok{here}\NormalTok{(}\StringTok{'functions'}\NormalTok{,}\StringTok{'Functions.R'}\NormalTok{))}
\end{Highlighting}
\end{Shaded}

\hypertarget{define-parameter-input-values}{%
\section{03 Define parameter input
values}\label{define-parameter-input-values}}

\begin{Shaded}
\begin{Highlighting}[]
\NormalTok{v_names_str   <-}\StringTok{ }\KeywordTok{c}\NormalTok{(}\StringTok{"No Tx"}\NormalTok{, }\StringTok{"Tx All"}\NormalTok{, }\StringTok{"Biopsy"}\NormalTok{)    }\CommentTok{# names of strategies}
\NormalTok{n_str         <-}\StringTok{ }\KeywordTok{length}\NormalTok{(v_names_str)               }\CommentTok{# number of strategies}
\NormalTok{wtp           <-}\StringTok{ }\DecValTok{100000}                            \CommentTok{# willingness to pay threshold}

\CommentTok{# Probabilities}
\NormalTok{p_HVE         <-}\StringTok{ }\FloatTok{0.52}   \CommentTok{# prevalence of HVE}
\NormalTok{p_HVE_comp    <-}\StringTok{ }\FloatTok{0.71}   \CommentTok{# complications with untreated HVE}
\NormalTok{p_OVE_comp    <-}\StringTok{ }\FloatTok{0.01}   \CommentTok{# complications with untreated OVE}
\NormalTok{p_HVE_comp_tx <-}\StringTok{ }\FloatTok{0.36}   \CommentTok{# complications with treated HVE}
\NormalTok{p_OVE_comp_tx <-}\StringTok{ }\FloatTok{0.20}   \CommentTok{# complications with treated OVE}
\NormalTok{p_biopsy_comp <-}\StringTok{ }\FloatTok{0.05}   \CommentTok{# probability of complications due to biopsy}

\CommentTok{# Costs}
\NormalTok{c_VE          <-}\StringTok{ }\DecValTok{1200}   \CommentTok{# cost of viral encephalitis care without complications}
\NormalTok{c_VE_comp     <-}\StringTok{ }\DecValTok{9000}   \CommentTok{# cost of viral encephalitis care with complications}
\NormalTok{c_tx          <-}\StringTok{ }\DecValTok{9500}   \CommentTok{# cost of treatment}
\NormalTok{c_biopsy      <-}\StringTok{ }\DecValTok{25000}  \CommentTok{# cost of brain biopsy}

\CommentTok{# QALYs}
\NormalTok{q_VE          <-}\StringTok{ }\DecValTok{20}     \CommentTok{# remaining QALYs for those without VE-related complications}
\NormalTok{q_VE_comp     <-}\StringTok{ }\DecValTok{19}     \CommentTok{# remaining QALYs for those with VE-related complications}
\NormalTok{q_loss_biopsy <-}\StringTok{ }\FloatTok{-0.01}  \CommentTok{# one-time QALY loss due to brain biopsy}

\CommentTok{# store the parameters into a list}
\NormalTok{l_params_all <-}\StringTok{ }\KeywordTok{as.list}\NormalTok{(}\KeywordTok{data.frame}\NormalTok{(p_HVE, p_HVE_comp, p_OVE_comp, p_HVE_comp_tx, p_OVE_comp_tx, p_biopsy_comp, }
\NormalTok{                                   c_VE, c_VE_comp, c_tx, c_biopsy, }
\NormalTok{                                   q_VE, q_VE_comp, q_loss_biopsy))}
\CommentTok{# store the names of the parameters into a vector}
\NormalTok{v_names_params <-}\StringTok{ }\KeywordTok{c}\NormalTok{(}\StringTok{'p_HVE'}\NormalTok{, }\StringTok{'p_HVE_comp'}\NormalTok{, }\StringTok{'p_OVE_comp'}\NormalTok{, }\StringTok{'p_HVE_comp_tx'}\NormalTok{, }\StringTok{'p_OVE_comp_tx'}\NormalTok{, }\StringTok{'p_biopsy_comp'}\NormalTok{, }
                    \StringTok{'c_VE'}\NormalTok{, }\StringTok{'c_VE_comp'}\NormalTok{,  }\StringTok{'c_tx'}\NormalTok{, }\StringTok{'c_biopsy'}\NormalTok{, }\StringTok{'q_VE'}\NormalTok{, }\StringTok{'q_VE_comp'}\NormalTok{, }\StringTok{'q_loss_biopsy'}\NormalTok{)}
\end{Highlighting}
\end{Shaded}

\hypertarget{create-and-run-decision-tree-model}{%
\section{04 Create and run decision tree
model}\label{create-and-run-decision-tree-model}}

\begin{Shaded}
\begin{Highlighting}[]
\NormalTok{decision_tree_HVE_output <-}\StringTok{ }\KeywordTok{with}\NormalTok{(}\KeywordTok{as.list}\NormalTok{(l_params_all), \{}
  
  \CommentTok{# Create vector of weights for each strategy }
  
\NormalTok{  v_w_no_tx  <-}\StringTok{ }\KeywordTok{c}\NormalTok{(  p_HVE  }\OperatorTok{*}\StringTok{    }\NormalTok{p_HVE_comp  ,  }\CommentTok{# HVE, complications}
\NormalTok{                    p_HVE  }\OperatorTok{*}\StringTok{ }\NormalTok{(}\DecValTok{1}\OperatorTok{-}\NormalTok{p_HVE_comp) ,  }\CommentTok{# HVE, no complications}
\NormalTok{                 (}\DecValTok{1}\OperatorTok{-}\NormalTok{p_HVE) }\OperatorTok{*}\StringTok{    }\NormalTok{p_OVE_comp  ,  }\CommentTok{# OVE, complications}
\NormalTok{                 (}\DecValTok{1}\OperatorTok{-}\NormalTok{p_HVE) }\OperatorTok{*}\StringTok{ }\NormalTok{(}\DecValTok{1}\OperatorTok{-}\NormalTok{p_OVE_comp))   }\CommentTok{# OVE, no complications}
  
\NormalTok{  v_w_tx     <-}\StringTok{ }\CommentTok{# your turn}
\StringTok{    }
\StringTok{  }\NormalTok{v_w_biopsy <-}\StringTok{ }\CommentTok{# your turn}
\StringTok{  }
\StringTok{  }\CommentTok{# Create vector of outcomes (QALYs) for each strategy }
\StringTok{  }
\StringTok{  }\NormalTok{v_qaly_no_tx  <-}\StringTok{ }\KeywordTok{c}\NormalTok{(q_VE_comp ,  }\CommentTok{# HVE, complications}
\NormalTok{                     q_VE      ,  }\CommentTok{# HVE, no complications}
\NormalTok{                     q_VE_comp ,  }\CommentTok{# OVE, complications}
\NormalTok{                     q_VE)        }\CommentTok{# OVE, no complications}
  
\NormalTok{  v_qaly_tx     <-}\StringTok{ }\CommentTok{# your turn}
\StringTok{    }
\StringTok{  }\NormalTok{v_qaly_biopsy <-}\StringTok{ }\CommentTok{# your turn}
\StringTok{  }
\StringTok{  }\CommentTok{# Create vector of costs for each strategy }

\StringTok{  }\NormalTok{v_cost_no_tx  <-}\StringTok{ }\KeywordTok{c}\NormalTok{(c_VE_comp ,  }\CommentTok{# HVE, complications}
\NormalTok{                     c_VE      ,  }\CommentTok{# HVE, no complications}
\NormalTok{                     c_VE_comp ,  }\CommentTok{# OVE, complications}
\NormalTok{                     c_VE)        }\CommentTok{# OVE, no complications}
  
\NormalTok{  v_costs_tx     <-}\StringTok{ }\CommentTok{# your turn}
\StringTok{    }
\StringTok{  }\NormalTok{v_costs_biopsy <-}\StringTok{ }\CommentTok{# your turn }
\StringTok{  }
\StringTok{  }\CommentTok{# Calculate total utilities for each strategy ####}
\StringTok{  }\NormalTok{total_qaly_no_tx  <-}\StringTok{ }\NormalTok{v_w_no_tx  }\OperatorTok\StringTok{  }\NormalTok{v_qaly_no_tx      }
\NormalTok{  total_qaly_tx     <-}\StringTok{ }\NormalTok{v_w_tx     }\OperatorTok\StringTok{  }\NormalTok{v_qaly_tx}
\NormalTok{  total_qaly_biopsy <-}\StringTok{ }\NormalTok{v_w_biopsy }\OperatorTok\StringTok{  }\NormalTok{v_qaly_biopsy}
  
  \CommentTok{# Calculate total costs for each strategy ####}
\NormalTok{  total_cost_no_tx  <-}\StringTok{ }\NormalTok{v_w_no_tx  }\OperatorTok\StringTok{  }\NormalTok{v_cost_no_tx    }
\NormalTok{  total_cost_tx     <-}\StringTok{ }\NormalTok{v_w_tx     }\OperatorTok\StringTok{  }\NormalTok{v_cost_tx}
\NormalTok{  total_cost_biopsy <-}\StringTok{ }\NormalTok{v_w_biopsy }\OperatorTok\StringTok{  }\NormalTok{v_cost_biopsy}
  
\NormalTok{  v_total_qaly <-}\StringTok{ }\KeywordTok{c}\NormalTok{(total_qaly_no_tx, total_qaly_tx, total_qaly_biopsy)  }\CommentTok{# vector of total QALYs}
\NormalTok{  v_total_cost <-}\StringTok{ }\KeywordTok{c}\NormalTok{(total_cost_no_tx, total_cost_tx, total_cost_biopsy)  }\CommentTok{# vector of total costs}
\NormalTok{  v_nmb        <-}\StringTok{ }\NormalTok{v_total_qaly }\OperatorTok{*}\StringTok{ }\NormalTok{wtp }\OperatorTok{-}\StringTok{ }\NormalTok{v_total_cost                      }\CommentTok{# calculate vector of nmb}
  
  \CommentTok{# Name outcomes}
  \KeywordTok{names}\NormalTok{(v_total_qaly) <-}\StringTok{ }\NormalTok{v_names_str  }\CommentTok{# names for the elements of the total QALYs vector}
  \KeywordTok{names}\NormalTok{(v_total_cost) <-}\StringTok{ }\NormalTok{v_names_str  }\CommentTok{# names for the elements of the total cost vector}
  \KeywordTok{names}\NormalTok{(v_nmb)        <-}\StringTok{ }\NormalTok{v_names_str  }\CommentTok{# names for the elements of the nmb vector}
  
\NormalTok{  df_output <-}\StringTok{ }\KeywordTok{data.frame}\NormalTok{(}\DataTypeTok{Strategy =}\NormalTok{  v_names_str,}
                          \DataTypeTok{Cost     =}\NormalTok{  v_total_cost,}
                          \DataTypeTok{Effect   =}\NormalTok{  v_total_qaly,}
                          \DataTypeTok{NMB      =}\NormalTok{  v_nmb)}
  \KeywordTok{return}\NormalTok{(df_output)}
\NormalTok{\})}
  
\CommentTok{# model output}
\NormalTok{decision_tree_HVE_output}
\end{Highlighting}
\end{Shaded}

\hypertarget{plot-the-decision-tree}{%
\subsection{04.1 Plot the decision tree}\label{plot-the-decision-tree}}

\begin{Shaded}
\begin{Highlighting}[]
\CommentTok{# your turn}
\end{Highlighting}
\end{Shaded}

\hypertarget{cost-effectiveness-analysis}{%
\subsection{05 Cost-Effectiveness
Analysis}\label{cost-effectiveness-analysis}}

\begin{Shaded}
\begin{Highlighting}[]
\CommentTok{# your turn}
\end{Highlighting}
\end{Shaded}

\hypertarget{plot-frontier-of-decision-tree}{%
\subsection{05.1 Plot frontier of Decision
Tree}\label{plot-frontier-of-decision-tree}}

\begin{Shaded}
\begin{Highlighting}[]
\CommentTok{# your turn}
\end{Highlighting}
\end{Shaded}

\hypertarget{deterministic-sensitivity-analysis}{%
\section{06 Deterministic Sensitivity
Analysis}\label{deterministic-sensitivity-analysis}}

\hypertarget{list-of-input-parameters}{%
\subsection{06.1 List of input
parameters}\label{list-of-input-parameters}}

\begin{Shaded}
\begin{Highlighting}[]
\CommentTok{# your turn}
\end{Highlighting}
\end{Shaded}

\hypertarget{load-decision-tree-model-function}{%
\subsection{06.2 Load decision tree model
function}\label{load-decision-tree-model-function}}

\begin{Shaded}
\begin{Highlighting}[]
\CommentTok{# your turn}
\end{Highlighting}
\end{Shaded}

\hypertarget{one-way-sensitivity-analysis-owsa}{%
\subsection{06.3 One-way sensitivity analysis
(OWSA)}\label{one-way-sensitivity-analysis-owsa}}

\begin{Shaded}
\begin{Highlighting}[]
\CommentTok{# your turn}
\end{Highlighting}
\end{Shaded}

\hypertarget{plot-owsa}{%
\subsection{06.3.1 Plot OWSA}\label{plot-owsa}}

\begin{Shaded}
\begin{Highlighting}[]
\CommentTok{# your turn}
\end{Highlighting}
\end{Shaded}

\hypertarget{optimal-strategy-with-owsa}{%
\subsection{06.3.2 Optimal strategy with
OWSA}\label{optimal-strategy-with-owsa}}

\begin{Shaded}
\begin{Highlighting}[]
\CommentTok{# your turn}
\end{Highlighting}
\end{Shaded}

\hypertarget{tornado-plot}{%
\subsection{06.3.3 Tornado plot}\label{tornado-plot}}

\begin{Shaded}
\begin{Highlighting}[]
\CommentTok{# your turn}
\end{Highlighting}
\end{Shaded}

\hypertarget{two-way-sensitivity-analysis-twsa}{%
\subsection{06.4 Two-way sensitivity analysis
(TWSA)}\label{two-way-sensitivity-analysis-twsa}}

\begin{Shaded}
\begin{Highlighting}[]
\CommentTok{# your turn}
\end{Highlighting}
\end{Shaded}

\hypertarget{plot-twsa}{%
\subsection{06.4.1 Plot TWSA}\label{plot-twsa}}

\begin{Shaded}
\begin{Highlighting}[]
\CommentTok{# your turn}
\end{Highlighting}
\end{Shaded}

\hypertarget{probabilistic-sensitivity-analysis-psa}{%
\section{07 Probabilistic Sensitivity Analysis
(PSA)}\label{probabilistic-sensitivity-analysis-psa}}

\begin{Shaded}
\begin{Highlighting}[]
\CommentTok{# your turn}
\end{Highlighting}
\end{Shaded}

\hypertarget{create-psa-object-for-dampack}{%
\subsection{07.2 Create PSA object for
dampack}\label{create-psa-object-for-dampack}}

\begin{Shaded}
\begin{Highlighting}[]
\CommentTok{# your turn}
\end{Highlighting}
\end{Shaded}

\hypertarget{save-psa-objects}{%
\subsection{07.2.1 Save PSA objects}\label{save-psa-objects}}

\begin{Shaded}
\begin{Highlighting}[]
\CommentTok{# your turn}
\end{Highlighting}
\end{Shaded}

\hypertarget{create-probabilistic-analysis-graphs}{%
\subsection{07.3 Create probabilistic analysis
graphs}\label{create-probabilistic-analysis-graphs}}

\begin{Shaded}
\begin{Highlighting}[]
\CommentTok{# your turn}
\end{Highlighting}
\end{Shaded}

Vector with willingness-to-pay (WTP) thresholds.

\begin{Shaded}
\begin{Highlighting}[]
\CommentTok{# your turn}
\end{Highlighting}
\end{Shaded}

\hypertarget{cost-effectiveness-scatter-plot}{%
\subsection{07.3.1 Cost-Effectiveness Scatter
plot}\label{cost-effectiveness-scatter-plot}}

\begin{Shaded}
\begin{Highlighting}[]
\CommentTok{# your turn}
\end{Highlighting}
\end{Shaded}

\hypertarget{conduct-cea-with-probabilistic-output}{%
\subsection{07.4 Conduct CEA with probabilistic
output}\label{conduct-cea-with-probabilistic-output}}

\begin{Shaded}
\begin{Highlighting}[]
\CommentTok{# your turn}
\end{Highlighting}
\end{Shaded}

\hypertarget{plot-cost-effectiveness-frontier}{%
\subsection{07.4.1 Plot cost-effectiveness
frontier}\label{plot-cost-effectiveness-frontier}}

\begin{Shaded}
\begin{Highlighting}[]
\CommentTok{# your turn}
\end{Highlighting}
\end{Shaded}

\hypertarget{cost-effectiveness-acceptability-curves-ceacs-and-frontier-ceaf}{%
\subsection{07.4.2 Cost-effectiveness acceptability curves (CEACs) and
frontier
(CEAF)}\label{cost-effectiveness-acceptability-curves-ceacs-and-frontier-ceaf}}

\begin{Shaded}
\begin{Highlighting}[]
\CommentTok{# your turn}
\end{Highlighting}
\end{Shaded}

\hypertarget{expected-loss-curves-elcs}{%
\subsection{07.4.3 Expected Loss Curves
(ELCs)}\label{expected-loss-curves-elcs}}

\begin{Shaded}
\begin{Highlighting}[]
\CommentTok{# your turn}
\end{Highlighting}
\end{Shaded}

\hypertarget{expected-value-of-perfect-information-evpi}{%
\subsection{07.4.4 Expected value of perfect information
(EVPI)}\label{expected-value-of-perfect-information-evpi}}

\begin{Shaded}
\begin{Highlighting}[]
\CommentTok{# your turn}
\end{Highlighting}
\end{Shaded}

\end{document}
