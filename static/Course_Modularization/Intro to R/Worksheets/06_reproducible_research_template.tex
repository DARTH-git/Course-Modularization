% Options for packages loaded elsewhere
\PassOptionsToPackage{unicode}{hyperref}
\PassOptionsToPackage{hyphens}{url}
%
\documentclass[
]{article}
\usepackage{lmodern}
\usepackage{amssymb,amsmath}
\usepackage{ifxetex,ifluatex}
\ifnum 0\ifxetex 1\fi\ifluatex 1\fi=0 % if pdftex
  \usepackage[T1]{fontenc}
  \usepackage[utf8]{inputenc}
  \usepackage{textcomp} % provide euro and other symbols
\else % if luatex or xetex
  \usepackage{unicode-math}
  \defaultfontfeatures{Scale=MatchLowercase}
  \defaultfontfeatures[\rmfamily]{Ligatures=TeX,Scale=1}
\fi
% Use upquote if available, for straight quotes in verbatim environments
\IfFileExists{upquote.sty}{\usepackage{upquote}}{}
\IfFileExists{microtype.sty}{% use microtype if available
  \usepackage[]{microtype}
  \UseMicrotypeSet[protrusion]{basicmath} % disable protrusion for tt fonts
}{}
\makeatletter
\@ifundefined{KOMAClassName}{% if non-KOMA class
  \IfFileExists{parskip.sty}{%
    \usepackage{parskip}
  }{% else
    \setlength{\parindent}{0pt}
    \setlength{\parskip}{6pt plus 2pt minus 1pt}}
}{% if KOMA class
  \KOMAoptions{parskip=half}}
\makeatother
\usepackage{xcolor}
\IfFileExists{xurl.sty}{\usepackage{xurl}}{} % add URL line breaks if available
\IfFileExists{bookmark.sty}{\usepackage{bookmark}}{\usepackage{hyperref}}
\hypersetup{
  pdftitle={Introduction to R for Health Services Research},
  pdfauthor={SickKids and DARTH},
  hidelinks,
  pdfcreator={LaTeX via pandoc}}
\urlstyle{same} % disable monospaced font for URLs
\usepackage[margin=1in]{geometry}
\usepackage{color}
\usepackage{fancyvrb}
\newcommand{\VerbBar}{|}
\newcommand{\VERB}{\Verb[commandchars=\\\{\}]}
\DefineVerbatimEnvironment{Highlighting}{Verbatim}{commandchars=\\\{\}}
% Add ',fontsize=\small' for more characters per line
\usepackage{framed}
\definecolor{shadecolor}{RGB}{248,248,248}
\newenvironment{Shaded}{\begin{snugshade}}{\end{snugshade}}
\newcommand{\AlertTok}[1]{\textcolor[rgb]{0.94,0.16,0.16}{#1}}
\newcommand{\AnnotationTok}[1]{\textcolor[rgb]{0.56,0.35,0.01}{\textbf{\textit{#1}}}}
\newcommand{\AttributeTok}[1]{\textcolor[rgb]{0.77,0.63,0.00}{#1}}
\newcommand{\BaseNTok}[1]{\textcolor[rgb]{0.00,0.00,0.81}{#1}}
\newcommand{\BuiltInTok}[1]{#1}
\newcommand{\CharTok}[1]{\textcolor[rgb]{0.31,0.60,0.02}{#1}}
\newcommand{\CommentTok}[1]{\textcolor[rgb]{0.56,0.35,0.01}{\textit{#1}}}
\newcommand{\CommentVarTok}[1]{\textcolor[rgb]{0.56,0.35,0.01}{\textbf{\textit{#1}}}}
\newcommand{\ConstantTok}[1]{\textcolor[rgb]{0.00,0.00,0.00}{#1}}
\newcommand{\ControlFlowTok}[1]{\textcolor[rgb]{0.13,0.29,0.53}{\textbf{#1}}}
\newcommand{\DataTypeTok}[1]{\textcolor[rgb]{0.13,0.29,0.53}{#1}}
\newcommand{\DecValTok}[1]{\textcolor[rgb]{0.00,0.00,0.81}{#1}}
\newcommand{\DocumentationTok}[1]{\textcolor[rgb]{0.56,0.35,0.01}{\textbf{\textit{#1}}}}
\newcommand{\ErrorTok}[1]{\textcolor[rgb]{0.64,0.00,0.00}{\textbf{#1}}}
\newcommand{\ExtensionTok}[1]{#1}
\newcommand{\FloatTok}[1]{\textcolor[rgb]{0.00,0.00,0.81}{#1}}
\newcommand{\FunctionTok}[1]{\textcolor[rgb]{0.00,0.00,0.00}{#1}}
\newcommand{\ImportTok}[1]{#1}
\newcommand{\InformationTok}[1]{\textcolor[rgb]{0.56,0.35,0.01}{\textbf{\textit{#1}}}}
\newcommand{\KeywordTok}[1]{\textcolor[rgb]{0.13,0.29,0.53}{\textbf{#1}}}
\newcommand{\NormalTok}[1]{#1}
\newcommand{\OperatorTok}[1]{\textcolor[rgb]{0.81,0.36,0.00}{\textbf{#1}}}
\newcommand{\OtherTok}[1]{\textcolor[rgb]{0.56,0.35,0.01}{#1}}
\newcommand{\PreprocessorTok}[1]{\textcolor[rgb]{0.56,0.35,0.01}{\textit{#1}}}
\newcommand{\RegionMarkerTok}[1]{#1}
\newcommand{\SpecialCharTok}[1]{\textcolor[rgb]{0.00,0.00,0.00}{#1}}
\newcommand{\SpecialStringTok}[1]{\textcolor[rgb]{0.31,0.60,0.02}{#1}}
\newcommand{\StringTok}[1]{\textcolor[rgb]{0.31,0.60,0.02}{#1}}
\newcommand{\VariableTok}[1]{\textcolor[rgb]{0.00,0.00,0.00}{#1}}
\newcommand{\VerbatimStringTok}[1]{\textcolor[rgb]{0.31,0.60,0.02}{#1}}
\newcommand{\WarningTok}[1]{\textcolor[rgb]{0.56,0.35,0.01}{\textbf{\textit{#1}}}}
\usepackage{graphicx,grffile}
\makeatletter
\def\maxwidth{\ifdim\Gin@nat@width>\linewidth\linewidth\else\Gin@nat@width\fi}
\def\maxheight{\ifdim\Gin@nat@height>\textheight\textheight\else\Gin@nat@height\fi}
\makeatother
% Scale images if necessary, so that they will not overflow the page
% margins by default, and it is still possible to overwrite the defaults
% using explicit options in \includegraphics[width, height, ...]{}
\setkeys{Gin}{width=\maxwidth,height=\maxheight,keepaspectratio}
% Set default figure placement to htbp
\makeatletter
\def\fps@figure{htbp}
\makeatother
\setlength{\emergencystretch}{3em} % prevent overfull lines
\providecommand{\tightlist}{%
  \setlength{\itemsep}{0pt}\setlength{\parskip}{0pt}}
\setcounter{secnumdepth}{-\maxdimen} % remove section numbering

\title{Introduction to R for Health Services Research}
\usepackage{etoolbox}
\makeatletter
\providecommand{\subtitle}[1]{% add subtitle to \maketitle
  \apptocmd{\@title}{\par {\large #1 \par}}{}{}
}
\makeatother
\subtitle{Reproducible Research}
\author{SickKids and DARTH}
\date{12/09/2019}

\begin{document}
\maketitle

This worksheet highlights some key settings where \texttt{R} can help
make your research easier to reproduce and present to key stakeholder.
This session is split into the following sections:

\begin{enumerate}
\def\labelenumi{\arabic{enumi}.}
\item
  R Markdown
\item
  R Script
\item
  R Shiny Applications
\end{enumerate}

Throughout the course, we will demonstrate code and leave some empty
\emph{code chunks} for you to fill in. We will also provide solutions
after the session.

Feel free to modify this document with your own comments and
clarifications.

\hypertarget{r-markdown}{%
\section{1. R Markdown}\label{r-markdown}}

Throughout this course, we have provided all materials using R Markdown
but we will now briefly highlight how R Markdown can improve the
reproducibility of research. In an R Markdown document, you can store,
comment on, and run code from \textbf{code chunks}. This document can be
shared with collaborators or other stakeholders so people can:

\begin{itemize}
\item
  Re-run your code and view the outputs and analysis results.
\item
  Read your free text that explains the analysis and the data.
\item
  Easily modify your code in any way they want.
\item
  Use your document as a template to build their own R Markdown
  document.
\end{itemize}

You can \textbf{Knit} R Markdown documents to other file types including
HTML, PDF and Word. These alternative outputs can be submitted to
journals so keep you analysis and manuscript writing in the same
document.

\begin{itemize}
\item
  To knit to a HTML document, you will need an Internet browser
  installed on your computer, but you do not need to connect to the
  Internet.
\item
  To knit to a Word document, you will need Microsoft Office Word
  installed on your computer.
\item
  To knit to a PDF document, you will need to install
  \href{https://miktex.org/download}{\textbf{MiKTex}}, an up-to-date
  implementation of TeX/LaTeX and related programs. You should download
  the \textbf{Net Intaller}, not the Basic Installer. You do not need to
  \emph{use} LaTeX.
\end{itemize}

\textbf{Inline Code Chunks}

R Markdown also contains an \emph{inline} feature that allows you to run
small pieces of code within the text. This is useful if you want to
refer to figures from your data/statistical analysis within the main
body of your text, e.g., 82\% of trial participants were successfully
sedated. Inline code is specified using the following structure: 3. In
this case, the \textbf{knited} R Markdown document will display the mean
of the number 1 to 5 (e.g., 3) in place of the code above. You can
\textbf{Knit} the document to see.

\hypertarget{r-script}{%
\section{2. R Script}\label{r-script}}

You can also use \texttt{R} with \emph{scripts}. These \emph{scripts}
are like a single code chunk from a Markdown document and can be used to
write \texttt{R} code for more complex operations or to save functions
that are reused across multiple documents. \texttt{R} scripts can be
used to write, store and run \texttt{R} code. To open a new \texttt{R}
script, click

\textbf{File -\textgreater{} New File -\textgreater{} R Script\ldots{}}

You can type code anywhere in a script. To run a line, you can click on
that line and either hit the \textbf{Run} button on the top right corner
of this window or hit CTRL + ENTER.

\textbf{EXERCISE 1}

\begin{itemize}
\item
  Create a new \texttt{R} script.
\item
  Run a few mathematical operations.
\item
  Type in and run \texttt{plot(iris\$Sepal.Width,iris\$Sepal.Length)}.
  Note that \texttt{iris} is a standard dataset that is loaded whenever
  you open \texttt{R}.
\end{itemize}

Note that when you run code from a \texttt{R} script the output
\emph{only} displays in the console and the plots display in the bottom
right-hand corner of the RStudio interface.

\textbf{Setting a working directory}

The \emph{working directory} in \texttt{R} is an important way to ensure
that your research results are easy to find and you have loaded the
correct dataset. The \emph{working directory} is the folder where all
data will be imported from and where all output, scripts and Markdown
documents should be stored. You can display the current working
directory using the function \texttt{getwd()}. You can change the
working directory either by clicking \textbf{Session - \textgreater{}
Set working directory -\textgreater{} Choose Directory} and setting your
working directory or writing the command
\texttt{setwd("path\ of\ your\ working\ directory")}.

Remember that you should use a forward slash to define your filenames
and the path to the folder, e.g.~(``C:/'').

You can also ask \texttt{R} to set the working directory to the folder
where you have saved your Markdown document or script using
\textbf{Session -\textgreater{} Set Working Directory -\textgreater{} To
Source File Location}. If you data files are not in the same folder as
your \texttt{R} script, you have to specify the full path to the data
files (as we did in Session 3).

\textbf{Cleaning the working space and the R memory}

To ensure your results are reproducible and not based on previous
analysis, it is advisable to begin each Markdown document or \texttt{R}
script with a command that clears \texttt{R}'s memory. This removes all
the variables that you used in your previous analysis and any
user-defined functions you created. The command to clear you memory is

\begin{Shaded}
\begin{Highlighting}[]
\KeywordTok{rm}\NormalTok{(}\DataTypeTok{list =} \KeywordTok{ls}\NormalTok{())}
\end{Highlighting}
\end{Shaded}

You can also clear your \texttt{R} console (your working space) using
type CTRL + L.

\textbf{EXERCISE 2} Create a few variables with different values in the
following \emph{code chunk}. Then clear the working space and the
\texttt{R} memory.

\textbf{Adding explanatory comments}

It is good practice in any programming language to write code clearly,
using logical steps and explaining clearly why those steps were taken.
This helps other researchers reproduce your results and review you
analysis. It also helps you remember what you did!

In R Markdown, you can explain your analysis using text and add short
comments to the code. We have used this method throughout the course.
Within an \texttt{R} script, you can only use comments, indicated using
the hashtag (\texttt{\#}) symbol. Any line with a \texttt{\#} symbol
will be ignored by \texttt{R}, e.g.

\begin{Shaded}
\begin{Highlighting}[]
\CommentTok{# cleaning the memory of R}
\KeywordTok{rm}\NormalTok{(}\DataTypeTok{list =} \KeywordTok{ls}\NormalTok{ ()) }
\end{Highlighting}
\end{Shaded}

You can also add comments in the same line as your code to explain what
analysis you are doing on each line. For example:

\begin{Shaded}
\begin{Highlighting}[]
\KeywordTok{rm}\NormalTok{(}\DataTypeTok{list =} \KeywordTok{ls}\NormalTok{ ()) }\CommentTok{# cleaning the memory of R}
\end{Highlighting}
\end{Shaded}

This commenting approach reduces the lines of code and results in more
condensed code but can be more challenging to read if the comments are
long.

\hypertarget{r-shiny-applications}{%
\section{\texorpdfstring{3. \texttt{R} Shiny
Applications}{3. R Shiny Applications}}\label{r-shiny-applications}}

\texttt{R} Shiny is a powerful and elegant package that that allows you
to build web-based applications using \texttt{R}. It lets you turn your
analyses into interactive web applications without requiring JavaScript,
CSS or HTML coding.

To create a new Shiny App, click

\textbf{File -\textgreater{} New File -\textgreater{} Shiny Web
App\ldots{}}

and choose a name for your App and the working directory it will be
saved in.

\textbf{EXERCISE 3} Create a new Shiny App file and name it ``demo''.

After you create this app, you will see \texttt{R} has opened up a file
name \texttt{app.R} --\textgreater{} this is the script for your App.

By default, \texttt{R} provides an example of a Shiny App which you can
use by clicking the \textbf{Run App} button on the top right corner of
the Script Editor. This App plots a histogram of the distribution of
waiting time to eruption of the geyser from a the Old Faithful Geyser
dataset, which is provided by standard in \texttt{R}.

This App interactively generates and displays a histogram with the
binwidth the user selects. The user of the App does not have to write or
run any \texttt{R} code as everything is done through clicks. This is
what makes Shiny Apps cool and powerful!

The following section briefly introduces how to build a shiny app for
your analysis. To begin, call the \texttt{shiny} library.

\begin{Shaded}
\begin{Highlighting}[]
\KeywordTok{library}\NormalTok{(shiny)}
\end{Highlighting}
\end{Shaded}

All shiny apps have two components: \textbf{UI} and \textbf{Server}.

\begin{itemize}
\item
  \textbf{UI} stands for ``User Interface'', this is where you code what
  the user can see and click on.
\item
  \textbf{Server} contains functions that produce the output that is
  displayed in or used by the UI.
\end{itemize}

It is easier to start with the Server since it feeds things into the UI.

In the demo App, the \textbf{Server}:

\begin{itemize}
\item
  Contains a function that first stores the 2nd column (eruption wait
  time) of the Old Faithful Geyser dataset into a variable called
  \texttt{x}.
\item
  Then, it takes different number of bins defined in the UI (called
  ``bins'') to create the \texttt{bins} variable.
\item
  Finally, it plots the histogram using the appropriate arguments
  (\texttt{x}, \texttt{bins}, etc).
\item
  The entire operation is wrapped by Shiny function
  \texttt{renderPlot()} (a function that takes plots produced in
  \texttt{R}) and saved into a variable called
  \texttt{output\$distPlot}.
\item
  \texttt{output} tells \texttt{R} that this is an output to be sent to
  the UI and \texttt{distPlot} is the name given to this operation.
\end{itemize}

Now we switch to the \textbf{UI},

\begin{itemize}
\item
  This UI has two parts: the side panel and the main panel.
\item
  The side panel contains a side bar with slider input that contains
  different number of bins (input name is ``bins'') ranging from 1 to
  50, with the default being 30.
\item
  The main panel contains a function called \texttt{plotOutput()} that
  takes our histogram and plots it on the main panel of the App.
\end{itemize}

To finalize the App, the command
\texttt{shinyApp(ui\ =\ ui,\ server\ =\ server)} is inserted at the very
end.

\textbf{EXERCISE 4} Follow the steps below to build a Shiny App that
makes stratified boxplots for the Framingham data:

\begin{itemize}
\item
  Open up the Shiny App file ``app.R'' in the ``framingham'' folder.
\item
  You will see that we have selected the categorical variables (the
  \texttt{x}'s) that we wish to stratify our continuous variables (the
  \texttt{y}'s) by.
\item
  \texttt{selectInput()} provides a dropdown menu with different
  user-defined selections.
\item
  Create a dropdown menu for the continuous variables. Include age, sex,
  current smoking status, diabetic status, prevalent myocardial
  infarction status and prevalent stroke status.
\item
  Write the code to draw the boxplot under the comment ``draw the
  boxplot of y stratified by x''. Make sure you have appropriate labels
  for the axes and different colours for the boxplots.
\item
  Run the App and test it.
\end{itemize}

\hypertarget{further-reading}{%
\section{4. Further Reading}\label{further-reading}}

There are loads of great resources available to learn more about R:

\begin{enumerate}
\def\labelenumi{\arabic{enumi}.}
\tightlist
\item
  Christopher Gandrud, Reproducible Research with R and RStudio
\item
  Yihui Xie, Dynamic Documents with R and knitr
\item
  The tidyverse website, \url{https://www.tidyverse.org/}
\item
  The R Markdown website, \url{https://rmarkdown.rstudio.com/}
\item
  Hadley Wickham, Advanced R.
\end{enumerate}

\end{document}
