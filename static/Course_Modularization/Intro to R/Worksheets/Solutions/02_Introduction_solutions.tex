% Options for packages loaded elsewhere
\PassOptionsToPackage{unicode}{hyperref}
\PassOptionsToPackage{hyphens}{url}
%
\documentclass[
]{article}
\usepackage{lmodern}
\usepackage{amssymb,amsmath}
\usepackage{ifxetex,ifluatex}
\ifnum 0\ifxetex 1\fi\ifluatex 1\fi=0 % if pdftex
  \usepackage[T1]{fontenc}
  \usepackage[utf8]{inputenc}
  \usepackage{textcomp} % provide euro and other symbols
\else % if luatex or xetex
  \usepackage{unicode-math}
  \defaultfontfeatures{Scale=MatchLowercase}
  \defaultfontfeatures[\rmfamily]{Ligatures=TeX,Scale=1}
\fi
% Use upquote if available, for straight quotes in verbatim environments
\IfFileExists{upquote.sty}{\usepackage{upquote}}{}
\IfFileExists{microtype.sty}{% use microtype if available
  \usepackage[]{microtype}
  \UseMicrotypeSet[protrusion]{basicmath} % disable protrusion for tt fonts
}{}
\makeatletter
\@ifundefined{KOMAClassName}{% if non-KOMA class
  \IfFileExists{parskip.sty}{%
    \usepackage{parskip}
  }{% else
    \setlength{\parindent}{0pt}
    \setlength{\parskip}{6pt plus 2pt minus 1pt}}
}{% if KOMA class
  \KOMAoptions{parskip=half}}
\makeatother
\usepackage{xcolor}
\IfFileExists{xurl.sty}{\usepackage{xurl}}{} % add URL line breaks if available
\IfFileExists{bookmark.sty}{\usepackage{bookmark}}{\usepackage{hyperref}}
\hypersetup{
  pdftitle={Intro to R for Decision Modeling},
  pdfauthor={SickKids and DARTH},
  hidelinks,
  pdfcreator={LaTeX via pandoc}}
\urlstyle{same} % disable monospaced font for URLs
\usepackage[margin=1in]{geometry}
\usepackage{color}
\usepackage{fancyvrb}
\newcommand{\VerbBar}{|}
\newcommand{\VERB}{\Verb[commandchars=\\\{\}]}
\DefineVerbatimEnvironment{Highlighting}{Verbatim}{commandchars=\\\{\}}
% Add ',fontsize=\small' for more characters per line
\usepackage{framed}
\definecolor{shadecolor}{RGB}{248,248,248}
\newenvironment{Shaded}{\begin{snugshade}}{\end{snugshade}}
\newcommand{\AlertTok}[1]{\textcolor[rgb]{0.94,0.16,0.16}{#1}}
\newcommand{\AnnotationTok}[1]{\textcolor[rgb]{0.56,0.35,0.01}{\textbf{\textit{#1}}}}
\newcommand{\AttributeTok}[1]{\textcolor[rgb]{0.77,0.63,0.00}{#1}}
\newcommand{\BaseNTok}[1]{\textcolor[rgb]{0.00,0.00,0.81}{#1}}
\newcommand{\BuiltInTok}[1]{#1}
\newcommand{\CharTok}[1]{\textcolor[rgb]{0.31,0.60,0.02}{#1}}
\newcommand{\CommentTok}[1]{\textcolor[rgb]{0.56,0.35,0.01}{\textit{#1}}}
\newcommand{\CommentVarTok}[1]{\textcolor[rgb]{0.56,0.35,0.01}{\textbf{\textit{#1}}}}
\newcommand{\ConstantTok}[1]{\textcolor[rgb]{0.00,0.00,0.00}{#1}}
\newcommand{\ControlFlowTok}[1]{\textcolor[rgb]{0.13,0.29,0.53}{\textbf{#1}}}
\newcommand{\DataTypeTok}[1]{\textcolor[rgb]{0.13,0.29,0.53}{#1}}
\newcommand{\DecValTok}[1]{\textcolor[rgb]{0.00,0.00,0.81}{#1}}
\newcommand{\DocumentationTok}[1]{\textcolor[rgb]{0.56,0.35,0.01}{\textbf{\textit{#1}}}}
\newcommand{\ErrorTok}[1]{\textcolor[rgb]{0.64,0.00,0.00}{\textbf{#1}}}
\newcommand{\ExtensionTok}[1]{#1}
\newcommand{\FloatTok}[1]{\textcolor[rgb]{0.00,0.00,0.81}{#1}}
\newcommand{\FunctionTok}[1]{\textcolor[rgb]{0.00,0.00,0.00}{#1}}
\newcommand{\ImportTok}[1]{#1}
\newcommand{\InformationTok}[1]{\textcolor[rgb]{0.56,0.35,0.01}{\textbf{\textit{#1}}}}
\newcommand{\KeywordTok}[1]{\textcolor[rgb]{0.13,0.29,0.53}{\textbf{#1}}}
\newcommand{\NormalTok}[1]{#1}
\newcommand{\OperatorTok}[1]{\textcolor[rgb]{0.81,0.36,0.00}{\textbf{#1}}}
\newcommand{\OtherTok}[1]{\textcolor[rgb]{0.56,0.35,0.01}{#1}}
\newcommand{\PreprocessorTok}[1]{\textcolor[rgb]{0.56,0.35,0.01}{\textit{#1}}}
\newcommand{\RegionMarkerTok}[1]{#1}
\newcommand{\SpecialCharTok}[1]{\textcolor[rgb]{0.00,0.00,0.00}{#1}}
\newcommand{\SpecialStringTok}[1]{\textcolor[rgb]{0.31,0.60,0.02}{#1}}
\newcommand{\StringTok}[1]{\textcolor[rgb]{0.31,0.60,0.02}{#1}}
\newcommand{\VariableTok}[1]{\textcolor[rgb]{0.00,0.00,0.00}{#1}}
\newcommand{\VerbatimStringTok}[1]{\textcolor[rgb]{0.31,0.60,0.02}{#1}}
\newcommand{\WarningTok}[1]{\textcolor[rgb]{0.56,0.35,0.01}{\textbf{\textit{#1}}}}
\usepackage{graphicx,grffile}
\makeatletter
\def\maxwidth{\ifdim\Gin@nat@width>\linewidth\linewidth\else\Gin@nat@width\fi}
\def\maxheight{\ifdim\Gin@nat@height>\textheight\textheight\else\Gin@nat@height\fi}
\makeatother
% Scale images if necessary, so that they will not overflow the page
% margins by default, and it is still possible to overwrite the defaults
% using explicit options in \includegraphics[width, height, ...]{}
\setkeys{Gin}{width=\maxwidth,height=\maxheight,keepaspectratio}
% Set default figure placement to htbp
\makeatletter
\def\fps@figure{htbp}
\makeatother
\setlength{\emergencystretch}{3em} % prevent overfull lines
\providecommand{\tightlist}{%
  \setlength{\itemsep}{0pt}\setlength{\parskip}{0pt}}
\setcounter{secnumdepth}{-\maxdimen} % remove section numbering

\title{Intro to R for Decision Modeling}
\usepackage{etoolbox}
\makeatletter
\providecommand{\subtitle}[1]{% add subtitle to \maketitle
  \apptocmd{\@title}{\par {\large #1 \par}}{}{}
}
\makeatother
\subtitle{Introduction}
\author{SickKids and DARTH}
\date{11/2/2020}

\begin{document}
\maketitle

Change \texttt{eval} to \texttt{TRUE} if you want to knit this
document.This worksheet provides an introduction to some key concepts in
\texttt{R}. This session is split into the following sections:

\begin{enumerate}
\def\labelenumi{\arabic{enumi}.}
\item
  Using \texttt{R} as a calculator
\item
  Assigning values in \texttt{R}
\item
  Creating vectors and matrices
\item
  Manipulating Matrices in \texttt{R}
\item
  Actions in \texttt{R}
\item
  Customized tools in \texttt{R}
\end{enumerate}

Throughout the course, we will demonstrate code and leave some empty
\emph{code chunks} for you to fill in. We will also provide solutions
after the session.

Feel free to modify this document with your own comments and
clarifications.

\hypertarget{using-r-as-a-calculator}{%
\section{\texorpdfstring{1. Using \texttt{R} as a
calculator}{1. Using R as a calculator}}\label{using-r-as-a-calculator}}

Although \texttt{R} has lots of capabilities, we will start by using it
as a calculator.

\textbf{EXERCISE 1} Type \texttt{2\ +\ 3} in the \emph{code chunk} below
and run. Remember that the code chunk can be run using the green arrow
at the right hand corner of the chunk or using the keyboard shortcut
CTRL + SHIFT + ENTER.

\begin{Shaded}
\begin{Highlighting}[]
\CommentTok{# Your turn}
\DecValTok{2} \OperatorTok{+}\StringTok{ }\DecValTok{3}
\end{Highlighting}
\end{Shaded}

You should now see that the calculation \texttt{2\ +\ 3} has been
completed as the output is shown in two places. In the console below and
in the R Markdown document directly above.

\textbf{EXERCISE 2} Perform a few more calculations in the \emph{code
chunk} below. You can use the following commands:

\begin{itemize}
\tightlist
\item
  \texttt{-}: subtract
\item
  \texttt{*}: multiply
\item
  \texttt{/}: divide
\item
  \texttt{\^{}2}: squared
\end{itemize}

\begin{Shaded}
\begin{Highlighting}[]
\CommentTok{# Your turn}
\DecValTok{5}\OperatorTok{^}\DecValTok{5}
\DecValTok{12}\OperatorTok{/}\DecValTok{4}
\NormalTok{(}\DecValTok{12}\OperatorTok{/}\DecValTok{4}\NormalTok{)}\OperatorTok{^}\DecValTok{2}
\NormalTok{(}\DecValTok{51-4}\NormalTok{)}\OperatorTok{*}\DecValTok{9}
\end{Highlighting}
\end{Shaded}

\hypertarget{assigning-values-in-r}{%
\section{\texorpdfstring{2. Assigning values in
\texttt{R}}{2. Assigning values in R}}\label{assigning-values-in-r}}

\texttt{R} performs data analysis by applying actions (known as
\emph{functions}) to \emph{objects}. One example of an \emph{object} is
a dataset that we load into \texttt{R}. Any function you wish to apply
(e.g.~finding the mean for each variable in your dataset) will be an
action on a particular object (the dataset in this case).

We define an object in \texttt{R} using the symbol
\texttt{\textless{}-}. So for example,

\begin{Shaded}
\begin{Highlighting}[]
\CommentTok{# Defining a}
\NormalTok{a <-}\StringTok{ }\DecValTok{5}
\end{Highlighting}
\end{Shaded}

means that we are defining an object called \texttt{a} and it is defined
as being equal to 5. \texttt{R} is case-sensitive so \texttt{a} and
\texttt{A} are different objects in \texttt{R}, if both defined.

\textbf{EXERCISE 3} Define two objects \texttt{alpha} equal to 7 and
\texttt{beta} equal to 2 in the \emph{code chunk} below and run. Note
that each object needs to be defined on a separate line within the
\emph{code chunk}.

\begin{Shaded}
\begin{Highlighting}[]
\CommentTok{# Your turn}
\NormalTok{alpha <-}\StringTok{ }\DecValTok{7}
\NormalTok{beta <-}\StringTok{ }\DecValTok{2}
\end{Highlighting}
\end{Shaded}

For the \emph{code chunk} above, notice that there was no output in the
R Markdown document. This is because we were only defining alpha and
beta and not asking \texttt{R} to do any calculations. If you look in
the console, you will see the two lines of code you just run are in the
console so we know that \texttt{R} has defined these objects.

As \texttt{R} now knows what we mean by \texttt{alpha} and
\texttt{beta}, we can use \texttt{R} to apply functions to these
objects. The calculations we use in Section 1 are examples of functions,
e.g.~run the following \emph{code chunk}

\begin{Shaded}
\begin{Highlighting}[]
\CommentTok{# Addition}
\NormalTok{alpha }\OperatorTok{+}\StringTok{ }\NormalTok{beta}
\end{Highlighting}
\end{Shaded}

\textbf{EXERCISE 4} Use the commands from the first section to undertake
some additional calculations with \texttt{alpha} and \texttt{beta} in
the following \emph{code chunk} and run.

\begin{Shaded}
\begin{Highlighting}[]
\CommentTok{# Your turn}
\NormalTok{alpha }\OperatorTok{-}\StringTok{ }\NormalTok{beta}
\NormalTok{alpha }\OperatorTok{*}\StringTok{ }\NormalTok{beta}
\NormalTok{alpha }\OperatorTok{/}\StringTok{ }\NormalTok{beta}
\NormalTok{alpha }\OperatorTok{^}\StringTok{ }\DecValTok{2}
\end{Highlighting}
\end{Shaded}

\hypertarget{types-of-data-in-r}{%
\section{3. Types of data in R}\label{types-of-data-in-r}}

An object doesn't have to be a number, common \texttt{R} objects include
numbers (as we defined above), a character string, a dataset, a vector,
a matrix, an array and a plot.

Vectors are a set of numbers with a specific set of properties. To
define a vector, we pass a list of numbers to the \texttt{c()} function
in \texttt{R}. For example, the following \emph{code chunk} defines a
vector that contains the ages for three different people:

\begin{Shaded}
\begin{Highlighting}[]
\CommentTok{# Defining a vector}
\NormalTok{ages <-}\StringTok{ }\KeywordTok{c}\NormalTok{(}\DecValTok{12}\NormalTok{, }\DecValTok{63}\NormalTok{, }\DecValTok{27}\NormalTok{)}
\end{Highlighting}
\end{Shaded}

\textbf{EXERCISE 5} In the \emph{code chunk} below, define
\texttt{vector1} that contains the numbers 3, 5 and 1 and
\texttt{vector2} that contains the numbers -1, 8, and -3. Run this
\emph{code chunk}.

\begin{Shaded}
\begin{Highlighting}[]
\CommentTok{# Your turn}
\NormalTok{vector1 <-}\StringTok{ }\KeywordTok{c}\NormalTok{(}\DecValTok{3}\NormalTok{, }\DecValTok{5}\NormalTok{, }\DecValTok{1}\NormalTok{)}
\NormalTok{vector2 <-}\StringTok{ }\KeywordTok{c}\NormalTok{(}\OperatorTok{-}\DecValTok{1}\NormalTok{, }\DecValTok{8}\NormalTok{, }\DecValTok{-3}\NormalTok{)}
\end{Highlighting}
\end{Shaded}

Character strings contain words, rather than numbers. To ensure that
\texttt{R} knows you are using a character string, you need to use
\texttt{""}, for example,

\begin{Shaded}
\begin{Highlighting}[]
\CommentTok{# Defining a character string}
\NormalTok{fruit <-}\StringTok{ "apple"}
\end{Highlighting}
\end{Shaded}

You can also combine character strings into a vector.

\textbf{EXERCISE 6} In the \emph{code chunk} below, define a vector
\texttt{fruits} that contains the strings ``apple'', ``banana'' and
``orange''. Run this \emph{code chunk}.

\begin{Shaded}
\begin{Highlighting}[]
\CommentTok{# Your turn}
\NormalTok{fruits <-}\StringTok{ }\KeywordTok{c}\NormalTok{(}\StringTok{"apple"}\NormalTok{, }\StringTok{"banana"}\NormalTok{, }\StringTok{"orange"}\NormalTok{)}
\end{Highlighting}
\end{Shaded}

Another important object is a matrix, which is a set of numbers or
characters with a specific shape. For example, the following \emph{code
chunk} creates a matrix called \texttt{matrixA} that contains the
numbers 1 to 9 in a 3 by 3 grid:

\begin{Shaded}
\begin{Highlighting}[]
\CommentTok{# Creating a matrix}
\NormalTok{matrixA <-}\StringTok{ }\KeywordTok{matrix}\NormalTok{(}\DataTypeTok{data =} \KeywordTok{c}\NormalTok{(}\DecValTok{1}\NormalTok{, }\DecValTok{2}\NormalTok{, }\DecValTok{3}\NormalTok{, }\DecValTok{4}\NormalTok{, }\DecValTok{5}\NormalTok{, }\DecValTok{6}\NormalTok{, }\DecValTok{7}\NormalTok{, }\DecValTok{8}\NormalTok{, }\DecValTok{9}\NormalTok{), }
                  \DataTypeTok{nrow =} \DecValTok{3}\NormalTok{, }
                  \DataTypeTok{ncol =} \DecValTok{3}\NormalTok{)}
\NormalTok{matrixA}
\end{Highlighting}
\end{Shaded}

In this \emph{code chunk}, we can see that the matrix \texttt{matrixA}
is printed in this R Markdown document above. This is because we added
the extra command \texttt{matrixA} to our \emph{code chunk}, this allows
us to see what the object \texttt{matrixA} is.

\textbf{EXERCISE 7} Use the following \emph{code chunk} to create a
matrix \texttt{matrixB} with 2 rows and 4 columns and a matrix
\texttt{matrixC} with 4 rows and 2 columns:

\begin{Shaded}
\begin{Highlighting}[]
\CommentTok{# Your turn}

\NormalTok{matrixB <-}\StringTok{ }\KeywordTok{matrix}\NormalTok{(}\DataTypeTok{data =} \KeywordTok{c}\NormalTok{(}\DecValTok{1}\NormalTok{, }\DecValTok{2}\NormalTok{, }\DecValTok{3}\NormalTok{, }\DecValTok{4}\NormalTok{, }\DecValTok{5}\NormalTok{, }\DecValTok{6}\NormalTok{, }\DecValTok{7}\NormalTok{, }\DecValTok{8}\NormalTok{), }
                  \DataTypeTok{nrow =} \DecValTok{2}\NormalTok{, }
                  \DataTypeTok{ncol =} \DecValTok{4}\NormalTok{)}
\NormalTok{matrixB}

\NormalTok{matrixC <-}\StringTok{ }\KeywordTok{matrix}\NormalTok{(}\DataTypeTok{data =} \KeywordTok{c}\NormalTok{(}\DecValTok{1}\NormalTok{, }\DecValTok{2}\NormalTok{, }\DecValTok{3}\NormalTok{, }\DecValTok{4}\NormalTok{, }\DecValTok{5}\NormalTok{, }\DecValTok{6}\NormalTok{, }\DecValTok{7}\NormalTok{, }\DecValTok{8}\NormalTok{), }
                  \DataTypeTok{nrow =} \DecValTok{4}\NormalTok{, }
                  \DataTypeTok{ncol =} \DecValTok{2}\NormalTok{)}
\NormalTok{matrixC}
\end{Highlighting}
\end{Shaded}

We can also use functions on vectors and matrices, for example, if we
use \texttt{vector1} and \texttt{vector2} that we defined above:

\begin{Shaded}
\begin{Highlighting}[]
\CommentTok{# Functions on vectors}
\NormalTok{vector1 }\OperatorTok{+}\StringTok{ }\NormalTok{vector2}
\end{Highlighting}
\end{Shaded}

\textbf{EXERCISE 8} Explore what happens when you use the commands from
section 1 on vectors in the \emph{code chunk} below:

\begin{Shaded}
\begin{Highlighting}[]
\CommentTok{# Your turn}
\end{Highlighting}
\end{Shaded}

What happens if we change the definition of \texttt{vector1}?

\textbf{EXERCISE 9} Run the following \emph{code chunk} to change the
definition of \texttt{vector1}:

\begin{Shaded}
\begin{Highlighting}[]
\CommentTok{# Your turn}
\NormalTok{vector1 <-}\StringTok{ }\KeywordTok{c}\NormalTok{(}\DecValTok{3}\NormalTok{, }\DecValTok{5}\NormalTok{, }\DecValTok{2}\NormalTok{)}
\end{Highlighting}
\end{Shaded}

The top right hand window will show the defintion of \texttt{vector1}.
Note that if you overwrite the definition of an object, you cannot
retrieve the previous defintion of the object unless you rerun the
initial code.

You can check the definition of an object by typing the name of an
object into a \emph{code chunk} and running that line of code. This
prints the object in the console and in the R Markdown so you can check
it is defined correctly. You can also check the defintion in the top
right-hand window that displays you \emph{workspace}.

\hypertarget{manipulating-matrices-in-r}{%
\section{\texorpdfstring{4. Manipulating Matrices in
\texttt{R}}{4. Manipulating Matrices in R}}\label{manipulating-matrices-in-r}}

In \texttt{R}, there are three key operations that we can do with
matrices, addition (\texttt{+}), multiplication (\texttt{*}) and matrix
multiplication (\texttt{\%*\%}). Each of these operations can only be
used if the matrices are compatible.

\textbf{EXERCISE 10} Run the following operations to determine which
matrices or vectors are compatible for matrix addition. Note that some
of them will give you errors, make a note of these errors and why they
occur to help you understand issues in you coding during the course.

\begin{Shaded}
\begin{Highlighting}[]
\CommentTok{# Matrix added to a number}
\NormalTok{matrixA }\OperatorTok{+}\StringTok{ }\DecValTok{5}

\CommentTok{# Matrix added to a vector}
\NormalTok{matrixA }\OperatorTok{+}\StringTok{ }\NormalTok{vector1}
\NormalTok{matrixB }\OperatorTok{+}\StringTok{ }\NormalTok{vector1}

\CommentTok{# Matrix addition}
\NormalTok{matrixA }\OperatorTok{+}\StringTok{ }\NormalTok{matrixA}
\NormalTok{matrixB }\OperatorTok{+}\StringTok{ }\NormalTok{matrixC}
\end{Highlighting}
\end{Shaded}

Notice that when you are adding a vector to a matrix, the addition is
done column-wise where the vector is added to each column of the matrix.
When adding matrices, the addition is performed element-wise, so the two
matrix entries in the same position are added together. Scalar
multiplcation \texttt{*} is similar to addition.

\textbf{EXERCISE 11} Copying the code above, or writing your own,
explore how scalar multiplication works with matrices and vectors:

\begin{Shaded}
\begin{Highlighting}[]
\CommentTok{# Matrix multipled by a number}
\NormalTok{matrixA }\OperatorTok{*}\StringTok{ }\DecValTok{5}

\CommentTok{# Matrix multipled by a vector}
\NormalTok{matrixA }\OperatorTok{*}\StringTok{ }\NormalTok{vector1}
\NormalTok{matrixB }\OperatorTok{*}\StringTok{ }\NormalTok{vector1}

\CommentTok{# Matrix with scalar multiplication}
\NormalTok{matrixA }\OperatorTok{*}\StringTok{ }\NormalTok{matrixA}
\NormalTok{matrixB }\OperatorTok{*}\StringTok{ }\NormalTok{matrixC}
\end{Highlighting}
\end{Shaded}

Matrix multiplication \texttt{\%*\%}, on the other hand, follows a
different set of rules, outlined in the pre-course material.

\textbf{EXERCISE 12} Run the following code to explore how matrix
multiplication works in R:

\begin{Shaded}
\begin{Highlighting}[]
\CommentTok{# Matrix Multiplication}
\NormalTok{matrixA }\OperatorTok\StringTok{ }\NormalTok{vector1}
\NormalTok{matrixA }\OperatorTok\StringTok{ }\NormalTok{matrixA}
\NormalTok{matrixB }\OperatorTok\StringTok{ }\NormalTok{matrixC}
\NormalTok{matrixC }\OperatorTok\StringTok{ }\NormalTok{matrixB}
\end{Highlighting}
\end{Shaded}

Notice that the order of the matrices matters for matrix multiplication
as the last two operations give different results. It is also important
to remember that matrix multiplication can only be undertaken if the
number of columns in the first matrix is equal to the number of rows in
the second matrix. The resulting matrix will have the same number of
rows as the first matrix and the same number of columns as the second
matrix.

\hypertarget{actions-in-r}{%
\section{\texorpdfstring{5. Actions in
\texttt{R}}{5. Actions in R}}\label{actions-in-r}}

\emph{Functions} in \texttt{R} are used to perform actions on the
different objects.

We have already seen some simple actions; +, *, - and / --- these are
calculator or arithmetic actions.

More complex functions are given in two parts.

\begin{itemize}
\item
  The name of the function, e.g.~\texttt{c}
\item
  The object to be acted on in brackets, e.g.~\texttt{(alpha)}
\end{itemize}

Within the brackets function specific options (formally called
``arguments'') can also be given after commas,
e.g.~\texttt{(alpha,\ beta)}

Some examples of function in \texttt{R} include:

\begin{itemize}
\item
  \texttt{log()}
\item
  \texttt{sum()}
\item
  \texttt{prod()}
\item
  \texttt{exp()}
\item
  \texttt{sort()}
\end{itemize}

\textbf{EXERCISE 13} Use \texttt{vector1}, \texttt{vector2} and
\texttt{matrixA} to investigate how these different functions are used
for vectors and matrices. You can use the \emph{code chunk} below:

\begin{Shaded}
\begin{Highlighting}[]
\CommentTok{# Your turn}
\NormalTok{vector1}
\KeywordTok{log}\NormalTok{(vector1)}
\KeywordTok{prod}\NormalTok{(vector1)}
\NormalTok{vector2}
\KeywordTok{sort}\NormalTok{(vector2)}
\KeywordTok{sum}\NormalTok{(vector2)}
\NormalTok{matrixA}
\KeywordTok{exp}\NormalTok{(matrixA)}
\end{Highlighting}
\end{Shaded}

Remember, you can run a single line in a \emph{code chunk} by
highlighting that line using CTRL + ENTER.

\begin{itemize}
\item
  To access help for a function you can type \texttt{?function} or
  \texttt{help(function)} into the \texttt{R} console, where
  \texttt{function} is the name of the function you need help for.
\item
  Some \emph{functions} can only be used for certain \emph{objects} and
  in the future sessions we will see some more complex \emph{functions}
  that can be used to manipulate our data.
\item
  Many \emph{functions} can take multiple arguments; simply separate
  them by \texttt{,} in your function call.
\end{itemize}

\textbf{EXERCISE 14} Open up and read the help document for the function
\texttt{sum()}.

There are some key functions that are very helpful when working with
matrices.

\begin{itemize}
\item
  \texttt{t()} creates the matrix transpose, i.e., the matrix where the
  element in the i-th row and j-th column of the original matrix is set
  to the j-th row and i-th column.
\item
  \texttt{diag(n)} creates a matrix of size \texttt{n} by \texttt{n}
  with 1s at each diagonal element and 0s everywhere else, i.e., the
  identity matrix.
\end{itemize}

\textbf{EXERCISE 15} Explore the transponse function using the code
provided below:

\begin{Shaded}
\begin{Highlighting}[]
\NormalTok{matrixD <-}\StringTok{ }\KeywordTok{matrix}\NormalTok{(}\KeywordTok{c}\NormalTok{(}\DecValTok{1}\NormalTok{, }\DecValTok{2}\NormalTok{, }\DecValTok{3}\NormalTok{, }\DecValTok{4}\NormalTok{), }\DataTypeTok{nrow =} \DecValTok{2}\NormalTok{, }\DataTypeTok{ncol =} \DecValTok{2}\NormalTok{)}

\CommentTok{# Determine the difference between matrixC and the transpose of matrixC}
\NormalTok{matrixC}
\KeywordTok{t}\NormalTok{(matrixC)}

\CommentTok{# Transpose can allow for matrix multiplication when they may not have been }
\CommentTok{# compatible before}
\NormalTok{matrixD }\OperatorTok\StringTok{ }\KeywordTok{t}\NormalTok{(matrixC)}
\NormalTok{matrixC }\OperatorTok\StringTok{ }\NormalTok{matrixD}
\KeywordTok{t}\NormalTok{(matrixB) }\OperatorTok\StringTok{ }\NormalTok{matrixD}
\NormalTok{matrixD }\OperatorTok\StringTok{ }\NormalTok{matrixB }
\end{Highlighting}
\end{Shaded}

\hypertarget{customized-tools-in-r}{%
\section{\texorpdfstring{5. Customized tools in
\texttt{R}}{5. Customized tools in R}}\label{customized-tools-in-r}}

\emph{Packages} give \texttt{R} extra capabilities as they contain
additional \emph{functions}. There are over 8000 additional packages in
\texttt{R} that can perform a huge range of additional \emph{functions}.

\emph{Packages} can be thought as customized tools that anyone can use
that have been developed by people who use \texttt{R}. This is one of
the many things that make \texttt{R} so great!

Before we can use all the \emph{functions} in a package, we need to load
those functions in \texttt{R} using the command \texttt{library}. For
example:

\texttt{library(dplyr)}

allows you to use all the \emph{functions} saved in the \texttt{dplyr}
package --- these functions are very useful for data manipulation.

The first time you use a package it may be necessary to download the
package to your RStudio. This is done using the command:

\texttt{install.packages("dplyr")}

Note the quotation marks \texttt{"} that surround the name of the
package.

\textbf{EXERCISE 16} Install \texttt{R} package \emph{dplyr}. Usually,
the command to install a package is run from the console as running it
from \emph{code chunks} can cause problems.

\end{document}
