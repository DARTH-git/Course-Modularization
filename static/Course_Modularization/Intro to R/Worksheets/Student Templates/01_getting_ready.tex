% Options for packages loaded elsewhere
\PassOptionsToPackage{unicode}{hyperref}
\PassOptionsToPackage{hyphens}{url}
%
\documentclass[
]{article}
\usepackage{lmodern}
\usepackage{amssymb,amsmath}
\usepackage{ifxetex,ifluatex}
\ifnum 0\ifxetex 1\fi\ifluatex 1\fi=0 % if pdftex
  \usepackage[T1]{fontenc}
  \usepackage[utf8]{inputenc}
  \usepackage{textcomp} % provide euro and other symbols
\else % if luatex or xetex
  \usepackage{unicode-math}
  \defaultfontfeatures{Scale=MatchLowercase}
  \defaultfontfeatures[\rmfamily]{Ligatures=TeX,Scale=1}
\fi
% Use upquote if available, for straight quotes in verbatim environments
\IfFileExists{upquote.sty}{\usepackage{upquote}}{}
\IfFileExists{microtype.sty}{% use microtype if available
  \usepackage[]{microtype}
  \UseMicrotypeSet[protrusion]{basicmath} % disable protrusion for tt fonts
}{}
\makeatletter
\@ifundefined{KOMAClassName}{% if non-KOMA class
  \IfFileExists{parskip.sty}{%
    \usepackage{parskip}
  }{% else
    \setlength{\parindent}{0pt}
    \setlength{\parskip}{6pt plus 2pt minus 1pt}}
}{% if KOMA class
  \KOMAoptions{parskip=half}}
\makeatother
\usepackage{xcolor}
\IfFileExists{xurl.sty}{\usepackage{xurl}}{} % add URL line breaks if available
\IfFileExists{bookmark.sty}{\usepackage{bookmark}}{\usepackage{hyperref}}
\hypersetup{
  pdftitle={Intro to R for Decision Modeling},
  pdfauthor={SickKids and DARTH},
  hidelinks,
  pdfcreator={LaTeX via pandoc}}
\urlstyle{same} % disable monospaced font for URLs
\usepackage[margin=1in]{geometry}
\usepackage{color}
\usepackage{fancyvrb}
\newcommand{\VerbBar}{|}
\newcommand{\VERB}{\Verb[commandchars=\\\{\}]}
\DefineVerbatimEnvironment{Highlighting}{Verbatim}{commandchars=\\\{\}}
% Add ',fontsize=\small' for more characters per line
\usepackage{framed}
\definecolor{shadecolor}{RGB}{248,248,248}
\newenvironment{Shaded}{\begin{snugshade}}{\end{snugshade}}
\newcommand{\AlertTok}[1]{\textcolor[rgb]{0.94,0.16,0.16}{#1}}
\newcommand{\AnnotationTok}[1]{\textcolor[rgb]{0.56,0.35,0.01}{\textbf{\textit{#1}}}}
\newcommand{\AttributeTok}[1]{\textcolor[rgb]{0.77,0.63,0.00}{#1}}
\newcommand{\BaseNTok}[1]{\textcolor[rgb]{0.00,0.00,0.81}{#1}}
\newcommand{\BuiltInTok}[1]{#1}
\newcommand{\CharTok}[1]{\textcolor[rgb]{0.31,0.60,0.02}{#1}}
\newcommand{\CommentTok}[1]{\textcolor[rgb]{0.56,0.35,0.01}{\textit{#1}}}
\newcommand{\CommentVarTok}[1]{\textcolor[rgb]{0.56,0.35,0.01}{\textbf{\textit{#1}}}}
\newcommand{\ConstantTok}[1]{\textcolor[rgb]{0.00,0.00,0.00}{#1}}
\newcommand{\ControlFlowTok}[1]{\textcolor[rgb]{0.13,0.29,0.53}{\textbf{#1}}}
\newcommand{\DataTypeTok}[1]{\textcolor[rgb]{0.13,0.29,0.53}{#1}}
\newcommand{\DecValTok}[1]{\textcolor[rgb]{0.00,0.00,0.81}{#1}}
\newcommand{\DocumentationTok}[1]{\textcolor[rgb]{0.56,0.35,0.01}{\textbf{\textit{#1}}}}
\newcommand{\ErrorTok}[1]{\textcolor[rgb]{0.64,0.00,0.00}{\textbf{#1}}}
\newcommand{\ExtensionTok}[1]{#1}
\newcommand{\FloatTok}[1]{\textcolor[rgb]{0.00,0.00,0.81}{#1}}
\newcommand{\FunctionTok}[1]{\textcolor[rgb]{0.00,0.00,0.00}{#1}}
\newcommand{\ImportTok}[1]{#1}
\newcommand{\InformationTok}[1]{\textcolor[rgb]{0.56,0.35,0.01}{\textbf{\textit{#1}}}}
\newcommand{\KeywordTok}[1]{\textcolor[rgb]{0.13,0.29,0.53}{\textbf{#1}}}
\newcommand{\NormalTok}[1]{#1}
\newcommand{\OperatorTok}[1]{\textcolor[rgb]{0.81,0.36,0.00}{\textbf{#1}}}
\newcommand{\OtherTok}[1]{\textcolor[rgb]{0.56,0.35,0.01}{#1}}
\newcommand{\PreprocessorTok}[1]{\textcolor[rgb]{0.56,0.35,0.01}{\textit{#1}}}
\newcommand{\RegionMarkerTok}[1]{#1}
\newcommand{\SpecialCharTok}[1]{\textcolor[rgb]{0.00,0.00,0.00}{#1}}
\newcommand{\SpecialStringTok}[1]{\textcolor[rgb]{0.31,0.60,0.02}{#1}}
\newcommand{\StringTok}[1]{\textcolor[rgb]{0.31,0.60,0.02}{#1}}
\newcommand{\VariableTok}[1]{\textcolor[rgb]{0.00,0.00,0.00}{#1}}
\newcommand{\VerbatimStringTok}[1]{\textcolor[rgb]{0.31,0.60,0.02}{#1}}
\newcommand{\WarningTok}[1]{\textcolor[rgb]{0.56,0.35,0.01}{\textbf{\textit{#1}}}}
\usepackage{graphicx,grffile}
\makeatletter
\def\maxwidth{\ifdim\Gin@nat@width>\linewidth\linewidth\else\Gin@nat@width\fi}
\def\maxheight{\ifdim\Gin@nat@height>\textheight\textheight\else\Gin@nat@height\fi}
\makeatother
% Scale images if necessary, so that they will not overflow the page
% margins by default, and it is still possible to overwrite the defaults
% using explicit options in \includegraphics[width, height, ...]{}
\setkeys{Gin}{width=\maxwidth,height=\maxheight,keepaspectratio}
% Set default figure placement to htbp
\makeatletter
\def\fps@figure{htbp}
\makeatother
\setlength{\emergencystretch}{3em} % prevent overfull lines
\providecommand{\tightlist}{%
  \setlength{\itemsep}{0pt}\setlength{\parskip}{0pt}}
\setcounter{secnumdepth}{-\maxdimen} % remove section numbering

\title{Intro to R for Decision Modeling}
\usepackage{etoolbox}
\makeatletter
\providecommand{\subtitle}[1]{% add subtitle to \maketitle
  \apptocmd{\@title}{\par {\large #1 \par}}{}{}
}
\makeatother
\subtitle{Getting Ready}
\author{SickKids and DARTH}
\date{11/2/2020}

\begin{document}
\maketitle

Change \texttt{eval} to \texttt{TRUE} if you want to knit this document.

\hypertarget{introduction-to-r}{%
\section{1. Introduction to R}\label{introduction-to-r}}

\texttt{R} is a specialized program for data manipulation, statistical
analysis, plotting and programming. \texttt{R} has several advantages,
including:

\begin{itemize}
\item
  \texttt{R} makes it easy to store and analyze large dataset,
\item
  \texttt{R} can perform data cleaning and ensure that these data
  cleaning steps are reproducible,
\item
  \texttt{R} includes a large, consistent, integrated collection of
  tools for data analysis,
\item
  \texttt{R} can create publication-quality graphics,
\item
  \texttt{R} can integrate text with data analysis to create Word
  documents and facilitate manuscript development,
\item
  \texttt{R} is based on a programming language that allows you to
  develop your own personalized methods to perform analysis,
\item
  \texttt{R} has a large number of user-developed \emph{packages} that
  can perform specific types of analyses, which can speed up the process
  of analysis significantly. These packages are developed by \texttt{R}
  users to help researchers use novel methods for presentation and
  analysis.
\end{itemize}

\hypertarget{introduction-to-rstudio}{%
\section{2. Introduction to RStudio}\label{introduction-to-rstudio}}

RStudio is a user-friendly interface for \texttt{R} that we will be
using throughout this course. Within RStudio, you will see that the
screen is divided into four parts:

\begin{itemize}
\item
  The top right hand corner contains the \emph{workspace}. The workspace
  is your control centre and gives an at-a-glance overview of what has
  been done so far in your \texttt{R} session.
\item
  The bottom right hand corner is used to load packages, view plotted
  figures and look at help files.
\item
  On the bottom left is \texttt{R} itself - the console - this is the
  machine that will crunch your data. All errors will be displayed here.
\item
  If a file is open in RStudio a text editor will appear in the top-left
  corner.
\end{itemize}

\hypertarget{introduction-to-r-markdown}{%
\section{3. Introduction to R
Markdown}\label{introduction-to-r-markdown}}

R Markdown is a convenient way to keep a record of all the analysis you
have done. R Markdown allows you to include text, \texttt{R} code and
output in your document. Once you are happy with your analysis you can
then output (or \emph{knit} ) your document to a PDF, Word document, or
HTML page. This means you can write your manuscript, perform your
analysis and output tables and graphics all within the same program and
then create a Word document or PDF report or a manuscript to submit to a
journal.

R markdown documents are stored by default by \texttt{R} using an
\texttt{.rmd} extension. This document is an example of such an R
Markdown document! To create a new R Markdown document in RStudio go to
\textbf{File -\textgreater{} New File -\textgreater{} R Markdown}. A
dialog box will open where you can name your \texttt{.rmd} file and
specify the file format you would like to use as your knitted output
(e.g.~\texttt{.doc}, \texttt{.pdf} or HTML).

As you can see at the top of this document, every R Markdown document
starts with a header surrounded by \texttt{-\/-\/-} to specify the
title, document format, and output type. This information is called
\emph{metadata}.

Regular text (like this) can be typed directly into the document while
\texttt{R} code is typed into and run from \emph{code chunks}. The
following section is an R Markdown \emph{code chunk} that calculates 2 +
2:

\begin{Shaded}
\begin{Highlighting}[]
\CommentTok{# test}
\DecValTok{2} \OperatorTok{+}\StringTok{ }\DecValTok{2}

\KeywordTok{message}\NormalTok{(}\StringTok{"a"}\NormalTok{)}
\end{Highlighting}
\end{Shaded}

You can include as many \emph{code chunks} as you want in your documents
so you can intersperse the text of your manuscript with the data
analysis. You chunks are defined using the two commands above that
designate the beginning and end of your chunk. You can insert a new
chunk using the \texttt{Insert} button at the top of this window or
writing the commands to define the beginning and end of the chunks
yourself.

You can run the code in a \emph{code chunk} by clicking the
\textbf{little green arrow} on the top right corner of the \emph{code
chunk}. YOu can run the whole chuck y also pressing CTRL + SHIFT +
ENTER. To only run specific lines of code in a \emph{code chunk}, you
can select them and press CTRL + ENTER.

Try running the code in the above \emph{code chunk}.

It is good practice to insert \emph{comments} in your \texttt{R} code to
explain what analysis you are doing. This makes it easier to navigate
your analysis (and remember what you were doing!). You can make a
\emph{comment} anywhere in the \emph{code chunk} by inserting a
\texttt{\#} before your ``comment''. In the above \emph{code chunk}, the
comment highlighted that we were doing a ``test''. There is a fine
balance between comments within a chunk and text outside it. The amount
of commenting is dependent on what you are aiming to do with the final
document. If you are writing a manuscript then the text should include
\emph{only} the information you wish to include in the manuscript and
the comments should be used to explain what you are doing in the code.
If you are using R Markdown to give a commented version of your data
analysis then you can use the text to describe you analysis and minimize
the number of comments within the code.

When you have more \emph{code chunks} in your R Markdown document, you
may need more flexible ways to run code and view outputs. At the top
right corner of this window, you can see a green arrow pointing towards
a button that says \textbf{``Run''}. If you click on this button, you
will find ways to run multiple \emph{code chunks} in your document.
Specifically, if you put your cursor in a block of text, you can
\textbf{Run All Chunks Above} or \textbf{Run All Chunks Below} your
cursor. You can also \textbf{Run All} the code chunks in your document.
There are several keyboard shortcuts that can be used to run chunks in
an R Markdown document. A great resource for those is the following
webpage: \url{https://rmd4sci.njtierney.com/keyboard-shortcuts}.

To knit the R Markdown document go to \textbf{File -\textgreater{} Knit
Document} or press \textbf{Knit} at the top left-hand corner of this
window. Try knitting this R Markdown document to a Word document.

To knit a R Markdown document to a PDF document, you will need to
download MiKTeX on your machine first. It is an up-to-date
implementation of TeX/LaTeX and related programs. MiKTeX can be
downloaded from the following webpage:
\url{https://miktex.org/download}.

Throughout this course, we will providing all the materials as R
Markdown documents.

\end{document}
