% Options for packages loaded elsewhere
\PassOptionsToPackage{unicode}{hyperref}
\PassOptionsToPackage{hyphens}{url}
%
\documentclass[
]{article}
\usepackage{lmodern}
\usepackage{amssymb,amsmath}
\usepackage{ifxetex,ifluatex}
\ifnum 0\ifxetex 1\fi\ifluatex 1\fi=0 % if pdftex
  \usepackage[T1]{fontenc}
  \usepackage[utf8]{inputenc}
  \usepackage{textcomp} % provide euro and other symbols
\else % if luatex or xetex
  \usepackage{unicode-math}
  \defaultfontfeatures{Scale=MatchLowercase}
  \defaultfontfeatures[\rmfamily]{Ligatures=TeX,Scale=1}
\fi
% Use upquote if available, for straight quotes in verbatim environments
\IfFileExists{upquote.sty}{\usepackage{upquote}}{}
\IfFileExists{microtype.sty}{% use microtype if available
  \usepackage[]{microtype}
  \UseMicrotypeSet[protrusion]{basicmath} % disable protrusion for tt fonts
}{}
\makeatletter
\@ifundefined{KOMAClassName}{% if non-KOMA class
  \IfFileExists{parskip.sty}{%
    \usepackage{parskip}
  }{% else
    \setlength{\parindent}{0pt}
    \setlength{\parskip}{6pt plus 2pt minus 1pt}}
}{% if KOMA class
  \KOMAoptions{parskip=half}}
\makeatother
\usepackage{xcolor}
\IfFileExists{xurl.sty}{\usepackage{xurl}}{} % add URL line breaks if available
\IfFileExists{bookmark.sty}{\usepackage{bookmark}}{\usepackage{hyperref}}
\hypersetup{
  pdftitle={Simple 3-state Markov model in R},
  pdfauthor={The DARTH workgroup},
  hidelinks,
  pdfcreator={LaTeX via pandoc}}
\urlstyle{same} % disable monospaced font for URLs
\usepackage[margin=1in]{geometry}
\usepackage{color}
\usepackage{fancyvrb}
\newcommand{\VerbBar}{|}
\newcommand{\VERB}{\Verb[commandchars=\\\{\}]}
\DefineVerbatimEnvironment{Highlighting}{Verbatim}{commandchars=\\\{\}}
% Add ',fontsize=\small' for more characters per line
\usepackage{framed}
\definecolor{shadecolor}{RGB}{248,248,248}
\newenvironment{Shaded}{\begin{snugshade}}{\end{snugshade}}
\newcommand{\AlertTok}[1]{\textcolor[rgb]{0.94,0.16,0.16}{#1}}
\newcommand{\AnnotationTok}[1]{\textcolor[rgb]{0.56,0.35,0.01}{\textbf{\textit{#1}}}}
\newcommand{\AttributeTok}[1]{\textcolor[rgb]{0.77,0.63,0.00}{#1}}
\newcommand{\BaseNTok}[1]{\textcolor[rgb]{0.00,0.00,0.81}{#1}}
\newcommand{\BuiltInTok}[1]{#1}
\newcommand{\CharTok}[1]{\textcolor[rgb]{0.31,0.60,0.02}{#1}}
\newcommand{\CommentTok}[1]{\textcolor[rgb]{0.56,0.35,0.01}{\textit{#1}}}
\newcommand{\CommentVarTok}[1]{\textcolor[rgb]{0.56,0.35,0.01}{\textbf{\textit{#1}}}}
\newcommand{\ConstantTok}[1]{\textcolor[rgb]{0.00,0.00,0.00}{#1}}
\newcommand{\ControlFlowTok}[1]{\textcolor[rgb]{0.13,0.29,0.53}{\textbf{#1}}}
\newcommand{\DataTypeTok}[1]{\textcolor[rgb]{0.13,0.29,0.53}{#1}}
\newcommand{\DecValTok}[1]{\textcolor[rgb]{0.00,0.00,0.81}{#1}}
\newcommand{\DocumentationTok}[1]{\textcolor[rgb]{0.56,0.35,0.01}{\textbf{\textit{#1}}}}
\newcommand{\ErrorTok}[1]{\textcolor[rgb]{0.64,0.00,0.00}{\textbf{#1}}}
\newcommand{\ExtensionTok}[1]{#1}
\newcommand{\FloatTok}[1]{\textcolor[rgb]{0.00,0.00,0.81}{#1}}
\newcommand{\FunctionTok}[1]{\textcolor[rgb]{0.00,0.00,0.00}{#1}}
\newcommand{\ImportTok}[1]{#1}
\newcommand{\InformationTok}[1]{\textcolor[rgb]{0.56,0.35,0.01}{\textbf{\textit{#1}}}}
\newcommand{\KeywordTok}[1]{\textcolor[rgb]{0.13,0.29,0.53}{\textbf{#1}}}
\newcommand{\NormalTok}[1]{#1}
\newcommand{\OperatorTok}[1]{\textcolor[rgb]{0.81,0.36,0.00}{\textbf{#1}}}
\newcommand{\OtherTok}[1]{\textcolor[rgb]{0.56,0.35,0.01}{#1}}
\newcommand{\PreprocessorTok}[1]{\textcolor[rgb]{0.56,0.35,0.01}{\textit{#1}}}
\newcommand{\RegionMarkerTok}[1]{#1}
\newcommand{\SpecialCharTok}[1]{\textcolor[rgb]{0.00,0.00,0.00}{#1}}
\newcommand{\SpecialStringTok}[1]{\textcolor[rgb]{0.31,0.60,0.02}{#1}}
\newcommand{\StringTok}[1]{\textcolor[rgb]{0.31,0.60,0.02}{#1}}
\newcommand{\VariableTok}[1]{\textcolor[rgb]{0.00,0.00,0.00}{#1}}
\newcommand{\VerbatimStringTok}[1]{\textcolor[rgb]{0.31,0.60,0.02}{#1}}
\newcommand{\WarningTok}[1]{\textcolor[rgb]{0.56,0.35,0.01}{\textbf{\textit{#1}}}}
\usepackage{graphicx,grffile}
\makeatletter
\def\maxwidth{\ifdim\Gin@nat@width>\linewidth\linewidth\else\Gin@nat@width\fi}
\def\maxheight{\ifdim\Gin@nat@height>\textheight\textheight\else\Gin@nat@height\fi}
\makeatother
% Scale images if necessary, so that they will not overflow the page
% margins by default, and it is still possible to overwrite the defaults
% using explicit options in \includegraphics[width, height, ...]{}
\setkeys{Gin}{width=\maxwidth,height=\maxheight,keepaspectratio}
% Set default figure placement to htbp
\makeatletter
\def\fps@figure{htbp}
\makeatother
\setlength{\emergencystretch}{3em} % prevent overfull lines
\providecommand{\tightlist}{%
  \setlength{\itemsep}{0pt}\setlength{\parskip}{0pt}}
\setcounter{secnumdepth}{-\maxdimen} % remove section numbering

\title{Simple 3-state Markov model in R}
\usepackage{etoolbox}
\makeatletter
\providecommand{\subtitle}[1]{% add subtitle to \maketitle
  \apptocmd{\@title}{\par {\large #1 \par}}{}{}
}
\makeatother
\subtitle{with age dependency}
\author{The DARTH workgroup}
\date{}

\begin{document}
\maketitle

Developed by the Decision Analysis in R for Technologies in Health
(DARTH) workgroup:

Fernando Alarid-Escudero, PhD (1)

Eva A. Enns, MS, PhD (2)

M.G. Myriam Hunink, MD, PhD (3,4)

Hawre J. Jalal, MD, PhD (5)

Eline M. Krijkamp, MSc (3)

Petros Pechlivanoglou, PhD (6,7)

Alan Yang, MSc (7)

In collaboration of:

\begin{enumerate}
\def\labelenumi{\arabic{enumi}.}
\tightlist
\item
  Division of Public Administration, Center for Research and Teaching in
  Economics (CIDE), Aguascalientes, Mexico
\item
  University of Minnesota School of Public Health, Minneapolis, MN, USA
\item
  Erasmus MC, Rotterdam, The Netherlands
\item
  Harvard T.H. Chan School of Public Health, Boston, USA
\item
  University of Pittsburgh Graduate School of Public Health, Pittsburgh,
  PA, USA
\item
  University of Toronto, Toronto ON, Canada
\item
  The Hospital for Sick Children, Toronto ON, Canada
\end{enumerate}

Please cite our publications when using this code:

\begin{itemize}
\item
  Jalal H, Pechlivanoglou P, Krijkamp E, Alarid-Escudero F, Enns E,
  Hunink MG. An Overview of R in Health Decision Sciences. Med Decis
  Making. 2017; 37(3): 735-746.
  \url{https://journals.sagepub.com/doi/abs/10.1177/0272989X16686559}
\item
  Alarid-Escudero F, Krijkamp EM, Enns EA, Yang A, Hunink MGM
  Pechlivanoglou P, Jalal H. Cohort State-Transition Models in R: A
  Tutorial. arXiv:200107824v2. 2020:1-48.
  \url{http://arxiv.org/abs/2001.07824}
\item
  Krijkamp EM, Alarid-Escudero F, Enns EA, Jalal HJ, Hunink MGM,
  Pechlivanoglou P. Microsimulation modeling for health decision
  sciences using R: A tutorial. Med Decis Making. 2018;38(3):400--22.
  \url{https://journals.sagepub.com/doi/abs/10.1177/0272989X18754513}
\item
  Krijkamp EM, Alarid-Escudero F, Enns E, Pechlivanoglou P, Hunink MM,
  Jalal H. A Multidimensional Array Representation of State-Transition
  Model Dynamics. Med Decis Making. 2020 Feb;40(2):242-248.
  \url{https://journals.sagepub.com/doi/10.1177/0272989X19893973}
\end{itemize}

Copyright 2017, THE HOSPITAL FOR SICK CHILDREN AND THE COLLABORATING
INSTITUTIONS. All rights reserved in Canada, the United States and
worldwide. Copyright, trademarks, trade names and any and all associated
intellectual property are exclusively owned by THE HOSPITAL FOR SICK
CHILDREN and the collaborating institutions. These materials may be
used, reproduced, modified, distributed and adapted with proper
attribution.

\newpage

Change \texttt{eval} to \texttt{TRUE} if you want to knit this document.

\begin{Shaded}
\begin{Highlighting}[]
\KeywordTok{rm}\NormalTok{(}\DataTypeTok{list =} \KeywordTok{ls}\NormalTok{())      }\CommentTok{# clear memory (removes all the variables from the workspace)}
\end{Highlighting}
\end{Shaded}

\hypertarget{load-packages}{%
\section{01 Load packages}\label{load-packages}}

\begin{Shaded}
\begin{Highlighting}[]
\ControlFlowTok{if}\NormalTok{ (}\OperatorTok{!}\KeywordTok{require}\NormalTok{(}\StringTok{'pacman'}\NormalTok{)) }\KeywordTok{install.packages}\NormalTok{(}\StringTok{'pacman'}\NormalTok{); }\KeywordTok{library}\NormalTok{(pacman) }\CommentTok{# use this package to conveniently install other packages}
\CommentTok{# load (install if required) packages from CRAN}
\KeywordTok{p_load}\NormalTok{(}\StringTok{"diagram"}\NormalTok{) }
\CommentTok{# install_github("DARTH-git/darthtools", force = TRUE) Uncomment if there is a newer version}
\KeywordTok{p_load_gh}\NormalTok{(}\StringTok{"DARTH-git/darthtools"}\NormalTok{)}
\end{Highlighting}
\end{Shaded}

\hypertarget{load-functions}{%
\section{02 Load functions}\label{load-functions}}

\begin{Shaded}
\begin{Highlighting}[]
\CommentTok{# No functions needed}
\end{Highlighting}
\end{Shaded}

\hypertarget{input-model-parameters}{%
\section{03 Input model parameters}\label{input-model-parameters}}

\begin{Shaded}
\begin{Highlighting}[]
\CommentTok{# Strategy names}
\NormalTok{v_names_str <-}\StringTok{ }\KeywordTok{c}\NormalTok{(}\StringTok{"Standard of Care"}\NormalTok{)  }

\CommentTok{# Markov model parameters}
\NormalTok{v_n  <-}\StringTok{ }\KeywordTok{c}\NormalTok{(}\StringTok{"Healthy"}\NormalTok{, }\StringTok{"Sick"}\NormalTok{, }\StringTok{"Dead"}\NormalTok{)  }\CommentTok{# state names}
\NormalTok{n_t  <-}\StringTok{ }\DecValTok{60}                            \CommentTok{# number of cycles}

\NormalTok{v_init <-}\StringTok{ }\KeywordTok{c}\NormalTok{(}\StringTok{"Healthy"}\NormalTok{ =}\StringTok{ }\DecValTok{1}\NormalTok{,}
            \StringTok{"Sick"}\NormalTok{    =}\StringTok{ }\DecValTok{0}\NormalTok{,}
            \StringTok{"Dead"}\NormalTok{    =}\StringTok{ }\DecValTok{0}\NormalTok{)            }\CommentTok{# initial cohort distribution (everyone allocated to the }
                                      \CommentTok{# "healthy" state)}

\CommentTok{# Transition probabilities}
\NormalTok{p_HD_min <-}\StringTok{ }\FloatTok{0.003}                     \CommentTok{# probability of dying when healthy at t = 0}
\NormalTok{p_HD_max <-}\StringTok{ }\FloatTok{0.01}                      \CommentTok{# probability of dying when health  at t = n.t}
\NormalTok{p_HS     <-}\StringTok{ }\FloatTok{0.05}                      \CommentTok{# probability of becoming sick when healthy, under standard of care}
\NormalTok{p_SD     <-}\StringTok{ }\FloatTok{0.1}                       \CommentTok{# probability of dying when sick}

\CommentTok{# Costs and utilities  }
\NormalTok{c_H      <-}\StringTok{ }\DecValTok{400}                       \CommentTok{# cost of one cycle in healthy state}
\NormalTok{c_S      <-}\StringTok{ }\DecValTok{1000}                      \CommentTok{# cost of one cycle in sick state}
\NormalTok{c_D      <-}\StringTok{ }\DecValTok{0}                         \CommentTok{# cost of one cycle in dead state}
\NormalTok{u_H      <-}\StringTok{ }\FloatTok{0.8}                       \CommentTok{# utility when healthy }
\NormalTok{u_S      <-}\StringTok{ }\FloatTok{0.5}                       \CommentTok{# utility when sick}
\NormalTok{u_D      <-}\StringTok{ }\DecValTok{0}                         \CommentTok{# utility when dead}
\NormalTok{d_e      <-}\StringTok{ }\NormalTok{d_c <-}\StringTok{ }\FloatTok{0.03}               \CommentTok{# discount rate per cycle equal discount of costs and QALYs by 3%}


\NormalTok{p_HD <-}\StringTok{ }\KeywordTok{seq}\NormalTok{(p_HD_min, p_HD_max, }\DataTypeTok{length.out =}\NormalTok{ n_t)  }\CommentTok{# probabilities of dying when healthy (age-dependent) - }
                                                   \CommentTok{# this is now a sequence of numbers, officially v_p_HD}

\NormalTok{n_str     <-}\StringTok{ }\KeywordTok{length}\NormalTok{(v_names_str)      }\CommentTok{# Number of strategies}
\NormalTok{n_states  <-}\StringTok{ }\KeywordTok{length}\NormalTok{(v_n)              }\CommentTok{# number of states}

\CommentTok{# Discount weights for costs and effects}
\NormalTok{v_dwc <-}\StringTok{ }\DecValTok{1} \OperatorTok{/}\StringTok{ }\NormalTok{(}\DecValTok{1} \OperatorTok{+}\StringTok{ }\NormalTok{d_c) }\OperatorTok{^}\StringTok{ }\NormalTok{(}\DecValTok{0}\OperatorTok{:}\NormalTok{n_t) }
\NormalTok{v_dwe <-}\StringTok{ }\DecValTok{1} \OperatorTok{/}\StringTok{ }\NormalTok{(}\DecValTok{1} \OperatorTok{+}\StringTok{ }\NormalTok{d_e) }\OperatorTok{^}\StringTok{ }\NormalTok{(}\DecValTok{0}\OperatorTok{:}\NormalTok{n_t) }
\end{Highlighting}
\end{Shaded}

\hypertarget{define-and-initialize-matrices-and-vectors}{%
\section{04 Define and initialize matrices and
vectors}\label{define-and-initialize-matrices-and-vectors}}

\hypertarget{cohort-trace}{%
\subsection{04.1 Cohort trace}\label{cohort-trace}}

\begin{Shaded}
\begin{Highlighting}[]
\CommentTok{# create the cohort trace}
\NormalTok{m_M <-}\StringTok{ }\KeywordTok{matrix}\NormalTok{(}\OtherTok{NA}\NormalTok{, }
              \DataTypeTok{nrow =}\NormalTok{ n_t }\OperatorTok{+}\StringTok{ }\DecValTok{1}\NormalTok{ ,  }\CommentTok{# create Markov trace (n.t + 1 because R doesn't }
                                \CommentTok{# understand cycle 0)}
              \DataTypeTok{ncol =}\NormalTok{ n_states, }
              \DataTypeTok{dimnames =} \KeywordTok{list}\NormalTok{(}\DecValTok{0}\OperatorTok{:}\NormalTok{n_t, v_n))}

\NormalTok{m_M[}\DecValTok{1}\NormalTok{, ] <-}\StringTok{ }\KeywordTok{c}\NormalTok{(}\DecValTok{1}\NormalTok{, }\DecValTok{0}\NormalTok{, }\DecValTok{0}\NormalTok{)          }\CommentTok{# initialize first cycle of Markov trace}
\end{Highlighting}
\end{Shaded}

\hypertarget{transition-probability-array}{%
\subsection{04.2 Transition probability
array}\label{transition-probability-array}}

\begin{Shaded}
\begin{Highlighting}[]
\CommentTok{# create the transition probability array}
\NormalTok{a_P <-}\StringTok{ }\KeywordTok{array}\NormalTok{(}\DecValTok{0}\NormalTok{,                                      }\CommentTok{# Create 3-D array}
             \DataTypeTok{dim =} \KeywordTok{c}\NormalTok{(n_states, n_states, n_t),}
             \DataTypeTok{dimnames =} \KeywordTok{list}\NormalTok{(v_n, v_n, }\DecValTok{0}\OperatorTok{:}\NormalTok{(n_t }\OperatorTok{-}\StringTok{ }\DecValTok{1}\NormalTok{))) }\CommentTok{# name the dimensions of the array }
\end{Highlighting}
\end{Shaded}

Fill in the transition probability array:

\begin{Shaded}
\begin{Highlighting}[]
\CommentTok{# from Healthy}
\NormalTok{a_P[}\StringTok{"Healthy"}\NormalTok{, }\StringTok{"Healthy"}\NormalTok{, ] <-}\StringTok{ }\NormalTok{(}\DecValTok{1} \OperatorTok{-}\StringTok{ }\NormalTok{p_HD) }\OperatorTok{*}\StringTok{ }\NormalTok{(}\DecValTok{1} \OperatorTok{-}\StringTok{ }\NormalTok{p_HS)}
\NormalTok{a_P[}\StringTok{"Healthy"}\NormalTok{, }\StringTok{"Sick"}\NormalTok{, ]    <-}\StringTok{ }\NormalTok{(}\DecValTok{1} \OperatorTok{-}\StringTok{ }\NormalTok{p_HD) }\OperatorTok{*}\StringTok{ }\NormalTok{p_HS}
\NormalTok{a_P[}\StringTok{"Healthy"}\NormalTok{, }\StringTok{"Dead"}\NormalTok{, ]    <-}\StringTok{  }\NormalTok{p_HD}

\CommentTok{# from Sick}
\NormalTok{a_P[}\StringTok{"Sick"}\NormalTok{, }\StringTok{"Sick"}\NormalTok{, ] <-}\StringTok{ }\DecValTok{1} \OperatorTok{-}\StringTok{ }\NormalTok{p_SD}
\NormalTok{a_P[}\StringTok{"Sick"}\NormalTok{, }\StringTok{"Dead"}\NormalTok{, ] <-}\StringTok{ }\NormalTok{p_SD}

\CommentTok{# from Dead}
\NormalTok{a_P[}\StringTok{"Dead"}\NormalTok{, }\StringTok{"Dead"}\NormalTok{, ] <-}\StringTok{ }\DecValTok{1}
\end{Highlighting}
\end{Shaded}

\hypertarget{check-if-transition-array-and-probabilities-are-valid}{%
\subsection{04.3 Check if transition array and probabilities are
valid}\label{check-if-transition-array-and-probabilities-are-valid}}

\begin{Shaded}
\begin{Highlighting}[]
\CommentTok{# Check that transition probabilities are in [0, 1]}
\KeywordTok{check_transition_probability}\NormalTok{(a_P, }\DataTypeTok{verbose =} \OtherTok{TRUE}\NormalTok{)}
\CommentTok{# Check that all rows sum to 1}
\KeywordTok{check_sum_of_transition_array}\NormalTok{(a_P, }\DataTypeTok{n_states =}\NormalTok{ n_states, }\DataTypeTok{n_cycles =}\NormalTok{ n_t, }\DataTypeTok{verbose =} \OtherTok{TRUE}\NormalTok{)}
\end{Highlighting}
\end{Shaded}

\hypertarget{run-markov-model}{%
\section{05 Run Markov model}\label{run-markov-model}}

\begin{Shaded}
\begin{Highlighting}[]
\ControlFlowTok{for}\NormalTok{ (t }\ControlFlowTok{in} \DecValTok{1}\OperatorTok{:}\NormalTok{n_t)\{ }\CommentTok{# t<-1                  # loop through the number of cycles}
\NormalTok{  m_M[t }\OperatorTok{+}\StringTok{ }\DecValTok{1}\NormalTok{, ] <-}\StringTok{ }\NormalTok{m_M[t, ] }\OperatorTok\StringTok{ }\NormalTok{a_P[, , t] }\CommentTok{# estimate the Markov trace for cycle t + 1 }
                                          \CommentTok{# using the t-th matrix from the }
                                          \CommentTok{# probability array }
\NormalTok{\}}
\KeywordTok{head}\NormalTok{(m_M)  }\CommentTok{# print the first lines of the matrix }
\end{Highlighting}
\end{Shaded}

\hypertarget{compute-and-plot-epidemiological-outcomes}{%
\section{06 Compute and Plot Epidemiological
Outcomes}\label{compute-and-plot-epidemiological-outcomes}}

\hypertarget{cohort-trace-1}{%
\subsection{06.1 Cohort trace}\label{cohort-trace-1}}

\begin{Shaded}
\begin{Highlighting}[]
\CommentTok{# create a plot of the data}
\KeywordTok{matplot}\NormalTok{(m_M, }\DataTypeTok{type =} \StringTok{'l'}\NormalTok{, }
        \DataTypeTok{ylab =} \StringTok{"Probability of state occupancy"}\NormalTok{,}
        \DataTypeTok{xlab =} \StringTok{"Cycle"}\NormalTok{,}
        \DataTypeTok{main =} \StringTok{"Cohort Trace"}\NormalTok{, }\DataTypeTok{lwd =} \DecValTok{2}\NormalTok{)              }
\CommentTok{# add a legend to the graph}
\KeywordTok{legend}\NormalTok{(}\StringTok{"right"}\NormalTok{, v_n, }\DataTypeTok{col =} \KeywordTok{c}\NormalTok{(}\StringTok{"black"}\NormalTok{, }\StringTok{"red"}\NormalTok{, }\StringTok{"green"}\NormalTok{), }\DataTypeTok{lty =} \DecValTok{1}\OperatorTok{:}\DecValTok{3}\NormalTok{, }\DataTypeTok{bty =} \StringTok{"n"}\NormalTok{)  }
\end{Highlighting}
\end{Shaded}

\hypertarget{overall-survival-os}{%
\subsection{06.2 Overall Survival (OS)}\label{overall-survival-os}}

\begin{Shaded}
\begin{Highlighting}[]
\NormalTok{v_os <-}\StringTok{ }\DecValTok{1} \OperatorTok{-}\StringTok{ }\NormalTok{m_M[, }\StringTok{"Dead"}\NormalTok{]   }\CommentTok{# calculate the overall survival (OS) probability}
\NormalTok{v_os <-}\StringTok{ }\KeywordTok{rowSums}\NormalTok{(m_M[, }\DecValTok{1}\OperatorTok{:}\DecValTok{2}\NormalTok{]) }\CommentTok{# alternative way of calculating the OS probability   }

\CommentTok{# create a simple plot showing the OS}
\KeywordTok{plot}\NormalTok{(v_os, }\DataTypeTok{type =} \StringTok{'l'}\NormalTok{, }
     \DataTypeTok{ylim =} \KeywordTok{c}\NormalTok{(}\DecValTok{0}\NormalTok{, }\DecValTok{1}\NormalTok{),}
     \DataTypeTok{ylab =} \StringTok{"Survival probability"}\NormalTok{,}
     \DataTypeTok{xlab =} \StringTok{"Cycle"}\NormalTok{,}
     \DataTypeTok{main =} \StringTok{"Overall Survival"}\NormalTok{)   }
\CommentTok{# add grid }
\KeywordTok{grid}\NormalTok{(}\DataTypeTok{nx =}\NormalTok{ n_t, }\DataTypeTok{ny =} \DecValTok{10}\NormalTok{, }
     \DataTypeTok{col =} \StringTok{"lightgray"}\NormalTok{, }\DataTypeTok{lty =} \StringTok{"dotted"}\NormalTok{, }\DataTypeTok{lwd =} \KeywordTok{par}\NormalTok{(}\StringTok{"lwd"}\NormalTok{), }
     \DataTypeTok{equilogs =} \OtherTok{TRUE}\NormalTok{) }
\end{Highlighting}
\end{Shaded}

\hypertarget{life-expectancy-le}{%
\subsection{06.2.1 Life Expectancy (LE)}\label{life-expectancy-le}}

\begin{Shaded}
\begin{Highlighting}[]
\NormalTok{v_le <-}\StringTok{ }\KeywordTok{sum}\NormalTok{(v_os) }\CommentTok{# summing probablity of OS over time  (i.e. life expectancy)}
\end{Highlighting}
\end{Shaded}

\hypertarget{disease-prevalence}{%
\subsection{06.3 Disease prevalence}\label{disease-prevalence}}

\begin{Shaded}
\begin{Highlighting}[]
\NormalTok{v_prev <-}\StringTok{ }\NormalTok{m_M[, }\StringTok{"Sick"}\NormalTok{]}\OperatorTok{/}\NormalTok{v_os}
\KeywordTok{plot}\NormalTok{(v_prev,}
     \DataTypeTok{ylim =} \KeywordTok{c}\NormalTok{(}\DecValTok{0}\NormalTok{, }\DecValTok{1}\NormalTok{),}
     \DataTypeTok{ylab =} \StringTok{"Prevalence"}\NormalTok{,}
     \DataTypeTok{xlab =} \StringTok{"Cycle"}\NormalTok{,}
     \DataTypeTok{main =} \StringTok{"Disease prevalence"}\NormalTok{)}
\end{Highlighting}
\end{Shaded}

\hypertarget{compute-cost-effectiveness-outcomes}{%
\section{07 Compute Cost-Effectiveness
Outcomes}\label{compute-cost-effectiveness-outcomes}}

\hypertarget{mean-costs-and-qalys}{%
\subsection{07.1 Mean Costs and QALYs}\label{mean-costs-and-qalys}}

\begin{Shaded}
\begin{Highlighting}[]
\CommentTok{# per cycle}
\CommentTok{# calculate expected costs by multiplying m_M with the cost vector for the different }
\CommentTok{# health states   }
\NormalTok{v_tc <-}\StringTok{ }\NormalTok{m_M }\OperatorTok\StringTok{ }\KeywordTok{c}\NormalTok{(c_H, c_S, c_D)  }
\CommentTok{# calculate expected QALYs by multiplying m_M with the utilities for the different }
\CommentTok{# health states   }
\NormalTok{v_tu <-}\StringTok{ }\NormalTok{m_M }\OperatorTok\StringTok{ }\KeywordTok{c}\NormalTok{(u_H, u_S, u_D)  }
\end{Highlighting}
\end{Shaded}

\hypertarget{discounted-mean-costs-and-qalys}{%
\subsection{07.2 Discounted Mean Costs and
QALYs}\label{discounted-mean-costs-and-qalys}}

\begin{Shaded}
\begin{Highlighting}[]
\CommentTok{# Discount costs by multiplying the cost vector with discount weights (v_dw) }
\NormalTok{v_tc_d <-}\StringTok{  }\KeywordTok{t}\NormalTok{(v_tc) }\OperatorTok\StringTok{ }\NormalTok{v_dwc}
\CommentTok{# Discount QALYS by multiplying the QALYs vector with discount weights (v_dw)}
\NormalTok{v_te_d <-}\StringTok{  }\KeywordTok{t}\NormalTok{(v_tu) }\OperatorTok\StringTok{ }\NormalTok{v_dwe}
\end{Highlighting}
\end{Shaded}

\hypertarget{store-results}{%
\subsection{07.3 Store Results}\label{store-results}}

\begin{Shaded}
\begin{Highlighting}[]
\NormalTok{df_ce <-}\StringTok{ }\KeywordTok{data.frame}\NormalTok{(}\StringTok{"Total Discounted Cost"}\NormalTok{  =}\StringTok{ }\NormalTok{v_tc_d, }
                    \StringTok{"Life Expectancy"}\NormalTok{        =}\StringTok{ }\NormalTok{v_le, }
                    \StringTok{"Total Discounted QALYs"}\NormalTok{ =}\StringTok{ }\NormalTok{v_te_d, }
                    \DataTypeTok{check.names =}\NormalTok{ F)}
\NormalTok{df_ce}
\end{Highlighting}
\end{Shaded}

\end{document}
