% Options for packages loaded elsewhere
\PassOptionsToPackage{unicode}{hyperref}
\PassOptionsToPackage{hyphens}{url}
%
\documentclass[
]{article}
\usepackage{lmodern}
\usepackage{amssymb,amsmath}
\usepackage{ifxetex,ifluatex}
\ifnum 0\ifxetex 1\fi\ifluatex 1\fi=0 % if pdftex
  \usepackage[T1]{fontenc}
  \usepackage[utf8]{inputenc}
  \usepackage{textcomp} % provide euro and other symbols
\else % if luatex or xetex
  \usepackage{unicode-math}
  \defaultfontfeatures{Scale=MatchLowercase}
  \defaultfontfeatures[\rmfamily]{Ligatures=TeX,Scale=1}
\fi
% Use upquote if available, for straight quotes in verbatim environments
\IfFileExists{upquote.sty}{\usepackage{upquote}}{}
\IfFileExists{microtype.sty}{% use microtype if available
  \usepackage[]{microtype}
  \UseMicrotypeSet[protrusion]{basicmath} % disable protrusion for tt fonts
}{}
\makeatletter
\@ifundefined{KOMAClassName}{% if non-KOMA class
  \IfFileExists{parskip.sty}{%
    \usepackage{parskip}
  }{% else
    \setlength{\parindent}{0pt}
    \setlength{\parskip}{6pt plus 2pt minus 1pt}}
}{% if KOMA class
  \KOMAoptions{parskip=half}}
\makeatother
\usepackage{xcolor}
\IfFileExists{xurl.sty}{\usepackage{xurl}}{} % add URL line breaks if available
\IfFileExists{bookmark.sty}{\usepackage{bookmark}}{\usepackage{hyperref}}
\hypersetup{
  pdftitle={PSA: Markov Sick-Sicker model in R},
  pdfauthor={The DARTH workgroup},
  hidelinks,
  pdfcreator={LaTeX via pandoc}}
\urlstyle{same} % disable monospaced font for URLs
\usepackage[margin=1in]{geometry}
\usepackage{color}
\usepackage{fancyvrb}
\newcommand{\VerbBar}{|}
\newcommand{\VERB}{\Verb[commandchars=\\\{\}]}
\DefineVerbatimEnvironment{Highlighting}{Verbatim}{commandchars=\\\{\}}
% Add ',fontsize=\small' for more characters per line
\usepackage{framed}
\definecolor{shadecolor}{RGB}{248,248,248}
\newenvironment{Shaded}{\begin{snugshade}}{\end{snugshade}}
\newcommand{\AlertTok}[1]{\textcolor[rgb]{0.94,0.16,0.16}{#1}}
\newcommand{\AnnotationTok}[1]{\textcolor[rgb]{0.56,0.35,0.01}{\textbf{\textit{#1}}}}
\newcommand{\AttributeTok}[1]{\textcolor[rgb]{0.77,0.63,0.00}{#1}}
\newcommand{\BaseNTok}[1]{\textcolor[rgb]{0.00,0.00,0.81}{#1}}
\newcommand{\BuiltInTok}[1]{#1}
\newcommand{\CharTok}[1]{\textcolor[rgb]{0.31,0.60,0.02}{#1}}
\newcommand{\CommentTok}[1]{\textcolor[rgb]{0.56,0.35,0.01}{\textit{#1}}}
\newcommand{\CommentVarTok}[1]{\textcolor[rgb]{0.56,0.35,0.01}{\textbf{\textit{#1}}}}
\newcommand{\ConstantTok}[1]{\textcolor[rgb]{0.00,0.00,0.00}{#1}}
\newcommand{\ControlFlowTok}[1]{\textcolor[rgb]{0.13,0.29,0.53}{\textbf{#1}}}
\newcommand{\DataTypeTok}[1]{\textcolor[rgb]{0.13,0.29,0.53}{#1}}
\newcommand{\DecValTok}[1]{\textcolor[rgb]{0.00,0.00,0.81}{#1}}
\newcommand{\DocumentationTok}[1]{\textcolor[rgb]{0.56,0.35,0.01}{\textbf{\textit{#1}}}}
\newcommand{\ErrorTok}[1]{\textcolor[rgb]{0.64,0.00,0.00}{\textbf{#1}}}
\newcommand{\ExtensionTok}[1]{#1}
\newcommand{\FloatTok}[1]{\textcolor[rgb]{0.00,0.00,0.81}{#1}}
\newcommand{\FunctionTok}[1]{\textcolor[rgb]{0.00,0.00,0.00}{#1}}
\newcommand{\ImportTok}[1]{#1}
\newcommand{\InformationTok}[1]{\textcolor[rgb]{0.56,0.35,0.01}{\textbf{\textit{#1}}}}
\newcommand{\KeywordTok}[1]{\textcolor[rgb]{0.13,0.29,0.53}{\textbf{#1}}}
\newcommand{\NormalTok}[1]{#1}
\newcommand{\OperatorTok}[1]{\textcolor[rgb]{0.81,0.36,0.00}{\textbf{#1}}}
\newcommand{\OtherTok}[1]{\textcolor[rgb]{0.56,0.35,0.01}{#1}}
\newcommand{\PreprocessorTok}[1]{\textcolor[rgb]{0.56,0.35,0.01}{\textit{#1}}}
\newcommand{\RegionMarkerTok}[1]{#1}
\newcommand{\SpecialCharTok}[1]{\textcolor[rgb]{0.00,0.00,0.00}{#1}}
\newcommand{\SpecialStringTok}[1]{\textcolor[rgb]{0.31,0.60,0.02}{#1}}
\newcommand{\StringTok}[1]{\textcolor[rgb]{0.31,0.60,0.02}{#1}}
\newcommand{\VariableTok}[1]{\textcolor[rgb]{0.00,0.00,0.00}{#1}}
\newcommand{\VerbatimStringTok}[1]{\textcolor[rgb]{0.31,0.60,0.02}{#1}}
\newcommand{\WarningTok}[1]{\textcolor[rgb]{0.56,0.35,0.01}{\textbf{\textit{#1}}}}
\usepackage{graphicx,grffile}
\makeatletter
\def\maxwidth{\ifdim\Gin@nat@width>\linewidth\linewidth\else\Gin@nat@width\fi}
\def\maxheight{\ifdim\Gin@nat@height>\textheight\textheight\else\Gin@nat@height\fi}
\makeatother
% Scale images if necessary, so that they will not overflow the page
% margins by default, and it is still possible to overwrite the defaults
% using explicit options in \includegraphics[width, height, ...]{}
\setkeys{Gin}{width=\maxwidth,height=\maxheight,keepaspectratio}
% Set default figure placement to htbp
\makeatletter
\def\fps@figure{htbp}
\makeatother
\setlength{\emergencystretch}{3em} % prevent overfull lines
\providecommand{\tightlist}{%
  \setlength{\itemsep}{0pt}\setlength{\parskip}{0pt}}
\setcounter{secnumdepth}{-\maxdimen} % remove section numbering

\title{PSA: Markov Sick-Sicker model in R}
\usepackage{etoolbox}
\makeatletter
\providecommand{\subtitle}[1]{% add subtitle to \maketitle
  \apptocmd{\@title}{\par {\large #1 \par}}{}{}
}
\makeatother
\subtitle{with age-specific mortality}
\author{The DARTH workgroup}
\date{}

\begin{document}
\maketitle

Developed by the Decision Analysis in R for Technologies in Health
(DARTH) workgroup:

Fernando Alarid-Escudero, PhD (1)

Eva A. Enns, MS, PhD (2)

M.G. Myriam Hunink, MD, PhD (3,4)

Hawre J. Jalal, MD, PhD (5)

Eline M. Krijkamp, MSc (3)

Petros Pechlivanoglou, PhD (6,7)

Alan Yang, MSc (7)

In collaboration of:

\begin{enumerate}
\def\labelenumi{\arabic{enumi}.}
\tightlist
\item
  Drug Policy Program, Center for Research and Teaching in Economics
  (CIDE) - CONACyT, Aguascalientes, Mexico
\item
  University of Minnesota School of Public Health, Minneapolis, MN, USA
\item
  Erasmus MC, Rotterdam, The Netherlands
\item
  Harvard T.H. Chan School of Public Health, Boston, USA
\item
  University of Pittsburgh Graduate School of Public Health, Pittsburgh,
  PA, USA
\item
  University of Toronto, Toronto ON, Canada
\item
  The Hospital for Sick Children, Toronto ON, Canada
\end{enumerate}

Please cite our publications when using this code:

\begin{itemize}
\item
  Jalal H, Pechlivanoglou P, Krijkamp E, Alarid-Escudero F, Enns E,
  Hunink MG. An Overview of R in Health Decision Sciences. Med Decis
  Making. 2017; 37(3): 735-746.
  \url{https://journals.sagepub.com/doi/abs/10.1177/0272989X16686559}
\item
  Krijkamp EM, Alarid-Escudero F, Enns EA, Jalal HJ, Hunink MGM,
  Pechlivanoglou P. Microsimulation modeling for health decision
  sciences using R: A tutorial. Med Decis Making. 2018;38(3):400--22.
  \url{https://journals.sagepub.com/doi/abs/10.1177/0272989X18754513}
\item
  Krijkamp EM, Alarid-Escudero F, Enns E, Pechlivanoglou P, Hunink MM,
  Jalal H. A Multidimensional Array Representation of State-Transition
  Model Dynamics. Med Decis Making. 2020 Online first.
  \url{https://doi.org/10.1177/0272989X19893973}
\end{itemize}

Copyright 2017, THE HOSPITAL FOR SICK CHILDREN AND THE COLLABORATING
INSTITUTIONS. All rights reserved in Canada, the United States and
worldwide. Copyright, trademarks, trade names and any and all associated
intellectual property are exclusively owned by THE HOSPITAL FOR Sick
CHILDREN and the collaborating institutions. These materials may be
used, reproduced, modified, distributed and adapted with proper
attribution.

\newpage

Change \texttt{eval} to \texttt{TRUE} if you want to knit this document.

\begin{Shaded}
\begin{Highlighting}[]
\KeywordTok{rm}\NormalTok{(}\DataTypeTok{list =} \KeywordTok{ls}\NormalTok{())      }\CommentTok{# clear memory (removes all the variables from the workspace)}
\end{Highlighting}
\end{Shaded}

\hypertarget{load-packages}{%
\section{01 Load packages}\label{load-packages}}

\begin{Shaded}
\begin{Highlighting}[]
\ControlFlowTok{if}\NormalTok{ (}\OperatorTok{!}\KeywordTok{require}\NormalTok{(}\StringTok{'pacman'}\NormalTok{)) }\KeywordTok{install.packages}\NormalTok{(}\StringTok{'pacman'}\NormalTok{); }\KeywordTok{library}\NormalTok{(pacman) }\CommentTok{# use this package to conveniently install other packages}
\CommentTok{# load (install if required) packages from CRAN}
\KeywordTok{p_load}\NormalTok{(}\StringTok{"here"}\NormalTok{, }\StringTok{"dplyr"}\NormalTok{, }\StringTok{"devtools"}\NormalTok{, }\StringTok{"scales"}\NormalTok{, }\StringTok{"ellipse"}\NormalTok{, }\StringTok{"ggplot2"}\NormalTok{, }\StringTok{"lazyeval"}\NormalTok{, }\StringTok{"igraph"}\NormalTok{, }\StringTok{"ggraph"}\NormalTok{, }
       \StringTok{"reshape2"}\NormalTok{, }\StringTok{"knitr"}\NormalTok{)                                               }
\CommentTok{# load (install if required) packages from GitHub}
\CommentTok{# install_github("DARTH-git/dampack", force = TRUE) Uncomment if there is a newer version}
\KeywordTok{p_load_gh}\NormalTok{(}\StringTok{"DARTH-git/dampack"}\NormalTok{) }
\end{Highlighting}
\end{Shaded}

\hypertarget{load-functions}{%
\section{02 Load functions}\label{load-functions}}

\begin{Shaded}
\begin{Highlighting}[]
\KeywordTok{source}\NormalTok{(}\KeywordTok{here}\NormalTok{(}\StringTok{"functions"}\NormalTok{,}\StringTok{"Functions.R"}\NormalTok{))}
\end{Highlighting}
\end{Shaded}

\hypertarget{input-model-parameters}{%
\section{03 Input model parameters}\label{input-model-parameters}}

\begin{Shaded}
\begin{Highlighting}[]
\CommentTok{# Strategy names}
\NormalTok{v_names_str <-}\StringTok{ }\KeywordTok{c}\NormalTok{(}\StringTok{"No Treatment"}\NormalTok{, }\StringTok{"Treatment"}\NormalTok{) }

\CommentTok{# Number of strategies}
\NormalTok{n_str <-}\StringTok{ }\KeywordTok{length}\NormalTok{(v_names_str)}

\CommentTok{# Markov model parameters}
\NormalTok{age     <-}\StringTok{ }\DecValTok{25}                       \CommentTok{# age at baseline}
\NormalTok{max_age <-}\StringTok{ }\DecValTok{55}                       \CommentTok{# maximum age of follow up}
\NormalTok{n_t     <-}\StringTok{ }\NormalTok{max_age }\OperatorTok{-}\StringTok{ }\NormalTok{age            }\CommentTok{# time horizon, number of cycles}
\NormalTok{v_n     <-}\StringTok{ }\KeywordTok{c}\NormalTok{(}\StringTok{"H"}\NormalTok{, }\StringTok{"S1"}\NormalTok{, }\StringTok{"S2"}\NormalTok{, }\StringTok{"D"}\NormalTok{)  }\CommentTok{# the 4 states of the model: Healthy (H), Sick (S1), }
                                    \CommentTok{# Sicker (S2), Dead (D)}
\NormalTok{n_s     <-}\StringTok{ }\KeywordTok{length}\NormalTok{(v_n)              }\CommentTok{# number of health states }

\CommentTok{# Transition probabilities (per cycle) and hazard ratios}
\CommentTok{# Read age-specific mortality rates from csv file}
\NormalTok{lt_usa_}\DecValTok{2005}\NormalTok{ <-}\StringTok{ }\KeywordTok{read.csv}\NormalTok{(}\KeywordTok{here}\NormalTok{(}\StringTok{"data"}\NormalTok{, }\StringTok{"HMD_USA_Mx_2015.csv"}\NormalTok{))}
\NormalTok{v_r_HD <-}\StringTok{ }\NormalTok{lt_usa_}\DecValTok{2005} \OperatorTok\StringTok{ }
\StringTok{  }\KeywordTok{filter}\NormalTok{(Age }\OperatorTok{>=}\StringTok{ }\NormalTok{age }\OperatorTok{&}\StringTok{ }\NormalTok{Age }\OperatorTok{<=}\StringTok{ }\NormalTok{(max_age}\DecValTok{-1}\NormalTok{)) }\OperatorTok
\StringTok{  }\KeywordTok{select}\NormalTok{(Total) }\OperatorTok
\StringTok{  }\KeywordTok{as.matrix}\NormalTok{()}

\NormalTok{p_HD    <-}\StringTok{ }\DecValTok{1} \OperatorTok{-}\StringTok{ }\KeywordTok{exp}\NormalTok{(}\OperatorTok{-}\StringTok{ }\NormalTok{v_r_HD)        }\CommentTok{# probability to die when healthy}
\NormalTok{p_HS1   <-}\StringTok{ }\FloatTok{0.15}                       \CommentTok{# probability to become sick when healthy}
\NormalTok{p_S1H   <-}\StringTok{ }\FloatTok{0.5}                        \CommentTok{# probability to become healthy when sick}
\NormalTok{p_S1S2  <-}\StringTok{ }\FloatTok{0.105}                      \CommentTok{# probability to become sicker when sick}
\NormalTok{hr_S1   <-}\StringTok{ }\DecValTok{3}                          \CommentTok{# hazard ratio of death in sick vs healthy}
\NormalTok{hr_S2   <-}\StringTok{ }\DecValTok{10}                         \CommentTok{# hazard ratio of death in sicker vs healthy }
\NormalTok{r_HD    <-}\StringTok{ }\OperatorTok{-}\StringTok{ }\KeywordTok{log}\NormalTok{(}\DecValTok{1} \OperatorTok{-}\StringTok{ }\NormalTok{p_HD)          }\CommentTok{# rate of death in healthy}
\NormalTok{r_S1D   <-}\StringTok{ }\NormalTok{hr_S1 }\OperatorTok{*}\StringTok{ }\NormalTok{r_HD               }\CommentTok{# rate of death in sick}
\NormalTok{r_S2D   <-}\StringTok{ }\NormalTok{hr_S2 }\OperatorTok{*}\StringTok{ }\NormalTok{r_HD               }\CommentTok{# rate of death in sicker}
\NormalTok{p_S1D   <-}\StringTok{ }\DecValTok{1} \OperatorTok{-}\StringTok{ }\KeywordTok{exp}\NormalTok{(}\OperatorTok{-}\NormalTok{r_S1D)          }\CommentTok{# probability to die in sick}
\NormalTok{p_S2D   <-}\StringTok{ }\DecValTok{1} \OperatorTok{-}\StringTok{ }\KeywordTok{exp}\NormalTok{(}\OperatorTok{-}\NormalTok{r_S2D)          }\CommentTok{# probability to die in sicker}

\CommentTok{# Cost and utility inputs }
\NormalTok{c_H     <-}\StringTok{ }\DecValTok{2000}                     \CommentTok{# cost of remaining one cycle in the healthy state}
\NormalTok{c_S1    <-}\StringTok{ }\DecValTok{4000}                     \CommentTok{# cost of remaining one cycle in the sick state}
\NormalTok{c_S2    <-}\StringTok{ }\DecValTok{15000}                    \CommentTok{# cost of remaining one cycle in the sicker state}
\NormalTok{c_trt   <-}\StringTok{ }\DecValTok{12000}                    \CommentTok{# cost of treatment(per cycle)}
\NormalTok{c_D     <-}\StringTok{ }\DecValTok{0}                        \CommentTok{# cost of being in the death state}
\NormalTok{u_H     <-}\StringTok{ }\DecValTok{1}                        \CommentTok{# utility when healthy}
\NormalTok{u_S1    <-}\StringTok{ }\FloatTok{0.75}                     \CommentTok{# utility when sick}
\NormalTok{u_S2    <-}\StringTok{ }\FloatTok{0.5}                      \CommentTok{# utility when sicker}
\NormalTok{u_D     <-}\StringTok{ }\DecValTok{0}                        \CommentTok{# utility when dead}
\NormalTok{u_trt   <-}\StringTok{ }\FloatTok{0.95}                     \CommentTok{# utility when being treated}

\CommentTok{# Discounting factor}
\NormalTok{d_r     <-}\StringTok{ }\FloatTok{0.03}                     \CommentTok{# equal discount of costs and QALYs by 3%}
\CommentTok{# calculate discount weights for costs for each cycle based on discount rate d_c}
\NormalTok{v_dwc   <-}\StringTok{ }\DecValTok{1} \OperatorTok{/}\StringTok{ }\NormalTok{(}\DecValTok{1} \OperatorTok{+}\StringTok{ }\NormalTok{d_r) }\OperatorTok{^}\StringTok{ }\NormalTok{(}\DecValTok{0}\OperatorTok{:}\NormalTok{n_t) }
\CommentTok{# calculate discount weights for effectiveness for each cycle based on discount rate d_e}
\NormalTok{v_dwe   <-}\StringTok{ }\DecValTok{1} \OperatorTok{/}\StringTok{ }\NormalTok{(}\DecValTok{1} \OperatorTok{+}\StringTok{ }\NormalTok{d_r) }\OperatorTok{^}\StringTok{ }\NormalTok{(}\DecValTok{0}\OperatorTok{:}\NormalTok{n_t) }
\end{Highlighting}
\end{Shaded}

\hypertarget{define-and-initialize-matrices-and-vectors}{%
\section{04 Define and initialize matrices and
vectors}\label{define-and-initialize-matrices-and-vectors}}

\hypertarget{cohort-trace}{%
\subsection{04.1 Cohort trace}\label{cohort-trace}}

\begin{Shaded}
\begin{Highlighting}[]
\CommentTok{# create the markov trace matrix M capturing the proportion of the cohort in each state }
\CommentTok{# at each cycle}
\NormalTok{m_M_notrt <-}\StringTok{ }\NormalTok{m_M_trt <-}\StringTok{ }\KeywordTok{matrix}\NormalTok{(}\OtherTok{NA}\NormalTok{, }
                                \DataTypeTok{nrow     =}\NormalTok{ n_t }\OperatorTok{+}\StringTok{ }\DecValTok{1}\NormalTok{, }\DataTypeTok{ncol =}\NormalTok{ n_s,}
                                \DataTypeTok{dimnames =} \KeywordTok{list}\NormalTok{(}\KeywordTok{paste}\NormalTok{(}\StringTok{"cycle"}\NormalTok{, }\DecValTok{0}\OperatorTok{:}\NormalTok{n_t, }\DataTypeTok{sep =} \StringTok{" "}\NormalTok{), v_n))}

\KeywordTok{head}\NormalTok{(m_M_notrt) }\CommentTok{# show first 6 rows of the matrix }

\CommentTok{# The cohort starts as healthy}
\NormalTok{m_M_notrt[}\DecValTok{1}\NormalTok{, ] <-}\StringTok{ }\NormalTok{m_M_trt[}\DecValTok{1}\NormalTok{, ] <-}\StringTok{ }\KeywordTok{c}\NormalTok{(}\DecValTok{1}\NormalTok{, }\DecValTok{0}\NormalTok{, }\DecValTok{0}\NormalTok{, }\DecValTok{0}\NormalTok{) }\CommentTok{# initiate first cycle of cohort trace }
\end{Highlighting}
\end{Shaded}

\hypertarget{transition-probability-array}{%
\subsection{04.2 Transition probability
array}\label{transition-probability-array}}

\begin{Shaded}
\begin{Highlighting}[]
\CommentTok{# create transition probability array for NO treatment}
\NormalTok{a_P_notrt <-}\StringTok{ }\KeywordTok{array}\NormalTok{(}\DecValTok{0}\NormalTok{,                                    }\CommentTok{# Create 3-D array}
                   \DataTypeTok{dim      =} \KeywordTok{c}\NormalTok{(n_s, n_s, n_t),}
                   \DataTypeTok{dimnames =} \KeywordTok{list}\NormalTok{(v_n, v_n, }\DecValTok{0}\OperatorTok{:}\NormalTok{(n_t}\DecValTok{-1}\NormalTok{))) }\CommentTok{# name dimensions of the array}
\end{Highlighting}
\end{Shaded}

Fill in the transition probability array:

\begin{Shaded}
\begin{Highlighting}[]
\CommentTok{# from Healthy}
\NormalTok{a_P_notrt[}\StringTok{"H"}\NormalTok{, }\StringTok{"H"}\NormalTok{, ]   <-}\StringTok{ }\DecValTok{1} \OperatorTok{-}\StringTok{ }\NormalTok{(p_HS1 }\OperatorTok{+}\StringTok{ }\NormalTok{p_HD)}
\NormalTok{a_P_notrt[}\StringTok{"H"}\NormalTok{, }\StringTok{"S1"}\NormalTok{, ]  <-}\StringTok{ }\NormalTok{p_HS1}
\NormalTok{a_P_notrt[}\StringTok{"H"}\NormalTok{, }\StringTok{"D"}\NormalTok{, ]   <-}\StringTok{ }\NormalTok{p_HD}
\CommentTok{# from Sick}
\NormalTok{a_P_notrt[}\StringTok{"S1"}\NormalTok{, }\StringTok{"H"}\NormalTok{, ]  <-}\StringTok{ }\NormalTok{p_S1H}
\NormalTok{a_P_notrt[}\StringTok{"S1"}\NormalTok{, }\StringTok{"S1"}\NormalTok{, ] <-}\StringTok{ }\DecValTok{1} \OperatorTok{-}\StringTok{ }\NormalTok{(p_S1H }\OperatorTok{+}\StringTok{ }\NormalTok{p_S1S2 }\OperatorTok{+}\StringTok{ }\NormalTok{p_S1D)}
\NormalTok{a_P_notrt[}\StringTok{"S1"}\NormalTok{, }\StringTok{"S2"}\NormalTok{, ] <-}\StringTok{ }\NormalTok{p_S1S2}
\NormalTok{a_P_notrt[}\StringTok{"S1"}\NormalTok{, }\StringTok{"D"}\NormalTok{, ]  <-}\StringTok{ }\NormalTok{p_S1D}
\CommentTok{# from Sicker}
\NormalTok{a_P_notrt[}\StringTok{"S2"}\NormalTok{, }\StringTok{"S2"}\NormalTok{, ] <-}\StringTok{ }\DecValTok{1} \OperatorTok{-}\StringTok{ }\NormalTok{p_S2D}
\NormalTok{a_P_notrt[}\StringTok{"S2"}\NormalTok{, }\StringTok{"D"}\NormalTok{, ]  <-}\StringTok{ }\NormalTok{p_S2D}
\CommentTok{# from Dead}
\NormalTok{a_P_notrt[}\StringTok{"D"}\NormalTok{, }\StringTok{"D"}\NormalTok{, ]   <-}\StringTok{ }\DecValTok{1}

\CommentTok{### Check if transition matrix is valid (i.e., each row should add up to 1)}
\NormalTok{valid <-}\StringTok{ }\KeywordTok{apply}\NormalTok{(a_P_notrt, }\DecValTok{3}\NormalTok{, }\ControlFlowTok{function}\NormalTok{(x) }\KeywordTok{sum}\NormalTok{(}\KeywordTok{rowSums}\NormalTok{(x))}\OperatorTok{==}\NormalTok{n_s)}
\ControlFlowTok{if}\NormalTok{ (}\OperatorTok{!}\KeywordTok{isTRUE}\NormalTok{(}\KeywordTok{all.equal}\NormalTok{(}\KeywordTok{as.numeric}\NormalTok{(}\KeywordTok{sum}\NormalTok{(valid)), }\KeywordTok{as.numeric}\NormalTok{(n_t)))) \{}
  \KeywordTok{stop}\NormalTok{(}\StringTok{"This is not a valid transition Matrix"}\NormalTok{)}
\NormalTok{\}}

\CommentTok{### create transition probability matrix for treatment same as NO treatment}
\NormalTok{a_P_trt <-}\StringTok{ }\NormalTok{a_P_notrt}
\end{Highlighting}
\end{Shaded}

\hypertarget{run-markov-model}{%
\section{05 Run Markov model}\label{run-markov-model}}

\begin{Shaded}
\begin{Highlighting}[]
\ControlFlowTok{for}\NormalTok{ (t }\ControlFlowTok{in} \DecValTok{1}\OperatorTok{:}\NormalTok{n_t)\{     }\CommentTok{# loop through the number of cycles}
\NormalTok{  m_M_notrt[t }\OperatorTok{+}\StringTok{ }\DecValTok{1}\NormalTok{, ] <-}\StringTok{ }\KeywordTok{t}\NormalTok{(m_M_notrt[t, ]) }\OperatorTok\StringTok{ }\NormalTok{a_P_notrt[,, t]    }\CommentTok{# estimate the Markov }
                                                                 \CommentTok{# trace for cycle the }
                                                                 \CommentTok{# next cycle (t + 1)}
  
\NormalTok{  m_M_trt[t }\OperatorTok{+}\StringTok{ }\DecValTok{1}\NormalTok{, ]    <-}\StringTok{ }\KeywordTok{t}\NormalTok{(m_M_trt[t, ])    }\OperatorTok\StringTok{ }\NormalTok{a_P_trt[,, t]    }\CommentTok{# estimate the Markov }
                                                                 \CommentTok{# trace for cycle the }
                                                                 \CommentTok{# next cycle (t + 1)}
\NormalTok{\} }\CommentTok{# close the loop}

\KeywordTok{head}\NormalTok{(m_M_notrt)  }\CommentTok{# show the first 6 lines of the matrix}
\end{Highlighting}
\end{Shaded}

\hypertarget{compute-and-plot-epidemiological-outcomes}{%
\section{06 Compute and Plot Epidemiological
Outcomes}\label{compute-and-plot-epidemiological-outcomes}}

\hypertarget{cohort-trace-1}{%
\subsection{06.1 Cohort trace}\label{cohort-trace-1}}

\begin{Shaded}
\begin{Highlighting}[]
\CommentTok{# create a plot of the data}
\KeywordTok{matplot}\NormalTok{(m_M_notrt, }\DataTypeTok{type =} \StringTok{'l'}\NormalTok{, }
        \DataTypeTok{ylab =} \StringTok{"Probability of state occupancy"}\NormalTok{,}
        \DataTypeTok{xlab =} \StringTok{"Cycle"}\NormalTok{,}
        \DataTypeTok{main =} \StringTok{"Cohort Trace"}\NormalTok{)                   }
\CommentTok{# add a legend to the graph }
\KeywordTok{legend}\NormalTok{(}\StringTok{"topright"}\NormalTok{, v_n, }\DataTypeTok{col =} \DecValTok{1}\OperatorTok{:}\NormalTok{n_s,}\DataTypeTok{lty =} \DecValTok{1}\OperatorTok{:}\NormalTok{n_s, }\DataTypeTok{bty =} \StringTok{"n"}\NormalTok{)  }
\end{Highlighting}
\end{Shaded}

\hypertarget{overall-survival-os}{%
\subsection{06.2 Overall Survival (OS)}\label{overall-survival-os}}

\begin{Shaded}
\begin{Highlighting}[]
\CommentTok{# calculate the overall survival (OS) probability for no treatment}
\NormalTok{v_os_notrt_td <-}\StringTok{ }\DecValTok{1} \OperatorTok{-}\StringTok{ }\NormalTok{m_M_notrt[, }\StringTok{"D"}\NormalTok{]         }
\CommentTok{# alternative way of calculating the OS probability   }
\NormalTok{v_os_notrt_td <-}\StringTok{ }\KeywordTok{rowSums}\NormalTok{(m_M_notrt[, }\DecValTok{1}\OperatorTok{:}\DecValTok{3}\NormalTok{])    }
\CommentTok{# create a simple plot showing the OS}
\KeywordTok{plot}\NormalTok{(age}\OperatorTok{:}\NormalTok{max_age, v_os_notrt_td, }\DataTypeTok{type =} \StringTok{'l'}\NormalTok{, }
     \DataTypeTok{ylim =} \KeywordTok{c}\NormalTok{(}\DecValTok{0}\NormalTok{, }\DecValTok{1}\NormalTok{),}
     \DataTypeTok{ylab =} \StringTok{"Survival probability"}\NormalTok{,}
     \DataTypeTok{xlab =} \StringTok{"Age"}\NormalTok{,}
     \DataTypeTok{main =} \StringTok{"Overall Survival Age-Dependent"}\NormalTok{)   }
\CommentTok{# add grid}
\KeywordTok{grid}\NormalTok{(}\DataTypeTok{nx =}\NormalTok{ n_t, }\DataTypeTok{ny =} \DecValTok{10}\NormalTok{, }\DataTypeTok{col =} \StringTok{"lightgray"}\NormalTok{, }\DataTypeTok{lty =} \StringTok{"dotted"}\NormalTok{, }\DataTypeTok{lwd =} \KeywordTok{par}\NormalTok{(}\StringTok{"lwd"}\NormalTok{), }
     \DataTypeTok{equilogs =} \OtherTok{TRUE}\NormalTok{) }
\end{Highlighting}
\end{Shaded}

\hypertarget{life-expectancy-le}{%
\subsection{06.2.1 Life Expectancy (LE)}\label{life-expectancy-le}}

\begin{Shaded}
\begin{Highlighting}[]
\NormalTok{v_le_td <-}\StringTok{ }\KeywordTok{sum}\NormalTok{(v_os_notrt_td) }\CommentTok{# summing probablity of OS over time (i_e_ life expectancy)}
\end{Highlighting}
\end{Shaded}

\hypertarget{disease-prevalence}{%
\subsection{06.3 Disease prevalence}\label{disease-prevalence}}

\begin{Shaded}
\begin{Highlighting}[]
\NormalTok{v_prev_td <-}\StringTok{ }\KeywordTok{rowSums}\NormalTok{(m_M_notrt[, }\KeywordTok{c}\NormalTok{(}\StringTok{"S1"}\NormalTok{, }\StringTok{"S2"}\NormalTok{)])}\OperatorTok{/}\NormalTok{v_os_notrt_td}
\KeywordTok{plot}\NormalTok{(v_prev_td,}
     \DataTypeTok{ylim =} \KeywordTok{c}\NormalTok{(}\DecValTok{0}\NormalTok{, }\DecValTok{1}\NormalTok{),}
     \DataTypeTok{ylab =} \StringTok{"Prevalence"}\NormalTok{,}
     \DataTypeTok{xlab =} \StringTok{"Cycle"}\NormalTok{,}
     \DataTypeTok{main =} \StringTok{"Disease prevalence"}\NormalTok{)}
\end{Highlighting}
\end{Shaded}

\hypertarget{ratio-of-sicks1-vs-sickers2}{%
\subsection{06.4 ratio of sick(S1) vs
sicker(S2)}\label{ratio-of-sicks1-vs-sickers2}}

\begin{Shaded}
\begin{Highlighting}[]
\NormalTok{v_ratio_S1S2_td <-}\StringTok{ }\NormalTok{m_M_notrt[, }\StringTok{"S1"}\NormalTok{] }\OperatorTok{/}\StringTok{ }\NormalTok{m_M_notrt[, }\StringTok{"S2"}\NormalTok{]}
\KeywordTok{plot}\NormalTok{(}\DecValTok{0}\OperatorTok{:}\NormalTok{n_t, v_ratio_S1S2_td,}
     \DataTypeTok{xlab =} \StringTok{"Cycle"}\NormalTok{, }
     \DataTypeTok{ylab =} \StringTok{"Ratio S1 vs S2"}\NormalTok{, }
     \DataTypeTok{main =} \StringTok{"Ratio of sick and sicker"}\NormalTok{, }
     \DataTypeTok{col =} \StringTok{"black"}\NormalTok{, }\DataTypeTok{type =} \StringTok{"l"}\NormalTok{)}
\end{Highlighting}
\end{Shaded}

\hypertarget{compute-cost-effectiveness-outcomes}{%
\section{07 Compute Cost-Effectiveness
Outcomes}\label{compute-cost-effectiveness-outcomes}}

\begin{Shaded}
\begin{Highlighting}[]
\CommentTok{# Vectors with costs and utilities by treatment}
\NormalTok{v_u_notrt  <-}\StringTok{ }\KeywordTok{c}\NormalTok{(u_H, u_S1, u_S2, u_D)}
\NormalTok{v_u_trt    <-}\StringTok{ }\KeywordTok{c}\NormalTok{(u_H, u_trt, u_S2, u_D)}

\NormalTok{v_c_notrt  <-}\StringTok{ }\KeywordTok{c}\NormalTok{(c_H, c_S1, c_S2, c_D)}
\NormalTok{v_c_trt    <-}\StringTok{ }\KeywordTok{c}\NormalTok{(c_H, c_S1 }\OperatorTok{+}\StringTok{ }\NormalTok{c_trt, c_S2 }\OperatorTok{+}\StringTok{ }\NormalTok{c_trt, c_D)}
\end{Highlighting}
\end{Shaded}

\hypertarget{mean-costs-and-qalys-for-treatment-and-no-treatment}{%
\subsection{07.1 Mean Costs and QALYs for Treatment and NO
Treatment}\label{mean-costs-and-qalys-for-treatment-and-no-treatment}}

\begin{Shaded}
\begin{Highlighting}[]
\NormalTok{v_tu_notrt   <-}\StringTok{ }\NormalTok{m_M_notrt   }\OperatorTok\StringTok{  }\NormalTok{v_u_notrt}
\NormalTok{v_tu_trt     <-}\StringTok{ }\NormalTok{m_M_trt     }\OperatorTok\StringTok{  }\NormalTok{v_u_trt}

\NormalTok{v_tc_notrt   <-}\StringTok{ }\NormalTok{m_M_notrt   }\OperatorTok\StringTok{  }\NormalTok{v_c_notrt}
\NormalTok{v_tc_trt     <-}\StringTok{ }\NormalTok{m_M_trt     }\OperatorTok\StringTok{  }\NormalTok{v_c_trt}
\end{Highlighting}
\end{Shaded}

\hypertarget{discounted-mean-costs-and-qalys}{%
\subsection{07.2 Discounted Mean Costs and
QALYs}\label{discounted-mean-costs-and-qalys}}

\begin{Shaded}
\begin{Highlighting}[]
\NormalTok{tu_d_notrt   <-}\StringTok{ }\KeywordTok{t}\NormalTok{(v_tu_notrt)   }\OperatorTok\StringTok{  }\NormalTok{v_dwe   }
\NormalTok{tu_d_trt     <-}\StringTok{ }\KeywordTok{t}\NormalTok{(v_tu_trt)     }\OperatorTok\StringTok{  }\NormalTok{v_dwe}

\NormalTok{tc_d_notrt   <-}\StringTok{ }\KeywordTok{t}\NormalTok{(v_tc_notrt)   }\OperatorTok\StringTok{  }\NormalTok{v_dwc}
\NormalTok{tc_d_trt     <-}\StringTok{ }\KeywordTok{t}\NormalTok{(v_tc_trt)     }\OperatorTok\StringTok{  }\NormalTok{v_dwc}

\CommentTok{# store them into a vector}
\NormalTok{v_tc_d       <-}\StringTok{ }\KeywordTok{c}\NormalTok{(tc_d_notrt, tc_d_trt)}
\NormalTok{v_tu_d       <-}\StringTok{ }\KeywordTok{c}\NormalTok{(tu_d_notrt, tu_d_trt)}

\CommentTok{# Dataframe with discounted costs and effectiveness}
\NormalTok{df_ce        <-}\StringTok{ }\KeywordTok{data.frame}\NormalTok{(}\DataTypeTok{Strategy =}\NormalTok{ v_names_str,}
                           \DataTypeTok{Cost     =}\NormalTok{ v_tc_d,}
                           \DataTypeTok{Effect   =}\NormalTok{ v_tu_d)}
\NormalTok{df_ce}
\end{Highlighting}
\end{Shaded}

\hypertarget{compute-icers-of-the-markov-model}{%
\subsection{07.3 Compute ICERs of the Markov
model}\label{compute-icers-of-the-markov-model}}

\begin{Shaded}
\begin{Highlighting}[]
\NormalTok{df_cea <-}\StringTok{ }\KeywordTok{calculate_icers}\NormalTok{(}\DataTypeTok{cost       =}\NormalTok{ df_ce}\OperatorTok{$}\NormalTok{Cost,}
                          \DataTypeTok{effect     =}\NormalTok{ df_ce}\OperatorTok{$}\NormalTok{Effect,}
                          \DataTypeTok{strategies =}\NormalTok{ df_ce}\OperatorTok{$}\NormalTok{Strategy)}
\end{Highlighting}
\end{Shaded}

\hypertarget{plot-frontier-of-the-markov-model}{%
\subsection{07.4 Plot frontier of the Markov
model}\label{plot-frontier-of-the-markov-model}}

\begin{Shaded}
\begin{Highlighting}[]
\KeywordTok{plot}\NormalTok{(df_cea, }\DataTypeTok{effect_units =} \StringTok{"Quality of Life"}\NormalTok{, }\DataTypeTok{xlim=}\KeywordTok{c}\NormalTok{(}\FloatTok{16.8}\NormalTok{,}\FloatTok{17.8}\NormalTok{))}
\end{Highlighting}
\end{Shaded}

\hypertarget{deterministic-sensitivity-analysis}{%
\section{08 Deterministic Sensitivity
Analysis}\label{deterministic-sensitivity-analysis}}

\hypertarget{list-of-input-parameters}{%
\subsection{08.1 List of input
parameters}\label{list-of-input-parameters}}

Create list ``l\_params\_all'' with all input probabilities, cost and
utilities.

\begin{Shaded}
\begin{Highlighting}[]
\NormalTok{l_params_all <-}\StringTok{ }\KeywordTok{as.list}\NormalTok{(}\KeywordTok{data.frame}\NormalTok{(}
  \DataTypeTok{p_HS1   =} \FloatTok{0.15}\NormalTok{,                }\CommentTok{# probability to become sick when healthy}
  \DataTypeTok{p_S1H   =} \FloatTok{0.5}\NormalTok{,               }\CommentTok{# probability to become healthy when sick}
  \DataTypeTok{p_S1S2  =} \FloatTok{0.105}\NormalTok{,             }\CommentTok{# probability to become sicker when sick}
  \DataTypeTok{hr_S1   =} \DecValTok{3}\NormalTok{,                 }\CommentTok{# hazard ratio of death in sick vs healthy}
  \DataTypeTok{hr_S2   =} \DecValTok{10}\NormalTok{,                }\CommentTok{# hazard ratio of death in sicker vs healthy}
  \DataTypeTok{c_H     =} \DecValTok{2000}\NormalTok{,              }\CommentTok{# cost of remaining one cycle in the healthy state}
  \DataTypeTok{c_S1    =} \DecValTok{4000}\NormalTok{,              }\CommentTok{# cost of remaining one cycle in the sick state}
  \DataTypeTok{c_S2    =} \DecValTok{15000}\NormalTok{,             }\CommentTok{# cost of remaining one cycle in the sicker state}
  \DataTypeTok{c_trt   =} \DecValTok{12000}\NormalTok{,             }\CommentTok{# cost of treatment(per cycle)}
  \DataTypeTok{c_D     =} \DecValTok{0}\NormalTok{,                 }\CommentTok{# cost of being in the death state}
  \DataTypeTok{u_H     =} \DecValTok{1}\NormalTok{,                 }\CommentTok{# utility when healthy}
  \DataTypeTok{u_S1    =} \FloatTok{0.75}\NormalTok{,              }\CommentTok{# utility when sick}
  \DataTypeTok{u_S2    =} \FloatTok{0.5}\NormalTok{,               }\CommentTok{# utility when sicker}
  \DataTypeTok{u_D     =} \DecValTok{0}\NormalTok{,                 }\CommentTok{# utility when dead}
  \DataTypeTok{u_trt   =} \FloatTok{0.95}\NormalTok{,              }\CommentTok{# utility when treated}
  \DataTypeTok{d_e    =} \FloatTok{0.03}\NormalTok{,               }\CommentTok{# discount factor for effectiveness}
  \DataTypeTok{d_c    =} \FloatTok{0.03}                \CommentTok{# discount factor for costs}
\NormalTok{))}

\CommentTok{# store the parameter names into a vector}
\NormalTok{v_names_params <-}\StringTok{ }\KeywordTok{c}\NormalTok{(}\StringTok{'p_HS1'}\NormalTok{, }\StringTok{'p_S1H'}\NormalTok{, }\StringTok{'p_S1S2'}\NormalTok{, }\StringTok{'hr_S1'}\NormalTok{, }\StringTok{'hr_S2'}\NormalTok{, }\StringTok{'c_H'}\NormalTok{, }\StringTok{'c_S1'}\NormalTok{, }\StringTok{'c_S2'}\NormalTok{, }\StringTok{'c_trt'}\NormalTok{, }\StringTok{'c_D'}\NormalTok{, }\StringTok{'u_H'}\NormalTok{, }\StringTok{'u_S1'}\NormalTok{,  }
                    \StringTok{'u_S2'}\NormalTok{, }\StringTok{'u_D'}\NormalTok{, }\StringTok{'u_trt'}\NormalTok{, }\StringTok{'d_e'}\NormalTok{, }\StringTok{'d_c'}\NormalTok{)}
\end{Highlighting}
\end{Shaded}

\hypertarget{load-sick-sicker-markov-model-function}{%
\subsection{08.2 Load Sick-Sicker Markov model
function}\label{load-sick-sicker-markov-model-function}}

\begin{Shaded}
\begin{Highlighting}[]
\KeywordTok{source}\NormalTok{(}\KeywordTok{here}\NormalTok{(}\StringTok{"functions"}\NormalTok{, }\StringTok{"Functions_markov_sick-sicker_time.R"}\NormalTok{))}
\CommentTok{# Test function}
\KeywordTok{calculate_ce_out}\NormalTok{(l_params_all)}
\end{Highlighting}
\end{Shaded}

\hypertarget{one-way-sensitivity-analysis-owsa}{%
\subsection{08.3 One-way sensitivity analysis
(OWSA)}\label{one-way-sensitivity-analysis-owsa}}

\begin{Shaded}
\begin{Highlighting}[]
\NormalTok{v_params_owsa <-}\KeywordTok{c}\NormalTok{(}\StringTok{"p_S1S2"}\NormalTok{, }\StringTok{"c_trt"}\NormalTok{, }\StringTok{"u_S1"}\NormalTok{, }\StringTok{"u_trt"}\NormalTok{) }\CommentTok{# vector of names of parameters of interest}

\CommentTok{# dataframe containing all parameters, their basecase values, and the min and max values of the parameters of interest }
\NormalTok{params_all_owsa  <-}\StringTok{ }\KeywordTok{data.frame}\NormalTok{(}\DataTypeTok{pars =}\NormalTok{ v_names_params, }\DataTypeTok{basecase =} \KeywordTok{as.numeric}\NormalTok{(l_params_all), }\DataTypeTok{min =} \KeywordTok{rep}\NormalTok{(}\OtherTok{NA}\NormalTok{, }\KeywordTok{length}\NormalTok{(v_names_params)), }\DataTypeTok{max  =} \KeywordTok{rep}\NormalTok{(}\OtherTok{NA}\NormalTok{, }\KeywordTok{length}\NormalTok{(v_names_params))) }
\NormalTok{params_all_owsa}\OperatorTok{$}\NormalTok{min[params_all_owsa}\OperatorTok{$}\NormalTok{pars }\OperatorTok\StringTok{ }\NormalTok{v_params_owsa] <-}\StringTok{  }\KeywordTok{c}\NormalTok{(}\FloatTok{0.05}\NormalTok{ ,  }\DecValTok{6000}\NormalTok{ , }\FloatTok{0.65}\NormalTok{, }\FloatTok{0.80}\NormalTok{)  }\CommentTok{# min parameter values}
\NormalTok{params_all_owsa}\OperatorTok{$}\NormalTok{max[params_all_owsa}\OperatorTok{$}\NormalTok{pars }\OperatorTok\StringTok{ }\NormalTok{v_params_owsa] <-}\StringTok{  }\KeywordTok{c}\NormalTok{(}\FloatTok{0.155}\NormalTok{, }\DecValTok{18000}\NormalTok{ , }\FloatTok{0.85}\NormalTok{, }\FloatTok{0.98}\NormalTok{)  }\CommentTok{# max parameter values}

\CommentTok{# list of all parameters with their basecase values}
\NormalTok{params_basecase_owsa <-}\StringTok{ }\KeywordTok{as.list}\NormalTok{(params_all_owsa}\OperatorTok{$}\NormalTok{basecase)}
\KeywordTok{names}\NormalTok{(params_basecase_owsa) <-}\StringTok{ }\KeywordTok{as.character}\NormalTok{(params_all_owsa}\OperatorTok{$}\NormalTok{pars)}

\CommentTok{# dataframe containing name, min and max of parameters of interest}
\NormalTok{df_params_owsa <-}\StringTok{ }\NormalTok{params_all_owsa[params_all_owsa}\OperatorTok{$}\NormalTok{pars }\OperatorTok\StringTok{ }\NormalTok{v_params_owsa, }\OperatorTok{!}\KeywordTok{colnames}\NormalTok{(params_all_owsa) }\OperatorTok\StringTok{ 'basecase'}\NormalTok{]}

\NormalTok{owsa_nmb  <-}\StringTok{ }\KeywordTok{run_owsa_det}\NormalTok{(}\DataTypeTok{params_range     =}\NormalTok{ df_params_owsa,        }\CommentTok{# parameters of interest}
                          \DataTypeTok{params_basecase  =}\NormalTok{ params_basecase_owsa,  }\CommentTok{# dataframe containing all parameter basecase values}
                          \DataTypeTok{nsamp            =} \DecValTok{100}\NormalTok{,                   }\CommentTok{# number of parameter values}
                          \DataTypeTok{FUN              =}\NormalTok{ calculate_ce_out,      }\CommentTok{# function to compute outputs}
                          \DataTypeTok{outcomes         =} \KeywordTok{c}\NormalTok{(}\StringTok{"NMB"}\NormalTok{),              }\CommentTok{# output to do the OWSA on}
                          \DataTypeTok{strategies       =}\NormalTok{ v_names_str,           }\CommentTok{# names of the strategies}
                          \DataTypeTok{n_wtp            =} \DecValTok{120000}\NormalTok{)                }\CommentTok{# extra argument to pass to FUN}
\end{Highlighting}
\end{Shaded}

\hypertarget{plot-owsa}{%
\subsection{08.3.1 Plot OWSA}\label{plot-owsa}}

\begin{Shaded}
\begin{Highlighting}[]
\KeywordTok{plot}\NormalTok{(owsa_nmb, }\DataTypeTok{txtsize =} \DecValTok{16}\NormalTok{, }\DataTypeTok{n_x_ticks =} \DecValTok{5}\NormalTok{, }
     \DataTypeTok{facet_scales =} \StringTok{"free"}\NormalTok{) }\OperatorTok{+}
\StringTok{     }\KeywordTok{theme}\NormalTok{(}\DataTypeTok{legend.position =} \StringTok{"bottom"}\NormalTok{)}
\end{Highlighting}
\end{Shaded}

\hypertarget{optimal-strategy-with-owsa}{%
\subsection{08.3.2 Optimal strategy with
OWSA}\label{optimal-strategy-with-owsa}}

\begin{Shaded}
\begin{Highlighting}[]
\KeywordTok{owsa_opt_strat}\NormalTok{(}\DataTypeTok{owsa =}\NormalTok{ owsa_nmb)}
\end{Highlighting}
\end{Shaded}

\hypertarget{tornado-plot}{%
\subsection{08.3.3 Tornado plot}\label{tornado-plot}}

\begin{Shaded}
\begin{Highlighting}[]
\KeywordTok{owsa_tornado}\NormalTok{(}\DataTypeTok{owsa =}\NormalTok{ owsa_nmb)}
\end{Highlighting}
\end{Shaded}

\hypertarget{two-way-sensitivity-analysis-twsa}{%
\subsection{08.4 Two-way sensitivity analysis
(TWSA)}\label{two-way-sensitivity-analysis-twsa}}

\begin{Shaded}
\begin{Highlighting}[]
\NormalTok{v_params_twsa     <-}\StringTok{ }\KeywordTok{c}\NormalTok{(}\StringTok{"c_trt"}\NormalTok{, }\StringTok{"u_trt"}\NormalTok{) }\CommentTok{# parameters of interest for TWSA}
\CommentTok{# dataframe containing all parameters, their basecase values, and the min and max values of the parameters of interest}
\NormalTok{params_all_twsa <-}\StringTok{ }\KeywordTok{data.frame}\NormalTok{(}\DataTypeTok{pars =}\NormalTok{ v_names_params, }
                              \DataTypeTok{basecase =} \KeywordTok{as.numeric}\NormalTok{(l_params_all), }
                              \DataTypeTok{min =} \KeywordTok{rep}\NormalTok{(}\OtherTok{NA}\NormalTok{, }\KeywordTok{length}\NormalTok{(v_names_params)), }
                              \DataTypeTok{max  =} \KeywordTok{rep}\NormalTok{(}\OtherTok{NA}\NormalTok{, }\KeywordTok{length}\NormalTok{(v_names_params))) }
\NormalTok{params_all_twsa}\OperatorTok{$}\NormalTok{min[params_all_twsa}\OperatorTok{$}\NormalTok{pars }\OperatorTok\StringTok{ }\NormalTok{v_params_twsa] <-}\StringTok{ }\KeywordTok{c}\NormalTok{( }\DecValTok{6000}\NormalTok{, }\FloatTok{0.80}\NormalTok{)  }\CommentTok{# min parameter values}
\NormalTok{params_all_twsa}\OperatorTok{$}\NormalTok{max[params_all_twsa}\OperatorTok{$}\NormalTok{pars }\OperatorTok\StringTok{ }\NormalTok{v_params_twsa] <-}\StringTok{ }\KeywordTok{c}\NormalTok{(}\DecValTok{18000}\NormalTok{, }\FloatTok{0.98}\NormalTok{)  }\CommentTok{# max parameter values}

\CommentTok{# list of all parameters with their basecase values}
\NormalTok{params_basecase_twsa <-}\StringTok{ }\KeywordTok{as.list}\NormalTok{(params_all_twsa}\OperatorTok{$}\NormalTok{basecase)}
\KeywordTok{names}\NormalTok{(params_basecase_twsa) <-}\StringTok{ }\KeywordTok{as.character}\NormalTok{(params_all_twsa}\OperatorTok{$}\NormalTok{pars)}

\CommentTok{# dataframe containing name, min and max of parameters of interest}
\NormalTok{df_params_twsa <-}\StringTok{ }\NormalTok{params_all_twsa[params_all_twsa}\OperatorTok{$}\NormalTok{pars }\OperatorTok\StringTok{ }\NormalTok{v_params_twsa, }\OperatorTok{!}\KeywordTok{colnames}\NormalTok{(params_all_twsa) }\OperatorTok\StringTok{ 'basecase'}\NormalTok{]}

\NormalTok{twsa_nmb <-}\StringTok{ }\KeywordTok{run_twsa_det}\NormalTok{(}\DataTypeTok{params_range    =}\NormalTok{ df_params_twsa,       }\CommentTok{# parameters of interest}
                         \DataTypeTok{params_basecase =}\NormalTok{ params_basecase_twsa, }\CommentTok{# dataframe containing all parameter basecase value}
                         \DataTypeTok{nsamp           =} \DecValTok{40}\NormalTok{,                   }\CommentTok{# number of parameter values}
                         \DataTypeTok{FUN             =}\NormalTok{ calculate_ce_out,     }\CommentTok{# function to compute outputs}
                         \DataTypeTok{outcomes        =} \KeywordTok{c}\NormalTok{(}\StringTok{"NMB"}\NormalTok{),             }\CommentTok{# output to do the twsa on}
                         \DataTypeTok{strategies      =}\NormalTok{ v_names_str,          }\CommentTok{# names of the strategies}
                         \DataTypeTok{n_wtp           =} \DecValTok{120000}\NormalTok{)               }\CommentTok{# extra argument to pass to FUN}
\end{Highlighting}
\end{Shaded}

\hypertarget{plot-twsa}{%
\subsection{08.4.1 Plot TWSA}\label{plot-twsa}}

\begin{Shaded}
\begin{Highlighting}[]
\KeywordTok{plot}\NormalTok{(twsa_nmb)}
\end{Highlighting}
\end{Shaded}

\hypertarget{probabilistic-sensitivity-analysis-psa}{%
\section{09 Probabilistic Sensitivity Analysis
(PSA)}\label{probabilistic-sensitivity-analysis-psa}}

\begin{Shaded}
\begin{Highlighting}[]
\CommentTok{# Function to generate PSA input dataset}
\NormalTok{generate_psa_params <-}\StringTok{ }\ControlFlowTok{function}\NormalTok{(}\DataTypeTok{n_sim =} \DecValTok{1000}\NormalTok{, }\DataTypeTok{seed =} \DecValTok{071818}\NormalTok{)\{}
  \KeywordTok{set.seed}\NormalTok{(seed) }\CommentTok{# set a seed to be able to reproduce the same results}
\NormalTok{  df_psa <-}\StringTok{ }\KeywordTok{data.frame}\NormalTok{(}
    \CommentTok{# Transition probabilities (per cycle)}
    \DataTypeTok{p_HS1   =} \KeywordTok{rbeta}\NormalTok{(n_sim, }\DecValTok{30}\NormalTok{, }\DecValTok{170}\NormalTok{),  }\CommentTok{# probability to become sick when healthy}
    \DataTypeTok{p_S1H   =} \KeywordTok{rbeta}\NormalTok{(n_sim, }\DecValTok{60}\NormalTok{, }\DecValTok{60}\NormalTok{) ,  }\CommentTok{# probability to become healthy when sick}
    \DataTypeTok{p_S1S2  =} \KeywordTok{rbeta}\NormalTok{(n_sim, }\DecValTok{84}\NormalTok{, }\DecValTok{716}\NormalTok{),  }\CommentTok{# probability to become sicker when sick}
    \DataTypeTok{hr_S1   =} \KeywordTok{rlnorm}\NormalTok{(n_sim, }\KeywordTok{log}\NormalTok{(}\DecValTok{3}\NormalTok{),  }\FloatTok{0.01}\NormalTok{), }\CommentTok{# rate ratio of death in S1 vs healthy}
    \DataTypeTok{hr_S2   =} \KeywordTok{rlnorm}\NormalTok{(n_sim, }\KeywordTok{log}\NormalTok{(}\DecValTok{10}\NormalTok{), }\FloatTok{0.02}\NormalTok{), }\CommentTok{# rate ratio of death in S2 vs healthy }
    
    \CommentTok{# State rewards}
    \CommentTok{# Costs}
    \DataTypeTok{c_H   =} \KeywordTok{rgamma}\NormalTok{(n_sim, }\DataTypeTok{shape =} \DecValTok{100}\NormalTok{, }\DataTypeTok{scale =} \DecValTok{20}\NormalTok{)    , }\CommentTok{# cost of remaining one cycle in state H}
    \DataTypeTok{c_S1  =} \KeywordTok{rgamma}\NormalTok{(n_sim, }\DataTypeTok{shape =} \FloatTok{177.8}\NormalTok{, }\DataTypeTok{scale =} \FloatTok{22.5}\NormalTok{), }\CommentTok{# cost of remaining one cycle in state S1}
    \DataTypeTok{c_S2  =} \KeywordTok{rgamma}\NormalTok{(n_sim, }\DataTypeTok{shape =} \DecValTok{225}\NormalTok{, }\DataTypeTok{scale =} \FloatTok{66.7}\NormalTok{)  , }\CommentTok{# cost of remaining one cycle in state S2}
    \DataTypeTok{c_Trt =} \KeywordTok{rgamma}\NormalTok{(n_sim, }\DataTypeTok{shape =} \FloatTok{73.5}\NormalTok{, }\DataTypeTok{scale =} \FloatTok{163.3}\NormalTok{), }\CommentTok{# cost of treatment (per cycle)}
    \DataTypeTok{c_D   =} \DecValTok{0}\NormalTok{                                         , }\CommentTok{# cost of being in the death state}
    
    \CommentTok{# Utilities}
    \DataTypeTok{u_H   =} \KeywordTok{rbeta}\NormalTok{(n_sim, }\DataTypeTok{shape1 =} \DecValTok{200}\NormalTok{, }\DataTypeTok{shape2 =} \DecValTok{3}\NormalTok{), }\CommentTok{# utility when healthy}
    \DataTypeTok{u_S1  =} \KeywordTok{rbeta}\NormalTok{(n_sim, }\DataTypeTok{shape1 =} \DecValTok{130}\NormalTok{, }\DataTypeTok{shape2 =} \DecValTok{45}\NormalTok{), }\CommentTok{# utility when sick}
    \DataTypeTok{u_S2  =} \KeywordTok{rbeta}\NormalTok{(n_sim, }\DataTypeTok{shape1 =} \DecValTok{230}\NormalTok{, }\DataTypeTok{shape2 =} \DecValTok{230}\NormalTok{), }\CommentTok{# utility when sicker}
    \DataTypeTok{u_D   =} \DecValTok{0}\NormalTok{                                               , }\CommentTok{# utility when dead}
    \DataTypeTok{u_Trt =} \KeywordTok{rbeta}\NormalTok{(n_sim, }\DataTypeTok{shape1 =} \DecValTok{300}\NormalTok{, }\DataTypeTok{shape2 =} \DecValTok{15}\NormalTok{), }\CommentTok{# utility when being treated}
    \DataTypeTok{d_e   =} \FloatTok{0.03}\NormalTok{,  }\CommentTok{# discount factor for effectiveness}
    \DataTypeTok{d_c   =} \FloatTok{0.03}   \CommentTok{# discount factor for costs}
\NormalTok{  )}
  \KeywordTok{return}\NormalTok{(df_psa)}
\NormalTok{\}}
\CommentTok{# Try it}
\KeywordTok{generate_psa_params}\NormalTok{(}\DecValTok{10}\NormalTok{) }

\CommentTok{# Number of simulations}
\NormalTok{n_sim <-}\StringTok{ }\DecValTok{1000}

\CommentTok{# Generate PSA input dataset}
\NormalTok{df_psa_input <-}\StringTok{ }\KeywordTok{generate_psa_params}\NormalTok{(}\DataTypeTok{n_sim =}\NormalTok{ n_sim)}
\CommentTok{# First six observations}
\KeywordTok{head}\NormalTok{(df_psa_input)}

\CommentTok{# Histogram of parameters}
\KeywordTok{ggplot}\NormalTok{(}\KeywordTok{melt}\NormalTok{(df_psa_input, }\DataTypeTok{variable.name =} \StringTok{"Parameter"}\NormalTok{), }\KeywordTok{aes}\NormalTok{(}\DataTypeTok{x =}\NormalTok{ value)) }\OperatorTok{+}
\StringTok{  }\KeywordTok{facet_wrap}\NormalTok{(}\OperatorTok{~}\NormalTok{Parameter, }\DataTypeTok{scales =} \StringTok{"free"}\NormalTok{) }\OperatorTok{+}
\StringTok{  }\KeywordTok{geom_histogram}\NormalTok{(}\KeywordTok{aes}\NormalTok{(}\DataTypeTok{y =}\NormalTok{ ..density..)) }\OperatorTok{+}
\StringTok{  }\KeywordTok{theme_bw}\NormalTok{(}\DataTypeTok{base_size =} \DecValTok{16}\NormalTok{)}

\CommentTok{# Initialize matrices with PSA output }
\CommentTok{# Dataframe of costs}
\NormalTok{df_c <-}\StringTok{ }\KeywordTok{as.data.frame}\NormalTok{(}\KeywordTok{matrix}\NormalTok{(}\DecValTok{0}\NormalTok{, }
                             \DataTypeTok{nrow =}\NormalTok{ n_sim,}
                             \DataTypeTok{ncol =}\NormalTok{ n_str))}
\KeywordTok{colnames}\NormalTok{(df_c) <-}\StringTok{ }\NormalTok{v_names_str}
\CommentTok{# Dataframe of effectiveness}
\NormalTok{df_e <-}\StringTok{ }\KeywordTok{as.data.frame}\NormalTok{(}\KeywordTok{matrix}\NormalTok{(}\DecValTok{0}\NormalTok{, }
                             \DataTypeTok{nrow =}\NormalTok{ n_sim,}
                             \DataTypeTok{ncol =}\NormalTok{ n_str))}
\KeywordTok{colnames}\NormalTok{(df_e) <-}\StringTok{ }\NormalTok{v_names_str}
\end{Highlighting}
\end{Shaded}

\hypertarget{conduct-probabilistic-sensitivity-analysis}{%
\subsection{09.1 Conduct probabilistic sensitivity
analysis}\label{conduct-probabilistic-sensitivity-analysis}}

\begin{Shaded}
\begin{Highlighting}[]
\CommentTok{# Run Markov model on each parameter set of PSA input dataset}
\ControlFlowTok{for}\NormalTok{(i }\ControlFlowTok{in} \DecValTok{1}\OperatorTok{:}\NormalTok{n_sim)\{}
\NormalTok{  l_out_temp <-}\StringTok{ }\KeywordTok{calculate_ce_out}\NormalTok{(df_psa_input[i, ])}
\NormalTok{  df_c[i, ]  <-}\StringTok{ }\NormalTok{l_out_temp}\OperatorTok{$}\NormalTok{Cost}
\NormalTok{  df_e[i, ]  <-}\StringTok{ }\NormalTok{l_out_temp}\OperatorTok{$}\NormalTok{Effect}
  \CommentTok{# Display simulation progress}
  \ControlFlowTok{if}\NormalTok{(i}\OperatorTok{/}\NormalTok{(n_sim}\OperatorTok{/}\DecValTok{10}\NormalTok{) }\OperatorTok{==}\StringTok{ }\KeywordTok{round}\NormalTok{(i}\OperatorTok{/}\NormalTok{(n_sim}\OperatorTok{/}\DecValTok{10}\NormalTok{),}\DecValTok{0}\NormalTok{)) \{ }\CommentTok{# display progress every 10%}
    \KeywordTok{cat}\NormalTok{(}\StringTok{'}\CharTok{\textbackslash{}r}\StringTok{'}\NormalTok{, }\KeywordTok{paste}\NormalTok{(i}\OperatorTok{/}\NormalTok{n_sim }\OperatorTok{*}\StringTok{ }\DecValTok{100}\NormalTok{, }\StringTok{"% done"}\NormalTok{, }\DataTypeTok{sep =} \StringTok{" "}\NormalTok{))}
\NormalTok{  \}}
\NormalTok{\}}
\end{Highlighting}
\end{Shaded}

\hypertarget{create-psa-object-for-dampack}{%
\subsection{09.2 Create PSA object for
dampack}\label{create-psa-object-for-dampack}}

\begin{Shaded}
\begin{Highlighting}[]
\NormalTok{l_psa <-}\StringTok{ }\KeywordTok{make_psa_obj}\NormalTok{(}\DataTypeTok{cost          =}\NormalTok{ df_c, }
                      \DataTypeTok{effectiveness =}\NormalTok{ df_e, }
                      \DataTypeTok{parameters    =}\NormalTok{ df_psa_input, }
                      \DataTypeTok{strategies    =}\NormalTok{ v_names_str)}
\end{Highlighting}
\end{Shaded}

\hypertarget{save-psa-objects}{%
\subsection{09.2.1 Save PSA objects}\label{save-psa-objects}}

\begin{Shaded}
\begin{Highlighting}[]
\KeywordTok{save}\NormalTok{(df_psa_input, df_c, df_e, v_names_str, n_str,}
\NormalTok{     l_psa,}
     \DataTypeTok{file =} \KeywordTok{here}\NormalTok{(}\StringTok{"output"}\NormalTok{, }\StringTok{"markov_sick-sicker_time_PSA_dataset.RData"}\NormalTok{))}
\end{Highlighting}
\end{Shaded}

\hypertarget{create-probabilistic-analysis-graphs}{%
\subsection{09.3 Create probabilistic analysis
graphs}\label{create-probabilistic-analysis-graphs}}

\begin{Shaded}
\begin{Highlighting}[]
\KeywordTok{load}\NormalTok{(}\DataTypeTok{file =} \KeywordTok{here}\NormalTok{(}\StringTok{"output"}\NormalTok{, }\StringTok{"markov_sick-sicker_time_PSA_dataset.RData"}\NormalTok{))}
\end{Highlighting}
\end{Shaded}

Vector with willingness-to-pay (WTP) thresholds.

\begin{Shaded}
\begin{Highlighting}[]
\NormalTok{v_wtp <-}\StringTok{ }\KeywordTok{seq}\NormalTok{(}\DecValTok{0}\NormalTok{, }\DecValTok{200000}\NormalTok{, }\DataTypeTok{by =} \DecValTok{10000}\NormalTok{)}
\end{Highlighting}
\end{Shaded}

\hypertarget{cost-effectiveness-scatter-plot}{%
\subsection{09.3.1 Cost-Effectiveness Scatter
plot}\label{cost-effectiveness-scatter-plot}}

\begin{Shaded}
\begin{Highlighting}[]
\KeywordTok{plot}\NormalTok{(l_psa)}
\end{Highlighting}
\end{Shaded}

\hypertarget{conduct-cea-with-probabilistic-output}{%
\subsection{09.4 Conduct CEA with probabilistic
output}\label{conduct-cea-with-probabilistic-output}}

\begin{Shaded}
\begin{Highlighting}[]
\CommentTok{# Compute expected costs and effects for each strategy from the PSA}
\NormalTok{df_out_ce_psa <-}\StringTok{ }\KeywordTok{summary}\NormalTok{(l_psa)}

\CommentTok{# Calculate incremental cost-effectiveness ratios (ICERs)}
\NormalTok{df_cea_psa <-}\StringTok{ }\KeywordTok{calculate_icers}\NormalTok{(}\DataTypeTok{cost       =}\NormalTok{ df_out_ce_psa}\OperatorTok{$}\NormalTok{meanCost, }
                              \DataTypeTok{effect     =}\NormalTok{ df_out_ce_psa}\OperatorTok{$}\NormalTok{meanEffect,}
                              \DataTypeTok{strategies =}\NormalTok{ df_out_ce_psa}\OperatorTok{$}\NormalTok{Strategy)}
\NormalTok{df_cea_psa}

\CommentTok{# Save CEA table with ICERs}
\CommentTok{# As .RData}
\KeywordTok{save}\NormalTok{(df_cea_psa, }
     \DataTypeTok{file =} \KeywordTok{here}\NormalTok{(}\StringTok{"tables"}\NormalTok{, }
                 \StringTok{"markov_sick-sicker_time_probabilistic_CEA_results.RData"}\NormalTok{))}
\CommentTok{# As .csv}
\KeywordTok{write.csv}\NormalTok{(df_cea_psa, }
          \DataTypeTok{file =} \KeywordTok{here}\NormalTok{(}\StringTok{"tables"}\NormalTok{, }
                      \StringTok{"markov_sick-sicker_time_probabilistic_CEA_results.csv"}\NormalTok{))}
\end{Highlighting}
\end{Shaded}

\hypertarget{plot-cost-effectiveness-frontier}{%
\subsection{09.4.1 Plot cost-effectiveness
frontier}\label{plot-cost-effectiveness-frontier}}

\begin{Shaded}
\begin{Highlighting}[]
\KeywordTok{plot}\NormalTok{(df_cea_psa)}
\end{Highlighting}
\end{Shaded}

\hypertarget{cost-effectiveness-acceptability-curves-ceacs-and-frontier-ceaf}{%
\subsection{09.4.2 Cost-effectiveness acceptability curves (CEACs) and
frontier
(CEAF)}\label{cost-effectiveness-acceptability-curves-ceacs-and-frontier-ceaf}}

\begin{Shaded}
\begin{Highlighting}[]
\NormalTok{ceac_obj <-}\StringTok{ }\KeywordTok{ceac}\NormalTok{(}\DataTypeTok{wtp =}\NormalTok{ v_wtp, }\DataTypeTok{psa =}\NormalTok{ l_psa)}
\CommentTok{# Regions of highest probability of cost-effectiveness for each strategy}
\KeywordTok{summary}\NormalTok{(ceac_obj)}
\CommentTok{# CEAC & CEAF plot}
\KeywordTok{plot}\NormalTok{(ceac_obj)}
\end{Highlighting}
\end{Shaded}

\hypertarget{expected-loss-curves-elcs}{%
\subsection{09.4.3 Expected Loss Curves
(ELCs)}\label{expected-loss-curves-elcs}}

\begin{Shaded}
\begin{Highlighting}[]
\NormalTok{elc_obj <-}\StringTok{ }\KeywordTok{calc_exp_loss}\NormalTok{(}\DataTypeTok{wtp =}\NormalTok{ v_wtp, }\DataTypeTok{psa =}\NormalTok{ l_psa)}
\NormalTok{elc_obj}
\CommentTok{# ELC plot}
\KeywordTok{plot}\NormalTok{(elc_obj, }\DataTypeTok{log_y =} \OtherTok{FALSE}\NormalTok{)}
\end{Highlighting}
\end{Shaded}

\hypertarget{expected-value-of-perfect-information-evpi}{%
\subsection{09.4.4 Expected value of perfect information
(EVPI)}\label{expected-value-of-perfect-information-evpi}}

\begin{Shaded}
\begin{Highlighting}[]
\NormalTok{evpi <-}\StringTok{ }\KeywordTok{calc_evpi}\NormalTok{(}\DataTypeTok{wtp =}\NormalTok{ v_wtp, }\DataTypeTok{psa =}\NormalTok{ l_psa)}
\CommentTok{# EVPI plot}
\KeywordTok{plot}\NormalTok{(evpi, }\DataTypeTok{effect_units =} \StringTok{"QALY"}\NormalTok{)}
\end{Highlighting}
\end{Shaded}

\end{document}
