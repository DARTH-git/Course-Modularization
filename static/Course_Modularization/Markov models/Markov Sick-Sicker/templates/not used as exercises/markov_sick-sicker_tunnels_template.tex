% Options for packages loaded elsewhere
\PassOptionsToPackage{unicode}{hyperref}
\PassOptionsToPackage{hyphens}{url}
%
\documentclass[
]{article}
\usepackage{lmodern}
\usepackage{amssymb,amsmath}
\usepackage{ifxetex,ifluatex}
\ifnum 0\ifxetex 1\fi\ifluatex 1\fi=0 % if pdftex
  \usepackage[T1]{fontenc}
  \usepackage[utf8]{inputenc}
  \usepackage{textcomp} % provide euro and other symbols
\else % if luatex or xetex
  \usepackage{unicode-math}
  \defaultfontfeatures{Scale=MatchLowercase}
  \defaultfontfeatures[\rmfamily]{Ligatures=TeX,Scale=1}
\fi
% Use upquote if available, for straight quotes in verbatim environments
\IfFileExists{upquote.sty}{\usepackage{upquote}}{}
\IfFileExists{microtype.sty}{% use microtype if available
  \usepackage[]{microtype}
  \UseMicrotypeSet[protrusion]{basicmath} % disable protrusion for tt fonts
}{}
\makeatletter
\@ifundefined{KOMAClassName}{% if non-KOMA class
  \IfFileExists{parskip.sty}{%
    \usepackage{parskip}
  }{% else
    \setlength{\parindent}{0pt}
    \setlength{\parskip}{6pt plus 2pt minus 1pt}}
}{% if KOMA class
  \KOMAoptions{parskip=half}}
\makeatother
\usepackage{xcolor}
\IfFileExists{xurl.sty}{\usepackage{xurl}}{} % add URL line breaks if available
\IfFileExists{bookmark.sty}{\usepackage{bookmark}}{\usepackage{hyperref}}
\hypersetup{
  pdftitle={Markov Sick-Sicker model in R},
  pdfauthor={The DARTH workgroup},
  hidelinks,
  pdfcreator={LaTeX via pandoc}}
\urlstyle{same} % disable monospaced font for URLs
\usepackage[margin=1in]{geometry}
\usepackage{color}
\usepackage{fancyvrb}
\newcommand{\VerbBar}{|}
\newcommand{\VERB}{\Verb[commandchars=\\\{\}]}
\DefineVerbatimEnvironment{Highlighting}{Verbatim}{commandchars=\\\{\}}
% Add ',fontsize=\small' for more characters per line
\usepackage{framed}
\definecolor{shadecolor}{RGB}{248,248,248}
\newenvironment{Shaded}{\begin{snugshade}}{\end{snugshade}}
\newcommand{\AlertTok}[1]{\textcolor[rgb]{0.94,0.16,0.16}{#1}}
\newcommand{\AnnotationTok}[1]{\textcolor[rgb]{0.56,0.35,0.01}{\textbf{\textit{#1}}}}
\newcommand{\AttributeTok}[1]{\textcolor[rgb]{0.77,0.63,0.00}{#1}}
\newcommand{\BaseNTok}[1]{\textcolor[rgb]{0.00,0.00,0.81}{#1}}
\newcommand{\BuiltInTok}[1]{#1}
\newcommand{\CharTok}[1]{\textcolor[rgb]{0.31,0.60,0.02}{#1}}
\newcommand{\CommentTok}[1]{\textcolor[rgb]{0.56,0.35,0.01}{\textit{#1}}}
\newcommand{\CommentVarTok}[1]{\textcolor[rgb]{0.56,0.35,0.01}{\textbf{\textit{#1}}}}
\newcommand{\ConstantTok}[1]{\textcolor[rgb]{0.00,0.00,0.00}{#1}}
\newcommand{\ControlFlowTok}[1]{\textcolor[rgb]{0.13,0.29,0.53}{\textbf{#1}}}
\newcommand{\DataTypeTok}[1]{\textcolor[rgb]{0.13,0.29,0.53}{#1}}
\newcommand{\DecValTok}[1]{\textcolor[rgb]{0.00,0.00,0.81}{#1}}
\newcommand{\DocumentationTok}[1]{\textcolor[rgb]{0.56,0.35,0.01}{\textbf{\textit{#1}}}}
\newcommand{\ErrorTok}[1]{\textcolor[rgb]{0.64,0.00,0.00}{\textbf{#1}}}
\newcommand{\ExtensionTok}[1]{#1}
\newcommand{\FloatTok}[1]{\textcolor[rgb]{0.00,0.00,0.81}{#1}}
\newcommand{\FunctionTok}[1]{\textcolor[rgb]{0.00,0.00,0.00}{#1}}
\newcommand{\ImportTok}[1]{#1}
\newcommand{\InformationTok}[1]{\textcolor[rgb]{0.56,0.35,0.01}{\textbf{\textit{#1}}}}
\newcommand{\KeywordTok}[1]{\textcolor[rgb]{0.13,0.29,0.53}{\textbf{#1}}}
\newcommand{\NormalTok}[1]{#1}
\newcommand{\OperatorTok}[1]{\textcolor[rgb]{0.81,0.36,0.00}{\textbf{#1}}}
\newcommand{\OtherTok}[1]{\textcolor[rgb]{0.56,0.35,0.01}{#1}}
\newcommand{\PreprocessorTok}[1]{\textcolor[rgb]{0.56,0.35,0.01}{\textit{#1}}}
\newcommand{\RegionMarkerTok}[1]{#1}
\newcommand{\SpecialCharTok}[1]{\textcolor[rgb]{0.00,0.00,0.00}{#1}}
\newcommand{\SpecialStringTok}[1]{\textcolor[rgb]{0.31,0.60,0.02}{#1}}
\newcommand{\StringTok}[1]{\textcolor[rgb]{0.31,0.60,0.02}{#1}}
\newcommand{\VariableTok}[1]{\textcolor[rgb]{0.00,0.00,0.00}{#1}}
\newcommand{\VerbatimStringTok}[1]{\textcolor[rgb]{0.31,0.60,0.02}{#1}}
\newcommand{\WarningTok}[1]{\textcolor[rgb]{0.56,0.35,0.01}{\textbf{\textit{#1}}}}
\usepackage{graphicx,grffile}
\makeatletter
\def\maxwidth{\ifdim\Gin@nat@width>\linewidth\linewidth\else\Gin@nat@width\fi}
\def\maxheight{\ifdim\Gin@nat@height>\textheight\textheight\else\Gin@nat@height\fi}
\makeatother
% Scale images if necessary, so that they will not overflow the page
% margins by default, and it is still possible to overwrite the defaults
% using explicit options in \includegraphics[width, height, ...]{}
\setkeys{Gin}{width=\maxwidth,height=\maxheight,keepaspectratio}
% Set default figure placement to htbp
\makeatletter
\def\fps@figure{htbp}
\makeatother
\setlength{\emergencystretch}{3em} % prevent overfull lines
\providecommand{\tightlist}{%
  \setlength{\itemsep}{0pt}\setlength{\parskip}{0pt}}
\setcounter{secnumdepth}{-\maxdimen} % remove section numbering

\title{Markov Sick-Sicker model in R}
\usepackage{etoolbox}
\makeatletter
\providecommand{\subtitle}[1]{% add subtitle to \maketitle
  \apptocmd{\@title}{\par {\large #1 \par}}{}{}
}
\makeatother
\subtitle{with dependency for time-since model start AND with state-residency
dependency}
\author{The DARTH workgroup}
\date{}

\begin{document}
\maketitle

Developed by the Decision Analysis in R for Technologies in Health
(DARTH) workgroup:

Fernando Alarid-Escudero, PhD (1)

Eva A. Enns, MS, PhD (2)

M.G. Myriam Hunink, MD, PhD (3,4)

Hawre J. Jalal, MD, PhD (5)

Eline M. Krijkamp, MSc (3)

Petros Pechlivanoglou, PhD (6,7)

Alan Yang, MSc (7)

In collaboration of:

\begin{enumerate}
\def\labelenumi{\arabic{enumi}.}
\tightlist
\item
  Drug Policy Program, Center for Research and Teaching in Economics
  (CIDE) - CONACyT, Aguascalientes, Mexico
\item
  University of Minnesota School of Public Health, Minneapolis, MN, USA
\item
  Erasmus MC, Rotterdam, The Netherlands
\item
  Harvard T.H. Chan School of Public Health, Boston, USA
\item
  University of Pittsburgh Graduate School of Public Health, Pittsburgh,
  PA, USA
\item
  University of Toronto, Toronto ON, Canada
\item
  The Hospital for Sick Children, Toronto ON, Canada
\end{enumerate}

Please cite our publications when using this code:

\begin{itemize}
\item
  Jalal H, Pechlivanoglou P, Krijkamp E, Alarid-Escudero F, Enns E,
  Hunink MG. An Overview of R in Health Decision Sciences. Med Decis
  Making. 2017; 37(3): 735-746.
  \url{https://journals.sagepub.com/doi/abs/10.1177/0272989X16686559}
\item
  Krijkamp EM, Alarid-Escudero F, Enns EA, Jalal HJ, Hunink MGM,
  Pechlivanoglou P. Microsimulation modeling for health decision
  sciences using R: A tutorial. Med Decis Making. 2018;38(3):400--22.
  \url{https://journals.sagepub.com/doi/abs/10.1177/0272989X18754513}
\item
  Krijkamp EM, Alarid-Escudero F, Enns E, Pechlivanoglou P, Hunink MM,
  Jalal H. A Multidimensional Array Representation of State-Transition
  Model Dynamics. BioRxiv 670612
  2019.https://www.biorxiv.org/content/10.1101/670612v1
\end{itemize}

Copyright 2017, THE HOSPITAL FOR SICK CHILDREN AND THE COLLABORATING
INSTITUTIONS. All rights reserved in Canada, the United States and
worldwide. Copyright, trademarks, trade names and any and all associated
intellectual property are exclusively owned by THE HOSPITAL FOR Sick
CHILDREN and the collaborating institutions. These materials may be
used, reproduced, modified, distributed and adapted with proper
attribution.

\newpage

Change \texttt{eval} to \texttt{TRUE} if you want to knit this document.

\begin{Shaded}
\begin{Highlighting}[]
\KeywordTok{rm}\NormalTok{(}\DataTypeTok{list =} \KeywordTok{ls}\NormalTok{())      }\CommentTok{# clear memory (removes all the variables from the workspace)}
\end{Highlighting}
\end{Shaded}

\hypertarget{load-packages}{%
\section{01 Load packages}\label{load-packages}}

\begin{Shaded}
\begin{Highlighting}[]
\ControlFlowTok{if}\NormalTok{ (}\OperatorTok{!}\KeywordTok{require}\NormalTok{(}\StringTok{'pacman'}\NormalTok{)) }\KeywordTok{install.packages}\NormalTok{(}\StringTok{'pacman'}\NormalTok{); }\KeywordTok{library}\NormalTok{(pacman) }\CommentTok{# use this package to conveniently install other packages}
\CommentTok{# load (install if required) packages from CRAN}
\KeywordTok{p_load}\NormalTok{(}\StringTok{"here"}\NormalTok{, }\StringTok{"dplyr"}\NormalTok{, }\StringTok{"devtools"}\NormalTok{, }\StringTok{"scales"}\NormalTok{, }\StringTok{"ellipse"}\NormalTok{, }\StringTok{"ggplot2"}\NormalTok{, }\StringTok{"lazyeval"}\NormalTok{, }\StringTok{"igraph"}\NormalTok{, }\StringTok{"truncnorm"}\NormalTok{, }\StringTok{"ggraph"}\NormalTok{, }\StringTok{"reshape2"}\NormalTok{, }\StringTok{"knitr"}\NormalTok{)                                               }
\CommentTok{# load (install if required) packages from GitHub}
\CommentTok{# install_github("DARTH-git/dampack", force = TRUE) Uncomment if there is a newer version}
\CommentTok{# install_github("DARTH-git/darthtools", force = TRUE) Uncomment if there is a newer version}
\KeywordTok{p_load_gh}\NormalTok{(}\StringTok{"DARTH-git/dampack"}\NormalTok{, }\StringTok{"DARTH-git/darthtools"}\NormalTok{)}
\end{Highlighting}
\end{Shaded}

\hypertarget{load-functions}{%
\section{02 Load functions}\label{load-functions}}

\begin{Shaded}
\begin{Highlighting}[]
\CommentTok{# No functions needed}
\end{Highlighting}
\end{Shaded}

\hypertarget{input-model-parameters}{%
\section{03 Input model parameters}\label{input-model-parameters}}

\begin{Shaded}
\begin{Highlighting}[]
\CommentTok{# Strategy names}
\NormalTok{v_names_str <-}\StringTok{ }\KeywordTok{c}\NormalTok{(}\StringTok{"No Treatment"}\NormalTok{, }\StringTok{"Treatment"}\NormalTok{) }

\CommentTok{# Number of strategies}
\NormalTok{n_str <-}\StringTok{ }\KeywordTok{length}\NormalTok{(v_names_str)}

\CommentTok{# Markov model parameters}
\NormalTok{age     <-}\StringTok{ }\DecValTok{25}                       \CommentTok{# age at baseline}
\NormalTok{max_age <-}\StringTok{ }\DecValTok{55}                       \CommentTok{# maximum age of follow up}
\NormalTok{n_t     <-}\StringTok{ }\NormalTok{max_age }\OperatorTok{-}\StringTok{ }\NormalTok{age            }\CommentTok{# time horizon, number of cycles}
\NormalTok{v_n     <-}\StringTok{ }\KeywordTok{c}\NormalTok{(}\StringTok{"H"}\NormalTok{, }\StringTok{"S1"}\NormalTok{, }\StringTok{"S2"}\NormalTok{, }\StringTok{"D"}\NormalTok{)  }\CommentTok{# the 4 states of the model: Healthy (H), Sick (S1), }
                                    \CommentTok{# Sicker (S2), Dead (D)}
\NormalTok{n_states <-}\StringTok{ }\KeywordTok{length}\NormalTok{(v_n)             }\CommentTok{# number of health states }

\CommentTok{# Tunnels}
\NormalTok{n_tunnel_size <-}\StringTok{ }\NormalTok{n_t}
\CommentTok{# Sick state}
\NormalTok{v_Sick_tunnels <-}\StringTok{ }\KeywordTok{paste}\NormalTok{(}\StringTok{"S1_"}\NormalTok{, }\KeywordTok{seq}\NormalTok{(}\DecValTok{1}\NormalTok{, n_tunnel_size), }\StringTok{"Yr"}\NormalTok{, }\DataTypeTok{sep =} \StringTok{""}\NormalTok{)}
\CommentTok{### Create variables for time-dependent model}
\NormalTok{v_n_tunnels <-}\StringTok{ }\KeywordTok{c}\NormalTok{(}\StringTok{"H"}\NormalTok{, v_Sick_tunnels, }\StringTok{"S2"}\NormalTok{, }\StringTok{"D"}\NormalTok{)  }\CommentTok{# state names}
\NormalTok{n_s_tunnels <-}\StringTok{ }\KeywordTok{length}\NormalTok{(v_n_tunnels)                }\CommentTok{# number of states}

\CommentTok{# Transition probabilities (per cycle) and hazard ratios}
\CommentTok{# Read age-specific mortality rates from csv file}
\NormalTok{lt_usa_}\DecValTok{2005}\NormalTok{ <-}\StringTok{ }\KeywordTok{read.csv}\NormalTok{(}\StringTok{"HMD_USA_Mx_2015.csv"}\NormalTok{)}
\NormalTok{v_r_HD <-}\StringTok{ }\NormalTok{lt_usa_}\DecValTok{2005} \OperatorTok\StringTok{ }
\StringTok{          }\KeywordTok{filter}\NormalTok{(Age }\OperatorTok{>=}\StringTok{ }\NormalTok{age }\OperatorTok{&}\StringTok{ }\NormalTok{Age }\OperatorTok{<=}\StringTok{ }\NormalTok{(max_age }\OperatorTok{-}\StringTok{ }\DecValTok{1}\NormalTok{)) }\OperatorTok
\StringTok{          }\KeywordTok{select}\NormalTok{(Total) }\OperatorTok
\StringTok{          }\KeywordTok{as.matrix}\NormalTok{()}

\NormalTok{p_HD    <-}\StringTok{ }\DecValTok{1} \OperatorTok{-}\StringTok{ }\KeywordTok{exp}\NormalTok{(}\OperatorTok{-}\StringTok{ }\NormalTok{v_r_HD)  }\CommentTok{# probability to die when healthy}
\NormalTok{p_HS1   <-}\StringTok{ }\FloatTok{0.15}                 \CommentTok{# probability to become sick when healthy}
\NormalTok{p_S1H   <-}\StringTok{ }\FloatTok{0.5}                  \CommentTok{# probability to become healthy when }

\CommentTok{# Weibull parameters}
\NormalTok{l       <-}\StringTok{ }\CommentTok{# <-  your turn  # scale parameter}
\NormalTok{g       <-}\StringTok{ }\CommentTok{# <-  your turn  # shape parameter}
\CommentTok{# Weibull function}
\NormalTok{p_S1S2  <-}\StringTok{ }\NormalTok{l }\OperatorTok{*}\StringTok{ }\NormalTok{g }\OperatorTok{*}\StringTok{ }\NormalTok{(}\DecValTok{1}\OperatorTok{:}\NormalTok{n_tunnel_size) }\OperatorTok{^}\StringTok{ }\NormalTok{\{g}\DecValTok{-1}\NormalTok{\} }\CommentTok{# probability to become sicker when sick }
                                             \CommentTok{# (time-dependent)}

\NormalTok{hr_S1   <-}\StringTok{ }\DecValTok{3}                  \CommentTok{# hazard ratio of death in sick vs healthy}
\NormalTok{hr_S2   <-}\StringTok{ }\DecValTok{10}                 \CommentTok{# hazard ratio of death in sicker vs healthy }
\NormalTok{r_HD    <-}\StringTok{ }\OperatorTok{-}\StringTok{ }\KeywordTok{log}\NormalTok{(}\DecValTok{1} \OperatorTok{-}\StringTok{ }\NormalTok{p_HD)  }\CommentTok{# rate of death in healthy}
\NormalTok{r_S1D   <-}\StringTok{ }\NormalTok{hr_S1 }\OperatorTok{*}\StringTok{ }\NormalTok{r_HD       }\CommentTok{# rate of death in sick}
\NormalTok{r_S2D   <-}\StringTok{ }\NormalTok{hr_S2 }\OperatorTok{*}\StringTok{ }\NormalTok{r_HD       }\CommentTok{# rate of death in sicker}
\NormalTok{p_S1D   <-}\StringTok{ }\DecValTok{1} \OperatorTok{-}\StringTok{ }\KeywordTok{exp}\NormalTok{(}\OperatorTok{-}\NormalTok{r_S1D)  }\CommentTok{# probability to die in sick}
\NormalTok{p_S2D   <-}\StringTok{ }\DecValTok{1} \OperatorTok{-}\StringTok{ }\KeywordTok{exp}\NormalTok{(}\OperatorTok{-}\NormalTok{r_S2D)  }\CommentTok{# probability to die in s}

\CommentTok{# Cost and utility inputs    }
\NormalTok{c_H     <-}\StringTok{ }\DecValTok{2000}             \CommentTok{# cost of remaining one cycle in the healthy state}
\NormalTok{c_S1    <-}\StringTok{ }\DecValTok{4000}             \CommentTok{# cost of remaining one cycle in the sick state}
\NormalTok{c_S2    <-}\StringTok{ }\DecValTok{15000}            \CommentTok{# cost of remaining one cycle in the sicker state}
\NormalTok{c_trt   <-}\StringTok{ }\DecValTok{12000}            \CommentTok{# cost of treatment(per cycle)}
\NormalTok{c_D     <-}\StringTok{ }\DecValTok{0}                \CommentTok{# cost of being in the death state}
\NormalTok{u_H     <-}\StringTok{ }\DecValTok{1}                \CommentTok{# utility when healthy}
\NormalTok{u_S1    <-}\StringTok{ }\FloatTok{0.75}             \CommentTok{# utility when sick}
\NormalTok{u_S2    <-}\StringTok{ }\FloatTok{0.5}              \CommentTok{# utility when sicker}
\NormalTok{u_D     <-}\StringTok{ }\DecValTok{0}                \CommentTok{# utility when dead}
\NormalTok{u_trt   <-}\StringTok{ }\FloatTok{0.95}             \CommentTok{# utility when being treated}

\NormalTok{d_e <-}\StringTok{ }\NormalTok{d_c <-}\StringTok{ }\FloatTok{0.03}          \CommentTok{# equal discount of costs and QALYs by 3%}

\CommentTok{# calculate discount weights for costs for each cycle based on discount rate d_c}
\NormalTok{v_dwc <-}\StringTok{ }\DecValTok{1} \OperatorTok{/}\StringTok{ }\NormalTok{(}\DecValTok{1} \OperatorTok{+}\StringTok{ }\NormalTok{d_e) }\OperatorTok{^}\StringTok{ }\NormalTok{(}\DecValTok{0}\OperatorTok{:}\NormalTok{n_t) }
\CommentTok{# calculate discount weights for effectiveness for each cycle based on discount rate d_e}
\NormalTok{v_dwe <-}\StringTok{ }\DecValTok{1} \OperatorTok{/}\StringTok{ }\NormalTok{(}\DecValTok{1} \OperatorTok{+}\StringTok{ }\NormalTok{d_c) }\OperatorTok{^}\StringTok{ }\NormalTok{(}\DecValTok{0}\OperatorTok{:}\NormalTok{n_t) }
\end{Highlighting}
\end{Shaded}

\hypertarget{define-and-initialize-matrices-and-vectors}{%
\section{04 Define and initialize matrices and
vectors}\label{define-and-initialize-matrices-and-vectors}}

\hypertarget{cohort-trace}{%
\subsection{04.1 Cohort trace}\label{cohort-trace}}

\begin{Shaded}
\begin{Highlighting}[]
\CommentTok{# create the markov trace matrix M capturing the proportion of the cohort in each state }
\CommentTok{# at each cycle}
\NormalTok{m_M_notrt <-}\StringTok{ }\NormalTok{m_M_trt  }\CommentTok{# <-  your turn}





\KeywordTok{head}\NormalTok{(m_M_notrt) }\CommentTok{# show first 6 rows of the matrix }

\CommentTok{# The cohort starts as healthy}
\CommentTok{# initialize first cycle of Markov trace accounting for the tunnels}
\NormalTok{m_M_notrt[}\DecValTok{1}\NormalTok{, ] <-}\StringTok{ }\NormalTok{m_M_trt[}\DecValTok{1}\NormalTok{, ] <-}\StringTok{ }\KeywordTok{c}\NormalTok{(}\DecValTok{1}\NormalTok{, }\KeywordTok{rep}\NormalTok{(}\DecValTok{0}\NormalTok{, n_tunnel_size), }\DecValTok{0}\NormalTok{, }\DecValTok{0}\NormalTok{) }
\end{Highlighting}
\end{Shaded}

\hypertarget{transition-probability-array}{%
\subsection{04.2 Transition probability
array}\label{transition-probability-array}}

\begin{Shaded}
\begin{Highlighting}[]
\CommentTok{# create the transition probability array for NO treatment}
\NormalTok{a_P_notrt }\CommentTok{# <- your turn}
\end{Highlighting}
\end{Shaded}

Fill in the transition probability array:

\begin{Shaded}
\begin{Highlighting}[]
\CommentTok{# from Healthy}

\CommentTok{# from Sick}

\CommentTok{# from Sicker}

\CommentTok{# from Dead}

\CommentTok{# create transition probability matrix for treatment same as NO treatment}
\NormalTok{a_P_trt <-}\StringTok{ }\NormalTok{a_P_notrt}
\end{Highlighting}
\end{Shaded}

\hypertarget{run-markov-model}{%
\section{05 Run Markov model}\label{run-markov-model}}

\begin{Shaded}
\begin{Highlighting}[]
\CommentTok{# your turn}
\end{Highlighting}
\end{Shaded}

\hypertarget{compute-and-plot-epidemiological-outcomes}{%
\section{06 Compute and Plot Epidemiological
Outcomes}\label{compute-and-plot-epidemiological-outcomes}}

\hypertarget{cohort-trace-1}{%
\subsection{06.1 Cohort trace}\label{cohort-trace-1}}

\begin{Shaded}
\begin{Highlighting}[]
\CommentTok{# your turn}
\end{Highlighting}
\end{Shaded}

\hypertarget{overall-survival-os}{%
\subsection{06.2 Overall Survival (OS)}\label{overall-survival-os}}

\begin{Shaded}
\begin{Highlighting}[]
\CommentTok{# your turn}
\end{Highlighting}
\end{Shaded}

\hypertarget{life-expectancy-le}{%
\subsection{06.2.1 Life Expectancy (LE)}\label{life-expectancy-le}}

\begin{Shaded}
\begin{Highlighting}[]
\CommentTok{# your turn}
\end{Highlighting}
\end{Shaded}

\hypertarget{disease-prevalence}{%
\subsection{06.3 Disease prevalence}\label{disease-prevalence}}

\begin{Shaded}
\begin{Highlighting}[]
\CommentTok{# your turn}
\end{Highlighting}
\end{Shaded}

\hypertarget{ratio-of-sicks1-vs-sickers2}{%
\subsection{06.4 ratio of sick(S1) vs
sicker(S2)}\label{ratio-of-sicks1-vs-sickers2}}

\begin{Shaded}
\begin{Highlighting}[]
\CommentTok{# your turn}
\end{Highlighting}
\end{Shaded}

\hypertarget{compute-cost-effectiveness-outcomes}{%
\section{07 Compute Cost-Effectiveness
Outcomes}\label{compute-cost-effectiveness-outcomes}}

\hypertarget{mean-costs-and-qalys-for-treatment-and-no-treatment}{%
\subsection{07.1 Mean Costs and QALYs for Treatment and NO
Treatment}\label{mean-costs-and-qalys-for-treatment-and-no-treatment}}

\begin{Shaded}
\begin{Highlighting}[]
\CommentTok{# your turn}
\end{Highlighting}
\end{Shaded}

\hypertarget{discounted-mean-costs-and-qalys}{%
\subsection{07.2 Discounted Mean Costs and
QALYs}\label{discounted-mean-costs-and-qalys}}

\begin{Shaded}
\begin{Highlighting}[]
\CommentTok{# your turn}

\CommentTok{# Store discounted costs and effectiveness for each strategy in a data frame}
\NormalTok{df_ce       <-}\StringTok{ }\KeywordTok{data.frame}\NormalTok{(}\DataTypeTok{Strategy =} \OtherTok{NA}\NormalTok{, }\CommentTok{# <-- your value (strategy names)}
                          \DataTypeTok{Cost     =} \OtherTok{NA}\NormalTok{, }\CommentTok{# <-- your value (strategy costs)}
                          \DataTypeTok{Effect   =} \OtherTok{NA}  \CommentTok{# <-- your value (strategy QALYs)}
\NormalTok{                          )}
\end{Highlighting}
\end{Shaded}

\hypertarget{compute-icers-of-the-markov-model}{%
\subsection{07.3 Compute ICERs of the Markov
model}\label{compute-icers-of-the-markov-model}}

\begin{Shaded}
\begin{Highlighting}[]
\CommentTok{# Use the function calculate_icers() from the dampack package}
\NormalTok{df_cea <-}\StringTok{ }\KeywordTok{calculate_icers}\NormalTok{(}\DataTypeTok{cost       =}\NormalTok{ df_ce}\OperatorTok{$}\NormalTok{Cost,}
                          \DataTypeTok{effect     =}\NormalTok{ df_ce}\OperatorTok{$}\NormalTok{Effect,}
                          \DataTypeTok{strategies =}\NormalTok{ df_ce}\OperatorTok{$}\NormalTok{Strategy)}
\NormalTok{df_cea }
\end{Highlighting}
\end{Shaded}

\hypertarget{plot-frontier-of-the-markov-model}{%
\subsection{07.4 Plot frontier of the Markov
model}\label{plot-frontier-of-the-markov-model}}

\begin{Shaded}
\begin{Highlighting}[]
\KeywordTok{plot}\NormalTok{(df_cea, }\DataTypeTok{effect_units =} \StringTok{"QALYs"}\NormalTok{)}
\end{Highlighting}
\end{Shaded}

\end{document}
