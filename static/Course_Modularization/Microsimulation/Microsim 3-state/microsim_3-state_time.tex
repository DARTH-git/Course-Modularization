\documentclass[]{article}
\usepackage{lmodern}
\usepackage{amssymb,amsmath}
\usepackage{ifxetex,ifluatex}
\usepackage{fixltx2e} % provides \textsubscript
\ifnum 0\ifxetex 1\fi\ifluatex 1\fi=0 % if pdftex
  \usepackage[T1]{fontenc}
  \usepackage[utf8]{inputenc}
\else % if luatex or xelatex
  \ifxetex
    \usepackage{mathspec}
  \else
    \usepackage{fontspec}
  \fi
  \defaultfontfeatures{Ligatures=TeX,Scale=MatchLowercase}
\fi
% use upquote if available, for straight quotes in verbatim environments
\IfFileExists{upquote.sty}{\usepackage{upquote}}{}
% use microtype if available
\IfFileExists{microtype.sty}{%
\usepackage[]{microtype}
\UseMicrotypeSet[protrusion]{basicmath} % disable protrusion for tt fonts
}{}
\PassOptionsToPackage{hyphens}{url} % url is loaded by hyperref
\usepackage[unicode=true]{hyperref}
\hypersetup{
            pdftitle={Simple 3-state microsimulation model},
            pdfauthor={The DARTH workgroup},
            pdfborder={0 0 0},
            breaklinks=true}
\urlstyle{same}  % don't use monospace font for urls
\usepackage[margin=1in]{geometry}
\usepackage{color}
\usepackage{fancyvrb}
\newcommand{\VerbBar}{|}
\newcommand{\VERB}{\Verb[commandchars=\\\{\}]}
\DefineVerbatimEnvironment{Highlighting}{Verbatim}{commandchars=\\\{\}}
% Add ',fontsize=\small' for more characters per line
\usepackage{framed}
\definecolor{shadecolor}{RGB}{248,248,248}
\newenvironment{Shaded}{\begin{snugshade}}{\end{snugshade}}
\newcommand{\KeywordTok}[1]{\textcolor[rgb]{0.13,0.29,0.53}{\textbf{#1}}}
\newcommand{\DataTypeTok}[1]{\textcolor[rgb]{0.13,0.29,0.53}{#1}}
\newcommand{\DecValTok}[1]{\textcolor[rgb]{0.00,0.00,0.81}{#1}}
\newcommand{\BaseNTok}[1]{\textcolor[rgb]{0.00,0.00,0.81}{#1}}
\newcommand{\FloatTok}[1]{\textcolor[rgb]{0.00,0.00,0.81}{#1}}
\newcommand{\ConstantTok}[1]{\textcolor[rgb]{0.00,0.00,0.00}{#1}}
\newcommand{\CharTok}[1]{\textcolor[rgb]{0.31,0.60,0.02}{#1}}
\newcommand{\SpecialCharTok}[1]{\textcolor[rgb]{0.00,0.00,0.00}{#1}}
\newcommand{\StringTok}[1]{\textcolor[rgb]{0.31,0.60,0.02}{#1}}
\newcommand{\VerbatimStringTok}[1]{\textcolor[rgb]{0.31,0.60,0.02}{#1}}
\newcommand{\SpecialStringTok}[1]{\textcolor[rgb]{0.31,0.60,0.02}{#1}}
\newcommand{\ImportTok}[1]{#1}
\newcommand{\CommentTok}[1]{\textcolor[rgb]{0.56,0.35,0.01}{\textit{#1}}}
\newcommand{\DocumentationTok}[1]{\textcolor[rgb]{0.56,0.35,0.01}{\textbf{\textit{#1}}}}
\newcommand{\AnnotationTok}[1]{\textcolor[rgb]{0.56,0.35,0.01}{\textbf{\textit{#1}}}}
\newcommand{\CommentVarTok}[1]{\textcolor[rgb]{0.56,0.35,0.01}{\textbf{\textit{#1}}}}
\newcommand{\OtherTok}[1]{\textcolor[rgb]{0.56,0.35,0.01}{#1}}
\newcommand{\FunctionTok}[1]{\textcolor[rgb]{0.00,0.00,0.00}{#1}}
\newcommand{\VariableTok}[1]{\textcolor[rgb]{0.00,0.00,0.00}{#1}}
\newcommand{\ControlFlowTok}[1]{\textcolor[rgb]{0.13,0.29,0.53}{\textbf{#1}}}
\newcommand{\OperatorTok}[1]{\textcolor[rgb]{0.81,0.36,0.00}{\textbf{#1}}}
\newcommand{\BuiltInTok}[1]{#1}
\newcommand{\ExtensionTok}[1]{#1}
\newcommand{\PreprocessorTok}[1]{\textcolor[rgb]{0.56,0.35,0.01}{\textit{#1}}}
\newcommand{\AttributeTok}[1]{\textcolor[rgb]{0.77,0.63,0.00}{#1}}
\newcommand{\RegionMarkerTok}[1]{#1}
\newcommand{\InformationTok}[1]{\textcolor[rgb]{0.56,0.35,0.01}{\textbf{\textit{#1}}}}
\newcommand{\WarningTok}[1]{\textcolor[rgb]{0.56,0.35,0.01}{\textbf{\textit{#1}}}}
\newcommand{\AlertTok}[1]{\textcolor[rgb]{0.94,0.16,0.16}{#1}}
\newcommand{\ErrorTok}[1]{\textcolor[rgb]{0.64,0.00,0.00}{\textbf{#1}}}
\newcommand{\NormalTok}[1]{#1}
\usepackage{graphicx,grffile}
\makeatletter
\def\maxwidth{\ifdim\Gin@nat@width>\linewidth\linewidth\else\Gin@nat@width\fi}
\def\maxheight{\ifdim\Gin@nat@height>\textheight\textheight\else\Gin@nat@height\fi}
\makeatother
% Scale images if necessary, so that they will not overflow the page
% margins by default, and it is still possible to overwrite the defaults
% using explicit options in \includegraphics[width, height, ...]{}
\setkeys{Gin}{width=\maxwidth,height=\maxheight,keepaspectratio}
\IfFileExists{parskip.sty}{%
\usepackage{parskip}
}{% else
\setlength{\parindent}{0pt}
\setlength{\parskip}{6pt plus 2pt minus 1pt}
}
\setlength{\emergencystretch}{3em}  % prevent overfull lines
\providecommand{\tightlist}{%
  \setlength{\itemsep}{0pt}\setlength{\parskip}{0pt}}
\setcounter{secnumdepth}{0}
% Redefines (sub)paragraphs to behave more like sections
\ifx\paragraph\undefined\else
\let\oldparagraph\paragraph
\renewcommand{\paragraph}[1]{\oldparagraph{#1}\mbox{}}
\fi
\ifx\subparagraph\undefined\else
\let\oldsubparagraph\subparagraph
\renewcommand{\subparagraph}[1]{\oldsubparagraph{#1}\mbox{}}
\fi

% set default figure placement to htbp
\makeatletter
\def\fps@figure{htbp}
\makeatother

\usepackage{etoolbox}
\makeatletter
\providecommand{\subtitle}[1]{% add subtitle to \maketitle
  \apptocmd{\@title}{\par {\large #1 \par}}{}{}
}
\makeatother

\title{Simple 3-state microsimulation model}
\providecommand{\subtitle}[1]{}
\subtitle{Includes sex specific probability of dying when healthy and state
occupation: probability of dying when sick depends on the time of being
sick}
\author{The DARTH workgroup}
\date{}

\begin{document}
\maketitle

Developed by the Decision Analysis in R for Technologies in Health
(DARTH) workgroup:

Fernando Alarid-Escudero, PhD (1)

Eva A. Enns, MS, PhD (2)

M.G. Myriam Hunink, MD, PhD (3,4)

Hawre J. Jalal, MD, PhD (5)

Eline M. Krijkamp, MSc (3)

Petros Pechlivanoglou, PhD (6,7)

Alan Yang, MSc (7)

In collaboration of:

\begin{enumerate}
\def\labelenumi{\arabic{enumi}.}
\tightlist
\item
  Drug Policy Program, Center for Research and Teaching in Economics
  (CIDE) - CONACyT, Aguascalientes, Mexico
\item
  University of Minnesota School of Public Health, Minneapolis, MN, USA
\item
  Erasmus MC, Rotterdam, The Netherlands
\item
  Harvard T.H. Chan School of Public Health, Boston, USA
\item
  University of Pittsburgh Graduate School of Public Health, Pittsburgh,
  PA, USA
\item
  University of Toronto, Toronto ON, Canada
\item
  The Hospital for Sick Children, Toronto ON, Canada
\end{enumerate}

Please cite our publications when using this code:

\begin{itemize}
\item
  Jalal H, Pechlivanoglou P, Krijkamp E, Alarid-Escudero F, Enns E,
  Hunink MG. An Overview of R in Health Decision Sciences. Med Decis
  Making. 2017; 37(3): 735-746.
  \url{https://journals.sagepub.com/doi/abs/10.1177/0272989X16686559}
\item
  Krijkamp EM, Alarid-Escudero F, Enns EA, Jalal HJ, Hunink MGM,
  Pechlivanoglou P. Microsimulation modeling for health decision
  sciences using R: A tutorial. Med Decis Making. 2018;38(3):400--22.
  \url{https://journals.sagepub.com/doi/abs/10.1177/0272989X18754513}
\item
  Krijkamp EM, Alarid-Escudero F, Enns E, Pechlivanoglou P, Hunink MM,
  Jalal H. A Multidimensional Array Representation of State-Transition
  Model Dynamics. Med Decis Making. 2020 Online first.
  \url{https://doi.org/10.1177/0272989X19893973}
\end{itemize}

\newpage

Change \texttt{eval} to \texttt{TRUE} if you want to knit this document.

\begin{Shaded}
\begin{Highlighting}[]
\KeywordTok{rm}\NormalTok{(}\DataTypeTok{list =} \KeywordTok{ls}\NormalTok{())      }\CommentTok{# clear memory (removes all the variables from the workspace)}
\end{Highlighting}
\end{Shaded}

\section{01 Load packages}\label{load-packages}

\begin{Shaded}
\begin{Highlighting}[]
\ControlFlowTok{if}\NormalTok{ (}\OperatorTok{!}\KeywordTok{require}\NormalTok{(}\StringTok{'pacman'}\NormalTok{)) }\KeywordTok{install.packages}\NormalTok{(}\StringTok{'pacman'}\NormalTok{); }\KeywordTok{library}\NormalTok{(pacman) }

\CommentTok{# load (install if required) packages from CRAN}
\KeywordTok{p_load}\NormalTok{(}\StringTok{"here"}\NormalTok{, }\StringTok{"devtools"}\NormalTok{, }\StringTok{"dplyr"}\NormalTok{, }\StringTok{"scales"}\NormalTok{, }\StringTok{"ellipse"}\NormalTok{, }\StringTok{"ggplot2"}\NormalTok{, }\StringTok{"lazyeval"}\NormalTok{, }\StringTok{"igraph"}\NormalTok{, }\StringTok{"truncnorm"}\NormalTok{, }\StringTok{"ggraph"}\NormalTok{, }\StringTok{"reshape2"}\NormalTok{, }\StringTok{"knitr"}\NormalTok{, }\StringTok{"markdown"}\NormalTok{, }\StringTok{"stringr"}\NormalTok{)}
\CommentTok{# load (install if required) packages from GitHub}
\CommentTok{# install_github("DARTH-git/dampack", force = TRUE) # Uncomment if there is a newer version}
\CommentTok{# install_github("DARTH-git/darthtools", force = TRUE) # Uncomment if there is a newer version}
\KeywordTok{p_load_gh}\NormalTok{(}\StringTok{"DARTH-git/dampack"}\NormalTok{, }\StringTok{"DARTH-git/darthtools"}\NormalTok{)}
\end{Highlighting}
\end{Shaded}

\section{02 Load functions}\label{load-functions}

\begin{Shaded}
\begin{Highlighting}[]
\CommentTok{# No functions needed}
\end{Highlighting}
\end{Shaded}

\section{03 Input model parameters}\label{input-model-parameters}

\begin{Shaded}
\begin{Highlighting}[]
\KeywordTok{set.seed}\NormalTok{(}\DecValTok{1}\NormalTok{)  }\CommentTok{# set the seed  }

\CommentTok{# Model structure}
\NormalTok{v_n        <-}\StringTok{ }\KeywordTok{c}\NormalTok{(}\StringTok{"healthy"}\NormalTok{, }\StringTok{"sick"}\NormalTok{, }\StringTok{"dead"}\NormalTok{)  }\CommentTok{# vector with state names}
\NormalTok{n_states   <-}\StringTok{ }\KeywordTok{length}\NormalTok{(v_n)                   }\CommentTok{# number of states}
\NormalTok{n_t        <-}\StringTok{ }\DecValTok{60}                            \CommentTok{# number of cycles}
\NormalTok{n_i        <-}\StringTok{ }\DecValTok{10000}                         \CommentTok{# number of individuals}
\NormalTok{d_e <-}\StringTok{ }\NormalTok{d_c <-}\StringTok{ }\FloatTok{0.03}                          \CommentTok{# equal discount of costs and QALYs by 3%}

\CommentTok{# calculate discount weights for costs for each cycle based on discount rate d_c}
\NormalTok{v_dwc <-}\StringTok{ }\DecValTok{1} \OperatorTok{/}\StringTok{ }\NormalTok{(}\DecValTok{1} \OperatorTok{+}\StringTok{ }\NormalTok{d_e) }\OperatorTok{^}\StringTok{ }\NormalTok{(}\DecValTok{0}\OperatorTok{:}\NormalTok{n_t) }
\CommentTok{# calculate discount weights for effectiveness for each cycle based on discount rate d_e}
\NormalTok{v_dwe <-}\StringTok{ }\DecValTok{1} \OperatorTok{/}\StringTok{ }\NormalTok{(}\DecValTok{1} \OperatorTok{+}\StringTok{ }\NormalTok{d_c) }\OperatorTok{^}\StringTok{ }\NormalTok{(}\DecValTok{0}\OperatorTok{:}\NormalTok{n_t) }

\NormalTok{#### Deterministic analysis ####}
\CommentTok{# Transition probabilities }
\CommentTok{# (all non-dead probabilities are conditional on survival)}
\NormalTok{p_HS        <-}\StringTok{ }\FloatTok{0.05}    \CommentTok{# probability healthy -> sick}
\NormalTok{p_HD_female <-}\StringTok{ }\FloatTok{0.0382}  \CommentTok{# probability health -> dead when female}
\NormalTok{p_HD_male   <-}\StringTok{ }\FloatTok{0.0463}  \CommentTok{# probability health -> dead when male}
\NormalTok{m_p_HD      <-}\StringTok{ }\KeywordTok{data.frame}\NormalTok{(}\DataTypeTok{Sex =} \KeywordTok{c}\NormalTok{(}\StringTok{"Female"}\NormalTok{, }\StringTok{"Male"}\NormalTok{), }\DataTypeTok{p_HD =} \KeywordTok{c}\NormalTok{(p_HD_female, p_HD_male))}

\CommentTok{# probability to die in sick state by cycle of being sick}
\NormalTok{p_SD <-}\StringTok{ }\KeywordTok{c}\NormalTok{(}\FloatTok{0.1}\NormalTok{, }\FloatTok{0.2}\NormalTok{, }\FloatTok{0.3}\NormalTok{, }\FloatTok{0.4}\NormalTok{, }\FloatTok{0.5}\NormalTok{, }\KeywordTok{rep}\NormalTok{(}\FloatTok{0.7}\NormalTok{, n_t }\OperatorTok{-}\StringTok{ }\DecValTok{5}\NormalTok{)) }

\CommentTok{# Costs inputs}
\NormalTok{c_H  <-}\StringTok{ }\DecValTok{1500}  \CommentTok{# cost of one cycle in healthy state}
\NormalTok{c_S  <-}\StringTok{ }\DecValTok{5000}  \CommentTok{# cost of one cycle in sick state}
\NormalTok{c_D  <-}\StringTok{ }\DecValTok{0}

\CommentTok{# utility inputs}
\NormalTok{u_H  <-}\StringTok{ }\DecValTok{1}     \CommentTok{# utility when healthy }
\NormalTok{u_S  <-}\StringTok{ }\FloatTok{0.85}  \CommentTok{# utility when sick }
\NormalTok{u_D  <-}\StringTok{ }\DecValTok{0}     \CommentTok{# utility when dead }
\end{Highlighting}
\end{Shaded}

\section{04 Sample individual level
characteristics}\label{sample-individual-level-characteristics}

\subsection{04.1 Static characteristics}\label{static-characteristics}

\begin{Shaded}
\begin{Highlighting}[]
\CommentTok{# randomly sample the sex of an individual (50% female)}
\NormalTok{v_sex <-}\StringTok{ }\KeywordTok{sample}\NormalTok{(}\DataTypeTok{x =} \KeywordTok{c}\NormalTok{(}\StringTok{"Female"}\NormalTok{, }\StringTok{"Male"}\NormalTok{), }\DataTypeTok{prob =} \KeywordTok{c}\NormalTok{(}\FloatTok{0.5}\NormalTok{, }\FloatTok{0.5}\NormalTok{), }\DataTypeTok{size =}\NormalTok{ n_i, }\DataTypeTok{replace =} \OtherTok{TRUE}\NormalTok{) }
\end{Highlighting}
\end{Shaded}

\subsection{04.2 Dynamic characteristics}\label{dynamic-characteristics}

\begin{Shaded}
\begin{Highlighting}[]
\CommentTok{# Specify the initial health state of the individuals }
\CommentTok{# everyone begins in the healthy state (in this example)}
\NormalTok{v_M_init  <-}\StringTok{ }\KeywordTok{rep}\NormalTok{(}\StringTok{"healthy"}\NormalTok{, }\DataTypeTok{times =}\NormalTok{ n_i)   }
\NormalTok{v_Ts_init <-}\StringTok{ }\KeywordTok{rep}\NormalTok{(}\DecValTok{0}\NormalTok{, n_i)  }\CommentTok{# a vector with the time of being sick at the start of the model }
\end{Highlighting}
\end{Shaded}

\subsection{04.3 Create a dataframe with the individual
characteristics}\label{create-a-dataframe-with-the-individual-characteristics}

\begin{Shaded}
\begin{Highlighting}[]
\CommentTok{# create a data frame with each individual's }
\CommentTok{# ID number, treatment effect modifier, age and initial time in sick state }
\NormalTok{df_X  <-}\StringTok{ }\KeywordTok{data.frame}\NormalTok{(}\DataTypeTok{ID =} \DecValTok{1}\OperatorTok{:}\NormalTok{n_i, }\DataTypeTok{Sex =}\NormalTok{ v_sex, }\DataTypeTok{Ts_init =}\NormalTok{ v_Ts_init, }\DataTypeTok{M_init =}\NormalTok{ v_M_init)}
\KeywordTok{head}\NormalTok{(df_X) }\CommentTok{# print the first rows of the dataframe}
\end{Highlighting}
\end{Shaded}

\section{05 Define Simulation
Functions}\label{define-simulation-functions}

\subsection{05.1 Probability function}\label{probability-function}

The \texttt{Probs} function updates the transition probabilities of
every cycle is shown below.

\begin{Shaded}
\begin{Highlighting}[]
\NormalTok{Probs <-}\StringTok{ }\ControlFlowTok{function}\NormalTok{(M_t, df_X, v_Ts) \{ }
  \CommentTok{# Arguments:}
    \CommentTok{# M_t: health state occupied at cycle t (character variable)}
    \CommentTok{# df_X: data frame with individual characteristics data }
    \CommentTok{# v_Ts: vector with the duration of being sick}
  \CommentTok{# Returns: }
    \CommentTok{# transition probabilities for that cycle}
  
  \CommentTok{# create matrix of state transition probabilities}
\NormalTok{  m_p_t           <-}\StringTok{ }\KeywordTok{matrix}\NormalTok{(}\DecValTok{0}\NormalTok{, }\DataTypeTok{nrow =}\NormalTok{ n_states, }\DataTypeTok{ncol =}\NormalTok{ n_i)  }
  \CommentTok{# give the state names to the rows}
  \KeywordTok{rownames}\NormalTok{(m_p_t) <-}\StringTok{  }\NormalTok{v_n                               }
  
  \CommentTok{# lookup baseline probability and rate of dying based on individual characteristics}
\NormalTok{  p_HD_all <-}\StringTok{ }\KeywordTok{inner_join}\NormalTok{(df_X, m_p_HD, }\DataTypeTok{by =} \KeywordTok{c}\NormalTok{(}\StringTok{"Sex"}\NormalTok{))}
\NormalTok{  p_HD     <-}\StringTok{ }\NormalTok{p_HD_all[M_t }\OperatorTok{==}\StringTok{ "healthy"}\NormalTok{, }\StringTok{"p_HD"}\NormalTok{]}
  
  \CommentTok{# update m_p_t with the appropriate probabilities }
  \CommentTok{# (all non-death probabilities are conditional on survival)}
  \CommentTok{# transition probabilities when healthy }
\NormalTok{  m_p_t[, M_t }\OperatorTok{==}\StringTok{ "healthy"}\NormalTok{] <-}\StringTok{ }\KeywordTok{rbind}\NormalTok{((}\DecValTok{1} \OperatorTok{-}\StringTok{ }\NormalTok{p_HD) }\OperatorTok{*}\StringTok{ }\NormalTok{(}\DecValTok{1} \OperatorTok{-}\StringTok{ }\NormalTok{p_HS) ,}
\NormalTok{                                     (}\DecValTok{1} \OperatorTok{-}\StringTok{ }\NormalTok{p_HD) }\OperatorTok{*}\StringTok{      }\NormalTok{p_HS  ,}
\NormalTok{                                          p_HD)    }
  \CommentTok{# transition probabilities when sick }
\NormalTok{  m_p_t[, M_t }\OperatorTok{==}\StringTok{ "sick"}\NormalTok{]    <-}\StringTok{ }\KeywordTok{rbind}\NormalTok{(}\DecValTok{0}\NormalTok{,}
                                     \DecValTok{1} \OperatorTok{-}\StringTok{ }\NormalTok{p_SD[v_Ts],}
\NormalTok{                                         p_SD[v_Ts])  }
  \CommentTok{# transition probabilities when dead     }
\NormalTok{  m_p_t[, M_t }\OperatorTok{==}\StringTok{ "dead"}\NormalTok{]    <-}\StringTok{ }\KeywordTok{rbind}\NormalTok{(}\DecValTok{0}\NormalTok{,}
                                     \DecValTok{0}\NormalTok{,}
                                     \DecValTok{1}\NormalTok{)                            }
  
  \KeywordTok{return}\NormalTok{(}\KeywordTok{t}\NormalTok{(m_p_t))}
\NormalTok{\}       }
\end{Highlighting}
\end{Shaded}

\subsection{05.2 Cost function}\label{cost-function}

The \texttt{Costs} function estimates the costs at every cycle.

\begin{Shaded}
\begin{Highlighting}[]
\NormalTok{Costs <-}\StringTok{ }\ControlFlowTok{function}\NormalTok{ (M_t) \{}
  \CommentTok{# Arguments:}
    \CommentTok{# M_t: health state occupied at cycle t (character variable)}
  \CommentTok{# Returns: }
    \CommentTok{# costs accrued in this cycle}
  
\NormalTok{  c_t <-}\StringTok{ }\KeywordTok{c}\NormalTok{()}
\NormalTok{  c_t[M_t }\OperatorTok{==}\StringTok{ "healthy"}\NormalTok{] <-}\StringTok{ }\NormalTok{c_H  }\CommentTok{# costs accrued by being healthy this cycle}
\NormalTok{  c_t[M_t }\OperatorTok{==}\StringTok{ "sick"}\NormalTok{]    <-}\StringTok{ }\NormalTok{c_S  }\CommentTok{# costs accrued by being sick this cycle}
\NormalTok{  c_t[M_t }\OperatorTok{==}\StringTok{ "dead"}\NormalTok{]    <-}\StringTok{ }\NormalTok{c_D  }\CommentTok{# costs at dead state}
  
  \KeywordTok{return}\NormalTok{(c_t) }
\NormalTok{\}}
\end{Highlighting}
\end{Shaded}

\subsection{05.3 Health outcome function}\label{health-outcome-function}

The \texttt{Effs} function to update the utilities at every cycle.

\begin{Shaded}
\begin{Highlighting}[]
\NormalTok{Effs <-}\StringTok{ }\ControlFlowTok{function}\NormalTok{ (M_t) \{}
  \CommentTok{# Arguments:}
    \CommentTok{# M_t: health state occupied at cycle t (character variable)}
  \CommentTok{# Returns: }
    \CommentTok{# QALYs accrued this cycle}
  
\NormalTok{  q_t <-}\StringTok{ }\KeywordTok{c}\NormalTok{() }
\NormalTok{  q_t[M_t }\OperatorTok{==}\StringTok{ "healthy"}\NormalTok{] <-}\StringTok{ }\NormalTok{u_H  }\CommentTok{# QALYs accrued by being healthy this cycle}
\NormalTok{  q_t[M_t }\OperatorTok{==}\StringTok{ "sick"}\NormalTok{]    <-}\StringTok{ }\NormalTok{u_S  }\CommentTok{# QALYs accrued by being sick this cycle}
\NormalTok{  q_t[M_t }\OperatorTok{==}\StringTok{ "dead"}\NormalTok{]    <-}\StringTok{ }\NormalTok{u_D  }\CommentTok{# QALYs at dead state}
  
  \KeywordTok{return}\NormalTok{(q_t)  }
\NormalTok{\}}
\end{Highlighting}
\end{Shaded}

\subsection{05.4 Microsimulation
function}\label{microsimulation-function}

Below we develop the microsimulation function that allows the model to
be run.

\begin{Shaded}
\begin{Highlighting}[]
\NormalTok{MicroSim <-}\StringTok{ }\ControlFlowTok{function}\NormalTok{(n_i, df_X, }\DataTypeTok{seed =} \DecValTok{1}\NormalTok{) \{}
  \CommentTok{# Arguments:  }
    \CommentTok{# n_i: number of individuals}
    \CommentTok{# df_X: data frame with individual data }
    \CommentTok{# seed: seed for the random number generator, default is 1}
  \CommentTok{# Returns:}
    \CommentTok{# results: data frame with total cost and QALYs}
  
  \KeywordTok{set.seed}\NormalTok{(seed) }\CommentTok{# set a seed to be able to reproduce the same results}
  
  \CommentTok{# create three matrices called m_M, m_C and m_E}
  \CommentTok{# number of rows is equal to the n_i, the number of columns is equal to n_t }
  \CommentTok{# (the initial state and all the n_t cycles)}
  \CommentTok{# m_M is used to store the health state information over time for every individual}
  \CommentTok{# m_C is used to store the costs information over time for every individual}
  \CommentTok{# m_E is used to store the effects information over time for every individual}
  
\NormalTok{  m_M <-}\StringTok{ }\NormalTok{m_C <-}\StringTok{ }\NormalTok{m_E <-}\StringTok{  }\KeywordTok{matrix}\NormalTok{(}\DataTypeTok{nrow =}\NormalTok{ n_i, }\DataTypeTok{ncol =}\NormalTok{ n_t }\OperatorTok{+}\StringTok{ }\DecValTok{1}\NormalTok{, }
                               \DataTypeTok{dimnames =} \KeywordTok{list}\NormalTok{(}\KeywordTok{paste}\NormalTok{(}\StringTok{"ind"}\NormalTok{  , }\DecValTok{1}\OperatorTok{:}\NormalTok{n_i, }\DataTypeTok{sep =} \StringTok{" "}\NormalTok{), }
                                               \KeywordTok{paste}\NormalTok{(}\StringTok{"cycle"}\NormalTok{, }\DecValTok{0}\OperatorTok{:}\NormalTok{n_t, }\DataTypeTok{sep =} \StringTok{" "}\NormalTok{)))  }
 
\NormalTok{  m_M[, }\DecValTok{1}\NormalTok{] <-}\StringTok{ }\KeywordTok{as.character}\NormalTok{(df_X}\OperatorTok{$}\NormalTok{M_init) }\CommentTok{# initial health state}
\NormalTok{  v_Ts     <-}\StringTok{ }\NormalTok{df_X}\OperatorTok{$}\NormalTok{Ts_init     }\CommentTok{# initialize time since illness onset}
\NormalTok{  m_C[, }\DecValTok{1}\NormalTok{] <-}\StringTok{ }\KeywordTok{Costs}\NormalTok{(m_M[, }\DecValTok{1}\NormalTok{])  }\CommentTok{# costs accrued during cycle 0}
\NormalTok{  m_E[, }\DecValTok{1}\NormalTok{] <-}\StringTok{ }\KeywordTok{Effs}\NormalTok{(m_M[, }\DecValTok{1}\NormalTok{])   }\CommentTok{# QALYs accrued during cycle 0}
  
  \CommentTok{# open a loop for time running cycles 1 to n_t }
  \ControlFlowTok{for}\NormalTok{ (t }\ControlFlowTok{in} \DecValTok{1}\OperatorTok{:}\NormalTok{n_t) \{}
    \CommentTok{# calculate the transition probabilities for the cycle based on health state t}
\NormalTok{    m_P <-}\StringTok{ }\KeywordTok{Probs}\NormalTok{(m_M[, t], df_X, v_Ts)}
    \CommentTok{# check if transition probabilities are between 0 and 1}
    \KeywordTok{check_transition_probability}\NormalTok{(m_P, }\DataTypeTok{verbose =} \OtherTok{TRUE}\NormalTok{)}
    \CommentTok{# check if each of the rows of the transition probabilities matrix sum to one}
    \KeywordTok{check_sum_of_transition_array}\NormalTok{(m_P, }\DataTypeTok{n_states =}\NormalTok{ n_i, }\DataTypeTok{n_t =}\NormalTok{ n_t, }\DataTypeTok{verbose =} \OtherTok{TRUE}\NormalTok{)}
    \CommentTok{# sample the next health state and store that state in matrix m_M}
\NormalTok{    m_M[, t }\OperatorTok{+}\StringTok{ }\DecValTok{1}\NormalTok{]  <-}\StringTok{ }\KeywordTok{samplev}\NormalTok{(m_P)    }
    \CommentTok{# calculate costs per individual during cycle t + 1}
\NormalTok{    m_C[, t }\OperatorTok{+}\StringTok{ }\DecValTok{1}\NormalTok{]  <-}\StringTok{ }\KeywordTok{Costs}\NormalTok{(m_M[, t }\OperatorTok{+}\StringTok{ }\DecValTok{1}\NormalTok{])  }
    \CommentTok{# calculate QALYs per individual during cycle t + 1}
\NormalTok{    m_E[, t }\OperatorTok{+}\StringTok{ }\DecValTok{1}\NormalTok{]  <-}\StringTok{ }\KeywordTok{Effs}\NormalTok{ (m_M[, t }\OperatorTok{+}\StringTok{ }\DecValTok{1}\NormalTok{])  }
    
    \CommentTok{# update time since illness onset for t + 1 }
\NormalTok{    v_Ts <-}\StringTok{ }\KeywordTok{if_else}\NormalTok{(m_M[, t }\OperatorTok{+}\StringTok{ }\DecValTok{1}\NormalTok{] }\OperatorTok{==}\StringTok{ "sick"}\NormalTok{, v_Ts }\OperatorTok{+}\StringTok{ }\DecValTok{1}\NormalTok{, }\DecValTok{0}\NormalTok{) }
    
    \CommentTok{# Display simulation progress}
    \ControlFlowTok{if}\NormalTok{(t}\OperatorTok{/}\NormalTok{(n_t}\OperatorTok{/}\DecValTok{10}\NormalTok{) }\OperatorTok{==}\StringTok{ }\KeywordTok{round}\NormalTok{(t}\OperatorTok{/}\NormalTok{(n_t}\OperatorTok{/}\DecValTok{10}\NormalTok{), }\DecValTok{0}\NormalTok{)) \{ }\CommentTok{# display progress every 10%}
      \KeywordTok{cat}\NormalTok{(}\StringTok{'}\CharTok{\textbackslash{}r}\StringTok{'}\NormalTok{, }\KeywordTok{paste}\NormalTok{(t}\OperatorTok{/}\NormalTok{n_t }\OperatorTok{*}\StringTok{ }\DecValTok{100}\NormalTok{, }\StringTok{"% done"}\NormalTok{, }\DataTypeTok{sep =} \StringTok{" "}\NormalTok{))}
\NormalTok{    \}}
    
\NormalTok{  \} }\CommentTok{# close the loop for the time points }
  
  \CommentTok{# calculate  }
\NormalTok{  tc <-}\StringTok{ }\NormalTok{m_C }\OperatorTok\StringTok{ }\NormalTok{v_dwc  }\CommentTok{# total (discounted) cost per individual}
\NormalTok{  te <-}\StringTok{ }\NormalTok{m_E }\OperatorTok\StringTok{ }\NormalTok{v_dwe  }\CommentTok{# total (discounted) QALYs per individual }
\NormalTok{  tc_hat <-}\StringTok{ }\KeywordTok{mean}\NormalTok{(tc)   }\CommentTok{# average (discounted) cost }
\NormalTok{  te_hat <-}\StringTok{ }\KeywordTok{mean}\NormalTok{(te)   }\CommentTok{# average (discounted) QAL  }
  \CommentTok{# store the results from the simulation in a list}
\NormalTok{  results <-}\StringTok{ }\KeywordTok{list}\NormalTok{(}\DataTypeTok{m_M =}\NormalTok{ m_M, }\DataTypeTok{m_C =}\NormalTok{ m_C, }\DataTypeTok{m_E =}\NormalTok{ m_E, }\DataTypeTok{tc =}\NormalTok{ tc , }\DataTypeTok{te =}\NormalTok{ te, }\DataTypeTok{tc_hat =}\NormalTok{ tc_hat, }
                  \DataTypeTok{te_hat =}\NormalTok{ te_hat)   }
  
  \KeywordTok{return}\NormalTok{(results)  }\CommentTok{# return the results}

\NormalTok{\} }\CommentTok{# end of the `MicroSim` function  }
\end{Highlighting}
\end{Shaded}

\subsection{06 Run Microsimulation}\label{run-microsimulation}

\begin{Shaded}
\begin{Highlighting}[]
\CommentTok{# 06 Run Microsimulation}

\CommentTok{# By specifying all the arguments in the `MicroSim()` the simulation can be started}

\CommentTok{# Run the simulation model}
\NormalTok{outcomes <-}\StringTok{ }\KeywordTok{MicroSim}\NormalTok{(}\DataTypeTok{n_i =}\NormalTok{ n_i, }\DataTypeTok{df_X =}\NormalTok{ df_X, }\DataTypeTok{seed =} \DecValTok{1}\NormalTok{)}

\CommentTok{# Show results}
\NormalTok{results  <-}\StringTok{ }\KeywordTok{data.frame}\NormalTok{(}\StringTok{"Total Cost"}\NormalTok{ =}\StringTok{ }\NormalTok{outcomes}\OperatorTok{$}\NormalTok{tc_hat, }\StringTok{"Total QALYs"}\NormalTok{ =}\StringTok{ }\NormalTok{outcomes}\OperatorTok{$}\NormalTok{te_hat)}
\NormalTok{results}
\end{Highlighting}
\end{Shaded}

\section{07 Visualize results}\label{visualize-results}

\begin{Shaded}
\begin{Highlighting}[]
\KeywordTok{options}\NormalTok{(}\DataTypeTok{scipen =} \DecValTok{999}\NormalTok{)   }\CommentTok{# disable scientific notation in R}
\KeywordTok{plot}\NormalTok{(}\KeywordTok{density}\NormalTok{(outcomes}\OperatorTok{$}\NormalTok{tc), }\DataTypeTok{main =} \KeywordTok{paste}\NormalTok{(}\StringTok{"Total cost per person"}\NormalTok{), }\DataTypeTok{xlab =} \StringTok{"Cost ($)"}\NormalTok{)}
\KeywordTok{plot}\NormalTok{(}\KeywordTok{density}\NormalTok{(outcomes}\OperatorTok{$}\NormalTok{te), }\DataTypeTok{main =} \KeywordTok{paste}\NormalTok{(}\StringTok{"Total QALYs per person"}\NormalTok{), }\DataTypeTok{xlab =} \StringTok{"QALYs"}\NormalTok{)}
\KeywordTok{plot_m_TR}\NormalTok{(outcomes}\OperatorTok{$}\NormalTok{m_M) }\CommentTok{# health state trace}
\end{Highlighting}
\end{Shaded}

\end{document}
