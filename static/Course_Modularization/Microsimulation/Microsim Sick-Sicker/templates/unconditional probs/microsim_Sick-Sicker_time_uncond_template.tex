% Options for packages loaded elsewhere
\PassOptionsToPackage{unicode}{hyperref}
\PassOptionsToPackage{hyphens}{url}
%
\documentclass[
]{article}
\usepackage{lmodern}
\usepackage{amssymb,amsmath}
\usepackage{ifxetex,ifluatex}
\ifnum 0\ifxetex 1\fi\ifluatex 1\fi=0 % if pdftex
  \usepackage[T1]{fontenc}
  \usepackage[utf8]{inputenc}
  \usepackage{textcomp} % provide euro and other symbols
\else % if luatex or xetex
  \usepackage{unicode-math}
  \defaultfontfeatures{Scale=MatchLowercase}
  \defaultfontfeatures[\rmfamily]{Ligatures=TeX,Scale=1}
\fi
% Use upquote if available, for straight quotes in verbatim environments
\IfFileExists{upquote.sty}{\usepackage{upquote}}{}
\IfFileExists{microtype.sty}{% use microtype if available
  \usepackage[]{microtype}
  \UseMicrotypeSet[protrusion]{basicmath} % disable protrusion for tt fonts
}{}
\makeatletter
\@ifundefined{KOMAClassName}{% if non-KOMA class
  \IfFileExists{parskip.sty}{%
    \usepackage{parskip}
  }{% else
    \setlength{\parindent}{0pt}
    \setlength{\parskip}{6pt plus 2pt minus 1pt}}
}{% if KOMA class
  \KOMAoptions{parskip=half}}
\makeatother
\usepackage{xcolor}
\IfFileExists{xurl.sty}{\usepackage{xurl}}{} % add URL line breaks if available
\IfFileExists{bookmark.sty}{\usepackage{bookmark}}{\usepackage{hyperref}}
\hypersetup{
  pdftitle={Microsimulation Sick-Sicker model with time dependency with PSA},
  pdfauthor={The DARTH workgroup},
  hidelinks,
  pdfcreator={LaTeX via pandoc}}
\urlstyle{same} % disable monospaced font for URLs
\usepackage[margin=1in]{geometry}
\usepackage{color}
\usepackage{fancyvrb}
\newcommand{\VerbBar}{|}
\newcommand{\VERB}{\Verb[commandchars=\\\{\}]}
\DefineVerbatimEnvironment{Highlighting}{Verbatim}{commandchars=\\\{\}}
% Add ',fontsize=\small' for more characters per line
\usepackage{framed}
\definecolor{shadecolor}{RGB}{248,248,248}
\newenvironment{Shaded}{\begin{snugshade}}{\end{snugshade}}
\newcommand{\AlertTok}[1]{\textcolor[rgb]{0.94,0.16,0.16}{#1}}
\newcommand{\AnnotationTok}[1]{\textcolor[rgb]{0.56,0.35,0.01}{\textbf{\textit{#1}}}}
\newcommand{\AttributeTok}[1]{\textcolor[rgb]{0.77,0.63,0.00}{#1}}
\newcommand{\BaseNTok}[1]{\textcolor[rgb]{0.00,0.00,0.81}{#1}}
\newcommand{\BuiltInTok}[1]{#1}
\newcommand{\CharTok}[1]{\textcolor[rgb]{0.31,0.60,0.02}{#1}}
\newcommand{\CommentTok}[1]{\textcolor[rgb]{0.56,0.35,0.01}{\textit{#1}}}
\newcommand{\CommentVarTok}[1]{\textcolor[rgb]{0.56,0.35,0.01}{\textbf{\textit{#1}}}}
\newcommand{\ConstantTok}[1]{\textcolor[rgb]{0.00,0.00,0.00}{#1}}
\newcommand{\ControlFlowTok}[1]{\textcolor[rgb]{0.13,0.29,0.53}{\textbf{#1}}}
\newcommand{\DataTypeTok}[1]{\textcolor[rgb]{0.13,0.29,0.53}{#1}}
\newcommand{\DecValTok}[1]{\textcolor[rgb]{0.00,0.00,0.81}{#1}}
\newcommand{\DocumentationTok}[1]{\textcolor[rgb]{0.56,0.35,0.01}{\textbf{\textit{#1}}}}
\newcommand{\ErrorTok}[1]{\textcolor[rgb]{0.64,0.00,0.00}{\textbf{#1}}}
\newcommand{\ExtensionTok}[1]{#1}
\newcommand{\FloatTok}[1]{\textcolor[rgb]{0.00,0.00,0.81}{#1}}
\newcommand{\FunctionTok}[1]{\textcolor[rgb]{0.00,0.00,0.00}{#1}}
\newcommand{\ImportTok}[1]{#1}
\newcommand{\InformationTok}[1]{\textcolor[rgb]{0.56,0.35,0.01}{\textbf{\textit{#1}}}}
\newcommand{\KeywordTok}[1]{\textcolor[rgb]{0.13,0.29,0.53}{\textbf{#1}}}
\newcommand{\NormalTok}[1]{#1}
\newcommand{\OperatorTok}[1]{\textcolor[rgb]{0.81,0.36,0.00}{\textbf{#1}}}
\newcommand{\OtherTok}[1]{\textcolor[rgb]{0.56,0.35,0.01}{#1}}
\newcommand{\PreprocessorTok}[1]{\textcolor[rgb]{0.56,0.35,0.01}{\textit{#1}}}
\newcommand{\RegionMarkerTok}[1]{#1}
\newcommand{\SpecialCharTok}[1]{\textcolor[rgb]{0.00,0.00,0.00}{#1}}
\newcommand{\SpecialStringTok}[1]{\textcolor[rgb]{0.31,0.60,0.02}{#1}}
\newcommand{\StringTok}[1]{\textcolor[rgb]{0.31,0.60,0.02}{#1}}
\newcommand{\VariableTok}[1]{\textcolor[rgb]{0.00,0.00,0.00}{#1}}
\newcommand{\VerbatimStringTok}[1]{\textcolor[rgb]{0.31,0.60,0.02}{#1}}
\newcommand{\WarningTok}[1]{\textcolor[rgb]{0.56,0.35,0.01}{\textbf{\textit{#1}}}}
\usepackage{graphicx,grffile}
\makeatletter
\def\maxwidth{\ifdim\Gin@nat@width>\linewidth\linewidth\else\Gin@nat@width\fi}
\def\maxheight{\ifdim\Gin@nat@height>\textheight\textheight\else\Gin@nat@height\fi}
\makeatother
% Scale images if necessary, so that they will not overflow the page
% margins by default, and it is still possible to overwrite the defaults
% using explicit options in \includegraphics[width, height, ...]{}
\setkeys{Gin}{width=\maxwidth,height=\maxheight,keepaspectratio}
% Set default figure placement to htbp
\makeatletter
\def\fps@figure{htbp}
\makeatother
\setlength{\emergencystretch}{3em} % prevent overfull lines
\providecommand{\tightlist}{%
  \setlength{\itemsep}{0pt}\setlength{\parskip}{0pt}}
\setcounter{secnumdepth}{-\maxdimen} % remove section numbering

\title{Microsimulation Sick-Sicker model with time dependency with PSA}
\usepackage{etoolbox}
\makeatletter
\providecommand{\subtitle}[1]{% add subtitle to \maketitle
  \apptocmd{\@title}{\par {\large #1 \par}}{}{}
}
\makeatother
\subtitle{Includes individual characteristics: age, age dependent mortality
probabilities, individual treatment effect modifyer, time dependency for
the sick (S1) state, increasing change of death in the first 6 year of
sickness (tunnel)}
\author{The DARTH workgroup}
\date{}

\begin{document}
\maketitle

Developed by the Decision Analysis in R for Technologies in Health
(DARTH) workgroup:

Fernando Alarid-Escudero, PhD (1)

Eva A. Enns, MS, PhD (2)

M.G. Myriam Hunink, MD, PhD (3,4)

Hawre J. Jalal, MD, PhD (5)

Eline M. Krijkamp, MSc (3)

Petros Pechlivanoglou, PhD (6,7)

Alan Yang, MSc (7)

In collaboration of:

\begin{enumerate}
\def\labelenumi{\arabic{enumi}.}
\tightlist
\item
  Drug Policy Program, Center for Research and Teaching in Economics
  (CIDE) - CONACyT, Aguascalientes, Mexico
\item
  University of Minnesota School of Public Health, Minneapolis, MN, USA
\item
  Erasmus MC, Rotterdam, The Netherlands
\item
  Harvard T.H. Chan School of Public Health, Boston, USA
\item
  University of Pittsburgh Graduate School of Public Health, Pittsburgh,
  PA, USA
\item
  University of Toronto, Toronto ON, Canada
\item
  The Hospital for Sick Children, Toronto ON, Canada
\end{enumerate}

Please cite our publications when using this code:

\begin{itemize}
\item
  Jalal H, Pechlivanoglou P, Krijkamp E, Alarid-Escudero F, Enns E,
  Hunink MG. An Overview of R in Health Decision Sciences. Med Decis
  Making. 2017; 37(3): 735-746.
  \url{https://journals.sagepub.com/doi/abs/10.1177/0272989X16686559}
\item
  Krijkamp EM, Alarid-Escudero F, Enns EA, Jalal HJ, Hunink MGM,
  Pechlivanoglou P. Microsimulation modeling for health decision
  sciences using R: A tutorial. Med Decis Making. 2018;38(3):400--22.
  \url{https://journals.sagepub.com/doi/abs/10.1177/0272989X18754513}
\item
  Krijkamp EM, Alarid-Escudero F, Enns E, Pechlivanoglou P, Hunink MM,
  Jalal H. A Multidimensional Array Representation of State-Transition
  Model Dynamics. Med Decis Making. 2020 Online first.
  \url{https://doi.org/10.1177/0272989X19893973}
\end{itemize}

Copyright 2017, THE HOSPITAL FOR SICK CHILDREN AND THE COLLABORATING
INSTITUTIONS. All rights reserved in Canada, the United States and
worldwide. Copyright, trademarks, trade names and any and all associated
intellectual property are exclusively owned by THE HOSPITAL FOR Sick
CHILDREN and the collaborating institutions. These materials may be
used, reproduced, modified, distributed and adapted with proper
attribution.

\newpage

Change \texttt{eval} to \texttt{TRUE} if you want to knit this document.

\begin{Shaded}
\begin{Highlighting}[]
\KeywordTok{rm}\NormalTok{(}\DataTypeTok{list =} \KeywordTok{ls}\NormalTok{())      }\CommentTok{# clear memory (removes all the variables from the workspace)}
\end{Highlighting}
\end{Shaded}

\hypertarget{load-packages}{%
\section{01 Load packages}\label{load-packages}}

\begin{Shaded}
\begin{Highlighting}[]
\ControlFlowTok{if}\NormalTok{ (}\OperatorTok{!}\KeywordTok{require}\NormalTok{(}\StringTok{'pacman'}\NormalTok{)) }\KeywordTok{install.packages}\NormalTok{(}\StringTok{'pacman'}\NormalTok{); }\KeywordTok{library}\NormalTok{(pacman) }\CommentTok{# use this package to conveniently install other packages}
\CommentTok{# load (install if required) packages from CRAN}
\KeywordTok{p_load}\NormalTok{(}\StringTok{"here"}\NormalTok{, }\StringTok{"dplyr"}\NormalTok{, }\StringTok{"devtools"}\NormalTok{, }\StringTok{"scales"}\NormalTok{, }\StringTok{"ellipse"}\NormalTok{, }\StringTok{"ggplot2"}\NormalTok{, }\StringTok{"lazyeval"}\NormalTok{, }\StringTok{"igraph"}\NormalTok{, }\StringTok{"ggraph"}\NormalTok{, }\StringTok{"reshape2"}\NormalTok{, }\StringTok{"knitr"}\NormalTok{)                                               }
\CommentTok{# load (install if required) packages from GitHub}
\CommentTok{# install_github("DARTH-git/dampack", force = TRUE) # Uncomment if there is a newer version}
\CommentTok{# install_github("DARTH-git/darthtools", force = TRUE) # Uncomment if there is a newer version}
\KeywordTok{p_load_gh}\NormalTok{(}\StringTok{"DARTH-git/dampack"}\NormalTok{, }\StringTok{"DARTH-git/darthtools"}\NormalTok{)}
\end{Highlighting}
\end{Shaded}

\hypertarget{load-functions}{%
\section{02 Load functions}\label{load-functions}}

\begin{Shaded}
\begin{Highlighting}[]
\CommentTok{# No functions needed}
\end{Highlighting}
\end{Shaded}

\hypertarget{input-model-parameters}{%
\section{03 Input model parameters}\label{input-model-parameters}}

\begin{Shaded}
\begin{Highlighting}[]
\KeywordTok{set.seed}\NormalTok{(}\DecValTok{1}\NormalTok{)  }\CommentTok{# set the seed  }

\CommentTok{# Model structure }
\NormalTok{n_t   <-}\StringTok{ }\DecValTok{30}                       \CommentTok{# time horizon, 30 cycles}
\NormalTok{n_i   <-}\StringTok{ }\DecValTok{100000}                   \CommentTok{# number of simulated individuals}
\NormalTok{v_n   <-}\StringTok{ }\KeywordTok{c}\NormalTok{(}\StringTok{"H"}\NormalTok{, }\StringTok{"S1"}\NormalTok{, }\StringTok{"S2"}\NormalTok{, }\StringTok{"D"}\NormalTok{)  }\CommentTok{# the model states names}
\NormalTok{n_s   <-}\StringTok{ }\KeywordTok{length}\NormalTok{(v_n)              }\CommentTok{# the number of health states}
\NormalTok{d_r   <-}\StringTok{ }\FloatTok{0.03}                     \CommentTok{# discount rate of 3% per cycle}
\NormalTok{v_dwe <-}\StringTok{ }\NormalTok{v_dwc <-}\StringTok{ }\DecValTok{1} \OperatorTok{/}\StringTok{ }\NormalTok{((}\DecValTok{1} \OperatorTok{+}\StringTok{ }\NormalTok{d_r) }\OperatorTok{^}\StringTok{ }\NormalTok{(}\DecValTok{0}\OperatorTok{:}\NormalTok{n_t))    }\CommentTok{# discount weight }
\NormalTok{v_names_str <-}\StringTok{ }\KeywordTok{c}\NormalTok{(}\StringTok{"no treatment"}\NormalTok{, }\StringTok{"treatment"}\NormalTok{)  }\CommentTok{# strategy names}
\NormalTok{n_str <-}\StringTok{ }\KeywordTok{length}\NormalTok{(v_names_str)      }\CommentTok{# number of strategies}

\CommentTok{### Event probabilities (per cycle)}
\CommentTok{# Annual transition probabilities}
\NormalTok{p_HS1   <-}\StringTok{ }\FloatTok{0.15}                   \CommentTok{# probability of becoming sick when healthy}
\NormalTok{p_S1H   <-}\StringTok{ }\FloatTok{0.5}                    \CommentTok{# probability of recovering to healthy when sick}
\NormalTok{p_S1S2  <-}\StringTok{ }\FloatTok{0.105}                  \CommentTok{# probability of becoming sicker when sick}

\CommentTok{# Annual probabilities of death}
\CommentTok{# load age dependent probability}
\NormalTok{p_mort   <-}\StringTok{ }\KeywordTok{read.csv}\NormalTok{(}\StringTok{"mortProb_age.csv"}\NormalTok{)}
\CommentTok{# load age distribution}
\NormalTok{dist_Age <-}\StringTok{ }\KeywordTok{read.csv}\NormalTok{(}\StringTok{"MyPopulation-AgeDistribution.csv"}\NormalTok{)}

\CommentTok{# probability to die in S1 by cycle }
\NormalTok{p_S1D    <-}\StringTok{ }\KeywordTok{c}\NormalTok{(}\FloatTok{0.0149}\NormalTok{, }\FloatTok{0.018}\NormalTok{, }\FloatTok{0.021}\NormalTok{, }\FloatTok{0.026}\NormalTok{, }\FloatTok{0.031}\NormalTok{, }\KeywordTok{rep}\NormalTok{(}\FloatTok{0.037}\NormalTok{, n_t }\OperatorTok{-}\StringTok{ }\DecValTok{5}\NormalTok{)) }
\NormalTok{p_S2D    <-}\StringTok{ }\FloatTok{0.048}           \CommentTok{# probability to die in S2}

\CommentTok{# Cost inputs}
\NormalTok{c_H     <-}\StringTok{ }\DecValTok{2000}             \CommentTok{# cost of one cycle in the healthy state}
\NormalTok{c_S1    <-}\StringTok{ }\DecValTok{4000}             \CommentTok{# cost of one cycle in the sick state}
\NormalTok{c_S2    <-}\StringTok{ }\DecValTok{15000}            \CommentTok{# cost of one cycle in the sicker state}
\NormalTok{c_D     <-}\StringTok{ }\DecValTok{0}                \CommentTok{# cost of one cycle in the dead state}
\NormalTok{c_Trt   <-}\StringTok{ }\DecValTok{12000}            \CommentTok{# cost of treatment (per cycle)}

\CommentTok{# Utility inputs}
\NormalTok{u_H     <-}\StringTok{ }\DecValTok{1}                \CommentTok{# utility when healthy }
\NormalTok{u_S1    <-}\StringTok{ }\FloatTok{0.75}             \CommentTok{# utility when sick }
\NormalTok{u_S2    <-}\StringTok{ }\FloatTok{0.5}              \CommentTok{# utility when sicker}
\NormalTok{u_D     <-}\StringTok{ }\DecValTok{0}                \CommentTok{# utility when dead}
\NormalTok{u_Trt   <-}\StringTok{ }\FloatTok{0.95}             \CommentTok{# utility when sick(er) and being treated}
\end{Highlighting}
\end{Shaded}

\hypertarget{sample-individual-level-characteristics}{%
\section{04 Sample individual level
characteristics}\label{sample-individual-level-characteristics}}

\hypertarget{static-characteristics}{%
\subsection{04.1 Static characteristics}\label{static-characteristics}}

\begin{Shaded}
\begin{Highlighting}[]
\CommentTok{# your turn}
\NormalTok{v_x     <-}\StringTok{ }\KeywordTok{runif}\NormalTok{(n_i, }\DataTypeTok{min =} \FloatTok{0.95}\NormalTok{, }\DataTypeTok{max =} \FloatTok{1.05}\NormalTok{) }\CommentTok{# treatment effect modifier at baseline }
\end{Highlighting}
\end{Shaded}

\hypertarget{dynamic-characteristics}{%
\subsection{04.2 Dynamic
characteristics}\label{dynamic-characteristics}}

\begin{Shaded}
\begin{Highlighting}[]
\CommentTok{# your turn}
\end{Highlighting}
\end{Shaded}

\hypertarget{create-a-dataframe-with-the-individual-characteristics}{%
\subsection{04.3 Create a dataframe with the individual
characteristics}\label{create-a-dataframe-with-the-individual-characteristics}}

\begin{Shaded}
\begin{Highlighting}[]
\NormalTok{df_X    <-}\StringTok{ }\CommentTok{#data.frame( # ADD ALL CHARACTERISTICS ) }
\end{Highlighting}
\end{Shaded}

\hypertarget{define-simulation-functions}{%
\section{05 Define Simulation
Functions}\label{define-simulation-functions}}

There is no need to make two functions for each strategy. We recomment
to make one \texttt{Probs()}, one \texttt{Cost()} and one
\texttt{Effs()} function and have an function argument
\texttt{Trt\ =\ FALSE} which you ``switch'' on and off of the strategy
of interest.

Please see the example below:

\begin{Shaded}
\begin{Highlighting}[]
\NormalTok{cost_stay <-}\StringTok{ }\ControlFlowTok{function}\NormalTok{ (}\DataTypeTok{days =} \DecValTok{0}\NormalTok{, }\DataTypeTok{Trt =} \OtherTok{FALSE}\NormalTok{) \{}
  \CommentTok{# days: days an individual is staying in a care facility}
  \CommentTok{# Trt:  is the individual treated? (default is FALSE) }
  
\NormalTok{  c_stay_day <-}\StringTok{ }\DecValTok{100}   \CommentTok{# the price to stay a day at the care facility}
\NormalTok{  c_Trt <-}\StringTok{ }\DecValTok{3000}       \CommentTok{# the price of treatment, total price for drug. Drug requires one dose}
  
\NormalTok{  cost <-}\StringTok{ }\NormalTok{c_stay_day }\OperatorTok{*}\StringTok{ }\NormalTok{days }\OperatorTok{+}\StringTok{ }\NormalTok{Trt }\OperatorTok{*}\StringTok{ }\NormalTok{c_Trt}

  \KeywordTok{return}\NormalTok{(cost)       }\CommentTok{# return the pric e}
\NormalTok{\}}

\NormalTok{cost_stay_noTrt <-}\StringTok{ }\KeywordTok{cost_stay}\NormalTok{(}\DataTypeTok{days =} \DecValTok{10}\NormalTok{, }\DataTypeTok{Trt =} \OtherTok{FALSE}\NormalTok{) }\CommentTok{# run the function for the no treatment strategy}
\NormalTok{cost_stay_Trt   <-}\StringTok{ }\KeywordTok{cost_stay}\NormalTok{(}\DataTypeTok{days =} \DecValTok{10}\NormalTok{, }\DataTypeTok{Trt =} \OtherTok{TRUE}\NormalTok{) }\CommentTok{# run the function for the treatment strategy}

\NormalTok{cost_stay_noTrt}
\NormalTok{cost_stay_Trt}
\end{Highlighting}
\end{Shaded}

\hypertarget{probability-function}{%
\subsection{05.1 Probability function}\label{probability-function}}

The function that updates the transition probabilities of every cycle is
shown below. Please make sure you correctly incorporate the time
dependency

\begin{Shaded}
\begin{Highlighting}[]
\CommentTok{# your turn}

\CommentTok{# In this function you have to incorporate age specific mortaility and incorporate the change in probability of the years spend in the sick state}
\end{Highlighting}
\end{Shaded}

\hypertarget{cost-function}{%
\subsection{05.2 Cost function}\label{cost-function}}

The \texttt{Costs} function estimates the costs at every cycle.

\begin{Shaded}
\begin{Highlighting}[]
\CommentTok{# your turn}

\CommentTok{# Make sure you incorporate the cost of the treatment in the treatment strategy}
\end{Highlighting}
\end{Shaded}

\hypertarget{health-outcome-function}{%
\subsection{05.3 Health outcome
function}\label{health-outcome-function}}

The \texttt{Effs} function to update the utilities at every cycle.

\begin{Shaded}
\begin{Highlighting}[]
\CommentTok{# your turn}

\CommentTok{# Make sure you incorporate the treatment effect modifier }
\end{Highlighting}
\end{Shaded}

\hypertarget{run-microsimulation}{%
\section{06 Run Microsimulation}\label{run-microsimulation}}

You have to run the function twice. Once for the treatment strategy and
once of the no-treatment strategy

\begin{Shaded}
\begin{Highlighting}[]
\CommentTok{# your turn}
\end{Highlighting}
\end{Shaded}

\hypertarget{visualize-results}{%
\section{07 Visualize results}\label{visualize-results}}

\begin{Shaded}
\begin{Highlighting}[]
\CommentTok{# your turn}
\end{Highlighting}
\end{Shaded}

\hypertarget{cost-effectiveness-analysis}{%
\section{08 Cost Effectiveness
Analysis}\label{cost-effectiveness-analysis}}

\begin{Shaded}
\begin{Highlighting}[]
\CommentTok{# your turn}
\end{Highlighting}
\end{Shaded}

\end{document}
