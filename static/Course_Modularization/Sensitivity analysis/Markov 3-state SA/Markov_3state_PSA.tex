% Options for packages loaded elsewhere
\PassOptionsToPackage{unicode}{hyperref}
\PassOptionsToPackage{hyphens}{url}
%
\documentclass[
]{article}
\usepackage{lmodern}
\usepackage{amssymb,amsmath}
\usepackage{ifxetex,ifluatex}
\ifnum 0\ifxetex 1\fi\ifluatex 1\fi=0 % if pdftex
  \usepackage[T1]{fontenc}
  \usepackage[utf8]{inputenc}
  \usepackage{textcomp} % provide euro and other symbols
\else % if luatex or xetex
  \usepackage{unicode-math}
  \defaultfontfeatures{Scale=MatchLowercase}
  \defaultfontfeatures[\rmfamily]{Ligatures=TeX,Scale=1}
\fi
% Use upquote if available, for straight quotes in verbatim environments
\IfFileExists{upquote.sty}{\usepackage{upquote}}{}
\IfFileExists{microtype.sty}{% use microtype if available
  \usepackage[]{microtype}
  \UseMicrotypeSet[protrusion]{basicmath} % disable protrusion for tt fonts
}{}
\makeatletter
\@ifundefined{KOMAClassName}{% if non-KOMA class
  \IfFileExists{parskip.sty}{%
    \usepackage{parskip}
  }{% else
    \setlength{\parindent}{0pt}
    \setlength{\parskip}{6pt plus 2pt minus 1pt}}
}{% if KOMA class
  \KOMAoptions{parskip=half}}
\makeatother
\usepackage{xcolor}
\IfFileExists{xurl.sty}{\usepackage{xurl}}{} % add URL line breaks if available
\IfFileExists{bookmark.sty}{\usepackage{bookmark}}{\usepackage{hyperref}}
\hypersetup{
  pdftitle={Simple 3-state Markov model in R},
  pdfauthor={The DARTH workgroup},
  hidelinks,
  pdfcreator={LaTeX via pandoc}}
\urlstyle{same} % disable monospaced font for URLs
\usepackage[margin=1in]{geometry}
\usepackage{color}
\usepackage{fancyvrb}
\newcommand{\VerbBar}{|}
\newcommand{\VERB}{\Verb[commandchars=\\\{\}]}
\DefineVerbatimEnvironment{Highlighting}{Verbatim}{commandchars=\\\{\}}
% Add ',fontsize=\small' for more characters per line
\usepackage{framed}
\definecolor{shadecolor}{RGB}{248,248,248}
\newenvironment{Shaded}{\begin{snugshade}}{\end{snugshade}}
\newcommand{\AlertTok}[1]{\textcolor[rgb]{0.94,0.16,0.16}{#1}}
\newcommand{\AnnotationTok}[1]{\textcolor[rgb]{0.56,0.35,0.01}{\textbf{\textit{#1}}}}
\newcommand{\AttributeTok}[1]{\textcolor[rgb]{0.77,0.63,0.00}{#1}}
\newcommand{\BaseNTok}[1]{\textcolor[rgb]{0.00,0.00,0.81}{#1}}
\newcommand{\BuiltInTok}[1]{#1}
\newcommand{\CharTok}[1]{\textcolor[rgb]{0.31,0.60,0.02}{#1}}
\newcommand{\CommentTok}[1]{\textcolor[rgb]{0.56,0.35,0.01}{\textit{#1}}}
\newcommand{\CommentVarTok}[1]{\textcolor[rgb]{0.56,0.35,0.01}{\textbf{\textit{#1}}}}
\newcommand{\ConstantTok}[1]{\textcolor[rgb]{0.00,0.00,0.00}{#1}}
\newcommand{\ControlFlowTok}[1]{\textcolor[rgb]{0.13,0.29,0.53}{\textbf{#1}}}
\newcommand{\DataTypeTok}[1]{\textcolor[rgb]{0.13,0.29,0.53}{#1}}
\newcommand{\DecValTok}[1]{\textcolor[rgb]{0.00,0.00,0.81}{#1}}
\newcommand{\DocumentationTok}[1]{\textcolor[rgb]{0.56,0.35,0.01}{\textbf{\textit{#1}}}}
\newcommand{\ErrorTok}[1]{\textcolor[rgb]{0.64,0.00,0.00}{\textbf{#1}}}
\newcommand{\ExtensionTok}[1]{#1}
\newcommand{\FloatTok}[1]{\textcolor[rgb]{0.00,0.00,0.81}{#1}}
\newcommand{\FunctionTok}[1]{\textcolor[rgb]{0.00,0.00,0.00}{#1}}
\newcommand{\ImportTok}[1]{#1}
\newcommand{\InformationTok}[1]{\textcolor[rgb]{0.56,0.35,0.01}{\textbf{\textit{#1}}}}
\newcommand{\KeywordTok}[1]{\textcolor[rgb]{0.13,0.29,0.53}{\textbf{#1}}}
\newcommand{\NormalTok}[1]{#1}
\newcommand{\OperatorTok}[1]{\textcolor[rgb]{0.81,0.36,0.00}{\textbf{#1}}}
\newcommand{\OtherTok}[1]{\textcolor[rgb]{0.56,0.35,0.01}{#1}}
\newcommand{\PreprocessorTok}[1]{\textcolor[rgb]{0.56,0.35,0.01}{\textit{#1}}}
\newcommand{\RegionMarkerTok}[1]{#1}
\newcommand{\SpecialCharTok}[1]{\textcolor[rgb]{0.00,0.00,0.00}{#1}}
\newcommand{\SpecialStringTok}[1]{\textcolor[rgb]{0.31,0.60,0.02}{#1}}
\newcommand{\StringTok}[1]{\textcolor[rgb]{0.31,0.60,0.02}{#1}}
\newcommand{\VariableTok}[1]{\textcolor[rgb]{0.00,0.00,0.00}{#1}}
\newcommand{\VerbatimStringTok}[1]{\textcolor[rgb]{0.31,0.60,0.02}{#1}}
\newcommand{\WarningTok}[1]{\textcolor[rgb]{0.56,0.35,0.01}{\textbf{\textit{#1}}}}
\usepackage{graphicx,grffile}
\makeatletter
\def\maxwidth{\ifdim\Gin@nat@width>\linewidth\linewidth\else\Gin@nat@width\fi}
\def\maxheight{\ifdim\Gin@nat@height>\textheight\textheight\else\Gin@nat@height\fi}
\makeatother
% Scale images if necessary, so that they will not overflow the page
% margins by default, and it is still possible to overwrite the defaults
% using explicit options in \includegraphics[width, height, ...]{}
\setkeys{Gin}{width=\maxwidth,height=\maxheight,keepaspectratio}
% Set default figure placement to htbp
\makeatletter
\def\fps@figure{htbp}
\makeatother
\setlength{\emergencystretch}{3em} % prevent overfull lines
\providecommand{\tightlist}{%
  \setlength{\itemsep}{0pt}\setlength{\parskip}{0pt}}
\setcounter{secnumdepth}{-\maxdimen} % remove section numbering

\title{Simple 3-state Markov model in R}
\usepackage{etoolbox}
\makeatletter
\providecommand{\subtitle}[1]{% add subtitle to \maketitle
  \apptocmd{\@title}{\par {\large #1 \par}}{}{}
}
\makeatother
\subtitle{With a probabilistic sensitivty analysis (PSA)}
\author{The DARTH workgroup}
\date{}

\begin{document}
\maketitle

Developed by the Decision Analysis in R for Technologies in Health
(DARTH) workgroup:

Fernando Alarid-Escudero, PhD (1)

Eva A. Enns, MS, PhD (2)

M.G. Myriam Hunink, MD, PhD (3,4)

Hawre J. Jalal, MD, PhD (5)

Eline M. Krijkamp, MSc (3)

Petros Pechlivanoglou, PhD (6,7)

Alan Yang, MSc (7)

In collaboration of:

\begin{enumerate}
\def\labelenumi{\arabic{enumi}.}
\tightlist
\item
  Division of Public Administration, Center for Research and Teaching in
  Economics (CIDE), Aguascalientes, Mexico
\item
  University of Minnesota School of Public Health, Minneapolis, MN, USA
\item
  Erasmus MC, Rotterdam, The Netherlands
\item
  Harvard T.H. Chan School of Public Health, Boston, USA
\item
  University of Pittsburgh Graduate School of Public Health, Pittsburgh,
  PA, USA
\item
  University of Toronto, Toronto ON, Canada
\item
  The Hospital for Sick Children, Toronto ON, Canada
\end{enumerate}

Please cite our publications when using this code:

\begin{itemize}
\item
  Jalal H, Pechlivanoglou P, Krijkamp E, Alarid-Escudero F, Enns E,
  Hunink MG. An Overview of R in Health Decision Sciences. Med Decis
  Making. 2017; 37(3): 735-746.
  \url{https://journals.sagepub.com/doi/abs/10.1177/0272989X16686559}
\item
  Alarid-Escudero F, Krijkamp EM, Enns EA, Yang A, Hunink MGM
  Pechlivanoglou P, Jalal H. Cohort State-Transition Models in R: A
  Tutorial. arXiv:200107824v2. 2020:1-48.
  \url{http://arxiv.org/abs/2001.07824}
\item
  Krijkamp EM, Alarid-Escudero F, Enns EA, Jalal HJ, Hunink MGM,
  Pechlivanoglou P. Microsimulation modeling for health decision
  sciences using R: A tutorial. Med Decis Making. 2018;38(3):400--22.
  \url{https://journals.sagepub.com/doi/abs/10.1177/0272989X18754513}
\item
  Krijkamp EM, Alarid-Escudero F, Enns E, Pechlivanoglou P, Hunink MM,
  Jalal H. A Multidimensional Array Representation of State-Transition
  Model Dynamics. Med Decis Making. Online First
  \url{https://doi.org/10.1177/0272989X19893973}
\end{itemize}

Copyright 2017, THE HOSPITAL FOR SICK CHILDREN AND THE COLLABORATING
INSTITUTIONS. All rights reserved in Canada, the United States and
worldwide. Copyright, trademarks, trade names and any and all associated
intellectual property are exclusively owned by THE HOSPITAL FOR Sick
CHILDREN and the collaborating institutions. These materials may be
used, reproduced, modified, distributed and adapted with proper
attribution.

\newpage

Change \texttt{eval} to \texttt{TRUE} if you want to knit this document.

\begin{Shaded}
\begin{Highlighting}[]
\KeywordTok{rm}\NormalTok{(}\DataTypeTok{list =} \KeywordTok{ls}\NormalTok{())      }\CommentTok{# clear memory (removes all the variables from the workspace)}
\end{Highlighting}
\end{Shaded}

\hypertarget{load-packages}{%
\section{01 Load packages}\label{load-packages}}

\begin{Shaded}
\begin{Highlighting}[]
\ControlFlowTok{if}\NormalTok{ (}\OperatorTok{!}\KeywordTok{require}\NormalTok{(}\StringTok{'pacman'}\NormalTok{)) }\KeywordTok{install.packages}\NormalTok{(}\StringTok{'pacman'}\NormalTok{); }\KeywordTok{library}\NormalTok{(pacman) }\CommentTok{# use this package to conveniently install other packages}
\CommentTok{# load (install if required) packages from CRAN}
\KeywordTok{p_load}\NormalTok{( }\StringTok{"dplyr"}\NormalTok{, }\StringTok{"devtools"}\NormalTok{, }\StringTok{"scales"}\NormalTok{, }\StringTok{"ellipse"}\NormalTok{, }\StringTok{"ggplot2"}\NormalTok{, }\StringTok{"lazyeval"}\NormalTok{, }\StringTok{"igraph"}\NormalTok{, }\StringTok{"ggraph"}\NormalTok{, }\StringTok{"reshape2"}\NormalTok{, }\StringTok{"knitr"}\NormalTok{, }\StringTok{"stringr"}\NormalTok{, }\StringTok{"diagram"}\NormalTok{)   }
\CommentTok{# load (install if required) packages from GitHub}
\KeywordTok{p_load_gh}\NormalTok{(}\StringTok{"DARTH-git/dampack"}\NormalTok{, }\StringTok{"DARTH-git/darthtools"}\NormalTok{)}
\end{Highlighting}
\end{Shaded}

\hypertarget{load-functions}{%
\section{02 Load functions}\label{load-functions}}

\begin{Shaded}
\begin{Highlighting}[]
\CommentTok{# all functions are in the darthtools package}
\end{Highlighting}
\end{Shaded}

\hypertarget{input-model-parameters}{%
\section{03 Input model parameters}\label{input-model-parameters}}

\begin{Shaded}
\begin{Highlighting}[]
\CommentTok{# Strategy names}
\NormalTok{v_names_str <-}\StringTok{ }\KeywordTok{c}\NormalTok{(}\StringTok{"Standard of Care"}\NormalTok{, }\StringTok{"Treatment"}\NormalTok{)  }

\CommentTok{# Number of strategies}
\NormalTok{n_str <-}\StringTok{ }\KeywordTok{length}\NormalTok{(v_names_str)}

\CommentTok{# Markov model parameters}
\NormalTok{v_n  <-}\StringTok{ }\KeywordTok{c}\NormalTok{(}\StringTok{"Healthy"}\NormalTok{, }\StringTok{"Sick"}\NormalTok{, }\StringTok{"Dead"}\NormalTok{)  }\CommentTok{# state names}
\NormalTok{n_states  <-}\StringTok{ }\KeywordTok{length}\NormalTok{(v_n)              }\CommentTok{# number of states}
\NormalTok{n_t  <-}\StringTok{ }\DecValTok{60}                            \CommentTok{# number of cycles}
\NormalTok{v_init <-}\StringTok{ }\KeywordTok{c}\NormalTok{(}\DecValTok{1}\NormalTok{, }\DecValTok{0}\NormalTok{, }\DecValTok{0}\NormalTok{)                  }\CommentTok{# initial cohort distribution (everyone allocated to the "healthy" state)}

\CommentTok{# Transition probabilities}
\NormalTok{p_HD <-}\StringTok{ }\FloatTok{0.02}                          \CommentTok{# probability of dying when healthy}
\NormalTok{p_HS <-}\StringTok{ }\FloatTok{0.05}                          \CommentTok{# probability of becoming sick when healthy, under standard of care}
\NormalTok{p_HS_trt <-}\StringTok{ }\FloatTok{0.03}                      \CommentTok{# probability of becoming sick when healthy, under treatment}
\NormalTok{p_SD <-}\StringTok{ }\FloatTok{0.1}                           \CommentTok{# probability of dying when sick}

\CommentTok{# Costs and utilities  }
\NormalTok{c_H   <-}\StringTok{ }\DecValTok{400}                          \CommentTok{# cost of one cycle in healthy state}
\NormalTok{c_S   <-}\StringTok{ }\DecValTok{1000}                         \CommentTok{# cost of one cycle in sick state}
\NormalTok{c_D   <-}\StringTok{ }\DecValTok{0}                            \CommentTok{# cost of one cycle in dead state}
\NormalTok{c_trt <-}\StringTok{ }\DecValTok{800}                          \CommentTok{# cost of treatment (per cycle)}
\NormalTok{u_H   <-}\StringTok{ }\FloatTok{0.8}                          \CommentTok{# utility when healthy }
\NormalTok{u_S   <-}\StringTok{ }\FloatTok{0.5}                          \CommentTok{# utility when sick}
\NormalTok{u_D   <-}\StringTok{ }\DecValTok{0}                            \CommentTok{# utility when dead}
\NormalTok{d_e   <-}\StringTok{ }\NormalTok{d_c <-}\StringTok{ }\FloatTok{0.03}                  \CommentTok{# discount rate per cycle equal discount of costs and QALYs by 3%}

\CommentTok{# calculate discount weights for costs for each cycle based on discount rate d_c}
\NormalTok{v_dwc <-}\StringTok{ }\DecValTok{1} \OperatorTok{/}\StringTok{ }\NormalTok{(}\DecValTok{1} \OperatorTok{+}\StringTok{ }\NormalTok{d_e) }\OperatorTok{^}\StringTok{ }\NormalTok{(}\DecValTok{0}\OperatorTok{:}\NormalTok{n_t) }
\CommentTok{# calculate discount weights for effectiveness for each cycle based on discount rate d_e}
\NormalTok{v_dwe <-}\StringTok{ }\DecValTok{1} \OperatorTok{/}\StringTok{ }\NormalTok{(}\DecValTok{1} \OperatorTok{+}\StringTok{ }\NormalTok{d_c) }\OperatorTok{^}\StringTok{ }\NormalTok{(}\DecValTok{0}\OperatorTok{:}\NormalTok{n_t) }
\end{Highlighting}
\end{Shaded}

\hypertarget{draw-the-state-transition-cohort-model}{%
\subsection{Draw the state-transition cohort
model}\label{draw-the-state-transition-cohort-model}}

\begin{Shaded}
\begin{Highlighting}[]
\NormalTok{m_P_diag <-}\StringTok{ }\KeywordTok{matrix}\NormalTok{(}\DecValTok{0}\NormalTok{, }\DataTypeTok{nrow =}\NormalTok{ n_states, }\DataTypeTok{ncol =}\NormalTok{ n_states, }\DataTypeTok{dimnames =} \KeywordTok{list}\NormalTok{(v_n, v_n))}
\NormalTok{m_P_diag[}\StringTok{"Healthy"}\NormalTok{, }\StringTok{"Sick"}\NormalTok{ ]     =}\StringTok{ ""} 
\NormalTok{m_P_diag[}\StringTok{"Healthy"}\NormalTok{, }\StringTok{"Dead"}\NormalTok{ ]     =}\StringTok{ ""}
\NormalTok{m_P_diag[}\StringTok{"Healthy"}\NormalTok{, }\StringTok{"Healthy"}\NormalTok{ ]  =}\StringTok{ ""}
\NormalTok{m_P_diag[}\StringTok{"Sick"}\NormalTok{   , }\StringTok{"Dead"}\NormalTok{ ]     =}\StringTok{ ""}
\NormalTok{m_P_diag[}\StringTok{"Sick"}\NormalTok{   , }\StringTok{"Sick"}\NormalTok{ ]     =}\StringTok{ ""}
\NormalTok{m_P_diag[}\StringTok{"Dead"}\NormalTok{   , }\StringTok{"Dead"}\NormalTok{ ]     =}\StringTok{ ""}
\NormalTok{layout.fig <-}\StringTok{ }\KeywordTok{c}\NormalTok{(}\DecValTok{2}\NormalTok{, }\DecValTok{1}\NormalTok{)}
\KeywordTok{plotmat}\NormalTok{(}\KeywordTok{t}\NormalTok{(m_P_diag), }\KeywordTok{t}\NormalTok{(layout.fig), }\DataTypeTok{self.cex =} \FloatTok{0.5}\NormalTok{, }\DataTypeTok{curve =} \DecValTok{0}\NormalTok{, }\DataTypeTok{arr.pos =} \FloatTok{0.8}\NormalTok{,  }
        \DataTypeTok{latex =}\NormalTok{ T, }\DataTypeTok{arr.type =} \StringTok{"curved"}\NormalTok{, }\DataTypeTok{relsize =} \FloatTok{0.85}\NormalTok{, }\DataTypeTok{box.prop =} \FloatTok{0.8}\NormalTok{, }
        \DataTypeTok{cex =} \FloatTok{0.8}\NormalTok{, }\DataTypeTok{box.cex =} \FloatTok{0.7}\NormalTok{, }\DataTypeTok{lwd =} \DecValTok{1}\NormalTok{)}
\end{Highlighting}
\end{Shaded}

\hypertarget{define-and-initialize-matrices-and-vectors}{%
\section{04 Define and initialize matrices and
vectors}\label{define-and-initialize-matrices-and-vectors}}

\hypertarget{cohort-trace}{%
\subsection{04.1 Cohort trace}\label{cohort-trace}}

\begin{Shaded}
\begin{Highlighting}[]
\CommentTok{# create the cohort trace}
\NormalTok{m_M <-}\StringTok{ }\NormalTok{m_M_trt <-}\StringTok{  }\KeywordTok{matrix}\NormalTok{(}\OtherTok{NA}\NormalTok{, }
                         \DataTypeTok{nrow =}\NormalTok{ n_t }\OperatorTok{+}\StringTok{ }\DecValTok{1}\NormalTok{ ,  }\CommentTok{# create Markov trace (n.t + 1 because R doesn't }
                                           \CommentTok{# understand Cycle 0)}
                         \DataTypeTok{ncol =}\NormalTok{ n_states, }
                        \DataTypeTok{dimnames =} \KeywordTok{list}\NormalTok{(}\DecValTok{0}\OperatorTok{:}\NormalTok{n_t, v_n))}

\NormalTok{m_M[}\DecValTok{1}\NormalTok{, ] <-}\StringTok{ }\NormalTok{m_M_trt[}\DecValTok{1}\NormalTok{, ] <-}\StringTok{ }\NormalTok{v_init         }\CommentTok{# initialize first cycle of Markov trace}
\end{Highlighting}
\end{Shaded}

\hypertarget{transition-probability-matrix}{%
\subsection{04.2 Transition probability
matrix}\label{transition-probability-matrix}}

\begin{Shaded}
\begin{Highlighting}[]
\CommentTok{# create the transition probability matrices}
\NormalTok{m_P  <-}\StringTok{ }\NormalTok{m_P_trt <-}\StringTok{ }\KeywordTok{matrix}\NormalTok{(}\DecValTok{0}\NormalTok{,}
                    \DataTypeTok{nrow =}\NormalTok{ n_states, }\DataTypeTok{ncol =}\NormalTok{ n_states,}
                    \DataTypeTok{dimnames =} \KeywordTok{list}\NormalTok{(v_n, v_n))  }\CommentTok{# name the columns and rows of the transition }
                                                \CommentTok{# probability matrices}
\NormalTok{m_P}
\end{Highlighting}
\end{Shaded}

Fill in the transition probability matrix:

\begin{Shaded}
\begin{Highlighting}[]
\CommentTok{# from Healthy}
\NormalTok{m_P[}\StringTok{"Healthy"}\NormalTok{, }\StringTok{"Healthy"}\NormalTok{] <-}\StringTok{ }\DecValTok{1} \OperatorTok{-}\StringTok{ }\NormalTok{p_HS }\OperatorTok{-}\StringTok{ }\NormalTok{p_HD}
\NormalTok{m_P[}\StringTok{"Healthy"}\NormalTok{, }\StringTok{"Sick"}\NormalTok{]    <-}\StringTok{ }\NormalTok{p_HS}
\NormalTok{m_P[}\StringTok{"Healthy"}\NormalTok{, }\StringTok{"Dead"}\NormalTok{]    <-}\StringTok{ }\NormalTok{p_HD}

\CommentTok{# from Sick}
\NormalTok{m_P[}\StringTok{"Sick"}\NormalTok{, }\StringTok{"Sick"}\NormalTok{] <-}\StringTok{ }\DecValTok{1} \OperatorTok{-}\StringTok{ }\NormalTok{p_SD}
\NormalTok{m_P[}\StringTok{"Sick"}\NormalTok{, }\StringTok{"Dead"}\NormalTok{] <-}\StringTok{ }\NormalTok{p_SD}

\CommentTok{# from Dead}
\NormalTok{m_P[}\StringTok{"Dead"}\NormalTok{, }\StringTok{"Dead"}\NormalTok{] <-}\StringTok{ }\DecValTok{1}

\CommentTok{# Under treatment}
\NormalTok{m_P_trt <-}\StringTok{ }\NormalTok{m_P}
\NormalTok{m_P_trt[}\StringTok{"Healthy"}\NormalTok{, }\StringTok{"Healthy"}\NormalTok{] <-}\StringTok{ }\DecValTok{1} \OperatorTok{-}\StringTok{ }\NormalTok{p_HS_trt }\OperatorTok{-}\StringTok{ }\NormalTok{p_HD}
\NormalTok{m_P_trt[}\StringTok{"Healthy"}\NormalTok{, }\StringTok{"Sick"}\NormalTok{]    <-}\StringTok{ }\NormalTok{p_HS_trt}
\end{Highlighting}
\end{Shaded}

\hypertarget{run-markov-model}{%
\section{05 Run Markov model}\label{run-markov-model}}

\begin{Shaded}
\begin{Highlighting}[]
\ControlFlowTok{for}\NormalTok{ (t }\ControlFlowTok{in} \DecValTok{1}\OperatorTok{:}\NormalTok{n_t)\{                               }\CommentTok{# loop through the number of cycles}
\NormalTok{  m_M[t }\OperatorTok{+}\StringTok{ }\DecValTok{1}\NormalTok{, ]     <-}\StringTok{ }\NormalTok{m_M[t, ]     }\OperatorTok\StringTok{ }\NormalTok{m_P      }\CommentTok{# estimate the state vector for the next cycle (t + 1)}
\NormalTok{  m_M_trt[t }\OperatorTok{+}\StringTok{ }\DecValTok{1}\NormalTok{, ] <-}\StringTok{ }\NormalTok{m_M_trt[t, ] }\OperatorTok\StringTok{ }\NormalTok{m_P_trt  }\CommentTok{# for treatment}
\NormalTok{\}}
\end{Highlighting}
\end{Shaded}

\hypertarget{compute-and-plot-epidemiological-outcomes}{%
\section{06 Compute and Plot Epidemiological
Outcomes}\label{compute-and-plot-epidemiological-outcomes}}

\hypertarget{cohort-trace-1}{%
\subsection{06.1 Cohort trace}\label{cohort-trace-1}}

Standard of Care:

\begin{Shaded}
\begin{Highlighting}[]
\KeywordTok{matplot}\NormalTok{(m_M, }\DataTypeTok{type =} \StringTok{'l'}\NormalTok{, }
        \DataTypeTok{ylab =} \StringTok{"Probability of state occupancy"}\NormalTok{,}
        \DataTypeTok{xlab =} \StringTok{"Cycle"}\NormalTok{,}
        \DataTypeTok{main =} \StringTok{"Cohort Trace - standard of care"}\NormalTok{, }\DataTypeTok{lwd =} \DecValTok{3}\NormalTok{)  }\CommentTok{# create a plot of the data}
\KeywordTok{legend}\NormalTok{(}\StringTok{"right"}\NormalTok{, v_n, }\DataTypeTok{col =} \KeywordTok{c}\NormalTok{(}\StringTok{"black"}\NormalTok{, }\StringTok{"red"}\NormalTok{, }\StringTok{"green"}\NormalTok{), }
       \DataTypeTok{lty =} \DecValTok{1}\OperatorTok{:}\DecValTok{3}\NormalTok{, }\DataTypeTok{bty =} \StringTok{"n"}\NormalTok{)                            }\CommentTok{# add a legend to the graph}

\KeywordTok{abline}\NormalTok{(}\DataTypeTok{v =} \KeywordTok{which.max}\NormalTok{(m_M[, }\StringTok{"Sick"}\NormalTok{]), }\DataTypeTok{col =} \StringTok{"gray"}\NormalTok{)      }\CommentTok{# plot a vertical line that helps identifying at which cycle the prevalence of sick is highest.  }
\end{Highlighting}
\end{Shaded}

Treatment:

\begin{Shaded}
\begin{Highlighting}[]
\KeywordTok{matplot}\NormalTok{(m_M_trt, }\DataTypeTok{type =} \StringTok{'l'}\NormalTok{, }
        \DataTypeTok{ylab =} \StringTok{"Probability of state occupancy"}\NormalTok{,}
        \DataTypeTok{xlab =} \StringTok{"Cycle"}\NormalTok{,}
        \DataTypeTok{main =} \StringTok{"Cohort Trace - treatment"}\NormalTok{, }\DataTypeTok{lwd =} \DecValTok{3}\NormalTok{)     }\CommentTok{# create a plot of the data}
\KeywordTok{legend}\NormalTok{(}\StringTok{"right"}\NormalTok{, v_n, }\DataTypeTok{col =} \KeywordTok{c}\NormalTok{(}\StringTok{"black"}\NormalTok{, }\StringTok{"red"}\NormalTok{, }\StringTok{"green"}\NormalTok{), }
       \DataTypeTok{lty =} \DecValTok{1}\OperatorTok{:}\DecValTok{3}\NormalTok{, }\DataTypeTok{bty =} \StringTok{"n"}\NormalTok{)                            }\CommentTok{# add a legend to the graph}

\KeywordTok{abline}\NormalTok{(}\DataTypeTok{v =} \KeywordTok{which.max}\NormalTok{(m_M[, }\StringTok{"Sick"}\NormalTok{]), }\DataTypeTok{col =} \StringTok{"gray"}\NormalTok{)      }\CommentTok{# plot a vertical line that helps identifying at which cycle the prevalence of sick is highest.  }
\end{Highlighting}
\end{Shaded}

\hypertarget{overall-survival-os}{%
\subsection{06.2 Overall Survival (OS)}\label{overall-survival-os}}

Standard of Care:

\begin{Shaded}
\begin{Highlighting}[]
\NormalTok{v_os <-}\StringTok{ }\DecValTok{1} \OperatorTok{-}\StringTok{ }\NormalTok{m_M[, }\StringTok{"Dead"}\NormalTok{]             }\CommentTok{# calculate the overall survival (OS) probability}
\NormalTok{v_os <-}\StringTok{ }\KeywordTok{rowSums}\NormalTok{(m_M[, }\DecValTok{1}\OperatorTok{:}\DecValTok{2}\NormalTok{])           }\CommentTok{# alternative way of calculating the OS probability   }

\KeywordTok{plot}\NormalTok{(v_os, }\DataTypeTok{type =} \StringTok{'l'}\NormalTok{, }
     \DataTypeTok{ylim =} \KeywordTok{c}\NormalTok{(}\DecValTok{0}\NormalTok{, }\DecValTok{1}\NormalTok{),}
     \DataTypeTok{ylab =} \StringTok{"Survival probability"}\NormalTok{,}
     \DataTypeTok{xlab =} \StringTok{"Cycle"}\NormalTok{,}
     \DataTypeTok{main =} \StringTok{"Overall Survival"}\NormalTok{)       }\CommentTok{# create a simple plot showing the OS}

\CommentTok{# add grid }
\KeywordTok{grid}\NormalTok{(}\DataTypeTok{nx =}\NormalTok{ n_t, }\DataTypeTok{ny =} \DecValTok{10}\NormalTok{, }\DataTypeTok{col =} \StringTok{"lightgray"}\NormalTok{, }\DataTypeTok{lty =} \StringTok{"dotted"}\NormalTok{, }\DataTypeTok{lwd =} \KeywordTok{par}\NormalTok{(}\StringTok{"lwd"}\NormalTok{), }
     \DataTypeTok{equilogs =} \OtherTok{TRUE}\NormalTok{) }
\end{Highlighting}
\end{Shaded}

Treatment:

\begin{Shaded}
\begin{Highlighting}[]
\NormalTok{v_os_trt <-}\StringTok{ }\DecValTok{1} \OperatorTok{-}\StringTok{ }\NormalTok{m_M_trt[, }\StringTok{"Dead"}\NormalTok{]     }\CommentTok{# calculate the overall survival (OS) probability}
\NormalTok{v_os_trt <-}\StringTok{ }\KeywordTok{rowSums}\NormalTok{(m_M_trt[, }\DecValTok{1}\OperatorTok{:}\DecValTok{2}\NormalTok{])   }\CommentTok{# alternative way of calculating the OS probability}

\KeywordTok{plot}\NormalTok{(v_os_trt, }\DataTypeTok{type =} \StringTok{'l'}\NormalTok{, }
     \DataTypeTok{ylim =} \KeywordTok{c}\NormalTok{(}\DecValTok{0}\NormalTok{, }\DecValTok{1}\NormalTok{),}
     \DataTypeTok{ylab =} \StringTok{"Survival probability"}\NormalTok{,}
     \DataTypeTok{xlab =} \StringTok{"Cycle"}\NormalTok{,}
     \DataTypeTok{main =} \StringTok{"Overall Survival"}\NormalTok{)       }\CommentTok{# create a simple plot showing the OS}

\CommentTok{# add grid }
\KeywordTok{grid}\NormalTok{(}\DataTypeTok{nx =}\NormalTok{ n_t, }\DataTypeTok{ny =} \DecValTok{10}\NormalTok{, }\DataTypeTok{col =} \StringTok{"lightgray"}\NormalTok{, }\DataTypeTok{lty =} \StringTok{"dotted"}\NormalTok{, }\DataTypeTok{lwd =} \KeywordTok{par}\NormalTok{(}\StringTok{"lwd"}\NormalTok{), }
     \DataTypeTok{equilogs =} \OtherTok{TRUE}\NormalTok{) }
\end{Highlighting}
\end{Shaded}

\hypertarget{life-expectancy-le}{%
\subsection{06.2.1 Life Expectancy (LE)}\label{life-expectancy-le}}

\begin{Shaded}
\begin{Highlighting}[]
\NormalTok{v_le     <-}\StringTok{ }\KeywordTok{sum}\NormalTok{(v_os)      }\CommentTok{# summing probability of OS over time (i.e. life expectancy)}
\NormalTok{v_le_trt <-}\StringTok{ }\KeywordTok{sum}\NormalTok{(v_os_trt)  }\CommentTok{# summing probability of OS over time (i.e. life expectancy), treatment}
\end{Highlighting}
\end{Shaded}

\hypertarget{disease-prevalence}{%
\subsection{06.3 Disease prevalence}\label{disease-prevalence}}

Standard of Care:

\begin{Shaded}
\begin{Highlighting}[]
\NormalTok{v_prev <-}\StringTok{ }\NormalTok{m_M[, }\StringTok{"Sick"}\NormalTok{]}\OperatorTok{/}\NormalTok{v_os}
\KeywordTok{plot}\NormalTok{(v_prev,}
     \DataTypeTok{ylim =} \KeywordTok{c}\NormalTok{(}\DecValTok{0}\NormalTok{, }\DecValTok{1}\NormalTok{),}
     \DataTypeTok{ylab =} \StringTok{"Prevalence"}\NormalTok{,}
     \DataTypeTok{xlab =} \StringTok{"Cycle"}\NormalTok{,}
     \DataTypeTok{main =} \StringTok{"Disease prevalence"}\NormalTok{)}
\end{Highlighting}
\end{Shaded}

Treatment:

\begin{Shaded}
\begin{Highlighting}[]
\NormalTok{v_prev_trt <-}\StringTok{ }\NormalTok{m_M_trt[, }\StringTok{"Sick"}\NormalTok{]}\OperatorTok{/}\NormalTok{v_os_trt}
\KeywordTok{plot}\NormalTok{(v_prev_trt,}
     \DataTypeTok{ylim =} \KeywordTok{c}\NormalTok{(}\DecValTok{0}\NormalTok{, }\DecValTok{1}\NormalTok{),}
     \DataTypeTok{ylab =} \StringTok{"Prevalence"}\NormalTok{,}
     \DataTypeTok{xlab =} \StringTok{"Cycle"}\NormalTok{,}
     \DataTypeTok{main =} \StringTok{"Disease prevalence"}\NormalTok{)}
\end{Highlighting}
\end{Shaded}

\hypertarget{compute-cost-effectiveness-outcomes}{%
\section{07 Compute Cost-Effectiveness
Outcomes}\label{compute-cost-effectiveness-outcomes}}

\hypertarget{mean-costs-and-qalys}{%
\subsection{07.1 Mean Costs and QALYs}\label{mean-costs-and-qalys}}

\begin{Shaded}
\begin{Highlighting}[]
\CommentTok{# per cycle}
\CommentTok{# calculate expected costs by multiplying m_M with the cost vector for the different }
\CommentTok{# health states   }
\NormalTok{v_tc     <-}\StringTok{ }\NormalTok{m_M     }\OperatorTok\StringTok{ }\KeywordTok{c}\NormalTok{(c_H, c_S, c_D)          }\CommentTok{# Standard of Care}
\NormalTok{v_tc_trt <-}\StringTok{ }\NormalTok{m_M_trt }\OperatorTok\StringTok{ }\KeywordTok{c}\NormalTok{(c_H, c_S }\OperatorTok{+}\StringTok{ }\NormalTok{c_trt, c_D)  }\CommentTok{# Treatment}
\CommentTok{# calculate expected QALYs  by multiplying m_M with the utilities for the different }
\CommentTok{# health states   }
\NormalTok{v_tu     <-}\StringTok{ }\NormalTok{m_M     }\OperatorTok\StringTok{ }\KeywordTok{c}\NormalTok{(u_H, u_S, u_D)          }\CommentTok{# Standard of Care}
\NormalTok{v_tu_trt <-}\StringTok{ }\NormalTok{m_M_trt }\OperatorTok\StringTok{ }\KeywordTok{c}\NormalTok{(u_H, u_S, u_D)          }\CommentTok{# Treatment}
\end{Highlighting}
\end{Shaded}

\hypertarget{discounted-mean-costs-and-qalys}{%
\subsection{07.2 Discounted Mean Costs and
QALYs}\label{discounted-mean-costs-and-qalys}}

\begin{Shaded}
\begin{Highlighting}[]
\CommentTok{# Discount costs by multiplying the cost vector with discount weights  }
\NormalTok{tc_d     <-}\StringTok{  }\KeywordTok{t}\NormalTok{(v_tc)     }\OperatorTok\StringTok{ }\NormalTok{v_dwc      }\CommentTok{# Standard of Care}
\NormalTok{tc_d_trt <-}\StringTok{  }\KeywordTok{t}\NormalTok{(v_tc_trt) }\OperatorTok\StringTok{ }\NormalTok{v_dwc      }\CommentTok{# Treatment}
\CommentTok{# Discount QALYS by multiplying the QALYs vector with discount weights }
\NormalTok{tu_d     <-}\StringTok{  }\KeywordTok{t}\NormalTok{(v_tu)     }\OperatorTok\StringTok{ }\NormalTok{v_dwe      }\CommentTok{# Standard of Care}
\NormalTok{tu_d_trt <-}\StringTok{  }\KeywordTok{t}\NormalTok{(v_tu_trt) }\OperatorTok\StringTok{ }\NormalTok{v_dwe      }\CommentTok{# Treatment}

\CommentTok{# store them into a vector}
\NormalTok{v_tc_d   <-}\StringTok{ }\KeywordTok{c}\NormalTok{(tc_d, tc_d_trt)}
\NormalTok{v_tu_d   <-}\StringTok{ }\KeywordTok{c}\NormalTok{(tu_d, tu_d_trt)}

\CommentTok{# Dataframe with discounted costs and effectiveness}
\NormalTok{df_ce       <-}\StringTok{ }\KeywordTok{data.frame}\NormalTok{(}\DataTypeTok{Strategy =}\NormalTok{ v_names_str,}
                          \DataTypeTok{Cost     =}\NormalTok{ v_tc_d,}
                          \DataTypeTok{Effect   =}\NormalTok{ v_tu_d}
\NormalTok{                          )}
\NormalTok{df_ce}
\end{Highlighting}
\end{Shaded}

\hypertarget{compute-icers-of-the-markov-model}{%
\subsection{07.3 Compute ICERs of the Markov
model}\label{compute-icers-of-the-markov-model}}

\begin{Shaded}
\begin{Highlighting}[]
\NormalTok{df_cea <-}\StringTok{ }\KeywordTok{calculate_icers}\NormalTok{(}\DataTypeTok{cost       =}\NormalTok{ df_ce}\OperatorTok{$}\NormalTok{Cost,}
                          \DataTypeTok{effect     =}\NormalTok{ df_ce}\OperatorTok{$}\NormalTok{Effect,}
                          \DataTypeTok{strategies =}\NormalTok{ df_ce}\OperatorTok{$}\NormalTok{Strategy}
\NormalTok{                          )}
\NormalTok{df_cea}
\end{Highlighting}
\end{Shaded}

\hypertarget{plot-frontier-of-the-markov-model}{%
\subsection{07.4 Plot frontier of the Markov
model}\label{plot-frontier-of-the-markov-model}}

\begin{Shaded}
\begin{Highlighting}[]
\KeywordTok{plot}\NormalTok{(df_cea, }\DataTypeTok{effect_units =} \StringTok{"QALYs"}\NormalTok{, }\DataTypeTok{xlim =} \KeywordTok{c}\NormalTok{(}\DecValTok{10}\NormalTok{, }\DecValTok{12}\NormalTok{))}
\end{Highlighting}
\end{Shaded}

\hypertarget{probabilistic-sensitivity-analysis}{%
\section{08 Probabilistic Sensitivity
Analysis}\label{probabilistic-sensitivity-analysis}}

\hypertarget{list-of-input-parameters}{%
\subsection{08.1 List of input
parameters}\label{list-of-input-parameters}}

Create list \texttt{l\_params\_all} with all input probabilities, cost
and utilities.

\begin{Shaded}
\begin{Highlighting}[]
\NormalTok{l_params_all <-}\StringTok{ }\KeywordTok{as.list}\NormalTok{(}\KeywordTok{data.frame}\NormalTok{(}
  \DataTypeTok{p_HD     =} \FloatTok{0.02}\NormalTok{,  }\CommentTok{# probability of dying when healthy}
  \DataTypeTok{p_HS     =} \FloatTok{0.05}\NormalTok{,  }\CommentTok{# probability of becoming sick when healthy, conditioned on not dying}
  \DataTypeTok{p_HS_trt =} \FloatTok{0.03}\NormalTok{,  }\CommentTok{# probability of becoming sick when healthy, conditioned on not dying}
  \DataTypeTok{p_SD     =} \FloatTok{0.1}\NormalTok{,   }\CommentTok{# probability of dying when sick}
  \DataTypeTok{c_H      =} \DecValTok{400}\NormalTok{,   }\CommentTok{# cost of one cycle in healthy state}
  \DataTypeTok{c_S      =} \DecValTok{1000}\NormalTok{,  }\CommentTok{# cost of one cycle in sick state}
  \DataTypeTok{c_D      =} \DecValTok{0}\NormalTok{,     }\CommentTok{# cost of one cycle in dead state}
  \DataTypeTok{c_trt    =} \DecValTok{800}\NormalTok{,   }\CommentTok{# cost of treatment (per cycle)}
  \DataTypeTok{u_H      =} \FloatTok{0.8}\NormalTok{,   }\CommentTok{# utility when healthy }
  \DataTypeTok{u_S      =} \FloatTok{0.5}\NormalTok{,   }\CommentTok{# utility when sick}
  \DataTypeTok{u_D      =} \DecValTok{0}\NormalTok{,     }\CommentTok{# utility when dead}
  \DataTypeTok{d_e      =} \FloatTok{0.03}\NormalTok{,  }\CommentTok{# discount factor for effectiveness}
  \DataTypeTok{d_c      =} \FloatTok{0.03}   \CommentTok{# discount factor for costs}
\NormalTok{))}

\CommentTok{# store the parameter names into a vector}
\NormalTok{v_names_params <-}\StringTok{ }\KeywordTok{names}\NormalTok{(l_params_all)}
\end{Highlighting}
\end{Shaded}

\hypertarget{load-sick-sicker-markov-model-function}{%
\subsection{08.2 Load Sick-Sicker Markov model
function}\label{load-sick-sicker-markov-model-function}}

\begin{Shaded}
\begin{Highlighting}[]
\KeywordTok{source}\NormalTok{(}\StringTok{"Functions_markov_3state.R"}\NormalTok{)}
\CommentTok{# Test function}
\KeywordTok{calculate_ce_out}\NormalTok{(l_params_all)}
\end{Highlighting}
\end{Shaded}

\hypertarget{generate-psa-datasets}{%
\subsection{08.3 Generate PSA datasets}\label{generate-psa-datasets}}

\begin{Shaded}
\begin{Highlighting}[]
\CommentTok{# Function to generate PSA input dataset}
\NormalTok{gen_psa <-}\StringTok{ }\ControlFlowTok{function}\NormalTok{(}\DataTypeTok{n_sim =} \DecValTok{1000}\NormalTok{, }\DataTypeTok{seed =} \DecValTok{071818}\NormalTok{)\{}
  \KeywordTok{set.seed}\NormalTok{(seed) }\CommentTok{# set a seed to be able to reproduce the same results}
\NormalTok{  df_psa <-}\StringTok{ }\KeywordTok{data.frame}\NormalTok{(}
    \CommentTok{# Transition probabilities (per cycle)}
    \CommentTok{# probability to become sick when healthy}
    \DataTypeTok{p_HS     =} \KeywordTok{rbeta}\NormalTok{(n_sim, }\DataTypeTok{shape1 =} \DecValTok{24}\NormalTok{, }\DataTypeTok{shape2 =} \DecValTok{450}\NormalTok{), }
    \DataTypeTok{p_HS_trt =} \KeywordTok{rbeta}\NormalTok{(n_sim, }\DataTypeTok{shape1 =} \DecValTok{9}\NormalTok{,  }\DataTypeTok{shape2 =} \DecValTok{281}\NormalTok{),   }\CommentTok{# under treatment}
    \CommentTok{# probability of dying when healthy}
    \DataTypeTok{p_HD     =} \KeywordTok{rbeta}\NormalTok{(n_sim, }\DataTypeTok{shape1 =} \DecValTok{16}\NormalTok{, }\DataTypeTok{shape2 =} \DecValTok{767}\NormalTok{),}
    \CommentTok{# probability of dying when sick}
    \DataTypeTok{p_SD     =} \KeywordTok{rbeta}\NormalTok{(n_sim, }\DataTypeTok{shape1 =} \FloatTok{22.4}\NormalTok{, }\DataTypeTok{shape2 =} \FloatTok{201.6}\NormalTok{), }

    \CommentTok{# Cost vectors with length n_sim}
    \CommentTok{# cost of remaining one cycle in state H}
    \DataTypeTok{c_H      =} \KeywordTok{rgamma}\NormalTok{(n_sim, }\DataTypeTok{shape =} \DecValTok{16}\NormalTok{, }\DataTypeTok{scale =} \DecValTok{25}\NormalTok{), }
    \CommentTok{# cost of remaining one cycle in state S1}
    \DataTypeTok{c_S      =} \KeywordTok{rgamma}\NormalTok{(n_sim, }\DataTypeTok{shape =} \DecValTok{100}\NormalTok{, }\DataTypeTok{scale =} \DecValTok{10}\NormalTok{), }
    \CommentTok{# cost of being in the death state}
    \DataTypeTok{c_D      =} \DecValTok{0}\NormalTok{, }
    \CommentTok{# cost of treatment (per cycle)}
    \DataTypeTok{c_trt    =} \KeywordTok{rgamma}\NormalTok{(n_sim, }\DataTypeTok{shape =} \DecValTok{64}\NormalTok{, }\DataTypeTok{scale =} \FloatTok{12.5}\NormalTok{),}
    
    \CommentTok{# Utility vectors with length n_sim }
    \CommentTok{# utility when healthy}
    \DataTypeTok{u_H      =} \KeywordTok{rbeta}\NormalTok{(n_sim, }\DataTypeTok{shape1 =}  \FloatTok{50.4}\NormalTok{, }\DataTypeTok{shape2 =} \FloatTok{12.6}\NormalTok{), }
    \CommentTok{# utility when sick}
    \DataTypeTok{u_S      =} \KeywordTok{rbeta}\NormalTok{(n_sim, }\DataTypeTok{shape1 =} \FloatTok{49.5}\NormalTok{, }\DataTypeTok{shape2 =} \FloatTok{49.5}\NormalTok{), }
    \CommentTok{# utility when dead}
    \DataTypeTok{u_D      =} \DecValTok{0}                                              
\NormalTok{  )}
  \KeywordTok{return}\NormalTok{(df_psa)}
\NormalTok{\}}
\CommentTok{# Try it}
\KeywordTok{gen_psa}\NormalTok{(}\DecValTok{10}\NormalTok{) }

\CommentTok{# Number of simulations}
\NormalTok{n_sim <-}\StringTok{ }\DecValTok{1000}

\CommentTok{# Generate PSA input dataset}
\NormalTok{df_psa_input <-}\StringTok{ }\KeywordTok{gen_psa}\NormalTok{(}\DataTypeTok{n_sim =}\NormalTok{ n_sim)}
\CommentTok{# First six observations}
\KeywordTok{head}\NormalTok{(df_psa_input)}

\CommentTok{# Histogram of parameters}
\KeywordTok{ggplot}\NormalTok{(}\KeywordTok{melt}\NormalTok{(df_psa_input, }\DataTypeTok{variable.name =} \StringTok{"Parameter"}\NormalTok{), }\KeywordTok{aes}\NormalTok{(}\DataTypeTok{x =}\NormalTok{ value)) }\OperatorTok{+}
\StringTok{       }\KeywordTok{facet_wrap}\NormalTok{(}\OperatorTok{~}\NormalTok{Parameter, }\DataTypeTok{scales =} \StringTok{"free"}\NormalTok{) }\OperatorTok{+}
\StringTok{       }\KeywordTok{geom_histogram}\NormalTok{(}\KeywordTok{aes}\NormalTok{(}\DataTypeTok{y =}\NormalTok{ ..density..)) }\OperatorTok{+}
\StringTok{       }\KeywordTok{theme_bw}\NormalTok{(}\DataTypeTok{base_size =} \DecValTok{16}\NormalTok{) }\OperatorTok{+}\StringTok{ }
\StringTok{       }\KeywordTok{theme}\NormalTok{(}\DataTypeTok{axis.text =} \KeywordTok{element_text}\NormalTok{(}\DataTypeTok{size=}\DecValTok{8}\NormalTok{))}

\CommentTok{# Initialize dataframes with PSA output }
\CommentTok{# Dataframe of costs}
\NormalTok{df_c <-}\StringTok{ }\KeywordTok{as.data.frame}\NormalTok{(}\KeywordTok{matrix}\NormalTok{(}\DecValTok{0}\NormalTok{, }
                      \DataTypeTok{nrow =}\NormalTok{ n_sim,}
                      \DataTypeTok{ncol =}\NormalTok{ n_str))}
\KeywordTok{colnames}\NormalTok{(df_c) <-}\StringTok{ }\NormalTok{v_names_str}
\CommentTok{# Dataframe of effectiveness}
\NormalTok{df_e <-}\StringTok{ }\KeywordTok{as.data.frame}\NormalTok{(}\KeywordTok{matrix}\NormalTok{(}\DecValTok{0}\NormalTok{, }
                      \DataTypeTok{nrow =}\NormalTok{ n_sim,}
                      \DataTypeTok{ncol =}\NormalTok{ n_str))}
\KeywordTok{colnames}\NormalTok{(df_e) <-}\StringTok{ }\NormalTok{v_names_str}
\end{Highlighting}
\end{Shaded}

\hypertarget{conduct-probabilistic-sensitivity-analysis}{%
\subsection{08.4 Conduct probabilistic sensitivity
analysis}\label{conduct-probabilistic-sensitivity-analysis}}

\begin{Shaded}
\begin{Highlighting}[]
\CommentTok{# Run Markov model on each parameter set of PSA input dataset}
\ControlFlowTok{for}\NormalTok{(i }\ControlFlowTok{in} \DecValTok{1}\OperatorTok{:}\NormalTok{n_sim)\{}
\NormalTok{  l_out_temp <-}\StringTok{ }\KeywordTok{calculate_ce_out}\NormalTok{(df_psa_input[i, ])}
\NormalTok{  df_c[i, ] <-}\StringTok{ }\NormalTok{l_out_temp}\OperatorTok{$}\NormalTok{Cost}
\NormalTok{  df_e[i, ] <-}\StringTok{ }\NormalTok{l_out_temp}\OperatorTok{$}\NormalTok{Effect}
  \CommentTok{# Display simulation progress}
  \ControlFlowTok{if}\NormalTok{(i}\OperatorTok{/}\NormalTok{(n_sim}\OperatorTok{/}\DecValTok{10}\NormalTok{) }\OperatorTok{==}\StringTok{ }\KeywordTok{round}\NormalTok{(i}\OperatorTok{/}\NormalTok{(n_sim}\OperatorTok{/}\DecValTok{10}\NormalTok{), }\DecValTok{0}\NormalTok{)) \{ }\CommentTok{# display progress every 10%}
    \KeywordTok{cat}\NormalTok{(}\StringTok{'}\CharTok{\textbackslash{}r}\StringTok{'}\NormalTok{, }\KeywordTok{paste}\NormalTok{(i}\OperatorTok{/}\NormalTok{n_sim }\OperatorTok{*}\StringTok{ }\DecValTok{100}\NormalTok{, }\StringTok{"% done"}\NormalTok{, }\DataTypeTok{sep =} \StringTok{" "}\NormalTok{))}
\NormalTok{  \}}
\NormalTok{\}}
\end{Highlighting}
\end{Shaded}

\hypertarget{analyze-and-visualize-psa-results-using-r-package-dampack}{%
\subsection{08.5 Analyze and visualize PSA results using R package:
dampack}\label{analyze-and-visualize-psa-results-using-r-package-dampack}}

Create PSA object for \texttt{dampack}

\begin{Shaded}
\begin{Highlighting}[]
\NormalTok{l_psa <-}\StringTok{ }\KeywordTok{make_psa_obj}\NormalTok{(}\DataTypeTok{cost          =}\NormalTok{ df_c, }
                      \DataTypeTok{effectiveness =}\NormalTok{ df_e, }
                      \DataTypeTok{parameters    =}\NormalTok{ df_psa_input, }
                      \DataTypeTok{strategies    =}\NormalTok{ v_names_str)}
\end{Highlighting}
\end{Shaded}

\hypertarget{save-psa-objects}{%
\subsection{08.5.1 Save PSA objects}\label{save-psa-objects}}

\begin{Shaded}
\begin{Highlighting}[]
\KeywordTok{save}\NormalTok{(df_psa_input, df_c, df_e, v_names_str, n_str, l_psa,}
     \DataTypeTok{file =} \StringTok{"markov_3state_PSA_dataset.RData"}\NormalTok{)}
\end{Highlighting}
\end{Shaded}

Vector with willingness-to-pay (WTP) thresholds.

\begin{Shaded}
\begin{Highlighting}[]
\NormalTok{v_wtp <-}\StringTok{ }\KeywordTok{seq}\NormalTok{(}\DecValTok{0}\NormalTok{, }\DecValTok{5000}\NormalTok{, }\DataTypeTok{by =} \DecValTok{1000}\NormalTok{)}
\end{Highlighting}
\end{Shaded}

\hypertarget{cost-effectiveness-scatter-plot}{%
\subsection{08.5.2 Cost-Effectiveness Scatter
plot}\label{cost-effectiveness-scatter-plot}}

\begin{Shaded}
\begin{Highlighting}[]
\KeywordTok{plot}\NormalTok{(l_psa)}
\end{Highlighting}
\end{Shaded}

\hypertarget{conduct-cea-with-probabilistic-output}{%
\subsection{08.5.3 Conduct CEA with probabilistic
output}\label{conduct-cea-with-probabilistic-output}}

\begin{Shaded}
\begin{Highlighting}[]
\CommentTok{# Compute expected costs and effects for each strategy from the PSA}
\NormalTok{df_out_ce_psa <-}\StringTok{ }\KeywordTok{summary}\NormalTok{(l_psa)}

\CommentTok{# Calculate incremental cost-effectiveness ratios (ICERs)}
\NormalTok{df_cea_psa <-}\StringTok{ }\KeywordTok{calculate_icers}\NormalTok{(}\DataTypeTok{cost       =}\NormalTok{ df_out_ce_psa}\OperatorTok{$}\NormalTok{meanCost, }
                              \DataTypeTok{effect     =}\NormalTok{ df_out_ce_psa}\OperatorTok{$}\NormalTok{meanEffect,}
                              \DataTypeTok{strategies =}\NormalTok{ df_out_ce_psa}\OperatorTok{$}\NormalTok{Strategy)}
\NormalTok{df_cea_psa}

\CommentTok{# Save CEA table with ICERs}
\CommentTok{# As .RData}
\KeywordTok{save}\NormalTok{(df_cea_psa, }
     \DataTypeTok{file =} \StringTok{"markov_3state_probabilistic_CEA_results.RData"}\NormalTok{)}
\CommentTok{# As .csv}
\KeywordTok{write.csv}\NormalTok{(df_cea_psa, }
          \DataTypeTok{file =} \StringTok{"markov_3state_probabilistic_CEA_results.csv"}\NormalTok{)}
\end{Highlighting}
\end{Shaded}

\hypertarget{plot-cost-effectiveness-frontier}{%
\subsection{08.5.4 Plot cost-effectiveness
frontier}\label{plot-cost-effectiveness-frontier}}

\begin{Shaded}
\begin{Highlighting}[]
\KeywordTok{plot}\NormalTok{(df_cea_psa)}
\end{Highlighting}
\end{Shaded}

\hypertarget{cost-effectiveness-acceptability-curves-ceacs-and-frontier-ceaf}{%
\subsection{08.5.6 Cost-effectiveness acceptability curves (CEACs) and
frontier
(CEAF)}\label{cost-effectiveness-acceptability-curves-ceacs-and-frontier-ceaf}}

\begin{Shaded}
\begin{Highlighting}[]
\NormalTok{ceac_obj <-}\StringTok{ }\KeywordTok{ceac}\NormalTok{(}\DataTypeTok{wtp =}\NormalTok{ v_wtp, }\DataTypeTok{psa =}\NormalTok{ l_psa)}
\CommentTok{# Regions of highest probability of cost-effectiveness for each strategy}
\KeywordTok{summary}\NormalTok{(ceac_obj)}
\CommentTok{# CEAC & CEAF plot}
\KeywordTok{plot}\NormalTok{(ceac_obj)}
\end{Highlighting}
\end{Shaded}

\hypertarget{expected-loss-curves-elcs}{%
\subsection{08.5.7 Expected Loss Curves
(ELCs)}\label{expected-loss-curves-elcs}}

The expected loss is the the quantification of the foregone benefits
when choosing a suboptimal strategy given current evidence.

\begin{Shaded}
\begin{Highlighting}[]
\NormalTok{elc_obj <-}\StringTok{ }\KeywordTok{calc_exp_loss}\NormalTok{(}\DataTypeTok{wtp =}\NormalTok{ v_wtp, }\DataTypeTok{psa =}\NormalTok{ l_psa)}
\NormalTok{elc_obj}
\CommentTok{# ELC plot}
\KeywordTok{plot}\NormalTok{(elc_obj, }\DataTypeTok{log_y =} \OtherTok{FALSE}\NormalTok{)}
\end{Highlighting}
\end{Shaded}

\hypertarget{expected-value-of-perfect-information-evpi}{%
\subsection{08.5.8 Expected value of perfect information
(EVPI)}\label{expected-value-of-perfect-information-evpi}}

\begin{Shaded}
\begin{Highlighting}[]
\NormalTok{evpi <-}\StringTok{ }\KeywordTok{calc_evpi}\NormalTok{(}\DataTypeTok{wtp =}\NormalTok{ v_wtp, }\DataTypeTok{psa =}\NormalTok{ l_psa)}
\CommentTok{# EVPI plot}
\KeywordTok{plot}\NormalTok{(evpi, }\DataTypeTok{effect_units =} \StringTok{"QALY"}\NormalTok{)}
\end{Highlighting}
\end{Shaded}

\end{document}
