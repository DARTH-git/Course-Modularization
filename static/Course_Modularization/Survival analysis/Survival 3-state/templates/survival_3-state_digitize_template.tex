% Options for packages loaded elsewhere
\PassOptionsToPackage{unicode}{hyperref}
\PassOptionsToPackage{hyphens}{url}
%
\documentclass[
]{article}
\usepackage{lmodern}
\usepackage{amssymb,amsmath}
\usepackage{ifxetex,ifluatex}
\ifnum 0\ifxetex 1\fi\ifluatex 1\fi=0 % if pdftex
  \usepackage[T1]{fontenc}
  \usepackage[utf8]{inputenc}
  \usepackage{textcomp} % provide euro and other symbols
\else % if luatex or xetex
  \usepackage{unicode-math}
  \defaultfontfeatures{Scale=MatchLowercase}
  \defaultfontfeatures[\rmfamily]{Ligatures=TeX,Scale=1}
\fi
% Use upquote if available, for straight quotes in verbatim environments
\IfFileExists{upquote.sty}{\usepackage{upquote}}{}
\IfFileExists{microtype.sty}{% use microtype if available
  \usepackage[]{microtype}
  \UseMicrotypeSet[protrusion]{basicmath} % disable protrusion for tt fonts
}{}
\makeatletter
\@ifundefined{KOMAClassName}{% if non-KOMA class
  \IfFileExists{parskip.sty}{%
    \usepackage{parskip}
  }{% else
    \setlength{\parindent}{0pt}
    \setlength{\parskip}{6pt plus 2pt minus 1pt}}
}{% if KOMA class
  \KOMAoptions{parskip=half}}
\makeatother
\usepackage{xcolor}
\IfFileExists{xurl.sty}{\usepackage{xurl}}{} % add URL line breaks if available
\IfFileExists{bookmark.sty}{\usepackage{bookmark}}{\usepackage{hyperref}}
\hypersetup{
  pdftitle={Simple 3-state Digitized Partitioned Survival model in R},
  pdfauthor={The DARTH workgroup},
  hidelinks,
  pdfcreator={LaTeX via pandoc}}
\urlstyle{same} % disable monospaced font for URLs
\usepackage[margin=1in]{geometry}
\usepackage{color}
\usepackage{fancyvrb}
\newcommand{\VerbBar}{|}
\newcommand{\VERB}{\Verb[commandchars=\\\{\}]}
\DefineVerbatimEnvironment{Highlighting}{Verbatim}{commandchars=\\\{\}}
% Add ',fontsize=\small' for more characters per line
\usepackage{framed}
\definecolor{shadecolor}{RGB}{248,248,248}
\newenvironment{Shaded}{\begin{snugshade}}{\end{snugshade}}
\newcommand{\AlertTok}[1]{\textcolor[rgb]{0.94,0.16,0.16}{#1}}
\newcommand{\AnnotationTok}[1]{\textcolor[rgb]{0.56,0.35,0.01}{\textbf{\textit{#1}}}}
\newcommand{\AttributeTok}[1]{\textcolor[rgb]{0.77,0.63,0.00}{#1}}
\newcommand{\BaseNTok}[1]{\textcolor[rgb]{0.00,0.00,0.81}{#1}}
\newcommand{\BuiltInTok}[1]{#1}
\newcommand{\CharTok}[1]{\textcolor[rgb]{0.31,0.60,0.02}{#1}}
\newcommand{\CommentTok}[1]{\textcolor[rgb]{0.56,0.35,0.01}{\textit{#1}}}
\newcommand{\CommentVarTok}[1]{\textcolor[rgb]{0.56,0.35,0.01}{\textbf{\textit{#1}}}}
\newcommand{\ConstantTok}[1]{\textcolor[rgb]{0.00,0.00,0.00}{#1}}
\newcommand{\ControlFlowTok}[1]{\textcolor[rgb]{0.13,0.29,0.53}{\textbf{#1}}}
\newcommand{\DataTypeTok}[1]{\textcolor[rgb]{0.13,0.29,0.53}{#1}}
\newcommand{\DecValTok}[1]{\textcolor[rgb]{0.00,0.00,0.81}{#1}}
\newcommand{\DocumentationTok}[1]{\textcolor[rgb]{0.56,0.35,0.01}{\textbf{\textit{#1}}}}
\newcommand{\ErrorTok}[1]{\textcolor[rgb]{0.64,0.00,0.00}{\textbf{#1}}}
\newcommand{\ExtensionTok}[1]{#1}
\newcommand{\FloatTok}[1]{\textcolor[rgb]{0.00,0.00,0.81}{#1}}
\newcommand{\FunctionTok}[1]{\textcolor[rgb]{0.00,0.00,0.00}{#1}}
\newcommand{\ImportTok}[1]{#1}
\newcommand{\InformationTok}[1]{\textcolor[rgb]{0.56,0.35,0.01}{\textbf{\textit{#1}}}}
\newcommand{\KeywordTok}[1]{\textcolor[rgb]{0.13,0.29,0.53}{\textbf{#1}}}
\newcommand{\NormalTok}[1]{#1}
\newcommand{\OperatorTok}[1]{\textcolor[rgb]{0.81,0.36,0.00}{\textbf{#1}}}
\newcommand{\OtherTok}[1]{\textcolor[rgb]{0.56,0.35,0.01}{#1}}
\newcommand{\PreprocessorTok}[1]{\textcolor[rgb]{0.56,0.35,0.01}{\textit{#1}}}
\newcommand{\RegionMarkerTok}[1]{#1}
\newcommand{\SpecialCharTok}[1]{\textcolor[rgb]{0.00,0.00,0.00}{#1}}
\newcommand{\SpecialStringTok}[1]{\textcolor[rgb]{0.31,0.60,0.02}{#1}}
\newcommand{\StringTok}[1]{\textcolor[rgb]{0.31,0.60,0.02}{#1}}
\newcommand{\VariableTok}[1]{\textcolor[rgb]{0.00,0.00,0.00}{#1}}
\newcommand{\VerbatimStringTok}[1]{\textcolor[rgb]{0.31,0.60,0.02}{#1}}
\newcommand{\WarningTok}[1]{\textcolor[rgb]{0.56,0.35,0.01}{\textbf{\textit{#1}}}}
\usepackage{graphicx,grffile}
\makeatletter
\def\maxwidth{\ifdim\Gin@nat@width>\linewidth\linewidth\else\Gin@nat@width\fi}
\def\maxheight{\ifdim\Gin@nat@height>\textheight\textheight\else\Gin@nat@height\fi}
\makeatother
% Scale images if necessary, so that they will not overflow the page
% margins by default, and it is still possible to overwrite the defaults
% using explicit options in \includegraphics[width, height, ...]{}
\setkeys{Gin}{width=\maxwidth,height=\maxheight,keepaspectratio}
% Set default figure placement to htbp
\makeatletter
\def\fps@figure{htbp}
\makeatother
\setlength{\emergencystretch}{3em} % prevent overfull lines
\providecommand{\tightlist}{%
  \setlength{\itemsep}{0pt}\setlength{\parskip}{0pt}}
\setcounter{secnumdepth}{-\maxdimen} % remove section numbering

\title{Simple 3-state Digitized Partitioned Survival model in R}
\author{The DARTH workgroup}
\date{}

\begin{document}
\maketitle

Developed by the Decision Analysis in R for Technologies in Health
(DARTH) workgroup:

Fernando Alarid-Escudero, PhD (1)

Eva A. Enns, MS, PhD (2)

M.G. Myriam Hunink, MD, PhD (3,4)

Hawre J. Jalal, MD, PhD (5)

Eline M. Krijkamp, MSc (3)

Petros Pechlivanoglou, PhD (6)

Alan Yang, MSc (7)

In collaboration of:

\begin{enumerate}
\def\labelenumi{\arabic{enumi}.}
\tightlist
\item
  Drug Policy Program, Center for Research and Teaching in Economics
  (CIDE) - CONACyT, Aguascalientes, Mexico
\item
  University of Minnesota School of Public Health, Minneapolis, MN, USA
\item
  Erasmus MC, Rotterdam, The Netherlands
\item
  Harvard T.H. Chan School of Public Health, Boston, USA
\item
  University of Pittsburgh Graduate School of Public Health, Pittsburgh,
  PA, USA
\item
  The Hospital for Sick Children, Toronto and University of Toronto,
  Toronto ON, Canada
\item
  The Hospital for Sick Children, Toronto ON, Canada
\end{enumerate}

Please cite our publications when using this code:

\begin{itemize}
\item
  Jalal H, Pechlivanoglou P, Krijkamp E, Alarid-Escudero F, Enns E,
  Hunink MG. An Overview of R in Health Decision Sciences. Med Decis
  Making. 2017; 37(3): 735-746.
  \url{https://journals.sagepub.com/doi/abs/10.1177/0272989X16686559}
\item
  Krijkamp EM, Alarid-Escudero F, Enns EA, Jalal HJ, Hunink MGM,
  Pechlivanoglou P. Microsimulation modeling for health decision
  sciences using R: A tutorial. Med Decis Making. 2018;38(3):400--22.
  \url{https://journals.sagepub.com/doi/abs/10.1177/0272989X18754513}
\item
  Krijkamp EM, Alarid-Escudero F, Enns E, Pechlivanoglou P, Hunink MM,
  Jalal H. A Multidimensional Array Representation of State-Transition
  Model Dynamics. BioRxiv 670612
  2019.https://www.biorxiv.org/content/10.1101/670612v1
\end{itemize}

Copyright 2017, THE HOSPITAL FOR SICK CHILDREN AND THE COLLABORATING
INSTITUTIONS. All rights reserved in Canada, the United States and
worldwide. Copyright, trademarks, trade names and any and all associated
intellectual property are exclusively owned by THE HOSPITAL FOR Sick
CHILDREN and the collaborating institutions. These materials may be
used, reproduced, modified, distributed and adapted with proper
attribution.

\newpage

Change \texttt{eval} to \texttt{TRUE} if you want to knit this document.

\begin{Shaded}
\begin{Highlighting}[]
\KeywordTok{rm}\NormalTok{(}\DataTypeTok{list =} \KeywordTok{ls}\NormalTok{())      }\CommentTok{# clear memory (removes all the variables from the workspace)}
\end{Highlighting}
\end{Shaded}

\hypertarget{load-packages}{%
\section{01 Load packages}\label{load-packages}}

\begin{Shaded}
\begin{Highlighting}[]
\ControlFlowTok{if}\NormalTok{ (}\OperatorTok{!}\KeywordTok{require}\NormalTok{(}\StringTok{'pacman'}\NormalTok{)) }\KeywordTok{install.packages}\NormalTok{(}\StringTok{'pacman'}\NormalTok{); }\KeywordTok{library}\NormalTok{(pacman) }\CommentTok{# use this package to conveniently install other packages}
\CommentTok{# load (install if required) packages from CRAN}
\KeywordTok{p_load}\NormalTok{(}\StringTok{"here"}\NormalTok{, }\StringTok{"dplyr"}\NormalTok{, }\StringTok{"devtools"}\NormalTok{, }\StringTok{"gems"}\NormalTok{, }\StringTok{"flexsurv"}\NormalTok{, }\StringTok{"survHE"}\NormalTok{, }\StringTok{"ggplot2"}\NormalTok{, }\StringTok{"msm"}\NormalTok{, }\StringTok{"igraph"}\NormalTok{, }\StringTok{"mstate"}\NormalTok{,}
        \StringTok{"reshape2"}\NormalTok{, }\StringTok{"knitr"}\NormalTok{)  }
\end{Highlighting}
\end{Shaded}

\hypertarget{load-functions}{%
\section{02 Load functions}\label{load-functions}}

\begin{Shaded}
\begin{Highlighting}[]
\KeywordTok{source}\NormalTok{(here}\OperatorTok{::}\KeywordTok{here}\NormalTok{(}\StringTok{"functions"}\NormalTok{, }\StringTok{"functions.R"}\NormalTok{))}
\end{Highlighting}
\end{Shaded}

\hypertarget{input-model-parameters}{%
\section{03 Input model parameters}\label{input-model-parameters}}

\begin{Shaded}
\begin{Highlighting}[]
\NormalTok{v_n       <-}\StringTok{ }\KeywordTok{c}\NormalTok{(}\StringTok{"healthy"}\NormalTok{, }\StringTok{"sick"}\NormalTok{, }\StringTok{"dead"}\NormalTok{)  }\CommentTok{# state names}
\NormalTok{n_s       <-}\StringTok{ }\KeywordTok{length}\NormalTok{(v_n)                   }\CommentTok{# No of states }
\NormalTok{c_l       <-}\StringTok{ }\DecValTok{1} \OperatorTok{/}\StringTok{ }\DecValTok{12}                        \CommentTok{# cycle length (a month)}
\NormalTok{n_t       <-}\StringTok{ }\DecValTok{20}                            \CommentTok{# number of years (20 years)}
\NormalTok{times     <-}\StringTok{ }\KeywordTok{seq}\NormalTok{(}\DecValTok{0}\NormalTok{, n_t, c_l)              }\CommentTok{# the cycles in years}
\NormalTok{d_r       <-}\StringTok{ }\FloatTok{0.03}                          \CommentTok{# discount rate}
\KeywordTok{set.seed}\NormalTok{(}\DecValTok{2009}\NormalTok{)                             }\CommentTok{# set the seed}

\NormalTok{c_H       <-}\StringTok{ }\DecValTok{200}                           \CommentTok{# cost of remaining one cycle healthy}
\NormalTok{c_S       <-}\StringTok{ }\DecValTok{500}                           \CommentTok{# cost of remaining one cycle sick}
\NormalTok{c_D       <-}\StringTok{ }\DecValTok{0}                             \CommentTok{# cost of remaining one cycle dead}
\NormalTok{v_c       <-}\StringTok{ }\KeywordTok{c}\NormalTok{(c_H, c_S, c_D)              }\CommentTok{# store in a vector}

\NormalTok{u_H       <-}\StringTok{ }\FloatTok{0.75}                          \CommentTok{# utility when healthy }
\NormalTok{u_S       <-}\StringTok{ }\FloatTok{0.30}                          \CommentTok{# utility when sick}
\NormalTok{u_D       <-}\StringTok{ }\DecValTok{0}                             \CommentTok{# utility when dead}
\NormalTok{v_u       <-}\StringTok{ }\KeywordTok{c}\NormalTok{(u_H, u_S, u_D)              }\CommentTok{# store in a vector}

\NormalTok{v_dw      <-}\StringTok{ }\DecValTok{1} \OperatorTok{/}\StringTok{ }\NormalTok{(}\DecValTok{1} \OperatorTok{+}\StringTok{ }\NormalTok{d_r) }\OperatorTok{^}\StringTok{ }\NormalTok{(times)       }\CommentTok{# discount weight }
\end{Highlighting}
\end{Shaded}

\hypertarget{digitized-data}{%
\section{04 Digitized Data}\label{digitized-data}}

Use the function \texttt{digitise()} to translate the digitised OS and
PFS data into patient level information.

\begin{Shaded}
\begin{Highlighting}[]
\CommentTok{# Create IPD and KM data for the OS curves}
\KeywordTok{digitise}\NormalTok{(here}\OperatorTok{::}\KeywordTok{here}\NormalTok{(}\StringTok{"data"}\NormalTok{,}\StringTok{"OS_Examp.txt"}\NormalTok{), }
\NormalTok{         here}\OperatorTok{::}\KeywordTok{here}\NormalTok{(}\StringTok{"data"}\NormalTok{, }\StringTok{"OS_Examp_AtRisk.txt"}\NormalTok{), }
         \DataTypeTok{nevent_inp =} \DecValTok{52}\NormalTok{, }
         \DataTypeTok{km_output  =}\NormalTok{ here}\OperatorTok{::}\KeywordTok{here}\NormalTok{(}\StringTok{"output"}\NormalTok{, }\StringTok{"KMdata_OS.txt"}\NormalTok{), }
         \DataTypeTok{ipd_output =}\NormalTok{ here}\OperatorTok{::}\KeywordTok{here}\NormalTok{(}\StringTok{"output"}\NormalTok{, }\StringTok{"IPDdata_OS.txt"}\NormalTok{))}

\CommentTok{# Create IPD and KM data for the PFS curves }
\CommentTok{# your turn}

\CommentTok{# Link the IPD files across the two arms of the trial for OS and PFS}
\NormalTok{IPD_OS  <-}\StringTok{ }\KeywordTok{make.ipd}\NormalTok{(}\DataTypeTok{ipd_files =} \KeywordTok{c}\NormalTok{(here}\OperatorTok{::}\KeywordTok{here}\NormalTok{(}\StringTok{"output"}\NormalTok{, }\StringTok{"IPDdata_OS.txt"}\NormalTok{)), }\DataTypeTok{ctr =} \DecValTok{1}\NormalTok{, }
                    \DataTypeTok{var.labs  =} \KeywordTok{c}\NormalTok{(}\StringTok{"time"}\NormalTok{,}\StringTok{"event"}\NormalTok{,}\StringTok{"arm"}\NormalTok{))}
\CommentTok{# repeat for PFS, your turn}
\end{Highlighting}
\end{Shaded}

\hypertarget{analysis}{%
\section{05 Analysis}\label{analysis}}

\hypertarget{partitioned-survival-model}{%
\subsection{05.1 Partitioned Survival
model}\label{partitioned-survival-model}}

\begin{Shaded}
\begin{Highlighting}[]
\CommentTok{# your turn}
\end{Highlighting}
\end{Shaded}

Calculate total cost and QALYs per cycle.

\begin{Shaded}
\begin{Highlighting}[]
\CommentTok{# your turn}
\end{Highlighting}
\end{Shaded}

\end{document}
