% Options for packages loaded elsewhere
\PassOptionsToPackage{unicode}{hyperref}
\PassOptionsToPackage{hyphens}{url}
%
\documentclass[
]{article}
\usepackage{lmodern}
\usepackage{amssymb,amsmath}
\usepackage{ifxetex,ifluatex}
\ifnum 0\ifxetex 1\fi\ifluatex 1\fi=0 % if pdftex
  \usepackage[T1]{fontenc}
  \usepackage[utf8]{inputenc}
  \usepackage{textcomp} % provide euro and other symbols
\else % if luatex or xetex
  \usepackage{unicode-math}
  \defaultfontfeatures{Scale=MatchLowercase}
  \defaultfontfeatures[\rmfamily]{Ligatures=TeX,Scale=1}
\fi
% Use upquote if available, for straight quotes in verbatim environments
\IfFileExists{upquote.sty}{\usepackage{upquote}}{}
\IfFileExists{microtype.sty}{% use microtype if available
  \usepackage[]{microtype}
  \UseMicrotypeSet[protrusion]{basicmath} % disable protrusion for tt fonts
}{}
\makeatletter
\@ifundefined{KOMAClassName}{% if non-KOMA class
  \IfFileExists{parskip.sty}{%
    \usepackage{parskip}
  }{% else
    \setlength{\parindent}{0pt}
    \setlength{\parskip}{6pt plus 2pt minus 1pt}}
}{% if KOMA class
  \KOMAoptions{parskip=half}}
\makeatother
\usepackage{xcolor}
\IfFileExists{xurl.sty}{\usepackage{xurl}}{} % add URL line breaks if available
\IfFileExists{bookmark.sty}{\usepackage{bookmark}}{\usepackage{hyperref}}
\hypersetup{
  pdftitle={Survival Analysis - Sick-Sicker model},
  pdfauthor={The DARTH workgroup},
  hidelinks,
  pdfcreator={LaTeX via pandoc}}
\urlstyle{same} % disable monospaced font for URLs
\usepackage[margin=1in]{geometry}
\usepackage{color}
\usepackage{fancyvrb}
\newcommand{\VerbBar}{|}
\newcommand{\VERB}{\Verb[commandchars=\\\{\}]}
\DefineVerbatimEnvironment{Highlighting}{Verbatim}{commandchars=\\\{\}}
% Add ',fontsize=\small' for more characters per line
\usepackage{framed}
\definecolor{shadecolor}{RGB}{248,248,248}
\newenvironment{Shaded}{\begin{snugshade}}{\end{snugshade}}
\newcommand{\AlertTok}[1]{\textcolor[rgb]{0.94,0.16,0.16}{#1}}
\newcommand{\AnnotationTok}[1]{\textcolor[rgb]{0.56,0.35,0.01}{\textbf{\textit{#1}}}}
\newcommand{\AttributeTok}[1]{\textcolor[rgb]{0.77,0.63,0.00}{#1}}
\newcommand{\BaseNTok}[1]{\textcolor[rgb]{0.00,0.00,0.81}{#1}}
\newcommand{\BuiltInTok}[1]{#1}
\newcommand{\CharTok}[1]{\textcolor[rgb]{0.31,0.60,0.02}{#1}}
\newcommand{\CommentTok}[1]{\textcolor[rgb]{0.56,0.35,0.01}{\textit{#1}}}
\newcommand{\CommentVarTok}[1]{\textcolor[rgb]{0.56,0.35,0.01}{\textbf{\textit{#1}}}}
\newcommand{\ConstantTok}[1]{\textcolor[rgb]{0.00,0.00,0.00}{#1}}
\newcommand{\ControlFlowTok}[1]{\textcolor[rgb]{0.13,0.29,0.53}{\textbf{#1}}}
\newcommand{\DataTypeTok}[1]{\textcolor[rgb]{0.13,0.29,0.53}{#1}}
\newcommand{\DecValTok}[1]{\textcolor[rgb]{0.00,0.00,0.81}{#1}}
\newcommand{\DocumentationTok}[1]{\textcolor[rgb]{0.56,0.35,0.01}{\textbf{\textit{#1}}}}
\newcommand{\ErrorTok}[1]{\textcolor[rgb]{0.64,0.00,0.00}{\textbf{#1}}}
\newcommand{\ExtensionTok}[1]{#1}
\newcommand{\FloatTok}[1]{\textcolor[rgb]{0.00,0.00,0.81}{#1}}
\newcommand{\FunctionTok}[1]{\textcolor[rgb]{0.00,0.00,0.00}{#1}}
\newcommand{\ImportTok}[1]{#1}
\newcommand{\InformationTok}[1]{\textcolor[rgb]{0.56,0.35,0.01}{\textbf{\textit{#1}}}}
\newcommand{\KeywordTok}[1]{\textcolor[rgb]{0.13,0.29,0.53}{\textbf{#1}}}
\newcommand{\NormalTok}[1]{#1}
\newcommand{\OperatorTok}[1]{\textcolor[rgb]{0.81,0.36,0.00}{\textbf{#1}}}
\newcommand{\OtherTok}[1]{\textcolor[rgb]{0.56,0.35,0.01}{#1}}
\newcommand{\PreprocessorTok}[1]{\textcolor[rgb]{0.56,0.35,0.01}{\textit{#1}}}
\newcommand{\RegionMarkerTok}[1]{#1}
\newcommand{\SpecialCharTok}[1]{\textcolor[rgb]{0.00,0.00,0.00}{#1}}
\newcommand{\SpecialStringTok}[1]{\textcolor[rgb]{0.31,0.60,0.02}{#1}}
\newcommand{\StringTok}[1]{\textcolor[rgb]{0.31,0.60,0.02}{#1}}
\newcommand{\VariableTok}[1]{\textcolor[rgb]{0.00,0.00,0.00}{#1}}
\newcommand{\VerbatimStringTok}[1]{\textcolor[rgb]{0.31,0.60,0.02}{#1}}
\newcommand{\WarningTok}[1]{\textcolor[rgb]{0.56,0.35,0.01}{\textbf{\textit{#1}}}}
\usepackage{graphicx,grffile}
\makeatletter
\def\maxwidth{\ifdim\Gin@nat@width>\linewidth\linewidth\else\Gin@nat@width\fi}
\def\maxheight{\ifdim\Gin@nat@height>\textheight\textheight\else\Gin@nat@height\fi}
\makeatother
% Scale images if necessary, so that they will not overflow the page
% margins by default, and it is still possible to overwrite the defaults
% using explicit options in \includegraphics[width, height, ...]{}
\setkeys{Gin}{width=\maxwidth,height=\maxheight,keepaspectratio}
% Set default figure placement to htbp
\makeatletter
\def\fps@figure{htbp}
\makeatother
\setlength{\emergencystretch}{3em} % prevent overfull lines
\providecommand{\tightlist}{%
  \setlength{\itemsep}{0pt}\setlength{\parskip}{0pt}}
\setcounter{secnumdepth}{-\maxdimen} % remove section numbering

\title{Survival Analysis - Sick-Sicker model}
\author{The DARTH workgroup}
\date{}

\begin{document}
\maketitle

Developed by the Decision Analysis in R for Technologies in Health
(DARTH) workgroup:

Fernando Alarid-Escudero, PhD (1)

Eva A. Enns, MS, PhD (2)

M.G. Myriam Hunink, MD, PhD (3,4)

Hawre J. Jalal, MD, PhD (5)

Eline M. Krijkamp, MSc (3)

Petros Pechlivanoglou, PhD (6)

Alan Yang, MSc (7)

In collaboration of:

\begin{enumerate}
\def\labelenumi{\arabic{enumi}.}
\tightlist
\item
  Drug Policy Program, Center for Research and Teaching in Economics
  (CIDE) - CONACyT, Aguascalientes, Mexico
\item
  University of Minnesota School of Public Health, Minneapolis, MN, USA
\item
  Erasmus MC, Rotterdam, The Netherlands
\item
  Harvard T.H. Chan School of Public Health, Boston, USA
\item
  University of Pittsburgh Graduate School of Public Health, Pittsburgh,
  PA, USA
\item
  The Hospital for Sick Children, Toronto and University of Toronto,
  Toronto ON, Canada
\item
  The Hospital for Sick Children, Toronto ON, Canada
\end{enumerate}

Please cite our publications when using this code:

\begin{itemize}
\item
  Jalal H, Pechlivanoglou P, Krijkamp E, Alarid-Escudero F, Enns E,
  Hunink MG. An Overview of R in Health Decision Sciences. Med Decis
  Making. 2017; 37(3): 735-746.
  \url{https://journals.sagepub.com/doi/abs/10.1177/0272989X16686559}
\item
  Krijkamp EM, Alarid-Escudero F, Enns EA, Jalal HJ, Hunink MGM,
  Pechlivanoglou P. Microsimulation modeling for health decision
  sciences using R: A tutorial. Med Decis Making. 2018;38(3):400--22.
  \url{https://journals.sagepub.com/doi/abs/10.1177/0272989X18754513}
\item
  Krijkamp EM, Alarid-Escudero F, Enns E, Pechlivanoglou P, Hunink MM,
  Jalal H. A Multidimensional Array Representation of State-Transition
  Model Dynamics. BioRxiv 670612
  2019.https://www.biorxiv.org/content/10.1101/670612v1
\end{itemize}

Copyright 2017, THE HOSPITAL FOR SICK CHILDREN AND THE COLLABORATING
INSTITUTIONS. All rights reserved in Canada, the United States and
worldwide. Copyright, trademarks, trade names and any and all associated
intellectual property are exclusively owned by THE HOSPITAL FOR Sick
CHILDREN and the collaborating institutions. These materials may be
used, reproduced, modified, distributed and adapted with proper
attribution.

\newpage

Change \texttt{eval} to \texttt{TRUE} if you want to knit this document.

\begin{Shaded}
\begin{Highlighting}[]
\KeywordTok{rm}\NormalTok{(}\DataTypeTok{list =} \KeywordTok{ls}\NormalTok{())      }\CommentTok{# clear memory (removes all the variables from the workspace)}
\end{Highlighting}
\end{Shaded}

\hypertarget{load-packages}{%
\section{01 Load packages}\label{load-packages}}

\begin{Shaded}
\begin{Highlighting}[]
\ControlFlowTok{if}\NormalTok{ (}\OperatorTok{!}\KeywordTok{require}\NormalTok{(}\StringTok{'pacman'}\NormalTok{)) }\KeywordTok{install.packages}\NormalTok{(}\StringTok{'pacman'}\NormalTok{); }\KeywordTok{library}\NormalTok{(pacman) }\CommentTok{# use this package to conveniently install other packages}
\CommentTok{# load (install if required) packages from CRAN}
\KeywordTok{p_load}\NormalTok{(}\StringTok{"here"}\NormalTok{, }\StringTok{"dplyr"}\NormalTok{, }\StringTok{"devtools"}\NormalTok{, }\StringTok{"gems"}\NormalTok{, }\StringTok{"flexsurv"}\NormalTok{, }\StringTok{"survHE"}\NormalTok{, }\StringTok{"ggplot2"}\NormalTok{, }\StringTok{"msm"}\NormalTok{, }\StringTok{"igraph"}\NormalTok{, }\StringTok{"mstate"}\NormalTok{, }\StringTok{"reshape2"}\NormalTok{, }\StringTok{"knitr"}\NormalTok{, }\StringTok{"darthtools"}\NormalTok{)  }
\CommentTok{# load (install if required) packages from GitHub}
\CommentTok{# install_github("DARTH-git/dampack", force = TRUE) Uncomment if there is a newer version}
\KeywordTok{p_load_gh}\NormalTok{(}\StringTok{"DARTH-git/dampack"}\NormalTok{)}
\end{Highlighting}
\end{Shaded}

\hypertarget{load-functions}{%
\section{02 Load functions}\label{load-functions}}

\begin{Shaded}
\begin{Highlighting}[]
\KeywordTok{source}\NormalTok{(}\StringTok{"survival_functions.R"}\NormalTok{)}
\end{Highlighting}
\end{Shaded}

\hypertarget{input-model-parameters}{%
\section{03 Input model parameters}\label{input-model-parameters}}

\begin{Shaded}
\begin{Highlighting}[]
\KeywordTok{set.seed}\NormalTok{(}\DecValTok{1}\NormalTok{)                    }\CommentTok{# set the seed  }
\NormalTok{v_n    <-}\StringTok{ }\KeywordTok{c}\NormalTok{(}\StringTok{"S1"}\NormalTok{, }\StringTok{"S2"}\NormalTok{, }\StringTok{"D"}\NormalTok{)   }\CommentTok{# the model states names}

\CommentTok{# Model structure }
\NormalTok{n_t    <-}\StringTok{ }\DecValTok{30}                   \CommentTok{# time horizon, 30 cycles}
\NormalTok{c_l    <-}\StringTok{ }\DecValTok{1}
\NormalTok{d_r    <-}\StringTok{ }\FloatTok{0.03}                 \CommentTok{# discount rate of 3% per cycle}

\NormalTok{p_S2D  <-}\StringTok{ }\FloatTok{0.2}                  \CommentTok{# probability of dying in sicker state}

\NormalTok{v_init <-}\StringTok{ }\KeywordTok{c}\NormalTok{(}\StringTok{"S1"}\NormalTok{ =}\StringTok{ }\DecValTok{1}\NormalTok{,}
            \StringTok{"S2"}\NormalTok{ =}\StringTok{ }\DecValTok{0}\NormalTok{,}
            \StringTok{"D"}\NormalTok{  =}\StringTok{ }\DecValTok{0}\NormalTok{)          }\CommentTok{# initial cohort distribution (everyone allocated to the }
                               \CommentTok{# "S1" state)}

\CommentTok{# Cost inputs}
\NormalTok{c_H   <-}\StringTok{ }\DecValTok{2000}                  \CommentTok{# cost of one cycle in the healthy state}
\NormalTok{c_S1  <-}\StringTok{ }\DecValTok{4000}                  \CommentTok{# cost of one cycle in the sick state}
\NormalTok{c_S2  <-}\StringTok{ }\DecValTok{15000}                 \CommentTok{# cost of one cycle in the sicker state}
\NormalTok{c_D   <-}\StringTok{ }\DecValTok{0}                     \CommentTok{# cost of one cycle in the dead state}
\NormalTok{c_Trt <-}\StringTok{ }\DecValTok{12000}                 \CommentTok{# cost of treatment (per cycle)}

\CommentTok{# Utility inputs}
\NormalTok{u_H   <-}\StringTok{ }\DecValTok{1}                     \CommentTok{# utility when healthy }
\NormalTok{u_S1  <-}\StringTok{ }\FloatTok{0.75}                  \CommentTok{# utility when sick }
\NormalTok{u_S2  <-}\StringTok{ }\FloatTok{0.5}                   \CommentTok{# utility when sicker}
\NormalTok{u_D   <-}\StringTok{ }\DecValTok{0}                     \CommentTok{# utility when dead}
\NormalTok{u_Trt <-}\StringTok{ }\FloatTok{0.95}                  \CommentTok{# utility when sick(er) and being treated}

\NormalTok{v_dw        <-}\StringTok{ }\DecValTok{1} \OperatorTok{/}\StringTok{ }\NormalTok{((}\DecValTok{1} \OperatorTok{+}\StringTok{ }\NormalTok{d_r) }\OperatorTok{^}\StringTok{ }\NormalTok{(}\DecValTok{0}\OperatorTok{:}\NormalTok{n_t))       }\CommentTok{# discount weight }
\NormalTok{n_states    <-}\StringTok{ }\KeywordTok{length}\NormalTok{(v_n)                     }\CommentTok{# the number of health states}
\NormalTok{v_names_str <-}\StringTok{ }\KeywordTok{c}\NormalTok{(}\StringTok{"no treatment"}\NormalTok{, }\StringTok{"treatment"}\NormalTok{)  }\CommentTok{# strategy names}

\NormalTok{times       <-}\StringTok{ }\KeywordTok{seq}\NormalTok{(}\DecValTok{0}\NormalTok{, n_t, c_l)                }\CommentTok{# the cycles in years}
\end{Highlighting}
\end{Shaded}

Survival analysis component

\begin{Shaded}
\begin{Highlighting}[]
\CommentTok{# Load the Sicker data }
\NormalTok{data_long <-}\StringTok{ }\KeywordTok{read.csv}\NormalTok{(}\StringTok{"data_long_Sicker.csv"}\NormalTok{)}
\KeywordTok{head}\NormalTok{(data_long)}

\CommentTok{# Models can be fitted independently for each transition. This is more flexible!}

\CommentTok{# Create subsets for each transition}
\NormalTok{data_S1D  <-}\StringTok{ }\NormalTok{data_long[data_long}\OperatorTok{$}\NormalTok{from }\OperatorTok{==}\StringTok{"S1"} \OperatorTok{&}\StringTok{ }\NormalTok{data_long}\OperatorTok{$}\NormalTok{to }\OperatorTok{==}\StringTok{"D"}\NormalTok{, ]}
\NormalTok{data_S1S2 <-}\StringTok{ }\NormalTok{data_long[data_long}\OperatorTok{$}\NormalTok{from }\OperatorTok{==}\StringTok{"S1"} \OperatorTok{&}\StringTok{ }\NormalTok{data_long}\OperatorTok{$}\NormalTok{to }\OperatorTok{==}\StringTok{"S2"}\NormalTok{,]}

\CommentTok{# Fit independent models for each transition and pick one that fits best}
\NormalTok{fit_S1S2 <-}\StringTok{ }\KeywordTok{fit.fun}\NormalTok{(}\DataTypeTok{time =} \StringTok{"time"}\NormalTok{, }\DataTypeTok{status =} \StringTok{"status"}\NormalTok{, }\DataTypeTok{data =}\NormalTok{ data_S1S2, }\DataTypeTok{times =}\NormalTok{ times)}
\NormalTok{fit_S1D  <-}\StringTok{ }\KeywordTok{fit.fun}\NormalTok{(}\DataTypeTok{time =} \StringTok{"time"}\NormalTok{, }\DataTypeTok{status =} \StringTok{"status"}\NormalTok{, }\DataTypeTok{data =}\NormalTok{ data_S1D,  }\DataTypeTok{times =}\NormalTok{ times)}

\CommentTok{# Extract the transition probabilities from the fitted survival models}
\NormalTok{p_S1S2 <-}\StringTok{ }\KeywordTok{trans_prob}\NormalTok{(fit_S1S2, }\DataTypeTok{choose_dist =}\StringTok{"log-Logistic"}\NormalTok{, }\DataTypeTok{times=}\NormalTok{times)}\OperatorTok{$}\NormalTok{t.p}
\NormalTok{p_S1D  <-}\StringTok{ }\KeywordTok{trans_prob}\NormalTok{(fit_S1D,  }\DataTypeTok{choose_dist =}\StringTok{"Exponential"}\NormalTok{, }\DataTypeTok{times=}\NormalTok{times)}\OperatorTok{$}\NormalTok{t.p}
\end{Highlighting}
\end{Shaded}

\hypertarget{define-and-initialize-matrices-and-vectors}{%
\section{04 Define and initialize matrices and
vectors}\label{define-and-initialize-matrices-and-vectors}}

\hypertarget{cohort-trace}{%
\subsection{04.1 Cohort trace}\label{cohort-trace}}

\begin{Shaded}
\begin{Highlighting}[]
\CommentTok{# create the cohort trace}
\NormalTok{m_M <-}\StringTok{ }\KeywordTok{matrix}\NormalTok{(}\OtherTok{NA}\NormalTok{, }
              \DataTypeTok{nrow =}\NormalTok{ n_t }\OperatorTok{+}\StringTok{ }\DecValTok{1}\NormalTok{ ,  }\CommentTok{# create Markov trace (n.t + 1 because R doesn't }
                                \CommentTok{# understand cycle 0)}
              \DataTypeTok{ncol =}\NormalTok{ n_states, }
              \DataTypeTok{dimnames =} \KeywordTok{list}\NormalTok{(}\DecValTok{0}\OperatorTok{:}\NormalTok{n_t, v_n))}

\NormalTok{m_M[}\DecValTok{1}\NormalTok{, ] <-}\StringTok{ }\NormalTok{v_init  }\CommentTok{# initialize first cycle of Markov trace}
\end{Highlighting}
\end{Shaded}

\hypertarget{transition-probability-array}{%
\subsection{04.2 Transition probability
array}\label{transition-probability-array}}

\begin{Shaded}
\begin{Highlighting}[]
\CommentTok{# create the transition probability array}
\NormalTok{a_P <-}\StringTok{ }\KeywordTok{array}\NormalTok{(}\DecValTok{0}\NormalTok{,                                      }\CommentTok{# Create 3-D array}
             \DataTypeTok{dim =} \KeywordTok{c}\NormalTok{(n_states, n_states, n_t),}
             \DataTypeTok{dimnames =} \KeywordTok{list}\NormalTok{(v_n, v_n, }\DecValTok{0}\OperatorTok{:}\NormalTok{(n_t }\OperatorTok{-}\StringTok{ }\DecValTok{1}\NormalTok{))) }\CommentTok{# name the dimensions of the array }
\end{Highlighting}
\end{Shaded}

Fill in the transition probability array:

\begin{Shaded}
\begin{Highlighting}[]
\CommentTok{# from Sick}
\NormalTok{a_P[}\StringTok{"S1"}\NormalTok{, }\StringTok{"S1"}\NormalTok{, ] <-}\StringTok{ }\NormalTok{(}\DecValTok{1} \OperatorTok{-}\StringTok{ }\NormalTok{p_S1D) }\OperatorTok{*}\StringTok{ }\NormalTok{(}\DecValTok{1} \OperatorTok{-}\StringTok{ }\NormalTok{p_S1S2)}
\NormalTok{a_P[}\StringTok{"S1"}\NormalTok{, }\StringTok{"S2"}\NormalTok{, ] <-}\StringTok{ }\NormalTok{(}\DecValTok{1} \OperatorTok{-}\StringTok{ }\NormalTok{p_S1D) }\OperatorTok{*}\StringTok{ }\NormalTok{p_S1S2}
\NormalTok{a_P[}\StringTok{"S1"}\NormalTok{, }\StringTok{"D"}\NormalTok{, ]  <-}\StringTok{  }\NormalTok{p_S1D}

\CommentTok{# from Sicker}
\NormalTok{a_P[}\StringTok{"S2"}\NormalTok{, }\StringTok{"S2"}\NormalTok{, ] <-}\StringTok{ }\DecValTok{1} \OperatorTok{-}\StringTok{ }\NormalTok{p_S2D}
\NormalTok{a_P[}\StringTok{"S2"}\NormalTok{, }\StringTok{"D"}\NormalTok{, ]  <-}\StringTok{ }\NormalTok{p_S2D}

\CommentTok{# from Dead}
\NormalTok{a_P[}\StringTok{"D"}\NormalTok{, }\StringTok{"D"}\NormalTok{, ] <-}\StringTok{ }\DecValTok{1}
\end{Highlighting}
\end{Shaded}

\hypertarget{check-if-transition-array-and-probabilities-are-valid}{%
\subsection{04.3 Check if transition array and probabilities are
valid}\label{check-if-transition-array-and-probabilities-are-valid}}

\begin{Shaded}
\begin{Highlighting}[]
\CommentTok{# Check that transition probabilities are in [0, 1]}
\KeywordTok{check_transition_probability}\NormalTok{(a_P, }\DataTypeTok{verbose =} \OtherTok{TRUE}\NormalTok{)}
\CommentTok{# Check that all rows sum to 1}
\KeywordTok{check_sum_of_transition_array}\NormalTok{(a_P, }\DataTypeTok{n_states =}\NormalTok{ n_states, }\DataTypeTok{n_cycles =}\NormalTok{ n_t, }\DataTypeTok{verbose =} \OtherTok{TRUE}\NormalTok{)}
\end{Highlighting}
\end{Shaded}

\hypertarget{run-markov-model}{%
\section{05 Run Markov model}\label{run-markov-model}}

\begin{Shaded}
\begin{Highlighting}[]
\ControlFlowTok{for}\NormalTok{ (t }\ControlFlowTok{in} \DecValTok{1}\OperatorTok{:}\NormalTok{n_t)\{ }\CommentTok{# t<-1                   # loop through the number of cycles}
\NormalTok{  m_M[t }\OperatorTok{+}\StringTok{ }\DecValTok{1}\NormalTok{, ] <-}\StringTok{ }\NormalTok{m_M[t, ] }\OperatorTok\StringTok{ }\NormalTok{a_P[, , t]  }\CommentTok{# estimate the Markov trace for cycle t + 1 }
                                           \CommentTok{# using the t-th matrix from the }
                                           \CommentTok{# probability array }
\NormalTok{\}}
\KeywordTok{head}\NormalTok{(m_M)  }\CommentTok{# print the first lines of the matrix }
\KeywordTok{matplot}\NormalTok{(m_M, }\DataTypeTok{type =}\StringTok{'l'}\NormalTok{)}
\end{Highlighting}
\end{Shaded}

\hypertarget{compute-cost-effectiveness-outcomes}{%
\section{07 Compute Cost-Effectiveness
Outcomes}\label{compute-cost-effectiveness-outcomes}}

\begin{Shaded}
\begin{Highlighting}[]
\CommentTok{# Vectors with costs and utilities by treatment}
\NormalTok{v_u_notrt   <-}\StringTok{ }\KeywordTok{c}\NormalTok{(u_S1,  u_S2, u_D)}
\NormalTok{v_u_trt     <-}\StringTok{ }\KeywordTok{c}\NormalTok{( u_Trt, u_S2, u_D)}

\NormalTok{v_c_notrt   <-}\StringTok{ }\KeywordTok{c}\NormalTok{( c_S1,         c_S2,         c_D)}
\NormalTok{v_c_trt     <-}\StringTok{ }\KeywordTok{c}\NormalTok{( c_S1 }\OperatorTok{+}\StringTok{ }\NormalTok{c_Trt, c_S2 }\OperatorTok{+}\StringTok{ }\NormalTok{c_Trt, c_D)}
\end{Highlighting}
\end{Shaded}

\hypertarget{mean-costs-and-qalys-for-treatment-and-no-treatment}{%
\subsection{07.1 Mean Costs and QALYs for Treatment and NO
Treatment}\label{mean-costs-and-qalys-for-treatment-and-no-treatment}}

\begin{Shaded}
\begin{Highlighting}[]
\NormalTok{v_tu_notrt  <-}\StringTok{ }\NormalTok{m_M   }\OperatorTok\StringTok{  }\NormalTok{v_u_notrt}
\NormalTok{v_tu_trt    <-}\StringTok{ }\NormalTok{m_M   }\OperatorTok\StringTok{  }\NormalTok{v_u_trt}

\NormalTok{v_tc_notrt  <-}\StringTok{ }\NormalTok{m_M   }\OperatorTok\StringTok{  }\NormalTok{v_c_notrt}
\NormalTok{v_tc_trt    <-}\StringTok{ }\NormalTok{m_M   }\OperatorTok\StringTok{  }\NormalTok{v_c_trt }
\end{Highlighting}
\end{Shaded}

\hypertarget{discounted-mean-costs-and-qalys}{%
\subsection{07.2 Discounted Mean Costs and
QALYs}\label{discounted-mean-costs-and-qalys}}

\begin{Shaded}
\begin{Highlighting}[]
\NormalTok{tu_d_notrt  <-}\StringTok{ }\KeywordTok{t}\NormalTok{(v_tu_notrt)   }\OperatorTok\StringTok{  }\NormalTok{v_dw   }
\NormalTok{tu_d_trt    <-}\StringTok{ }\KeywordTok{t}\NormalTok{(v_tu_trt)     }\OperatorTok\StringTok{  }\NormalTok{v_dw}

\NormalTok{tc_d_notrt  <-}\StringTok{ }\KeywordTok{t}\NormalTok{(v_tc_notrt)   }\OperatorTok\StringTok{  }\NormalTok{v_dw}
\NormalTok{tc_d_trt    <-}\StringTok{ }\KeywordTok{t}\NormalTok{(v_tc_trt)     }\OperatorTok\StringTok{  }\NormalTok{v_dw}

\CommentTok{# store them into a vector}
\NormalTok{v_tc_d      <-}\StringTok{ }\KeywordTok{c}\NormalTok{(tc_d_notrt, tc_d_trt)}
\NormalTok{v_tu_d      <-}\StringTok{ }\KeywordTok{c}\NormalTok{(tu_d_notrt, tu_d_trt)}

\CommentTok{# Dataframe with discounted costs and effectiveness}
\NormalTok{df_ce       <-}\StringTok{ }\KeywordTok{data.frame}\NormalTok{(}\DataTypeTok{Strategy =}\NormalTok{ v_names_str,}
                          \DataTypeTok{Cost     =}\NormalTok{ v_tc_d,}
                          \DataTypeTok{Effect   =}\NormalTok{ v_tu_d}
\NormalTok{                          )}
\NormalTok{df_ce}
\end{Highlighting}
\end{Shaded}

\hypertarget{compute-icers-of-the-markov-model}{%
\subsection{07.3 Compute ICERs of the Markov
model}\label{compute-icers-of-the-markov-model}}

\begin{Shaded}
\begin{Highlighting}[]
\NormalTok{df_cea <-}\StringTok{ }\KeywordTok{calculate_icers}\NormalTok{(}\DataTypeTok{cost       =}\NormalTok{ df_ce}\OperatorTok{$}\NormalTok{Cost,}
                          \DataTypeTok{effect     =}\NormalTok{ df_ce}\OperatorTok{$}\NormalTok{Effect,}
                          \DataTypeTok{strategies =}\NormalTok{ df_ce}\OperatorTok{$}\NormalTok{Strategy}
\NormalTok{                          )}
\NormalTok{df_cea}
\end{Highlighting}
\end{Shaded}

\hypertarget{plot-frontier-of-the-markov-model}{%
\subsection{07.4 Plot frontier of the Markov
model}\label{plot-frontier-of-the-markov-model}}

\begin{Shaded}
\begin{Highlighting}[]
\KeywordTok{plot}\NormalTok{(df_cea, }\DataTypeTok{effect_units =} \StringTok{"QALYs"}\NormalTok{)}
\end{Highlighting}
\end{Shaded}

\end{document}
