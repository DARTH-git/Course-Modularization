% Options for packages loaded elsewhere
\PassOptionsToPackage{unicode}{hyperref}
\PassOptionsToPackage{hyphens}{url}
%
\documentclass[
]{article}
\usepackage{lmodern}
\usepackage{amssymb,amsmath}
\usepackage{ifxetex,ifluatex}
\ifnum 0\ifxetex 1\fi\ifluatex 1\fi=0 % if pdftex
  \usepackage[T1]{fontenc}
  \usepackage[utf8]{inputenc}
  \usepackage{textcomp} % provide euro and other symbols
\else % if luatex or xetex
  \usepackage{unicode-math}
  \defaultfontfeatures{Scale=MatchLowercase}
  \defaultfontfeatures[\rmfamily]{Ligatures=TeX,Scale=1}
\fi
% Use upquote if available, for straight quotes in verbatim environments
\IfFileExists{upquote.sty}{\usepackage{upquote}}{}
\IfFileExists{microtype.sty}{% use microtype if available
  \usepackage[]{microtype}
  \UseMicrotypeSet[protrusion]{basicmath} % disable protrusion for tt fonts
}{}
\makeatletter
\@ifundefined{KOMAClassName}{% if non-KOMA class
  \IfFileExists{parskip.sty}{%
    \usepackage{parskip}
  }{% else
    \setlength{\parindent}{0pt}
    \setlength{\parskip}{6pt plus 2pt minus 1pt}}
}{% if KOMA class
  \KOMAoptions{parskip=half}}
\makeatother
\usepackage{xcolor}
\IfFileExists{xurl.sty}{\usepackage{xurl}}{} % add URL line breaks if available
\IfFileExists{bookmark.sty}{\usepackage{bookmark}}{\usepackage{hyperref}}
\hypersetup{
  pdftitle={Survival Analysis - Sick-Sicker model},
  pdfauthor={The DARTH workgroup},
  hidelinks,
  pdfcreator={LaTeX via pandoc}}
\urlstyle{same} % disable monospaced font for URLs
\usepackage[margin=1in]{geometry}
\usepackage{color}
\usepackage{fancyvrb}
\newcommand{\VerbBar}{|}
\newcommand{\VERB}{\Verb[commandchars=\\\{\}]}
\DefineVerbatimEnvironment{Highlighting}{Verbatim}{commandchars=\\\{\}}
% Add ',fontsize=\small' for more characters per line
\usepackage{framed}
\definecolor{shadecolor}{RGB}{248,248,248}
\newenvironment{Shaded}{\begin{snugshade}}{\end{snugshade}}
\newcommand{\AlertTok}[1]{\textcolor[rgb]{0.94,0.16,0.16}{#1}}
\newcommand{\AnnotationTok}[1]{\textcolor[rgb]{0.56,0.35,0.01}{\textbf{\textit{#1}}}}
\newcommand{\AttributeTok}[1]{\textcolor[rgb]{0.77,0.63,0.00}{#1}}
\newcommand{\BaseNTok}[1]{\textcolor[rgb]{0.00,0.00,0.81}{#1}}
\newcommand{\BuiltInTok}[1]{#1}
\newcommand{\CharTok}[1]{\textcolor[rgb]{0.31,0.60,0.02}{#1}}
\newcommand{\CommentTok}[1]{\textcolor[rgb]{0.56,0.35,0.01}{\textit{#1}}}
\newcommand{\CommentVarTok}[1]{\textcolor[rgb]{0.56,0.35,0.01}{\textbf{\textit{#1}}}}
\newcommand{\ConstantTok}[1]{\textcolor[rgb]{0.00,0.00,0.00}{#1}}
\newcommand{\ControlFlowTok}[1]{\textcolor[rgb]{0.13,0.29,0.53}{\textbf{#1}}}
\newcommand{\DataTypeTok}[1]{\textcolor[rgb]{0.13,0.29,0.53}{#1}}
\newcommand{\DecValTok}[1]{\textcolor[rgb]{0.00,0.00,0.81}{#1}}
\newcommand{\DocumentationTok}[1]{\textcolor[rgb]{0.56,0.35,0.01}{\textbf{\textit{#1}}}}
\newcommand{\ErrorTok}[1]{\textcolor[rgb]{0.64,0.00,0.00}{\textbf{#1}}}
\newcommand{\ExtensionTok}[1]{#1}
\newcommand{\FloatTok}[1]{\textcolor[rgb]{0.00,0.00,0.81}{#1}}
\newcommand{\FunctionTok}[1]{\textcolor[rgb]{0.00,0.00,0.00}{#1}}
\newcommand{\ImportTok}[1]{#1}
\newcommand{\InformationTok}[1]{\textcolor[rgb]{0.56,0.35,0.01}{\textbf{\textit{#1}}}}
\newcommand{\KeywordTok}[1]{\textcolor[rgb]{0.13,0.29,0.53}{\textbf{#1}}}
\newcommand{\NormalTok}[1]{#1}
\newcommand{\OperatorTok}[1]{\textcolor[rgb]{0.81,0.36,0.00}{\textbf{#1}}}
\newcommand{\OtherTok}[1]{\textcolor[rgb]{0.56,0.35,0.01}{#1}}
\newcommand{\PreprocessorTok}[1]{\textcolor[rgb]{0.56,0.35,0.01}{\textit{#1}}}
\newcommand{\RegionMarkerTok}[1]{#1}
\newcommand{\SpecialCharTok}[1]{\textcolor[rgb]{0.00,0.00,0.00}{#1}}
\newcommand{\SpecialStringTok}[1]{\textcolor[rgb]{0.31,0.60,0.02}{#1}}
\newcommand{\StringTok}[1]{\textcolor[rgb]{0.31,0.60,0.02}{#1}}
\newcommand{\VariableTok}[1]{\textcolor[rgb]{0.00,0.00,0.00}{#1}}
\newcommand{\VerbatimStringTok}[1]{\textcolor[rgb]{0.31,0.60,0.02}{#1}}
\newcommand{\WarningTok}[1]{\textcolor[rgb]{0.56,0.35,0.01}{\textbf{\textit{#1}}}}
\usepackage{graphicx,grffile}
\makeatletter
\def\maxwidth{\ifdim\Gin@nat@width>\linewidth\linewidth\else\Gin@nat@width\fi}
\def\maxheight{\ifdim\Gin@nat@height>\textheight\textheight\else\Gin@nat@height\fi}
\makeatother
% Scale images if necessary, so that they will not overflow the page
% margins by default, and it is still possible to overwrite the defaults
% using explicit options in \includegraphics[width, height, ...]{}
\setkeys{Gin}{width=\maxwidth,height=\maxheight,keepaspectratio}
% Set default figure placement to htbp
\makeatletter
\def\fps@figure{htbp}
\makeatother
\setlength{\emergencystretch}{3em} % prevent overfull lines
\providecommand{\tightlist}{%
  \setlength{\itemsep}{0pt}\setlength{\parskip}{0pt}}
\setcounter{secnumdepth}{-\maxdimen} % remove section numbering

\title{Survival Analysis - Sick-Sicker model}
\author{The DARTH workgroup}
\date{}

\begin{document}
\maketitle

Developed by the Decision Analysis in R for Technologies in Health
(DARTH) workgroup:

Fernando Alarid-Escudero, PhD (1)

Eva A. Enns, MS, PhD (2)

M.G. Myriam Hunink, MD, PhD (3,4)

Hawre J. Jalal, MD, PhD (5)

Eline M. Krijkamp, MSc (3)

Petros Pechlivanoglou, PhD (6)

Alan Yang, MSc (7)

In collaboration of:

\begin{enumerate}
\def\labelenumi{\arabic{enumi}.}
\tightlist
\item
  Drug Policy Program, Center for Research and Teaching in Economics
  (CIDE) - CONACyT, Aguascalientes, Mexico
\item
  University of Minnesota School of Public Health, Minneapolis, MN, USA
\item
  Erasmus MC, Rotterdam, The Netherlands
\item
  Harvard T.H. Chan School of Public Health, Boston, USA
\item
  University of Pittsburgh Graduate School of Public Health, Pittsburgh,
  PA, USA
\item
  The Hospital for Sick Children, Toronto and University of Toronto,
  Toronto ON, Canada
\item
  The Hospital for Sick Children, Toronto ON, Canada
\end{enumerate}

Please cite our publications when using this code:

\begin{itemize}
\item
  Jalal H, Pechlivanoglou P, Krijkamp E, Alarid-Escudero F, Enns E,
  Hunink MG. An Overview of R in Health Decision Sciences. Med Decis
  Making. 2017; 37(3): 735-746.
  \url{https://journals.sagepub.com/doi/abs/10.1177/0272989X16686559}
\item
  Krijkamp EM, Alarid-Escudero F, Enns EA, Jalal HJ, Hunink MGM,
  Pechlivanoglou P. Microsimulation modeling for health decision
  sciences using R: A tutorial. Med Decis Making. 2018;38(3):400--22.
  \url{https://journals.sagepub.com/doi/abs/10.1177/0272989X18754513}
\item
  Krijkamp EM, Alarid-Escudero F, Enns E, Pechlivanoglou P, Hunink MM,
  Jalal H. A Multidimensional Array Representation of State-Transition
  Model Dynamics. BioRxiv 670612
  2019.https://www.biorxiv.org/content/10.1101/670612v1
\end{itemize}

Copyright 2017, THE HOSPITAL FOR SICK CHILDREN AND THE COLLABORATING
INSTITUTIONS. All rights reserved in Canada, the United States and
worldwide. Copyright, trademarks, trade names and any and all associated
intellectual property are exclusively owned by THE HOSPITAL FOR Sick
CHILDREN and the collaborating institutions. These materials may be
used, reproduced, modified, distributed and adapted with proper
attribution.

\newpage

Change \texttt{eval} to \texttt{TRUE} if you want to knit this document.

\begin{Shaded}
\begin{Highlighting}[]
\KeywordTok{rm}\NormalTok{(}\DataTypeTok{list =} \KeywordTok{ls}\NormalTok{())      }\CommentTok{# clear memory (removes all the variables from the workspace)}
\end{Highlighting}
\end{Shaded}

\hypertarget{load-packages}{%
\section{01 Load packages}\label{load-packages}}

\begin{Shaded}
\begin{Highlighting}[]
\ControlFlowTok{if}\NormalTok{ (}\OperatorTok{!}\KeywordTok{require}\NormalTok{(}\StringTok{'pacman'}\NormalTok{)) }\KeywordTok{install.packages}\NormalTok{(}\StringTok{'pacman'}\NormalTok{); }\KeywordTok{library}\NormalTok{(pacman) }\CommentTok{# use this package to conveniently install other packages}
\CommentTok{# load (install if required) packages from CRAN}
\KeywordTok{p_load}\NormalTok{(}\StringTok{"here"}\NormalTok{, }\StringTok{"dplyr"}\NormalTok{, }\StringTok{"devtools"}\NormalTok{, }\StringTok{"matrixStats"}\NormalTok{, }\StringTok{"scales"}\NormalTok{, }\StringTok{"ggplot2"}\NormalTok{, }\StringTok{"grid"}\NormalTok{, }\StringTok{"mgcv"}\NormalTok{, }\StringTok{"gridExtra"}\NormalTok{, }\StringTok{"gdata"}\NormalTok{, }\StringTok{"reshape2"}\NormalTok{, }\StringTok{"knitr"}\NormalTok{)       }
\CommentTok{# load (install if required) packages from GitHub}
\CommentTok{# install_github("DARTH-git/dampack", force = TRUE) Uncomment if there is a newer version}
\KeywordTok{p_load_gh}\NormalTok{(}\StringTok{"DARTH-git/dampack"}\NormalTok{) }
\end{Highlighting}
\end{Shaded}

\hypertarget{load-functions}{%
\section{02 Load functions}\label{load-functions}}

\begin{Shaded}
\begin{Highlighting}[]
\KeywordTok{source}\NormalTok{(}\KeywordTok{here}\NormalTok{(}\StringTok{"functions"}\NormalTok{, }\StringTok{"VOI_Functions.R"}\NormalTok{))}
\KeywordTok{source}\NormalTok{(}\KeywordTok{here}\NormalTok{(}\StringTok{"functions"}\NormalTok{, }\StringTok{"GA_functions.R"}\NormalTok{))}
\end{Highlighting}
\end{Shaded}

\hypertarget{input-model-parameters}{%
\section{03 Input model parameters}\label{input-model-parameters}}

\begin{Shaded}
\begin{Highlighting}[]
\CommentTok{# Load simulation file}
\CommentTok{# Read the `.csv` simulation file into `R`.}
\NormalTok{toy <-}\StringTok{ }\KeywordTok{read.csv}\NormalTok{(}\KeywordTok{here}\NormalTok{(}\StringTok{"data"}\NormalTok{, }\StringTok{"psa_sick_sicker.csv"}\NormalTok{), }\DataTypeTok{header =} \OtherTok{TRUE}\NormalTok{)[, }\DecValTok{-1}\NormalTok{]}
\NormalTok{n_sim <-}\StringTok{ }\KeywordTok{nrow}\NormalTok{(toy)}

\CommentTok{# Display first five observations of the data fram using the command `head`}
\KeywordTok{head}\NormalTok{(toy)}

\CommentTok{# Net Monetary Benefit (NMB) }
\CommentTok{# Create NMB matrix}
\NormalTok{wtp <-}\StringTok{ }\DecValTok{120000}
\NormalTok{toy}\OperatorTok{$}\NormalTok{NMB_NoTrt <-}\StringTok{ }\NormalTok{wtp }\OperatorTok{*}\StringTok{ }\NormalTok{toy}\OperatorTok{$}\NormalTok{QALY_NoTrt }\OperatorTok{-}\StringTok{ }\NormalTok{toy}\OperatorTok{$}\NormalTok{Cost_NoTrt}
\NormalTok{toy}\OperatorTok{$}\NormalTok{NMB_Trt <-}\StringTok{ }\NormalTok{wtp }\OperatorTok{*}\StringTok{ }\NormalTok{toy}\OperatorTok{$}\NormalTok{QALY_Trt }\OperatorTok{-}\StringTok{ }\NormalTok{toy}\OperatorTok{$}\NormalTok{Cost_Trt}

\NormalTok{nmb <-}\StringTok{ }\NormalTok{toy[, }\KeywordTok{c}\NormalTok{(}\StringTok{"NMB_NoTrt"}\NormalTok{, }\StringTok{"NMB_Trt"}\NormalTok{)]}
\KeywordTok{head}\NormalTok{(nmb)}

\CommentTok{# Number of Strategies}
\NormalTok{n_strategies <-}\StringTok{ }\KeywordTok{ncol}\NormalTok{(nmb)}
\NormalTok{n_strategies}

\CommentTok{# Assign name of strategies}
\NormalTok{strategies <-}\StringTok{ }\KeywordTok{c}\NormalTok{(}\StringTok{"No Trt"}\NormalTok{, }\StringTok{"Trt"}\NormalTok{)}
\KeywordTok{colnames}\NormalTok{(nmb) <-}\StringTok{ }\NormalTok{strategies}
\KeywordTok{head}\NormalTok{(nmb)}

\CommentTok{# Format data frame suitably for plotting}
\NormalTok{nmb_gg <-}\StringTok{ }\KeywordTok{melt}\NormalTok{(nmb,  }
               \DataTypeTok{variable.name =} \StringTok{"Strategy"}\NormalTok{, }
               \DataTypeTok{value.name =} \StringTok{"NMB"}\NormalTok{)}

\CommentTok{# Plot NMB for different strategies}
\CommentTok{# Faceted plot by Strategy}
\KeywordTok{ggplot}\NormalTok{(nmb_gg, }\KeywordTok{aes}\NormalTok{(}\DataTypeTok{x =}\NormalTok{ NMB}\OperatorTok{/}\DecValTok{1000}\NormalTok{)) }\OperatorTok{+}
\StringTok{  }\KeywordTok{geom_histogram}\NormalTok{(}\KeywordTok{aes}\NormalTok{(}\DataTypeTok{y =}\NormalTok{..density..), }\DataTypeTok{col=}\StringTok{"black"}\NormalTok{, }\DataTypeTok{fill =} \StringTok{"gray"}\NormalTok{) }\OperatorTok{+}
\StringTok{  }\KeywordTok{geom_density}\NormalTok{(}\DataTypeTok{color =} \StringTok{"red"}\NormalTok{) }\OperatorTok{+}
\StringTok{  }\KeywordTok{facet_wrap}\NormalTok{(}\OperatorTok{~}\StringTok{ }\NormalTok{Strategy, }\DataTypeTok{scales =} \StringTok{"free_y"}\NormalTok{) }\OperatorTok{+}
\StringTok{  }\KeywordTok{xlab}\NormalTok{(}\StringTok{"Net Monetary Benefit (NMB) x10^3"}\NormalTok{) }\OperatorTok{+}
\StringTok{  }\KeywordTok{scale_x_continuous}\NormalTok{(}\DataTypeTok{breaks =} \KeywordTok{number_ticks}\NormalTok{(}\DecValTok{5}\NormalTok{), }\DataTypeTok{labels =}\NormalTok{ dollar) }\OperatorTok{+}\StringTok{ }
\StringTok{  }\KeywordTok{scale_y_continuous}\NormalTok{(}\DataTypeTok{breaks =} \KeywordTok{number_ticks}\NormalTok{(}\DecValTok{5}\NormalTok{)) }\OperatorTok{+}\StringTok{ }
\StringTok{  }\KeywordTok{theme_bw}\NormalTok{()}
\end{Highlighting}
\end{Shaded}

\hypertarget{incremental-nmb-inmb}{%
\section{04 Incremental NMB (INMB)}\label{incremental-nmb-inmb}}

\begin{Shaded}
\begin{Highlighting}[]
\CommentTok{# Calculate INMB of NoTrt vs Trt}
\NormalTok{inmb <-}\StringTok{ }\KeywordTok{data.frame}\NormalTok{(}\DataTypeTok{Simulation =} \DecValTok{1}\OperatorTok{:}\NormalTok{n_sim,}
                   \StringTok{`}\DataTypeTok{Trt vs. No Trt}\StringTok{`}\NormalTok{ =}\StringTok{ }\NormalTok{nmb}\OperatorTok{$}\NormalTok{Trt }\OperatorTok{-}\StringTok{ }\NormalTok{nmb}\OperatorTok{$}\StringTok{`}\DataTypeTok{No Trt}\StringTok{`}\NormalTok{) }

\CommentTok{## Format data frame suitably for plotting}
\NormalTok{inmb_gg <-}\StringTok{ }\KeywordTok{melt}\NormalTok{(inmb, }\DataTypeTok{id.vars =} \StringTok{"Simulation"}\NormalTok{, }
                \DataTypeTok{variable.name =} \StringTok{"Comparison"}\NormalTok{, }
                \DataTypeTok{value.name =} \StringTok{"INMB"}\NormalTok{)}
\NormalTok{txtsize<-}\DecValTok{16}

\CommentTok{# Plot INMB}
\KeywordTok{ggplot}\NormalTok{(inmb_gg, }\KeywordTok{aes}\NormalTok{(}\DataTypeTok{x =}\NormalTok{ INMB}\OperatorTok{/}\DecValTok{1000}\NormalTok{)) }\OperatorTok{+}
\StringTok{  }\KeywordTok{geom_histogram}\NormalTok{(}\KeywordTok{aes}\NormalTok{(}\DataTypeTok{y =}\NormalTok{..density..), }\DataTypeTok{col=}\StringTok{"black"}\NormalTok{, }\DataTypeTok{fill =} \StringTok{"gray"}\NormalTok{) }\OperatorTok{+}
\StringTok{  }\KeywordTok{geom_density}\NormalTok{(}\DataTypeTok{color =} \StringTok{"red"}\NormalTok{) }\OperatorTok{+}
\StringTok{  }\KeywordTok{geom_vline}\NormalTok{(}\DataTypeTok{xintercept =} \DecValTok{0}\NormalTok{, }\DataTypeTok{col =} \DecValTok{4}\NormalTok{, }\DataTypeTok{size =} \FloatTok{1.5}\NormalTok{, }\DataTypeTok{linetype =} \StringTok{"dashed"}\NormalTok{) }\OperatorTok{+}
\StringTok{  }\KeywordTok{facet_wrap}\NormalTok{(}\OperatorTok{~}\StringTok{ }\NormalTok{Comparison, }\DataTypeTok{scales =} \StringTok{"free_y"}\NormalTok{) }\OperatorTok{+}
\StringTok{  }\KeywordTok{xlab}\NormalTok{(}\StringTok{"Incremental Net Monetary Benefit (INMB) in thousand $"}\NormalTok{) }\OperatorTok{+}
\StringTok{  }\KeywordTok{scale_x_continuous}\NormalTok{(}\DataTypeTok{breaks =} \KeywordTok{number_ticks}\NormalTok{(}\DecValTok{5}\NormalTok{), }\DataTypeTok{limits =} \KeywordTok{c}\NormalTok{(}\OperatorTok{-}\DecValTok{100}\NormalTok{, }\DecValTok{100}\NormalTok{)) }\OperatorTok{+}\StringTok{ }
\StringTok{  }\KeywordTok{scale_y_continuous}\NormalTok{(}\DataTypeTok{breaks =} \KeywordTok{number_ticks}\NormalTok{(}\DecValTok{5}\NormalTok{)) }\OperatorTok{+}\StringTok{ }
\StringTok{  }\KeywordTok{theme_bw}\NormalTok{(}\DataTypeTok{base_size =}\NormalTok{ txtsize)}
\end{Highlighting}
\end{Shaded}

\hypertarget{loss-matrix}{%
\section{05 Loss Matrix}\label{loss-matrix}}

\begin{Shaded}
\begin{Highlighting}[]
\CommentTok{# Find optimal strategy (d*) based on the highest expected NMB}
\NormalTok{d_star <-}\StringTok{ }\KeywordTok{which.max}\NormalTok{(}\KeywordTok{colMeans}\NormalTok{(nmb))}
\NormalTok{d_star}

\CommentTok{# Compute Loss matrix iterating over all strategies}
\NormalTok{loss <-}\StringTok{ }\KeywordTok{as.matrix}\NormalTok{(nmb }\OperatorTok{-}\StringTok{ }\NormalTok{nmb[, d_star])}
\KeywordTok{head}\NormalTok{(loss)}
\end{Highlighting}
\end{Shaded}

\hypertarget{evpi}{%
\section{06 EVPI}\label{evpi}}

\begin{Shaded}
\begin{Highlighting}[]
\CommentTok{# Find maximum loss overall strategies at each state of the world }
\CommentTok{# (i.e., PSA sample)}
\NormalTok{max_loss_i <-}\StringTok{ }\KeywordTok{rowMaxs}\NormalTok{(loss)}
\KeywordTok{head}\NormalTok{(max_loss_i)}

\CommentTok{# Average across all states of the world}
\NormalTok{evpi <-}\StringTok{ }\KeywordTok{mean}\NormalTok{(max_loss_i)}
\NormalTok{evpi}
\end{Highlighting}
\end{Shaded}

\hypertarget{evppi}{%
\section{07 EVPPI}\label{evppi}}

\begin{Shaded}
\begin{Highlighting}[]
\CommentTok{# Matrix with parameters}
\NormalTok{x <-}\StringTok{ }\NormalTok{toy[, }\KeywordTok{c}\NormalTok{(}\DecValTok{1}\OperatorTok{:}\DecValTok{14}\NormalTok{)]}
\KeywordTok{head}\NormalTok{(x)}

\CommentTok{# Number and names of parameters}
\NormalTok{n_params <-}\StringTok{ }\KeywordTok{ncol}\NormalTok{(x)}
\NormalTok{n_params}
\NormalTok{names_params <-}\StringTok{ }\KeywordTok{colnames}\NormalTok{(x) }
\NormalTok{names_params}

\CommentTok{# Histogram of parameters}
\CommentTok{# Format data suitably for plotting}
\NormalTok{params <-}\StringTok{ }\KeywordTok{melt}\NormalTok{(x, }\DataTypeTok{variable.name =} \StringTok{"Parameter"}\NormalTok{)}
\KeywordTok{head}\NormalTok{(params)}
\CommentTok{# Make parameter names as factors (helps with plotting formatting)}
\NormalTok{params}\OperatorTok{$}\NormalTok{Parameter <-}\StringTok{ }\KeywordTok{factor}\NormalTok{(params}\OperatorTok{$}\NormalTok{Parameter, }
                           \DataTypeTok{levels =}\NormalTok{ names_params, }
                           \DataTypeTok{labels =}\NormalTok{ names_params)}

\CommentTok{# Facet plot of parameter distributions}
\KeywordTok{ggplot}\NormalTok{(params, }\KeywordTok{aes}\NormalTok{(}\DataTypeTok{x =}\NormalTok{ value)) }\OperatorTok{+}\StringTok{ }
\StringTok{  }\KeywordTok{geom_histogram}\NormalTok{(}\KeywordTok{aes}\NormalTok{(}\DataTypeTok{y =}\NormalTok{..density..), }\DataTypeTok{col=}\StringTok{"black"}\NormalTok{, }\DataTypeTok{fill =} \StringTok{"gray"}\NormalTok{) }\OperatorTok{+}
\StringTok{  }\KeywordTok{geom_density}\NormalTok{(}\DataTypeTok{color =} \StringTok{"red"}\NormalTok{) }\OperatorTok{+}
\StringTok{  }\KeywordTok{facet_wrap}\NormalTok{(}\OperatorTok{~}\StringTok{ }\NormalTok{Parameter, }\DataTypeTok{scales =} \StringTok{"free"}\NormalTok{) }\OperatorTok{+}
\StringTok{  }\KeywordTok{scale_x_continuous}\NormalTok{(}\DataTypeTok{breaks =} \KeywordTok{number_ticks}\NormalTok{(}\DecValTok{5}\NormalTok{)) }\OperatorTok{+}\StringTok{ }
\StringTok{  }\KeywordTok{scale_y_continuous}\NormalTok{(}\DataTypeTok{breaks =} \KeywordTok{number_ticks}\NormalTok{(}\DecValTok{5}\NormalTok{)) }\OperatorTok{+}\StringTok{ }
\StringTok{  }\KeywordTok{theme_bw}\NormalTok{(}\DataTypeTok{base_size =} \DecValTok{14}\NormalTok{)}
\end{Highlighting}
\end{Shaded}

Construct Spline metamodel.

\begin{Shaded}
\begin{Highlighting}[]
\CommentTok{# Splines}
\CommentTok{# Initialize EVPPI vector }
\NormalTok{evppi_splines <-}\StringTok{ }\KeywordTok{matrix}\NormalTok{(}\DecValTok{0}\NormalTok{, n_params)}
\NormalTok{lmm1 <-}\StringTok{ }\KeywordTok{vector}\NormalTok{(}\StringTok{"list"}\NormalTok{, n_params)}
\NormalTok{lmm2 <-}\StringTok{ }\KeywordTok{vector}\NormalTok{(}\StringTok{"list"}\NormalTok{, n_params)}
\ControlFlowTok{for}\NormalTok{ (p }\ControlFlowTok{in} \DecValTok{1}\OperatorTok{:}\NormalTok{n_params)\{ }\CommentTok{# p <- 1}
  \KeywordTok{print}\NormalTok{(}\KeywordTok{paste}\NormalTok{(}\StringTok{"Computing EVPPI of parameter"}\NormalTok{, names_params[p]))}
  \CommentTok{# Estimate Splines}
\NormalTok{  lmm1[[p]] <-}\StringTok{ }\KeywordTok{gam}\NormalTok{(loss[, }\DecValTok{1}\NormalTok{] }\OperatorTok{~}\StringTok{ }\KeywordTok{s}\NormalTok{(x[, p]))}
\NormalTok{  lmm2[[p]] <-}\StringTok{ }\KeywordTok{gam}\NormalTok{(loss[, }\DecValTok{2}\NormalTok{] }\OperatorTok{~}\StringTok{ }\KeywordTok{s}\NormalTok{(x[, p]))}
  
  \CommentTok{# Predict Loss using Splines}
\NormalTok{  Lhat_splines <-}\StringTok{ }\KeywordTok{cbind}\NormalTok{(lmm1[[p]]}\OperatorTok{$}\NormalTok{fitted, lmm2[[p]]}\OperatorTok{$}\NormalTok{fitted)}
  
  \CommentTok{# Compute EVPPI}
\NormalTok{  evppi_splines[p] <-}\StringTok{ }\KeywordTok{mean}\NormalTok{(}\KeywordTok{rowMaxs}\NormalTok{(Lhat_splines))}
\NormalTok{\}}

\CommentTok{# Ploting EVPPI using of order polynomial}
\NormalTok{evppi_splines_gg <-}\StringTok{ }\KeywordTok{data.frame}\NormalTok{(}\DataTypeTok{Parameter =}\NormalTok{ names_params, }\DataTypeTok{EVPPI =}\NormalTok{ evppi_splines)}
\NormalTok{evppi_splines_gg}\OperatorTok{$}\NormalTok{Parameter <-}\StringTok{ }\KeywordTok{factor}\NormalTok{((evppi_splines_gg}\OperatorTok{$}\NormalTok{Parameter), }
                                     \DataTypeTok{levels =}\NormalTok{ names_params[}\KeywordTok{order}\NormalTok{(evppi_splines_gg}\OperatorTok{$}\NormalTok{EVPPI, }
                                                                 \DataTypeTok{decreasing =} \OtherTok{TRUE}\NormalTok{)])}

\CommentTok{# Plot EVPPI using ggplot2 package}
\KeywordTok{ggplot}\NormalTok{(}\DataTypeTok{data =}\NormalTok{ evppi_splines_gg, }\KeywordTok{aes}\NormalTok{(}\DataTypeTok{x =}\NormalTok{ Parameter, }\DataTypeTok{y =}\NormalTok{ EVPPI)) }\OperatorTok{+}
\StringTok{  }\KeywordTok{geom_bar}\NormalTok{(}\DataTypeTok{stat =} \StringTok{"identity"}\NormalTok{) }\OperatorTok{+}
\StringTok{  }\KeywordTok{ylab}\NormalTok{(}\StringTok{"EVPPI ($)"}\NormalTok{) }\OperatorTok{+}
\StringTok{  }\KeywordTok{scale_y_continuous}\NormalTok{(}\DataTypeTok{breaks =} \KeywordTok{number_ticks}\NormalTok{(}\DecValTok{6}\NormalTok{), }\DataTypeTok{labels =}\NormalTok{ comma) }\OperatorTok{+}
\StringTok{  }\KeywordTok{theme_bw}\NormalTok{(}\DataTypeTok{base_size =} \DecValTok{14}\NormalTok{)}
\end{Highlighting}
\end{Shaded}

\hypertarget{evsi}{%
\section{08 EVSI}\label{evsi}}

\begin{Shaded}
\begin{Highlighting}[]
\CommentTok{# Select parameters with positive EVPPI}
\NormalTok{sel_params <-}\StringTok{ }\KeywordTok{c}\NormalTok{(}\DecValTok{3}\NormalTok{, }\DecValTok{4}\NormalTok{, }\DecValTok{10}\NormalTok{, }\DecValTok{12}\NormalTok{, }\DecValTok{14}\NormalTok{)}
\NormalTok{n_params <-}\StringTok{ }\KeywordTok{length}\NormalTok{(sel_params)}
\CommentTok{# Effective (prior) Sample size}
\NormalTok{n0 <-}\StringTok{ }\KeywordTok{numeric}\NormalTok{(}\KeywordTok{length}\NormalTok{(sel_params))}
\NormalTok{n0[}\DecValTok{1}\NormalTok{] <-}\StringTok{ }\DecValTok{84}\OperatorTok{+}\DecValTok{800}    \CommentTok{# p.S1S2 ~ Beta(84, 800)}
\NormalTok{n0[}\DecValTok{2}\NormalTok{] <-}\StringTok{ }\DecValTok{10}\OperatorTok{+}\DecValTok{2000}   \CommentTok{# p.HD ~ Beta(10,2000)}
\NormalTok{n0[}\DecValTok{3}\NormalTok{] <-}\StringTok{ }\FloatTok{73.5}      \CommentTok{# cTrt ~ Gamma(73.5, 163.3) -> likelihood ~ Exponential}
\NormalTok{n0[}\DecValTok{4}\NormalTok{] <-}\StringTok{ }\DecValTok{50}        \CommentTok{# u.S1 ~ N(.75, .02 / sqrt(50) = )}
\NormalTok{n0[}\DecValTok{5}\NormalTok{] <-}\StringTok{ }\DecValTok{20}        \CommentTok{# u.Trt ~ N(.95, 0.02)}

\NormalTok{n <-}\StringTok{ }\KeywordTok{c}\NormalTok{(}\DecValTok{0}\NormalTok{, }\DecValTok{100}\NormalTok{, }\KeywordTok{seq}\NormalTok{(}\DecValTok{200}\NormalTok{, }\DecValTok{2000}\NormalTok{, }\DataTypeTok{by =} \DecValTok{200}\NormalTok{))}
\NormalTok{n_samples <-}\StringTok{ }\KeywordTok{length}\NormalTok{(n)}

\CommentTok{# Each parameter individually (only assuming linear relationship)}
\CommentTok{# Initialize EVSI matrix for each parameters}
\NormalTok{evsi <-}\StringTok{ }\KeywordTok{data.frame}\NormalTok{(}\DataTypeTok{N =}\NormalTok{ n, }\KeywordTok{matrix}\NormalTok{(}\DecValTok{0}\NormalTok{, }\DataTypeTok{nrow =}\NormalTok{ n_samples, }\DataTypeTok{ncol =}\NormalTok{ n_params))}

\CommentTok{# Name columns of EVPSI matrix with parameter names}
\KeywordTok{colnames}\NormalTok{(evsi)[}\OperatorTok{-}\DecValTok{1}\NormalTok{] <-}\StringTok{ }\NormalTok{names_params[sel_params]}

\CommentTok{# Compute EVSI for all parameters separately}
\ControlFlowTok{for}\NormalTok{ (p }\ControlFlowTok{in} \DecValTok{1}\OperatorTok{:}\NormalTok{n_params)\{ }\CommentTok{# p <- 1}
  \KeywordTok{print}\NormalTok{(}\KeywordTok{paste}\NormalTok{(}\StringTok{"Computing EVSI of parameter"}\NormalTok{, names_params[p]))}
  \CommentTok{# Update loss based on gaussian approximation for each sample of interest}
  \ControlFlowTok{for}\NormalTok{ (nSamp }\ControlFlowTok{in} \DecValTok{1}\OperatorTok{:}\NormalTok{n_samples)\{ }\CommentTok{# nSamp <- 10}
\NormalTok{    Ltilde1 <-}\StringTok{ }\KeywordTok{predict.ga}\NormalTok{(lmm1[[sel_params[p]]], }\DataTypeTok{n =}\NormalTok{ n[nSamp], }\DataTypeTok{n0 =}\NormalTok{ n0[p])}
\NormalTok{    Ltilde2 <-}\StringTok{ }\KeywordTok{predict.ga}\NormalTok{(lmm2[[sel_params[p]]], }\DataTypeTok{n =}\NormalTok{ n[nSamp], }\DataTypeTok{n0 =}\NormalTok{ n0[p])}
    \CommentTok{## Combine losses into one matrix}
\NormalTok{    Ltilde <-}\StringTok{ }\KeywordTok{cbind}\NormalTok{(Ltilde1, Ltilde2)}
    \CommentTok{### Apply EVSI equation}
\NormalTok{    evsi[nSamp, p}\OperatorTok{+}\DecValTok{1}\NormalTok{] <-}\StringTok{ }\KeywordTok{mean}\NormalTok{(}\KeywordTok{rowMaxs}\NormalTok{(Ltilde))}
\NormalTok{  \}}
\NormalTok{\}}

\CommentTok{# Plotting EVSI}
\CommentTok{# Create EVSI data frame for plotting in decreasing order of EVPPI}
\NormalTok{evsi_gg <-}\StringTok{ }\KeywordTok{melt}\NormalTok{(evsi, }\DataTypeTok{id.vars =} \StringTok{"N"}\NormalTok{, }
                \DataTypeTok{variable.name =} \StringTok{"Parameter"}\NormalTok{, }
                \DataTypeTok{value.name =} \StringTok{"evsi"}\NormalTok{)}
\NormalTok{evsi_gg}\OperatorTok{$}\NormalTok{Parameter <-}\StringTok{ }\KeywordTok{factor}\NormalTok{((evsi_gg}\OperatorTok{$}\NormalTok{Parameter), }
                            \DataTypeTok{levels =}\NormalTok{ names_params[}\KeywordTok{order}\NormalTok{(evppi_splines_gg}\OperatorTok{$}\NormalTok{EVPPI, }\DataTypeTok{decreasing =} \OtherTok{TRUE}\NormalTok{)])}

\CommentTok{# Plot evsi using ggplot2 package}
\KeywordTok{ggplot}\NormalTok{(evsi_gg, }\KeywordTok{aes}\NormalTok{(}\DataTypeTok{x =}\NormalTok{ N, }\DataTypeTok{y =}\NormalTok{ evsi)) }\OperatorTok{+}\StringTok{  }\CommentTok{# colour = Parameter}
\StringTok{  }\KeywordTok{geom_line}\NormalTok{() }\OperatorTok{+}
\StringTok{  }\KeywordTok{geom_point}\NormalTok{() }\OperatorTok{+}
\StringTok{  }\KeywordTok{facet_wrap}\NormalTok{(}\OperatorTok{~}\StringTok{ }\NormalTok{Parameter) }\OperatorTok{+}\StringTok{  }\CommentTok{# scales = "free_y"}
\StringTok{  }\KeywordTok{ggtitle}\NormalTok{(}\StringTok{"Expected Value of Sample Information (EVSI)"}\NormalTok{) }\OperatorTok{+}
\StringTok{  }\KeywordTok{xlab}\NormalTok{(}\StringTok{"Sample size (n)"}\NormalTok{) }\OperatorTok{+}
\StringTok{  }\KeywordTok{ylab}\NormalTok{(}\StringTok{"$"}\NormalTok{) }\OperatorTok{+}
\StringTok{  }\KeywordTok{scale_x_continuous}\NormalTok{(}\DataTypeTok{breaks =} \KeywordTok{number_ticks}\NormalTok{(}\DecValTok{5}\NormalTok{)) }\OperatorTok{+}\StringTok{ }
\StringTok{  }\KeywordTok{scale_y_continuous}\NormalTok{(}\DataTypeTok{breaks =} \KeywordTok{number_ticks}\NormalTok{(}\DecValTok{6}\NormalTok{), }\DataTypeTok{labels =}\NormalTok{ dollar) }\OperatorTok{+}\StringTok{ }
\StringTok{  }\KeywordTok{theme_bw}\NormalTok{(}\DataTypeTok{base_size =} \DecValTok{14}\NormalTok{)}

\CommentTok{# Adding EVPPI }
\KeywordTok{ggplot}\NormalTok{(evsi_gg, }\KeywordTok{aes}\NormalTok{(}\DataTypeTok{x =}\NormalTok{ N, }\DataTypeTok{y =}\NormalTok{ evsi)) }\OperatorTok{+}\StringTok{  }\CommentTok{# colour = Parameter}
\StringTok{  }\KeywordTok{geom_line}\NormalTok{(}\KeywordTok{aes}\NormalTok{(}\DataTypeTok{linetype =} \StringTok{"EVSI"}\NormalTok{)) }\OperatorTok{+}
\StringTok{  }\KeywordTok{geom_point}\NormalTok{() }\OperatorTok{+}
\StringTok{  }\KeywordTok{facet_wrap}\NormalTok{(}\OperatorTok{~}\StringTok{ }\NormalTok{Parameter) }\OperatorTok{+}\StringTok{  }\CommentTok{# scales = "free_y"}
\StringTok{  }\KeywordTok{geom_hline}\NormalTok{(}\KeywordTok{aes}\NormalTok{(}\DataTypeTok{yintercept =}\NormalTok{ EVPPI, }\DataTypeTok{linetype =} \StringTok{"EVPPI"}\NormalTok{), }\DataTypeTok{data =}\NormalTok{ evppi_splines_gg[sel_params, ]) }\OperatorTok{+}
\StringTok{  }\KeywordTok{scale_linetype_manual}\NormalTok{(}\DataTypeTok{name=}\StringTok{""}\NormalTok{, }
                        \DataTypeTok{values =} \KeywordTok{c}\NormalTok{(}\StringTok{"EVSI"}\NormalTok{ =}\StringTok{ "solid"}\NormalTok{, }\StringTok{"EVPPI"}\NormalTok{ =}\StringTok{ "dashed"}\NormalTok{)) }\OperatorTok{+}
\StringTok{  }\KeywordTok{xlab}\NormalTok{(}\StringTok{"Sample size (n)"}\NormalTok{) }\OperatorTok{+}
\StringTok{  }\KeywordTok{ylab}\NormalTok{(}\StringTok{"$"}\NormalTok{) }\OperatorTok{+}
\StringTok{  }\CommentTok{#ggtitle("Expected Value of Sample Information (EVSI)") +}
\StringTok{  }\KeywordTok{scale_x_continuous}\NormalTok{(}\DataTypeTok{breaks =} \KeywordTok{number_ticks}\NormalTok{(}\DecValTok{5}\NormalTok{)) }\OperatorTok{+}\StringTok{ }
\StringTok{  }\KeywordTok{scale_y_continuous}\NormalTok{(}\DataTypeTok{breaks =} \KeywordTok{number_ticks}\NormalTok{(}\DecValTok{6}\NormalTok{), }\DataTypeTok{labels =}\NormalTok{ dollar) }\OperatorTok{+}\StringTok{ }
\StringTok{  }\KeywordTok{theme_bw}\NormalTok{(}\DataTypeTok{base_size =} \DecValTok{14}\NormalTok{)}
\end{Highlighting}
\end{Shaded}

\end{document}
